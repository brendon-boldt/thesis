% \appendix

\section{Algorithm}%
\label{app:alg}

\subsection{Candidate generation}%
\label{app:candidate}

For simplicity's sake (and inductive bias), we limit the candidate generation functions to all non-empty substrings for forms and all non-empty subsets for meanings.
Nevertheless, we could extend form candidate generation to non-contiguous forms to detect non-concatenative morphology (e.g., the form ``\mbox{x.z}'' matching ``\mbox{xyz}'' and ``\mbox{xwz}'').
In fact, we could could use arbitrary regular expressions to represent forms (or meanings) such as ``\^{}..x'' or ``x+'' to represent absolute position and optional repetitions, respectively.
We could consider empty forms and empty meanings to explicitly identify forms and meanings which do not have mappings (as opposed to implicitly not including them in the morphology).

Of course, part of the difficulty of extending the complexity of the candidate generation is that it expands the already (sometimes intractably) large search space.
One method of making this tractable, though, is adding heuristics that determine which form candidates should be considered rather than considering every possible candidate.


\subsection{Ambiguous pair application}%
\label{app:ambig-app}

In some cases of applying a morpheme to record in the dataset, there are multiple applications possible.
Say we have the utterance ``x y z x y'' meaning $\{A,B\}$ and we want to apply the morpheme (``x y'', $\{A\}$).
The form matches two substrings in the utterance, so there are two possible ways to apply the morpheme.
As a heuristic for selecting the best application, CSAR break ties by selecting the substring least likely to be a morpheme (as determined by the morpheme weights).
Going back to the above example, if it is the case the morpheme (``z x y'', $\{B\}$) has a higher weight than (``x y z'', $\{B\}$), then CSAR will apply (``x y'', $\{A\}$) to the first instance of ``x y'' instead of the second.

This search can be very computationally expensive since it can entail going through a large number of morpheme candidates.
Thus for the experiments with human language data, we do not perform this search and select the best form quasirandomly.

\subsection{Heuristic optimizations}%
\label{app:opt}

Below we include a summary of heuristic optimizations available in CSAR\@:
\smallskip
\begin{description}[nosep,itemindent=-1em]
  \item[max input records] Only consider a certain number of records from the input data; $20\,000$ for machine translation, image captions, and ShapeWorld.
  \item[max inventory size] Stop after inducing a certain number of morphemes; $300$ for image captions and machine translation settings.
  \item[\textit{n}-gram semantics] Treat complete meanings as ordered and generate meaning candidates identically to forms (i.e., as $n$-grams); used for machine translation data where the ``meanings'' are sentences.
  \item[max form/meaning size] Only consider form/meaning candidates up to a certain size; $3$ for machine translation (form and meaning) and image captions (form only), $2$ for image captions meaning.
  \item[no search best form] When ablating a form with multiple matches in an utterance, do not search for best form, simply choose it randomly; no search for image captions and machine translation.
  \item[form/meaning vocabulary size] Only consider the most common form/meaning candidates; $100\,000$ for image captions and machine translation.
  \item[token vocabulary size] Only consider the most common form/meaning tokens and ignore an form meaning candidates which contain an unknown token; $1000$ for image captions and $500$ for machine translation.
  \item[co-occurrence threshold] Zero out any co-occurrences which fall below a certain threshold (e.g., if a form and meaning candidate only occur once, treat it as never co-occurring); $1$ for ShapeWorld, $10$ for image captions, and $100$ for machine translation.
\end{description}


\section{Empirical Validation}%
\label{app:emp-val}


\subsection{Procedural dataset hyperparameters}%
\label{app:synth-hparams}

The following hyperparameters were used for generating the procedural datasets.
Each dataset uses $4$ attributes and $4$ values except for the sparse setting which uses $8$ independent values.
\begin{description}[nosep,itemindent=-1em]
  \item[Synonymy] $\{1,3\}$; forms per meaning
  \item[Polysemy] $\{0, 0.15\}$; proportion of meanings mapped to an already-used form
  \item[Multi-token forms] $\{\{1\}, \{1,2,3,4\}\}$; possible tokens per form
  \item[Vocab size] $\{10, 50\}$; only applies to non-unity multi-token forms
  \item[Sparse meanings] $\{\text{true}, \text{false}\}$
  \item[Distribution imbalance]  $\{\text{true}, \text{false}\}$; non-uniform distribution is based on the ramp function, i.e., probability of given value for an attribute is proportional to its $\text{index} +1$.
  \item[Dataset size] $\{50, 500\}$
  \item[Noise forms] $\{0, 0.5\}$; $1-p$ of parameter of geometric distribution
  \item[Shuffle form] $\{\text{true}, \text{false}\}$
  \item[Non-compositionality] $\{\text{true}, \text{false}\}$
  \item[Random seeds] $3$ per hyperparameter setting
\end{description}
Non-unity polysemy and synonymy rates for the non-compositional dataset implementation were not implemented and are excluded from the above grid.

\subsection{Tokenizer vocabulary size}%
\label{app:vocab-size}

The heuristic for the tokenizer vocabulary size is as follows:
\begin{align}
  |V| &= \left\lfloor
    \frac{|\mathcal T_\text{meaning}|}{|\mathcal R|} \sum_{r\in\mathcal R} \frac{|r_\text{form}|}{|r_\text{meaning}|}
  \right\rfloor
    + |\mathcal T_\text{form}|
  ,
\end{align}
where
  $\mathcal T_\text{meaning}$ is the set of all meaning tokens in the dataset (likewise for $\mathcal T_\text{form}$),
  $\mathcal R$ is the multiset of records in dataset,
  $r_\text{form}$ is the particular form (utterance) for an individual record (likewise for $r_\text{meaning}$.
This heuristic can be interpreted as the mean form tokens per meaning tokens times the number unique meaning tokens added to the number of unique form tokens (since each of them will automatically be included in the vocabulary).

\subsection{Additional procedural dataset results}%
\label{app:proc-table}

\Cref{tab:proc-all} shows all results of baseline methods on the procedural datasets.
\Cref{fig:baseline-exact} visualizes the results of the baseline methods with exact $F_1$ score.

\begin{table*}
\centering
\begin{tabular}{lrrrrrrr}
\toprule
 & CSAR & IBM Model 1 & IBM Model 3 & Morfessor & BPE & ULM & Records \\
\midrule
Exact $F_1$, form & 0.868 & 0.616 & 0.595 & 0.827 & 0.624 & 0.670 & 0.133 \\
Fuzzy $F_1$, form & 0.960 & 0.899 & 0.893 & 0.949 & 0.890 & 0.891 & 0.637 \\
Fuzzy prec., form & 0.954 & 0.855 & 0.850 & 0.933 & 0.852 & 0.853 & 0.597 \\
Fuzzy recall, form & 0.967 & 0.952 & 0.946 & 0.967 & 0.934 & 0.938 & 0.701 \\
Exact $F_1$ & 0.788 & 0.375 & 0.379 & 0.000 & 0.000 & 0.000 & 0.101 \\
Fuzzy $F_1$ & 0.899 & 0.721 & 0.726 & 0.000 & 0.000 & 0.000 & 0.441 \\
Fuzzy prec. & 0.881 & 0.641 & 0.640 & 0.000 & 0.000 & 0.000 & 0.390 \\
Fuzzy recall & 0.921 & 0.855 & 0.866 & 0.000 & 0.000 & 0.000 & 0.543 \\
\bottomrule
\end{tabular}

\caption{Results of baseline methods on the procedural datasets.}%
\label{tab:proc-all}
\end{table*}

\begin{figure}
\centering
%% Creator: Matplotlib, PGF backend
%%
%% To include the figure in your LaTeX document, write
%%   \input{<filename>.pgf}
%%
%% Make sure the required packages are loaded in your preamble
%%   \usepackage{pgf}
%%
%% Also ensure that all the required font packages are loaded; for instance,
%% the lmodern package is sometimes necessary when using math font.
%%   \usepackage{lmodern}
%%
%% Figures using additional raster images can only be included by \input if
%% they are in the same directory as the main LaTeX file. For loading figures
%% from other directories you can use the `import` package
%%   \usepackage{import}
%%
%% and then include the figures with
%%   \import{<path to file>}{<filename>.pgf}
%%
%% Matplotlib used the following preamble
%%   \def\mathdefault#1{#1}
%%   \everymath=\expandafter{\the\everymath\displaystyle}
%%   \IfFileExists{scrextend.sty}{
%%     \usepackage[fontsize=8.160000pt]{scrextend}
%%   }{
%%     \renewcommand{\normalsize}{\fontsize{8.160000}{9.792000}\selectfont}
%%     \normalsize
%%   }
%%   
%%   \makeatletter\@ifpackageloaded{underscore}{}{\usepackage[strings]{underscore}}\makeatother
%%
\begingroup%
\makeatletter%
\begin{pgfpicture}%
\pgfpathrectangle{\pgfpointorigin}{\pgfqpoint{2.942522in}{3.100000in}}%
\pgfusepath{use as bounding box, clip}%
\begin{pgfscope}%
\pgfsetbuttcap%
\pgfsetmiterjoin%
\definecolor{currentfill}{rgb}{1.000000,1.000000,1.000000}%
\pgfsetfillcolor{currentfill}%
\pgfsetlinewidth{0.000000pt}%
\definecolor{currentstroke}{rgb}{1.000000,1.000000,1.000000}%
\pgfsetstrokecolor{currentstroke}%
\pgfsetdash{}{0pt}%
\pgfpathmoveto{\pgfqpoint{0.000000in}{0.000000in}}%
\pgfpathlineto{\pgfqpoint{2.942522in}{0.000000in}}%
\pgfpathlineto{\pgfqpoint{2.942522in}{3.100000in}}%
\pgfpathlineto{\pgfqpoint{0.000000in}{3.100000in}}%
\pgfpathlineto{\pgfqpoint{0.000000in}{0.000000in}}%
\pgfpathclose%
\pgfusepath{fill}%
\end{pgfscope}%
\begin{pgfscope}%
\pgfsetbuttcap%
\pgfsetmiterjoin%
\definecolor{currentfill}{rgb}{1.000000,1.000000,1.000000}%
\pgfsetfillcolor{currentfill}%
\pgfsetlinewidth{0.000000pt}%
\definecolor{currentstroke}{rgb}{0.000000,0.000000,0.000000}%
\pgfsetstrokecolor{currentstroke}%
\pgfsetstrokeopacity{0.000000}%
\pgfsetdash{}{0pt}%
\pgfpathmoveto{\pgfqpoint{0.793996in}{0.443060in}}%
\pgfpathlineto{\pgfqpoint{2.835606in}{0.443060in}}%
\pgfpathlineto{\pgfqpoint{2.835606in}{3.020667in}}%
\pgfpathlineto{\pgfqpoint{0.793996in}{3.020667in}}%
\pgfpathlineto{\pgfqpoint{0.793996in}{0.443060in}}%
\pgfpathclose%
\pgfusepath{fill}%
\end{pgfscope}%
\begin{pgfscope}%
\pgfpathrectangle{\pgfqpoint{0.793996in}{0.443060in}}{\pgfqpoint{2.041610in}{2.577607in}}%
\pgfusepath{clip}%
\pgfsetroundcap%
\pgfsetroundjoin%
\pgfsetlinewidth{0.803000pt}%
\definecolor{currentstroke}{rgb}{0.800000,0.800000,0.800000}%
\pgfsetstrokecolor{currentstroke}%
\pgfsetdash{}{0pt}%
\pgfpathmoveto{\pgfqpoint{0.793996in}{0.443060in}}%
\pgfpathlineto{\pgfqpoint{0.793996in}{3.020667in}}%
\pgfusepath{stroke}%
\end{pgfscope}%
\begin{pgfscope}%
\definecolor{textcolor}{rgb}{0.150000,0.150000,0.150000}%
\pgfsetstrokecolor{textcolor}%
\pgfsetfillcolor{textcolor}%
\pgftext[x=0.793996in,y=0.327782in,,top]{\color{textcolor}{\sffamily\fontsize{7.480000}{8.976000}\selectfont\catcode`\^=\active\def^{\ifmmode\sp\else\^{}\fi}\catcode`\%=\active\def%{\%}0.0}}%
\end{pgfscope}%
\begin{pgfscope}%
\pgfpathrectangle{\pgfqpoint{0.793996in}{0.443060in}}{\pgfqpoint{2.041610in}{2.577607in}}%
\pgfusepath{clip}%
\pgfsetroundcap%
\pgfsetroundjoin%
\pgfsetlinewidth{0.803000pt}%
\definecolor{currentstroke}{rgb}{0.800000,0.800000,0.800000}%
\pgfsetstrokecolor{currentstroke}%
\pgfsetdash{}{0pt}%
\pgfpathmoveto{\pgfqpoint{1.202318in}{0.443060in}}%
\pgfpathlineto{\pgfqpoint{1.202318in}{3.020667in}}%
\pgfusepath{stroke}%
\end{pgfscope}%
\begin{pgfscope}%
\definecolor{textcolor}{rgb}{0.150000,0.150000,0.150000}%
\pgfsetstrokecolor{textcolor}%
\pgfsetfillcolor{textcolor}%
\pgftext[x=1.202318in,y=0.327782in,,top]{\color{textcolor}{\sffamily\fontsize{7.480000}{8.976000}\selectfont\catcode`\^=\active\def^{\ifmmode\sp\else\^{}\fi}\catcode`\%=\active\def%{\%}0.2}}%
\end{pgfscope}%
\begin{pgfscope}%
\pgfpathrectangle{\pgfqpoint{0.793996in}{0.443060in}}{\pgfqpoint{2.041610in}{2.577607in}}%
\pgfusepath{clip}%
\pgfsetroundcap%
\pgfsetroundjoin%
\pgfsetlinewidth{0.803000pt}%
\definecolor{currentstroke}{rgb}{0.800000,0.800000,0.800000}%
\pgfsetstrokecolor{currentstroke}%
\pgfsetdash{}{0pt}%
\pgfpathmoveto{\pgfqpoint{1.610640in}{0.443060in}}%
\pgfpathlineto{\pgfqpoint{1.610640in}{3.020667in}}%
\pgfusepath{stroke}%
\end{pgfscope}%
\begin{pgfscope}%
\definecolor{textcolor}{rgb}{0.150000,0.150000,0.150000}%
\pgfsetstrokecolor{textcolor}%
\pgfsetfillcolor{textcolor}%
\pgftext[x=1.610640in,y=0.327782in,,top]{\color{textcolor}{\sffamily\fontsize{7.480000}{8.976000}\selectfont\catcode`\^=\active\def^{\ifmmode\sp\else\^{}\fi}\catcode`\%=\active\def%{\%}0.4}}%
\end{pgfscope}%
\begin{pgfscope}%
\pgfpathrectangle{\pgfqpoint{0.793996in}{0.443060in}}{\pgfqpoint{2.041610in}{2.577607in}}%
\pgfusepath{clip}%
\pgfsetroundcap%
\pgfsetroundjoin%
\pgfsetlinewidth{0.803000pt}%
\definecolor{currentstroke}{rgb}{0.800000,0.800000,0.800000}%
\pgfsetstrokecolor{currentstroke}%
\pgfsetdash{}{0pt}%
\pgfpathmoveto{\pgfqpoint{2.018962in}{0.443060in}}%
\pgfpathlineto{\pgfqpoint{2.018962in}{3.020667in}}%
\pgfusepath{stroke}%
\end{pgfscope}%
\begin{pgfscope}%
\definecolor{textcolor}{rgb}{0.150000,0.150000,0.150000}%
\pgfsetstrokecolor{textcolor}%
\pgfsetfillcolor{textcolor}%
\pgftext[x=2.018962in,y=0.327782in,,top]{\color{textcolor}{\sffamily\fontsize{7.480000}{8.976000}\selectfont\catcode`\^=\active\def^{\ifmmode\sp\else\^{}\fi}\catcode`\%=\active\def%{\%}0.6}}%
\end{pgfscope}%
\begin{pgfscope}%
\pgfpathrectangle{\pgfqpoint{0.793996in}{0.443060in}}{\pgfqpoint{2.041610in}{2.577607in}}%
\pgfusepath{clip}%
\pgfsetroundcap%
\pgfsetroundjoin%
\pgfsetlinewidth{0.803000pt}%
\definecolor{currentstroke}{rgb}{0.800000,0.800000,0.800000}%
\pgfsetstrokecolor{currentstroke}%
\pgfsetdash{}{0pt}%
\pgfpathmoveto{\pgfqpoint{2.427284in}{0.443060in}}%
\pgfpathlineto{\pgfqpoint{2.427284in}{3.020667in}}%
\pgfusepath{stroke}%
\end{pgfscope}%
\begin{pgfscope}%
\definecolor{textcolor}{rgb}{0.150000,0.150000,0.150000}%
\pgfsetstrokecolor{textcolor}%
\pgfsetfillcolor{textcolor}%
\pgftext[x=2.427284in,y=0.327782in,,top]{\color{textcolor}{\sffamily\fontsize{7.480000}{8.976000}\selectfont\catcode`\^=\active\def^{\ifmmode\sp\else\^{}\fi}\catcode`\%=\active\def%{\%}0.8}}%
\end{pgfscope}%
\begin{pgfscope}%
\pgfpathrectangle{\pgfqpoint{0.793996in}{0.443060in}}{\pgfqpoint{2.041610in}{2.577607in}}%
\pgfusepath{clip}%
\pgfsetroundcap%
\pgfsetroundjoin%
\pgfsetlinewidth{0.803000pt}%
\definecolor{currentstroke}{rgb}{0.800000,0.800000,0.800000}%
\pgfsetstrokecolor{currentstroke}%
\pgfsetdash{}{0pt}%
\pgfpathmoveto{\pgfqpoint{2.835606in}{0.443060in}}%
\pgfpathlineto{\pgfqpoint{2.835606in}{3.020667in}}%
\pgfusepath{stroke}%
\end{pgfscope}%
\begin{pgfscope}%
\definecolor{textcolor}{rgb}{0.150000,0.150000,0.150000}%
\pgfsetstrokecolor{textcolor}%
\pgfsetfillcolor{textcolor}%
\pgftext[x=2.835606in,y=0.327782in,,top]{\color{textcolor}{\sffamily\fontsize{7.480000}{8.976000}\selectfont\catcode`\^=\active\def^{\ifmmode\sp\else\^{}\fi}\catcode`\%=\active\def%{\%}1.0}}%
\end{pgfscope}%
\begin{pgfscope}%
\definecolor{textcolor}{rgb}{0.150000,0.150000,0.150000}%
\pgfsetstrokecolor{textcolor}%
\pgfsetfillcolor{textcolor}%
\pgftext[x=1.814801in,y=0.179973in,,top]{\color{textcolor}{\sffamily\fontsize{8.160000}{9.792000}\selectfont\catcode`\^=\active\def^{\ifmmode\sp\else\^{}\fi}\catcode`\%=\active\def%{\%}F1-score}}%
\end{pgfscope}%
\begin{pgfscope}%
\definecolor{textcolor}{rgb}{0.150000,0.150000,0.150000}%
\pgfsetstrokecolor{textcolor}%
\pgfsetfillcolor{textcolor}%
\pgftext[x=0.401855in, y=2.800502in, left, base]{\color{textcolor}{\sffamily\fontsize{7.480000}{8.976000}\selectfont\catcode`\^=\active\def^{\ifmmode\sp\else\^{}\fi}\catcode`\%=\active\def%{\%}CSAR}}%
\end{pgfscope}%
\begin{pgfscope}%
\definecolor{textcolor}{rgb}{0.150000,0.150000,0.150000}%
\pgfsetstrokecolor{textcolor}%
\pgfsetfillcolor{textcolor}%
\pgftext[x=0.478213in, y=2.478399in, left, base]{\color{textcolor}{\sffamily\fontsize{7.480000}{8.976000}\selectfont\catcode`\^=\active\def^{\ifmmode\sp\else\^{}\fi}\catcode`\%=\active\def%{\%}IBM}}%
\end{pgfscope}%
\begin{pgfscope}%
\definecolor{textcolor}{rgb}{0.150000,0.150000,0.150000}%
\pgfsetstrokecolor{textcolor}%
\pgfsetfillcolor{textcolor}%
\pgftext[x=0.299836in, y=2.371726in, left, base]{\color{textcolor}{\sffamily\fontsize{7.480000}{8.976000}\selectfont\catcode`\^=\active\def^{\ifmmode\sp\else\^{}\fi}\catcode`\%=\active\def%{\%}Model 1}}%
\end{pgfscope}%
\begin{pgfscope}%
\definecolor{textcolor}{rgb}{0.150000,0.150000,0.150000}%
\pgfsetstrokecolor{textcolor}%
\pgfsetfillcolor{textcolor}%
\pgftext[x=0.478213in, y=2.110170in, left, base]{\color{textcolor}{\sffamily\fontsize{7.480000}{8.976000}\selectfont\catcode`\^=\active\def^{\ifmmode\sp\else\^{}\fi}\catcode`\%=\active\def%{\%}IBM}}%
\end{pgfscope}%
\begin{pgfscope}%
\definecolor{textcolor}{rgb}{0.150000,0.150000,0.150000}%
\pgfsetstrokecolor{textcolor}%
\pgfsetfillcolor{textcolor}%
\pgftext[x=0.299836in, y=2.003497in, left, base]{\color{textcolor}{\sffamily\fontsize{7.480000}{8.976000}\selectfont\catcode`\^=\active\def^{\ifmmode\sp\else\^{}\fi}\catcode`\%=\active\def%{\%}Model 3}}%
\end{pgfscope}%
\begin{pgfscope}%
\definecolor{textcolor}{rgb}{0.150000,0.150000,0.150000}%
\pgfsetstrokecolor{textcolor}%
\pgfsetfillcolor{textcolor}%
\pgftext[x=0.235529in, y=1.695814in, left, base]{\color{textcolor}{\sffamily\fontsize{7.480000}{8.976000}\selectfont\catcode`\^=\active\def^{\ifmmode\sp\else\^{}\fi}\catcode`\%=\active\def%{\%}Morfessor}}%
\end{pgfscope}%
\begin{pgfscope}%
\definecolor{textcolor}{rgb}{0.150000,0.150000,0.150000}%
\pgfsetstrokecolor{textcolor}%
\pgfsetfillcolor{textcolor}%
\pgftext[x=0.468447in, y=1.327584in, left, base]{\color{textcolor}{\sffamily\fontsize{7.480000}{8.976000}\selectfont\catcode`\^=\active\def^{\ifmmode\sp\else\^{}\fi}\catcode`\%=\active\def%{\%}BPE}}%
\end{pgfscope}%
\begin{pgfscope}%
\definecolor{textcolor}{rgb}{0.150000,0.150000,0.150000}%
\pgfsetstrokecolor{textcolor}%
\pgfsetfillcolor{textcolor}%
\pgftext[x=0.446631in, y=0.959355in, left, base]{\color{textcolor}{\sffamily\fontsize{7.480000}{8.976000}\selectfont\catcode`\^=\active\def^{\ifmmode\sp\else\^{}\fi}\catcode`\%=\active\def%{\%}ULM}}%
\end{pgfscope}%
\begin{pgfscope}%
\definecolor{textcolor}{rgb}{0.150000,0.150000,0.150000}%
\pgfsetstrokecolor{textcolor}%
\pgfsetfillcolor{textcolor}%
\pgftext[x=0.320094in, y=0.591125in, left, base]{\color{textcolor}{\sffamily\fontsize{7.480000}{8.976000}\selectfont\catcode`\^=\active\def^{\ifmmode\sp\else\^{}\fi}\catcode`\%=\active\def%{\%}Records}}%
\end{pgfscope}%
\begin{pgfscope}%
\definecolor{textcolor}{rgb}{0.150000,0.150000,0.150000}%
\pgfsetstrokecolor{textcolor}%
\pgfsetfillcolor{textcolor}%
\pgftext[x=0.179973in,y=1.731863in,,bottom,rotate=90.000000]{\color{textcolor}{\sffamily\fontsize{8.160000}{9.792000}\selectfont\catcode`\^=\active\def^{\ifmmode\sp\else\^{}\fi}\catcode`\%=\active\def%{\%}Model}}%
\end{pgfscope}%
\begin{pgfscope}%
\pgfpathrectangle{\pgfqpoint{0.793996in}{0.443060in}}{\pgfqpoint{2.041610in}{2.577607in}}%
\pgfusepath{clip}%
\pgfsetbuttcap%
\pgfsetroundjoin%
\definecolor{currentfill}{rgb}{0.848437,0.867532,0.899724}%
\pgfsetfillcolor{currentfill}%
\pgfsetlinewidth{0.602250pt}%
\definecolor{currentstroke}{rgb}{0.296471,0.296471,0.296471}%
\pgfsetstrokecolor{currentstroke}%
\pgfsetdash{}{0pt}%
\pgfpathmoveto{\pgfqpoint{1.307085in}{2.910485in}}%
\pgfpathlineto{\pgfqpoint{1.420989in}{2.910485in}}%
\pgfpathlineto{\pgfqpoint{1.420989in}{2.909910in}}%
\pgfpathlineto{\pgfqpoint{1.307085in}{2.909910in}}%
\pgfpathlineto{\pgfqpoint{1.307085in}{2.910485in}}%
\pgfpathclose%
\pgfusepath{stroke,fill}%
\end{pgfscope}%
\begin{pgfscope}%
\pgfpathrectangle{\pgfqpoint{0.793996in}{0.443060in}}{\pgfqpoint{2.041610in}{2.577607in}}%
\pgfusepath{clip}%
\pgfsetbuttcap%
\pgfsetroundjoin%
\definecolor{currentfill}{rgb}{0.825117,0.848522,0.887698}%
\pgfsetfillcolor{currentfill}%
\pgfsetlinewidth{0.602250pt}%
\definecolor{currentstroke}{rgb}{0.296471,0.296471,0.296471}%
\pgfsetstrokecolor{currentstroke}%
\pgfsetdash{}{0pt}%
\pgfpathmoveto{\pgfqpoint{1.420989in}{2.910773in}}%
\pgfpathlineto{\pgfqpoint{1.430791in}{2.910773in}}%
\pgfpathlineto{\pgfqpoint{1.430791in}{2.909622in}}%
\pgfpathlineto{\pgfqpoint{1.420989in}{2.909622in}}%
\pgfpathlineto{\pgfqpoint{1.420989in}{2.910773in}}%
\pgfpathclose%
\pgfusepath{stroke,fill}%
\end{pgfscope}%
\begin{pgfscope}%
\pgfpathrectangle{\pgfqpoint{0.793996in}{0.443060in}}{\pgfqpoint{2.041610in}{2.577607in}}%
\pgfusepath{clip}%
\pgfsetbuttcap%
\pgfsetroundjoin%
\definecolor{currentfill}{rgb}{0.792469,0.821908,0.870863}%
\pgfsetfillcolor{currentfill}%
\pgfsetlinewidth{0.602250pt}%
\definecolor{currentstroke}{rgb}{0.296471,0.296471,0.296471}%
\pgfsetstrokecolor{currentstroke}%
\pgfsetdash{}{0pt}%
\pgfpathmoveto{\pgfqpoint{1.430791in}{2.911349in}}%
\pgfpathlineto{\pgfqpoint{1.443362in}{2.911349in}}%
\pgfpathlineto{\pgfqpoint{1.443362in}{2.909047in}}%
\pgfpathlineto{\pgfqpoint{1.430791in}{2.909047in}}%
\pgfpathlineto{\pgfqpoint{1.430791in}{2.911349in}}%
\pgfpathclose%
\pgfusepath{stroke,fill}%
\end{pgfscope}%
\begin{pgfscope}%
\pgfpathrectangle{\pgfqpoint{0.793996in}{0.443060in}}{\pgfqpoint{2.041610in}{2.577607in}}%
\pgfusepath{clip}%
\pgfsetbuttcap%
\pgfsetroundjoin%
\definecolor{currentfill}{rgb}{0.755157,0.791493,0.851622}%
\pgfsetfillcolor{currentfill}%
\pgfsetlinewidth{0.602250pt}%
\definecolor{currentstroke}{rgb}{0.296471,0.296471,0.296471}%
\pgfsetstrokecolor{currentstroke}%
\pgfsetdash{}{0pt}%
\pgfpathmoveto{\pgfqpoint{1.443362in}{2.912499in}}%
\pgfpathlineto{\pgfqpoint{1.507630in}{2.912499in}}%
\pgfpathlineto{\pgfqpoint{1.507630in}{2.907896in}}%
\pgfpathlineto{\pgfqpoint{1.443362in}{2.907896in}}%
\pgfpathlineto{\pgfqpoint{1.443362in}{2.912499in}}%
\pgfpathclose%
\pgfusepath{stroke,fill}%
\end{pgfscope}%
\begin{pgfscope}%
\pgfpathrectangle{\pgfqpoint{0.793996in}{0.443060in}}{\pgfqpoint{2.041610in}{2.577607in}}%
\pgfusepath{clip}%
\pgfsetbuttcap%
\pgfsetroundjoin%
\definecolor{currentfill}{rgb}{0.706185,0.751573,0.826368}%
\pgfsetfillcolor{currentfill}%
\pgfsetlinewidth{0.602250pt}%
\definecolor{currentstroke}{rgb}{0.296471,0.296471,0.296471}%
\pgfsetstrokecolor{currentstroke}%
\pgfsetdash{}{0pt}%
\pgfpathmoveto{\pgfqpoint{1.507630in}{2.914801in}}%
\pgfpathlineto{\pgfqpoint{1.552232in}{2.914801in}}%
\pgfpathlineto{\pgfqpoint{1.552232in}{2.905595in}}%
\pgfpathlineto{\pgfqpoint{1.507630in}{2.905595in}}%
\pgfpathlineto{\pgfqpoint{1.507630in}{2.914801in}}%
\pgfpathclose%
\pgfusepath{stroke,fill}%
\end{pgfscope}%
\begin{pgfscope}%
\pgfpathrectangle{\pgfqpoint{0.793996in}{0.443060in}}{\pgfqpoint{2.041610in}{2.577607in}}%
\pgfusepath{clip}%
\pgfsetbuttcap%
\pgfsetroundjoin%
\definecolor{currentfill}{rgb}{0.643221,0.700246,0.793900}%
\pgfsetfillcolor{currentfill}%
\pgfsetlinewidth{0.602250pt}%
\definecolor{currentstroke}{rgb}{0.296471,0.296471,0.296471}%
\pgfsetstrokecolor{currentstroke}%
\pgfsetdash{}{0pt}%
\pgfpathmoveto{\pgfqpoint{1.552232in}{2.919404in}}%
\pgfpathlineto{\pgfqpoint{1.635977in}{2.919404in}}%
\pgfpathlineto{\pgfqpoint{1.635977in}{2.900992in}}%
\pgfpathlineto{\pgfqpoint{1.552232in}{2.900992in}}%
\pgfpathlineto{\pgfqpoint{1.552232in}{2.919404in}}%
\pgfpathclose%
\pgfusepath{stroke,fill}%
\end{pgfscope}%
\begin{pgfscope}%
\pgfpathrectangle{\pgfqpoint{0.793996in}{0.443060in}}{\pgfqpoint{2.041610in}{2.577607in}}%
\pgfusepath{clip}%
\pgfsetbuttcap%
\pgfsetroundjoin%
\definecolor{currentfill}{rgb}{0.566266,0.637515,0.754216}%
\pgfsetfillcolor{currentfill}%
\pgfsetlinewidth{0.602250pt}%
\definecolor{currentstroke}{rgb}{0.296471,0.296471,0.296471}%
\pgfsetstrokecolor{currentstroke}%
\pgfsetdash{}{0pt}%
\pgfpathmoveto{\pgfqpoint{1.635977in}{2.928609in}}%
\pgfpathlineto{\pgfqpoint{1.812517in}{2.928609in}}%
\pgfpathlineto{\pgfqpoint{1.812517in}{2.891786in}}%
\pgfpathlineto{\pgfqpoint{1.635977in}{2.891786in}}%
\pgfpathlineto{\pgfqpoint{1.635977in}{2.928609in}}%
\pgfpathclose%
\pgfusepath{stroke,fill}%
\end{pgfscope}%
\begin{pgfscope}%
\pgfpathrectangle{\pgfqpoint{0.793996in}{0.443060in}}{\pgfqpoint{2.041610in}{2.577607in}}%
\pgfusepath{clip}%
\pgfsetbuttcap%
\pgfsetroundjoin%
\definecolor{currentfill}{rgb}{0.468322,0.557674,0.703709}%
\pgfsetfillcolor{currentfill}%
\pgfsetlinewidth{0.602250pt}%
\definecolor{currentstroke}{rgb}{0.296471,0.296471,0.296471}%
\pgfsetstrokecolor{currentstroke}%
\pgfsetdash{}{0pt}%
\pgfpathmoveto{\pgfqpoint{1.812517in}{2.947021in}}%
\pgfpathlineto{\pgfqpoint{2.085120in}{2.947021in}}%
\pgfpathlineto{\pgfqpoint{2.085120in}{2.873375in}}%
\pgfpathlineto{\pgfqpoint{1.812517in}{2.873375in}}%
\pgfpathlineto{\pgfqpoint{1.812517in}{2.947021in}}%
\pgfpathclose%
\pgfusepath{stroke,fill}%
\end{pgfscope}%
\begin{pgfscope}%
\pgfpathrectangle{\pgfqpoint{0.793996in}{0.443060in}}{\pgfqpoint{2.041610in}{2.577607in}}%
\pgfusepath{clip}%
\pgfsetbuttcap%
\pgfsetroundjoin%
\definecolor{currentfill}{rgb}{0.347059,0.458824,0.641176}%
\pgfsetfillcolor{currentfill}%
\pgfsetlinewidth{0.602250pt}%
\definecolor{currentstroke}{rgb}{0.296471,0.296471,0.296471}%
\pgfsetstrokecolor{currentstroke}%
\pgfsetdash{}{0pt}%
\pgfpathmoveto{\pgfqpoint{2.085120in}{2.983844in}}%
\pgfpathlineto{\pgfqpoint{2.802413in}{2.983844in}}%
\pgfpathlineto{\pgfqpoint{2.802413in}{2.836552in}}%
\pgfpathlineto{\pgfqpoint{2.085120in}{2.836552in}}%
\pgfpathlineto{\pgfqpoint{2.085120in}{2.983844in}}%
\pgfpathclose%
\pgfusepath{stroke,fill}%
\end{pgfscope}%
\begin{pgfscope}%
\pgfpathrectangle{\pgfqpoint{0.793996in}{0.443060in}}{\pgfqpoint{2.041610in}{2.577607in}}%
\pgfusepath{clip}%
\pgfsetbuttcap%
\pgfsetroundjoin%
\definecolor{currentfill}{rgb}{0.468322,0.557674,0.703709}%
\pgfsetfillcolor{currentfill}%
\pgfsetlinewidth{0.602250pt}%
\definecolor{currentstroke}{rgb}{0.296471,0.296471,0.296471}%
\pgfsetstrokecolor{currentstroke}%
\pgfsetdash{}{0pt}%
\pgfpathmoveto{\pgfqpoint{2.802413in}{2.947021in}}%
\pgfpathlineto{\pgfqpoint{2.835606in}{2.947021in}}%
\pgfpathlineto{\pgfqpoint{2.835606in}{2.873375in}}%
\pgfpathlineto{\pgfqpoint{2.802413in}{2.873375in}}%
\pgfpathlineto{\pgfqpoint{2.802413in}{2.947021in}}%
\pgfpathclose%
\pgfusepath{stroke,fill}%
\end{pgfscope}%
\begin{pgfscope}%
\pgfpathrectangle{\pgfqpoint{0.793996in}{0.443060in}}{\pgfqpoint{2.041610in}{2.577607in}}%
\pgfusepath{clip}%
\pgfsetbuttcap%
\pgfsetroundjoin%
\definecolor{currentfill}{rgb}{0.566266,0.637515,0.754216}%
\pgfsetfillcolor{currentfill}%
\pgfsetlinewidth{0.602250pt}%
\definecolor{currentstroke}{rgb}{0.296471,0.296471,0.296471}%
\pgfsetstrokecolor{currentstroke}%
\pgfsetdash{}{0pt}%
\pgfpathmoveto{\pgfqpoint{2.835606in}{2.928609in}}%
\pgfpathlineto{\pgfqpoint{2.835606in}{2.928609in}}%
\pgfpathlineto{\pgfqpoint{2.835606in}{2.891786in}}%
\pgfpathlineto{\pgfqpoint{2.835606in}{2.891786in}}%
\pgfpathlineto{\pgfqpoint{2.835606in}{2.928609in}}%
\pgfpathclose%
\pgfusepath{stroke,fill}%
\end{pgfscope}%
\begin{pgfscope}%
\pgfpathrectangle{\pgfqpoint{0.793996in}{0.443060in}}{\pgfqpoint{2.041610in}{2.577607in}}%
\pgfusepath{clip}%
\pgfsetbuttcap%
\pgfsetroundjoin%
\definecolor{currentfill}{rgb}{0.643221,0.700246,0.793900}%
\pgfsetfillcolor{currentfill}%
\pgfsetlinewidth{0.602250pt}%
\definecolor{currentstroke}{rgb}{0.296471,0.296471,0.296471}%
\pgfsetstrokecolor{currentstroke}%
\pgfsetdash{}{0pt}%
\pgfpathmoveto{\pgfqpoint{2.835606in}{2.919404in}}%
\pgfpathlineto{\pgfqpoint{2.835606in}{2.919404in}}%
\pgfpathlineto{\pgfqpoint{2.835606in}{2.900992in}}%
\pgfpathlineto{\pgfqpoint{2.835606in}{2.900992in}}%
\pgfpathlineto{\pgfqpoint{2.835606in}{2.919404in}}%
\pgfpathclose%
\pgfusepath{stroke,fill}%
\end{pgfscope}%
\begin{pgfscope}%
\pgfpathrectangle{\pgfqpoint{0.793996in}{0.443060in}}{\pgfqpoint{2.041610in}{2.577607in}}%
\pgfusepath{clip}%
\pgfsetbuttcap%
\pgfsetroundjoin%
\definecolor{currentfill}{rgb}{0.706185,0.751573,0.826368}%
\pgfsetfillcolor{currentfill}%
\pgfsetlinewidth{0.602250pt}%
\definecolor{currentstroke}{rgb}{0.296471,0.296471,0.296471}%
\pgfsetstrokecolor{currentstroke}%
\pgfsetdash{}{0pt}%
\pgfpathmoveto{\pgfqpoint{2.835606in}{2.914801in}}%
\pgfpathlineto{\pgfqpoint{2.835606in}{2.914801in}}%
\pgfpathlineto{\pgfqpoint{2.835606in}{2.905595in}}%
\pgfpathlineto{\pgfqpoint{2.835606in}{2.905595in}}%
\pgfpathlineto{\pgfqpoint{2.835606in}{2.914801in}}%
\pgfpathclose%
\pgfusepath{stroke,fill}%
\end{pgfscope}%
\begin{pgfscope}%
\pgfpathrectangle{\pgfqpoint{0.793996in}{0.443060in}}{\pgfqpoint{2.041610in}{2.577607in}}%
\pgfusepath{clip}%
\pgfsetbuttcap%
\pgfsetroundjoin%
\definecolor{currentfill}{rgb}{0.755157,0.791493,0.851622}%
\pgfsetfillcolor{currentfill}%
\pgfsetlinewidth{0.602250pt}%
\definecolor{currentstroke}{rgb}{0.296471,0.296471,0.296471}%
\pgfsetstrokecolor{currentstroke}%
\pgfsetdash{}{0pt}%
\pgfpathmoveto{\pgfqpoint{2.835606in}{2.912499in}}%
\pgfpathlineto{\pgfqpoint{2.835606in}{2.912499in}}%
\pgfpathlineto{\pgfqpoint{2.835606in}{2.907896in}}%
\pgfpathlineto{\pgfqpoint{2.835606in}{2.907896in}}%
\pgfpathlineto{\pgfqpoint{2.835606in}{2.912499in}}%
\pgfpathclose%
\pgfusepath{stroke,fill}%
\end{pgfscope}%
\begin{pgfscope}%
\pgfpathrectangle{\pgfqpoint{0.793996in}{0.443060in}}{\pgfqpoint{2.041610in}{2.577607in}}%
\pgfusepath{clip}%
\pgfsetbuttcap%
\pgfsetroundjoin%
\definecolor{currentfill}{rgb}{0.792469,0.821908,0.870863}%
\pgfsetfillcolor{currentfill}%
\pgfsetlinewidth{0.602250pt}%
\definecolor{currentstroke}{rgb}{0.296471,0.296471,0.296471}%
\pgfsetstrokecolor{currentstroke}%
\pgfsetdash{}{0pt}%
\pgfpathmoveto{\pgfqpoint{2.835606in}{2.911349in}}%
\pgfpathlineto{\pgfqpoint{2.835606in}{2.911349in}}%
\pgfpathlineto{\pgfqpoint{2.835606in}{2.909047in}}%
\pgfpathlineto{\pgfqpoint{2.835606in}{2.909047in}}%
\pgfpathlineto{\pgfqpoint{2.835606in}{2.911349in}}%
\pgfpathclose%
\pgfusepath{stroke,fill}%
\end{pgfscope}%
\begin{pgfscope}%
\pgfpathrectangle{\pgfqpoint{0.793996in}{0.443060in}}{\pgfqpoint{2.041610in}{2.577607in}}%
\pgfusepath{clip}%
\pgfsetbuttcap%
\pgfsetroundjoin%
\definecolor{currentfill}{rgb}{0.825117,0.848522,0.887698}%
\pgfsetfillcolor{currentfill}%
\pgfsetlinewidth{0.602250pt}%
\definecolor{currentstroke}{rgb}{0.296471,0.296471,0.296471}%
\pgfsetstrokecolor{currentstroke}%
\pgfsetdash{}{0pt}%
\pgfpathmoveto{\pgfqpoint{2.835606in}{2.910773in}}%
\pgfpathlineto{\pgfqpoint{2.835606in}{2.910773in}}%
\pgfpathlineto{\pgfqpoint{2.835606in}{2.909622in}}%
\pgfpathlineto{\pgfqpoint{2.835606in}{2.909622in}}%
\pgfpathlineto{\pgfqpoint{2.835606in}{2.910773in}}%
\pgfpathclose%
\pgfusepath{stroke,fill}%
\end{pgfscope}%
\begin{pgfscope}%
\pgfpathrectangle{\pgfqpoint{0.793996in}{0.443060in}}{\pgfqpoint{2.041610in}{2.577607in}}%
\pgfusepath{clip}%
\pgfsetbuttcap%
\pgfsetroundjoin%
\definecolor{currentfill}{rgb}{0.848437,0.867532,0.899724}%
\pgfsetfillcolor{currentfill}%
\pgfsetlinewidth{0.602250pt}%
\definecolor{currentstroke}{rgb}{0.296471,0.296471,0.296471}%
\pgfsetstrokecolor{currentstroke}%
\pgfsetdash{}{0pt}%
\pgfpathmoveto{\pgfqpoint{2.835606in}{2.910485in}}%
\pgfpathlineto{\pgfqpoint{2.835606in}{2.910485in}}%
\pgfpathlineto{\pgfqpoint{2.835606in}{2.909910in}}%
\pgfpathlineto{\pgfqpoint{2.835606in}{2.909910in}}%
\pgfpathlineto{\pgfqpoint{2.835606in}{2.910485in}}%
\pgfpathclose%
\pgfusepath{stroke,fill}%
\end{pgfscope}%
\begin{pgfscope}%
\pgfpathrectangle{\pgfqpoint{0.793996in}{0.443060in}}{\pgfqpoint{2.041610in}{2.577607in}}%
\pgfusepath{clip}%
\pgfsetbuttcap%
\pgfsetroundjoin%
\pgfsetlinewidth{0.803000pt}%
\definecolor{currentstroke}{rgb}{0.450000,0.450000,0.450000}%
\pgfsetstrokecolor{currentstroke}%
\pgfsetdash{}{0pt}%
\pgfpathmoveto{\pgfqpoint{0.000000in}{-0.034722in}}%
\pgfpathcurveto{\pgfqpoint{0.009208in}{-0.034722in}}{\pgfqpoint{0.018041in}{-0.031064in}}{\pgfqpoint{0.024552in}{-0.024552in}}%
\pgfpathcurveto{\pgfqpoint{0.031064in}{-0.018041in}}{\pgfqpoint{0.034722in}{-0.009208in}}{\pgfqpoint{0.034722in}{0.000000in}}%
\pgfpathcurveto{\pgfqpoint{0.034722in}{0.009208in}}{\pgfqpoint{0.031064in}{0.018041in}}{\pgfqpoint{0.024552in}{0.024552in}}%
\pgfpathcurveto{\pgfqpoint{0.018041in}{0.031064in}}{\pgfqpoint{0.009208in}{0.034722in}}{\pgfqpoint{0.000000in}{0.034722in}}%
\pgfpathcurveto{\pgfqpoint{-0.009208in}{0.034722in}}{\pgfqpoint{-0.018041in}{0.031064in}}{\pgfqpoint{-0.024552in}{0.024552in}}%
\pgfpathcurveto{\pgfqpoint{-0.031064in}{0.018041in}}{\pgfqpoint{-0.034722in}{0.009208in}}{\pgfqpoint{-0.034722in}{0.000000in}}%
\pgfpathcurveto{\pgfqpoint{-0.034722in}{-0.009208in}}{\pgfqpoint{-0.031064in}{-0.018041in}}{\pgfqpoint{-0.024552in}{-0.024552in}}%
\pgfpathcurveto{\pgfqpoint{-0.018041in}{-0.031064in}}{\pgfqpoint{-0.009208in}{-0.034722in}}{\pgfqpoint{0.000000in}{-0.034722in}}%
\pgfusepath{stroke}%
\end{pgfscope}%
\begin{pgfscope}%
\pgfpathrectangle{\pgfqpoint{0.793996in}{0.443060in}}{\pgfqpoint{2.041610in}{2.577607in}}%
\pgfusepath{clip}%
\pgfsetbuttcap%
\pgfsetroundjoin%
\definecolor{currentfill}{rgb}{0.919097,0.862812,0.832112}%
\pgfsetfillcolor{currentfill}%
\pgfsetlinewidth{0.602250pt}%
\definecolor{currentstroke}{rgb}{0.296471,0.296471,0.296471}%
\pgfsetstrokecolor{currentstroke}%
\pgfsetdash{}{0pt}%
\pgfpathmoveto{\pgfqpoint{1.749413in}{2.763481in}}%
\pgfpathlineto{\pgfqpoint{1.757617in}{2.763481in}}%
\pgfpathlineto{\pgfqpoint{1.757617in}{2.762331in}}%
\pgfpathlineto{\pgfqpoint{1.749413in}{2.762331in}}%
\pgfpathlineto{\pgfqpoint{1.749413in}{2.763481in}}%
\pgfpathclose%
\pgfusepath{stroke,fill}%
\end{pgfscope}%
\begin{pgfscope}%
\pgfpathrectangle{\pgfqpoint{0.793996in}{0.443060in}}{\pgfqpoint{2.041610in}{2.577607in}}%
\pgfusepath{clip}%
\pgfsetbuttcap%
\pgfsetroundjoin%
\definecolor{currentfill}{rgb}{0.910863,0.840546,0.801899}%
\pgfsetfillcolor{currentfill}%
\pgfsetlinewidth{0.602250pt}%
\definecolor{currentstroke}{rgb}{0.296471,0.296471,0.296471}%
\pgfsetstrokecolor{currentstroke}%
\pgfsetdash{}{0pt}%
\pgfpathmoveto{\pgfqpoint{1.757617in}{2.764057in}}%
\pgfpathlineto{\pgfqpoint{1.854791in}{2.764057in}}%
\pgfpathlineto{\pgfqpoint{1.854791in}{2.761755in}}%
\pgfpathlineto{\pgfqpoint{1.757617in}{2.761755in}}%
\pgfpathlineto{\pgfqpoint{1.757617in}{2.764057in}}%
\pgfpathclose%
\pgfusepath{stroke,fill}%
\end{pgfscope}%
\begin{pgfscope}%
\pgfpathrectangle{\pgfqpoint{0.793996in}{0.443060in}}{\pgfqpoint{2.041610in}{2.577607in}}%
\pgfusepath{clip}%
\pgfsetbuttcap%
\pgfsetroundjoin%
\definecolor{currentfill}{rgb}{0.901453,0.815098,0.767370}%
\pgfsetfillcolor{currentfill}%
\pgfsetlinewidth{0.602250pt}%
\definecolor{currentstroke}{rgb}{0.296471,0.296471,0.296471}%
\pgfsetstrokecolor{currentstroke}%
\pgfsetdash{}{0pt}%
\pgfpathmoveto{\pgfqpoint{1.854791in}{2.765207in}}%
\pgfpathlineto{\pgfqpoint{1.875216in}{2.765207in}}%
\pgfpathlineto{\pgfqpoint{1.875216in}{2.760605in}}%
\pgfpathlineto{\pgfqpoint{1.854791in}{2.760605in}}%
\pgfpathlineto{\pgfqpoint{1.854791in}{2.765207in}}%
\pgfpathclose%
\pgfusepath{stroke,fill}%
\end{pgfscope}%
\begin{pgfscope}%
\pgfpathrectangle{\pgfqpoint{0.793996in}{0.443060in}}{\pgfqpoint{2.041610in}{2.577607in}}%
\pgfusepath{clip}%
\pgfsetbuttcap%
\pgfsetroundjoin%
\definecolor{currentfill}{rgb}{0.889102,0.781698,0.722050}%
\pgfsetfillcolor{currentfill}%
\pgfsetlinewidth{0.602250pt}%
\definecolor{currentstroke}{rgb}{0.296471,0.296471,0.296471}%
\pgfsetstrokecolor{currentstroke}%
\pgfsetdash{}{0pt}%
\pgfpathmoveto{\pgfqpoint{1.875216in}{2.767509in}}%
\pgfpathlineto{\pgfqpoint{1.928532in}{2.767509in}}%
\pgfpathlineto{\pgfqpoint{1.928532in}{2.758303in}}%
\pgfpathlineto{\pgfqpoint{1.875216in}{2.758303in}}%
\pgfpathlineto{\pgfqpoint{1.875216in}{2.767509in}}%
\pgfpathclose%
\pgfusepath{stroke,fill}%
\end{pgfscope}%
\begin{pgfscope}%
\pgfpathrectangle{\pgfqpoint{0.793996in}{0.443060in}}{\pgfqpoint{2.041610in}{2.577607in}}%
\pgfusepath{clip}%
\pgfsetbuttcap%
\pgfsetroundjoin%
\definecolor{currentfill}{rgb}{0.873223,0.738755,0.663782}%
\pgfsetfillcolor{currentfill}%
\pgfsetlinewidth{0.602250pt}%
\definecolor{currentstroke}{rgb}{0.296471,0.296471,0.296471}%
\pgfsetstrokecolor{currentstroke}%
\pgfsetdash{}{0pt}%
\pgfpathmoveto{\pgfqpoint{1.928532in}{2.772112in}}%
\pgfpathlineto{\pgfqpoint{2.034973in}{2.772112in}}%
\pgfpathlineto{\pgfqpoint{2.034973in}{2.753700in}}%
\pgfpathlineto{\pgfqpoint{1.928532in}{2.753700in}}%
\pgfpathlineto{\pgfqpoint{1.928532in}{2.772112in}}%
\pgfpathclose%
\pgfusepath{stroke,fill}%
\end{pgfscope}%
\begin{pgfscope}%
\pgfpathrectangle{\pgfqpoint{0.793996in}{0.443060in}}{\pgfqpoint{2.041610in}{2.577607in}}%
\pgfusepath{clip}%
\pgfsetbuttcap%
\pgfsetroundjoin%
\definecolor{currentfill}{rgb}{0.853814,0.686269,0.592565}%
\pgfsetfillcolor{currentfill}%
\pgfsetlinewidth{0.602250pt}%
\definecolor{currentstroke}{rgb}{0.296471,0.296471,0.296471}%
\pgfsetstrokecolor{currentstroke}%
\pgfsetdash{}{0pt}%
\pgfpathmoveto{\pgfqpoint{2.034973in}{2.781317in}}%
\pgfpathlineto{\pgfqpoint{2.147229in}{2.781317in}}%
\pgfpathlineto{\pgfqpoint{2.147229in}{2.744494in}}%
\pgfpathlineto{\pgfqpoint{2.034973in}{2.744494in}}%
\pgfpathlineto{\pgfqpoint{2.034973in}{2.781317in}}%
\pgfpathclose%
\pgfusepath{stroke,fill}%
\end{pgfscope}%
\begin{pgfscope}%
\pgfpathrectangle{\pgfqpoint{0.793996in}{0.443060in}}{\pgfqpoint{2.041610in}{2.577607in}}%
\pgfusepath{clip}%
\pgfsetbuttcap%
\pgfsetroundjoin%
\definecolor{currentfill}{rgb}{0.829112,0.619469,0.501926}%
\pgfsetfillcolor{currentfill}%
\pgfsetlinewidth{0.602250pt}%
\definecolor{currentstroke}{rgb}{0.296471,0.296471,0.296471}%
\pgfsetstrokecolor{currentstroke}%
\pgfsetdash{}{0pt}%
\pgfpathmoveto{\pgfqpoint{2.147229in}{2.799729in}}%
\pgfpathlineto{\pgfqpoint{2.360200in}{2.799729in}}%
\pgfpathlineto{\pgfqpoint{2.360200in}{2.726083in}}%
\pgfpathlineto{\pgfqpoint{2.147229in}{2.726083in}}%
\pgfpathlineto{\pgfqpoint{2.147229in}{2.799729in}}%
\pgfpathclose%
\pgfusepath{stroke,fill}%
\end{pgfscope}%
\begin{pgfscope}%
\pgfpathrectangle{\pgfqpoint{0.793996in}{0.443060in}}{\pgfqpoint{2.041610in}{2.577607in}}%
\pgfusepath{clip}%
\pgfsetbuttcap%
\pgfsetroundjoin%
\definecolor{currentfill}{rgb}{0.798529,0.536765,0.389706}%
\pgfsetfillcolor{currentfill}%
\pgfsetlinewidth{0.602250pt}%
\definecolor{currentstroke}{rgb}{0.296471,0.296471,0.296471}%
\pgfsetstrokecolor{currentstroke}%
\pgfsetdash{}{0pt}%
\pgfpathmoveto{\pgfqpoint{2.360200in}{2.836552in}}%
\pgfpathlineto{\pgfqpoint{2.835606in}{2.836552in}}%
\pgfpathlineto{\pgfqpoint{2.835606in}{2.689260in}}%
\pgfpathlineto{\pgfqpoint{2.360200in}{2.689260in}}%
\pgfpathlineto{\pgfqpoint{2.360200in}{2.836552in}}%
\pgfpathclose%
\pgfusepath{stroke,fill}%
\end{pgfscope}%
\begin{pgfscope}%
\pgfpathrectangle{\pgfqpoint{0.793996in}{0.443060in}}{\pgfqpoint{2.041610in}{2.577607in}}%
\pgfusepath{clip}%
\pgfsetbuttcap%
\pgfsetroundjoin%
\definecolor{currentfill}{rgb}{0.829112,0.619469,0.501926}%
\pgfsetfillcolor{currentfill}%
\pgfsetlinewidth{0.602250pt}%
\definecolor{currentstroke}{rgb}{0.296471,0.296471,0.296471}%
\pgfsetstrokecolor{currentstroke}%
\pgfsetdash{}{0pt}%
\pgfpathmoveto{\pgfqpoint{2.835606in}{2.799729in}}%
\pgfpathlineto{\pgfqpoint{2.835606in}{2.799729in}}%
\pgfpathlineto{\pgfqpoint{2.835606in}{2.726083in}}%
\pgfpathlineto{\pgfqpoint{2.835606in}{2.726083in}}%
\pgfpathlineto{\pgfqpoint{2.835606in}{2.799729in}}%
\pgfpathclose%
\pgfusepath{stroke,fill}%
\end{pgfscope}%
\begin{pgfscope}%
\pgfpathrectangle{\pgfqpoint{0.793996in}{0.443060in}}{\pgfqpoint{2.041610in}{2.577607in}}%
\pgfusepath{clip}%
\pgfsetbuttcap%
\pgfsetroundjoin%
\definecolor{currentfill}{rgb}{0.853814,0.686269,0.592565}%
\pgfsetfillcolor{currentfill}%
\pgfsetlinewidth{0.602250pt}%
\definecolor{currentstroke}{rgb}{0.296471,0.296471,0.296471}%
\pgfsetstrokecolor{currentstroke}%
\pgfsetdash{}{0pt}%
\pgfpathmoveto{\pgfqpoint{2.835606in}{2.781317in}}%
\pgfpathlineto{\pgfqpoint{2.835606in}{2.781317in}}%
\pgfpathlineto{\pgfqpoint{2.835606in}{2.744494in}}%
\pgfpathlineto{\pgfqpoint{2.835606in}{2.744494in}}%
\pgfpathlineto{\pgfqpoint{2.835606in}{2.781317in}}%
\pgfpathclose%
\pgfusepath{stroke,fill}%
\end{pgfscope}%
\begin{pgfscope}%
\pgfpathrectangle{\pgfqpoint{0.793996in}{0.443060in}}{\pgfqpoint{2.041610in}{2.577607in}}%
\pgfusepath{clip}%
\pgfsetbuttcap%
\pgfsetroundjoin%
\definecolor{currentfill}{rgb}{0.873223,0.738755,0.663782}%
\pgfsetfillcolor{currentfill}%
\pgfsetlinewidth{0.602250pt}%
\definecolor{currentstroke}{rgb}{0.296471,0.296471,0.296471}%
\pgfsetstrokecolor{currentstroke}%
\pgfsetdash{}{0pt}%
\pgfpathmoveto{\pgfqpoint{2.835606in}{2.772112in}}%
\pgfpathlineto{\pgfqpoint{2.835606in}{2.772112in}}%
\pgfpathlineto{\pgfqpoint{2.835606in}{2.753700in}}%
\pgfpathlineto{\pgfqpoint{2.835606in}{2.753700in}}%
\pgfpathlineto{\pgfqpoint{2.835606in}{2.772112in}}%
\pgfpathclose%
\pgfusepath{stroke,fill}%
\end{pgfscope}%
\begin{pgfscope}%
\pgfpathrectangle{\pgfqpoint{0.793996in}{0.443060in}}{\pgfqpoint{2.041610in}{2.577607in}}%
\pgfusepath{clip}%
\pgfsetbuttcap%
\pgfsetroundjoin%
\definecolor{currentfill}{rgb}{0.889102,0.781698,0.722050}%
\pgfsetfillcolor{currentfill}%
\pgfsetlinewidth{0.602250pt}%
\definecolor{currentstroke}{rgb}{0.296471,0.296471,0.296471}%
\pgfsetstrokecolor{currentstroke}%
\pgfsetdash{}{0pt}%
\pgfpathmoveto{\pgfqpoint{2.835606in}{2.767509in}}%
\pgfpathlineto{\pgfqpoint{2.835606in}{2.767509in}}%
\pgfpathlineto{\pgfqpoint{2.835606in}{2.758303in}}%
\pgfpathlineto{\pgfqpoint{2.835606in}{2.758303in}}%
\pgfpathlineto{\pgfqpoint{2.835606in}{2.767509in}}%
\pgfpathclose%
\pgfusepath{stroke,fill}%
\end{pgfscope}%
\begin{pgfscope}%
\pgfpathrectangle{\pgfqpoint{0.793996in}{0.443060in}}{\pgfqpoint{2.041610in}{2.577607in}}%
\pgfusepath{clip}%
\pgfsetbuttcap%
\pgfsetroundjoin%
\definecolor{currentfill}{rgb}{0.901453,0.815098,0.767370}%
\pgfsetfillcolor{currentfill}%
\pgfsetlinewidth{0.602250pt}%
\definecolor{currentstroke}{rgb}{0.296471,0.296471,0.296471}%
\pgfsetstrokecolor{currentstroke}%
\pgfsetdash{}{0pt}%
\pgfpathmoveto{\pgfqpoint{2.835606in}{2.765207in}}%
\pgfpathlineto{\pgfqpoint{2.835606in}{2.765207in}}%
\pgfpathlineto{\pgfqpoint{2.835606in}{2.760605in}}%
\pgfpathlineto{\pgfqpoint{2.835606in}{2.760605in}}%
\pgfpathlineto{\pgfqpoint{2.835606in}{2.765207in}}%
\pgfpathclose%
\pgfusepath{stroke,fill}%
\end{pgfscope}%
\begin{pgfscope}%
\pgfpathrectangle{\pgfqpoint{0.793996in}{0.443060in}}{\pgfqpoint{2.041610in}{2.577607in}}%
\pgfusepath{clip}%
\pgfsetbuttcap%
\pgfsetroundjoin%
\definecolor{currentfill}{rgb}{0.910863,0.840546,0.801899}%
\pgfsetfillcolor{currentfill}%
\pgfsetlinewidth{0.602250pt}%
\definecolor{currentstroke}{rgb}{0.296471,0.296471,0.296471}%
\pgfsetstrokecolor{currentstroke}%
\pgfsetdash{}{0pt}%
\pgfpathmoveto{\pgfqpoint{2.835606in}{2.764057in}}%
\pgfpathlineto{\pgfqpoint{2.835606in}{2.764057in}}%
\pgfpathlineto{\pgfqpoint{2.835606in}{2.761755in}}%
\pgfpathlineto{\pgfqpoint{2.835606in}{2.761755in}}%
\pgfpathlineto{\pgfqpoint{2.835606in}{2.764057in}}%
\pgfpathclose%
\pgfusepath{stroke,fill}%
\end{pgfscope}%
\begin{pgfscope}%
\pgfpathrectangle{\pgfqpoint{0.793996in}{0.443060in}}{\pgfqpoint{2.041610in}{2.577607in}}%
\pgfusepath{clip}%
\pgfsetbuttcap%
\pgfsetroundjoin%
\definecolor{currentfill}{rgb}{0.919097,0.862812,0.832112}%
\pgfsetfillcolor{currentfill}%
\pgfsetlinewidth{0.602250pt}%
\definecolor{currentstroke}{rgb}{0.296471,0.296471,0.296471}%
\pgfsetstrokecolor{currentstroke}%
\pgfsetdash{}{0pt}%
\pgfpathmoveto{\pgfqpoint{2.835606in}{2.763481in}}%
\pgfpathlineto{\pgfqpoint{2.835606in}{2.763481in}}%
\pgfpathlineto{\pgfqpoint{2.835606in}{2.762331in}}%
\pgfpathlineto{\pgfqpoint{2.835606in}{2.762331in}}%
\pgfpathlineto{\pgfqpoint{2.835606in}{2.763481in}}%
\pgfpathclose%
\pgfusepath{stroke,fill}%
\end{pgfscope}%
\begin{pgfscope}%
\pgfpathrectangle{\pgfqpoint{0.793996in}{0.443060in}}{\pgfqpoint{2.041610in}{2.577607in}}%
\pgfusepath{clip}%
\pgfsetbuttcap%
\pgfsetroundjoin%
\pgfsetlinewidth{0.803000pt}%
\definecolor{currentstroke}{rgb}{0.450000,0.450000,0.450000}%
\pgfsetstrokecolor{currentstroke}%
\pgfsetdash{}{0pt}%
\pgfpathmoveto{\pgfqpoint{0.000000in}{-0.034722in}}%
\pgfpathcurveto{\pgfqpoint{0.009208in}{-0.034722in}}{\pgfqpoint{0.018041in}{-0.031064in}}{\pgfqpoint{0.024552in}{-0.024552in}}%
\pgfpathcurveto{\pgfqpoint{0.031064in}{-0.018041in}}{\pgfqpoint{0.034722in}{-0.009208in}}{\pgfqpoint{0.034722in}{0.000000in}}%
\pgfpathcurveto{\pgfqpoint{0.034722in}{0.009208in}}{\pgfqpoint{0.031064in}{0.018041in}}{\pgfqpoint{0.024552in}{0.024552in}}%
\pgfpathcurveto{\pgfqpoint{0.018041in}{0.031064in}}{\pgfqpoint{0.009208in}{0.034722in}}{\pgfqpoint{0.000000in}{0.034722in}}%
\pgfpathcurveto{\pgfqpoint{-0.009208in}{0.034722in}}{\pgfqpoint{-0.018041in}{0.031064in}}{\pgfqpoint{-0.024552in}{0.024552in}}%
\pgfpathcurveto{\pgfqpoint{-0.031064in}{0.018041in}}{\pgfqpoint{-0.034722in}{0.009208in}}{\pgfqpoint{-0.034722in}{0.000000in}}%
\pgfpathcurveto{\pgfqpoint{-0.034722in}{-0.009208in}}{\pgfqpoint{-0.031064in}{-0.018041in}}{\pgfqpoint{-0.024552in}{-0.024552in}}%
\pgfpathcurveto{\pgfqpoint{-0.018041in}{-0.031064in}}{\pgfqpoint{-0.009208in}{-0.034722in}}{\pgfqpoint{0.000000in}{-0.034722in}}%
\pgfusepath{stroke}%
\end{pgfscope}%
\begin{pgfscope}%
\pgfpathrectangle{\pgfqpoint{0.793996in}{0.443060in}}{\pgfqpoint{2.041610in}{2.577607in}}%
\pgfusepath{clip}%
\pgfsetbuttcap%
\pgfsetroundjoin%
\definecolor{currentfill}{rgb}{0.848437,0.867532,0.899724}%
\pgfsetfillcolor{currentfill}%
\pgfsetlinewidth{0.602250pt}%
\definecolor{currentstroke}{rgb}{0.296471,0.296471,0.296471}%
\pgfsetstrokecolor{currentstroke}%
\pgfsetdash{}{0pt}%
\pgfpathmoveto{\pgfqpoint{0.876678in}{2.542256in}}%
\pgfpathlineto{\pgfqpoint{0.881229in}{2.542256in}}%
\pgfpathlineto{\pgfqpoint{0.881229in}{2.541681in}}%
\pgfpathlineto{\pgfqpoint{0.876678in}{2.541681in}}%
\pgfpathlineto{\pgfqpoint{0.876678in}{2.542256in}}%
\pgfpathclose%
\pgfusepath{stroke,fill}%
\end{pgfscope}%
\begin{pgfscope}%
\pgfpathrectangle{\pgfqpoint{0.793996in}{0.443060in}}{\pgfqpoint{2.041610in}{2.577607in}}%
\pgfusepath{clip}%
\pgfsetbuttcap%
\pgfsetroundjoin%
\definecolor{currentfill}{rgb}{0.825117,0.848522,0.887698}%
\pgfsetfillcolor{currentfill}%
\pgfsetlinewidth{0.602250pt}%
\definecolor{currentstroke}{rgb}{0.296471,0.296471,0.296471}%
\pgfsetstrokecolor{currentstroke}%
\pgfsetdash{}{0pt}%
\pgfpathmoveto{\pgfqpoint{0.881229in}{2.542544in}}%
\pgfpathlineto{\pgfqpoint{0.882624in}{2.542544in}}%
\pgfpathlineto{\pgfqpoint{0.882624in}{2.541393in}}%
\pgfpathlineto{\pgfqpoint{0.881229in}{2.541393in}}%
\pgfpathlineto{\pgfqpoint{0.881229in}{2.542544in}}%
\pgfpathclose%
\pgfusepath{stroke,fill}%
\end{pgfscope}%
\begin{pgfscope}%
\pgfpathrectangle{\pgfqpoint{0.793996in}{0.443060in}}{\pgfqpoint{2.041610in}{2.577607in}}%
\pgfusepath{clip}%
\pgfsetbuttcap%
\pgfsetroundjoin%
\definecolor{currentfill}{rgb}{0.792469,0.821908,0.870863}%
\pgfsetfillcolor{currentfill}%
\pgfsetlinewidth{0.602250pt}%
\definecolor{currentstroke}{rgb}{0.296471,0.296471,0.296471}%
\pgfsetstrokecolor{currentstroke}%
\pgfsetdash{}{0pt}%
\pgfpathmoveto{\pgfqpoint{0.882624in}{2.543119in}}%
\pgfpathlineto{\pgfqpoint{0.885313in}{2.543119in}}%
\pgfpathlineto{\pgfqpoint{0.885313in}{2.540817in}}%
\pgfpathlineto{\pgfqpoint{0.882624in}{2.540817in}}%
\pgfpathlineto{\pgfqpoint{0.882624in}{2.543119in}}%
\pgfpathclose%
\pgfusepath{stroke,fill}%
\end{pgfscope}%
\begin{pgfscope}%
\pgfpathrectangle{\pgfqpoint{0.793996in}{0.443060in}}{\pgfqpoint{2.041610in}{2.577607in}}%
\pgfusepath{clip}%
\pgfsetbuttcap%
\pgfsetroundjoin%
\definecolor{currentfill}{rgb}{0.755157,0.791493,0.851622}%
\pgfsetfillcolor{currentfill}%
\pgfsetlinewidth{0.602250pt}%
\definecolor{currentstroke}{rgb}{0.296471,0.296471,0.296471}%
\pgfsetstrokecolor{currentstroke}%
\pgfsetdash{}{0pt}%
\pgfpathmoveto{\pgfqpoint{0.885313in}{2.544270in}}%
\pgfpathlineto{\pgfqpoint{0.905280in}{2.544270in}}%
\pgfpathlineto{\pgfqpoint{0.905280in}{2.539667in}}%
\pgfpathlineto{\pgfqpoint{0.885313in}{2.539667in}}%
\pgfpathlineto{\pgfqpoint{0.885313in}{2.544270in}}%
\pgfpathclose%
\pgfusepath{stroke,fill}%
\end{pgfscope}%
\begin{pgfscope}%
\pgfpathrectangle{\pgfqpoint{0.793996in}{0.443060in}}{\pgfqpoint{2.041610in}{2.577607in}}%
\pgfusepath{clip}%
\pgfsetbuttcap%
\pgfsetroundjoin%
\definecolor{currentfill}{rgb}{0.706185,0.751573,0.826368}%
\pgfsetfillcolor{currentfill}%
\pgfsetlinewidth{0.602250pt}%
\definecolor{currentstroke}{rgb}{0.296471,0.296471,0.296471}%
\pgfsetstrokecolor{currentstroke}%
\pgfsetdash{}{0pt}%
\pgfpathmoveto{\pgfqpoint{0.905280in}{2.546571in}}%
\pgfpathlineto{\pgfqpoint{0.917730in}{2.546571in}}%
\pgfpathlineto{\pgfqpoint{0.917730in}{2.537365in}}%
\pgfpathlineto{\pgfqpoint{0.905280in}{2.537365in}}%
\pgfpathlineto{\pgfqpoint{0.905280in}{2.546571in}}%
\pgfpathclose%
\pgfusepath{stroke,fill}%
\end{pgfscope}%
\begin{pgfscope}%
\pgfpathrectangle{\pgfqpoint{0.793996in}{0.443060in}}{\pgfqpoint{2.041610in}{2.577607in}}%
\pgfusepath{clip}%
\pgfsetbuttcap%
\pgfsetroundjoin%
\definecolor{currentfill}{rgb}{0.643221,0.700246,0.793900}%
\pgfsetfillcolor{currentfill}%
\pgfsetlinewidth{0.602250pt}%
\definecolor{currentstroke}{rgb}{0.296471,0.296471,0.296471}%
\pgfsetstrokecolor{currentstroke}%
\pgfsetdash{}{0pt}%
\pgfpathmoveto{\pgfqpoint{0.917730in}{2.551174in}}%
\pgfpathlineto{\pgfqpoint{0.950740in}{2.551174in}}%
\pgfpathlineto{\pgfqpoint{0.950740in}{2.532762in}}%
\pgfpathlineto{\pgfqpoint{0.917730in}{2.532762in}}%
\pgfpathlineto{\pgfqpoint{0.917730in}{2.551174in}}%
\pgfpathclose%
\pgfusepath{stroke,fill}%
\end{pgfscope}%
\begin{pgfscope}%
\pgfpathrectangle{\pgfqpoint{0.793996in}{0.443060in}}{\pgfqpoint{2.041610in}{2.577607in}}%
\pgfusepath{clip}%
\pgfsetbuttcap%
\pgfsetroundjoin%
\definecolor{currentfill}{rgb}{0.566266,0.637515,0.754216}%
\pgfsetfillcolor{currentfill}%
\pgfsetlinewidth{0.602250pt}%
\definecolor{currentstroke}{rgb}{0.296471,0.296471,0.296471}%
\pgfsetstrokecolor{currentstroke}%
\pgfsetdash{}{0pt}%
\pgfpathmoveto{\pgfqpoint{0.950740in}{2.560380in}}%
\pgfpathlineto{\pgfqpoint{0.994835in}{2.560380in}}%
\pgfpathlineto{\pgfqpoint{0.994835in}{2.523557in}}%
\pgfpathlineto{\pgfqpoint{0.950740in}{2.523557in}}%
\pgfpathlineto{\pgfqpoint{0.950740in}{2.560380in}}%
\pgfpathclose%
\pgfusepath{stroke,fill}%
\end{pgfscope}%
\begin{pgfscope}%
\pgfpathrectangle{\pgfqpoint{0.793996in}{0.443060in}}{\pgfqpoint{2.041610in}{2.577607in}}%
\pgfusepath{clip}%
\pgfsetbuttcap%
\pgfsetroundjoin%
\definecolor{currentfill}{rgb}{0.468322,0.557674,0.703709}%
\pgfsetfillcolor{currentfill}%
\pgfsetlinewidth{0.602250pt}%
\definecolor{currentstroke}{rgb}{0.296471,0.296471,0.296471}%
\pgfsetstrokecolor{currentstroke}%
\pgfsetdash{}{0pt}%
\pgfpathmoveto{\pgfqpoint{0.994835in}{2.578791in}}%
\pgfpathlineto{\pgfqpoint{1.106619in}{2.578791in}}%
\pgfpathlineto{\pgfqpoint{1.106619in}{2.505145in}}%
\pgfpathlineto{\pgfqpoint{0.994835in}{2.505145in}}%
\pgfpathlineto{\pgfqpoint{0.994835in}{2.578791in}}%
\pgfpathclose%
\pgfusepath{stroke,fill}%
\end{pgfscope}%
\begin{pgfscope}%
\pgfpathrectangle{\pgfqpoint{0.793996in}{0.443060in}}{\pgfqpoint{2.041610in}{2.577607in}}%
\pgfusepath{clip}%
\pgfsetbuttcap%
\pgfsetroundjoin%
\definecolor{currentfill}{rgb}{0.347059,0.458824,0.641176}%
\pgfsetfillcolor{currentfill}%
\pgfsetlinewidth{0.602250pt}%
\definecolor{currentstroke}{rgb}{0.296471,0.296471,0.296471}%
\pgfsetstrokecolor{currentstroke}%
\pgfsetdash{}{0pt}%
\pgfpathmoveto{\pgfqpoint{1.106619in}{2.615614in}}%
\pgfpathlineto{\pgfqpoint{1.877335in}{2.615614in}}%
\pgfpathlineto{\pgfqpoint{1.877335in}{2.468322in}}%
\pgfpathlineto{\pgfqpoint{1.106619in}{2.468322in}}%
\pgfpathlineto{\pgfqpoint{1.106619in}{2.615614in}}%
\pgfpathclose%
\pgfusepath{stroke,fill}%
\end{pgfscope}%
\begin{pgfscope}%
\pgfpathrectangle{\pgfqpoint{0.793996in}{0.443060in}}{\pgfqpoint{2.041610in}{2.577607in}}%
\pgfusepath{clip}%
\pgfsetbuttcap%
\pgfsetroundjoin%
\definecolor{currentfill}{rgb}{0.468322,0.557674,0.703709}%
\pgfsetfillcolor{currentfill}%
\pgfsetlinewidth{0.602250pt}%
\definecolor{currentstroke}{rgb}{0.296471,0.296471,0.296471}%
\pgfsetstrokecolor{currentstroke}%
\pgfsetdash{}{0pt}%
\pgfpathmoveto{\pgfqpoint{1.877335in}{2.578791in}}%
\pgfpathlineto{\pgfqpoint{2.443112in}{2.578791in}}%
\pgfpathlineto{\pgfqpoint{2.443112in}{2.505145in}}%
\pgfpathlineto{\pgfqpoint{1.877335in}{2.505145in}}%
\pgfpathlineto{\pgfqpoint{1.877335in}{2.578791in}}%
\pgfpathclose%
\pgfusepath{stroke,fill}%
\end{pgfscope}%
\begin{pgfscope}%
\pgfpathrectangle{\pgfqpoint{0.793996in}{0.443060in}}{\pgfqpoint{2.041610in}{2.577607in}}%
\pgfusepath{clip}%
\pgfsetbuttcap%
\pgfsetroundjoin%
\definecolor{currentfill}{rgb}{0.566266,0.637515,0.754216}%
\pgfsetfillcolor{currentfill}%
\pgfsetlinewidth{0.602250pt}%
\definecolor{currentstroke}{rgb}{0.296471,0.296471,0.296471}%
\pgfsetstrokecolor{currentstroke}%
\pgfsetdash{}{0pt}%
\pgfpathmoveto{\pgfqpoint{2.443112in}{2.560380in}}%
\pgfpathlineto{\pgfqpoint{2.754531in}{2.560380in}}%
\pgfpathlineto{\pgfqpoint{2.754531in}{2.523557in}}%
\pgfpathlineto{\pgfqpoint{2.443112in}{2.523557in}}%
\pgfpathlineto{\pgfqpoint{2.443112in}{2.560380in}}%
\pgfpathclose%
\pgfusepath{stroke,fill}%
\end{pgfscope}%
\begin{pgfscope}%
\pgfpathrectangle{\pgfqpoint{0.793996in}{0.443060in}}{\pgfqpoint{2.041610in}{2.577607in}}%
\pgfusepath{clip}%
\pgfsetbuttcap%
\pgfsetroundjoin%
\definecolor{currentfill}{rgb}{0.643221,0.700246,0.793900}%
\pgfsetfillcolor{currentfill}%
\pgfsetlinewidth{0.602250pt}%
\definecolor{currentstroke}{rgb}{0.296471,0.296471,0.296471}%
\pgfsetstrokecolor{currentstroke}%
\pgfsetdash{}{0pt}%
\pgfpathmoveto{\pgfqpoint{2.754531in}{2.551174in}}%
\pgfpathlineto{\pgfqpoint{2.835606in}{2.551174in}}%
\pgfpathlineto{\pgfqpoint{2.835606in}{2.532762in}}%
\pgfpathlineto{\pgfqpoint{2.754531in}{2.532762in}}%
\pgfpathlineto{\pgfqpoint{2.754531in}{2.551174in}}%
\pgfpathclose%
\pgfusepath{stroke,fill}%
\end{pgfscope}%
\begin{pgfscope}%
\pgfpathrectangle{\pgfqpoint{0.793996in}{0.443060in}}{\pgfqpoint{2.041610in}{2.577607in}}%
\pgfusepath{clip}%
\pgfsetbuttcap%
\pgfsetroundjoin%
\definecolor{currentfill}{rgb}{0.706185,0.751573,0.826368}%
\pgfsetfillcolor{currentfill}%
\pgfsetlinewidth{0.602250pt}%
\definecolor{currentstroke}{rgb}{0.296471,0.296471,0.296471}%
\pgfsetstrokecolor{currentstroke}%
\pgfsetdash{}{0pt}%
\pgfpathmoveto{\pgfqpoint{2.835606in}{2.546571in}}%
\pgfpathlineto{\pgfqpoint{2.835606in}{2.546571in}}%
\pgfpathlineto{\pgfqpoint{2.835606in}{2.537365in}}%
\pgfpathlineto{\pgfqpoint{2.835606in}{2.537365in}}%
\pgfpathlineto{\pgfqpoint{2.835606in}{2.546571in}}%
\pgfpathclose%
\pgfusepath{stroke,fill}%
\end{pgfscope}%
\begin{pgfscope}%
\pgfpathrectangle{\pgfqpoint{0.793996in}{0.443060in}}{\pgfqpoint{2.041610in}{2.577607in}}%
\pgfusepath{clip}%
\pgfsetbuttcap%
\pgfsetroundjoin%
\definecolor{currentfill}{rgb}{0.755157,0.791493,0.851622}%
\pgfsetfillcolor{currentfill}%
\pgfsetlinewidth{0.602250pt}%
\definecolor{currentstroke}{rgb}{0.296471,0.296471,0.296471}%
\pgfsetstrokecolor{currentstroke}%
\pgfsetdash{}{0pt}%
\pgfpathmoveto{\pgfqpoint{2.835606in}{2.544270in}}%
\pgfpathlineto{\pgfqpoint{2.835606in}{2.544270in}}%
\pgfpathlineto{\pgfqpoint{2.835606in}{2.539667in}}%
\pgfpathlineto{\pgfqpoint{2.835606in}{2.539667in}}%
\pgfpathlineto{\pgfqpoint{2.835606in}{2.544270in}}%
\pgfpathclose%
\pgfusepath{stroke,fill}%
\end{pgfscope}%
\begin{pgfscope}%
\pgfpathrectangle{\pgfqpoint{0.793996in}{0.443060in}}{\pgfqpoint{2.041610in}{2.577607in}}%
\pgfusepath{clip}%
\pgfsetbuttcap%
\pgfsetroundjoin%
\definecolor{currentfill}{rgb}{0.792469,0.821908,0.870863}%
\pgfsetfillcolor{currentfill}%
\pgfsetlinewidth{0.602250pt}%
\definecolor{currentstroke}{rgb}{0.296471,0.296471,0.296471}%
\pgfsetstrokecolor{currentstroke}%
\pgfsetdash{}{0pt}%
\pgfpathmoveto{\pgfqpoint{2.835606in}{2.543119in}}%
\pgfpathlineto{\pgfqpoint{2.835606in}{2.543119in}}%
\pgfpathlineto{\pgfqpoint{2.835606in}{2.540817in}}%
\pgfpathlineto{\pgfqpoint{2.835606in}{2.540817in}}%
\pgfpathlineto{\pgfqpoint{2.835606in}{2.543119in}}%
\pgfpathclose%
\pgfusepath{stroke,fill}%
\end{pgfscope}%
\begin{pgfscope}%
\pgfpathrectangle{\pgfqpoint{0.793996in}{0.443060in}}{\pgfqpoint{2.041610in}{2.577607in}}%
\pgfusepath{clip}%
\pgfsetbuttcap%
\pgfsetroundjoin%
\definecolor{currentfill}{rgb}{0.825117,0.848522,0.887698}%
\pgfsetfillcolor{currentfill}%
\pgfsetlinewidth{0.602250pt}%
\definecolor{currentstroke}{rgb}{0.296471,0.296471,0.296471}%
\pgfsetstrokecolor{currentstroke}%
\pgfsetdash{}{0pt}%
\pgfpathmoveto{\pgfqpoint{2.835606in}{2.542544in}}%
\pgfpathlineto{\pgfqpoint{2.835606in}{2.542544in}}%
\pgfpathlineto{\pgfqpoint{2.835606in}{2.541393in}}%
\pgfpathlineto{\pgfqpoint{2.835606in}{2.541393in}}%
\pgfpathlineto{\pgfqpoint{2.835606in}{2.542544in}}%
\pgfpathclose%
\pgfusepath{stroke,fill}%
\end{pgfscope}%
\begin{pgfscope}%
\pgfpathrectangle{\pgfqpoint{0.793996in}{0.443060in}}{\pgfqpoint{2.041610in}{2.577607in}}%
\pgfusepath{clip}%
\pgfsetbuttcap%
\pgfsetroundjoin%
\definecolor{currentfill}{rgb}{0.848437,0.867532,0.899724}%
\pgfsetfillcolor{currentfill}%
\pgfsetlinewidth{0.602250pt}%
\definecolor{currentstroke}{rgb}{0.296471,0.296471,0.296471}%
\pgfsetstrokecolor{currentstroke}%
\pgfsetdash{}{0pt}%
\pgfpathmoveto{\pgfqpoint{2.835606in}{2.542256in}}%
\pgfpathlineto{\pgfqpoint{2.835606in}{2.542256in}}%
\pgfpathlineto{\pgfqpoint{2.835606in}{2.541681in}}%
\pgfpathlineto{\pgfqpoint{2.835606in}{2.541681in}}%
\pgfpathlineto{\pgfqpoint{2.835606in}{2.542256in}}%
\pgfpathclose%
\pgfusepath{stroke,fill}%
\end{pgfscope}%
\begin{pgfscope}%
\pgfpathrectangle{\pgfqpoint{0.793996in}{0.443060in}}{\pgfqpoint{2.041610in}{2.577607in}}%
\pgfusepath{clip}%
\pgfsetbuttcap%
\pgfsetroundjoin%
\pgfsetlinewidth{0.803000pt}%
\definecolor{currentstroke}{rgb}{0.450000,0.450000,0.450000}%
\pgfsetstrokecolor{currentstroke}%
\pgfsetdash{}{0pt}%
\pgfpathmoveto{\pgfqpoint{0.000000in}{-0.034722in}}%
\pgfpathcurveto{\pgfqpoint{0.009208in}{-0.034722in}}{\pgfqpoint{0.018041in}{-0.031064in}}{\pgfqpoint{0.024552in}{-0.024552in}}%
\pgfpathcurveto{\pgfqpoint{0.031064in}{-0.018041in}}{\pgfqpoint{0.034722in}{-0.009208in}}{\pgfqpoint{0.034722in}{0.000000in}}%
\pgfpathcurveto{\pgfqpoint{0.034722in}{0.009208in}}{\pgfqpoint{0.031064in}{0.018041in}}{\pgfqpoint{0.024552in}{0.024552in}}%
\pgfpathcurveto{\pgfqpoint{0.018041in}{0.031064in}}{\pgfqpoint{0.009208in}{0.034722in}}{\pgfqpoint{0.000000in}{0.034722in}}%
\pgfpathcurveto{\pgfqpoint{-0.009208in}{0.034722in}}{\pgfqpoint{-0.018041in}{0.031064in}}{\pgfqpoint{-0.024552in}{0.024552in}}%
\pgfpathcurveto{\pgfqpoint{-0.031064in}{0.018041in}}{\pgfqpoint{-0.034722in}{0.009208in}}{\pgfqpoint{-0.034722in}{0.000000in}}%
\pgfpathcurveto{\pgfqpoint{-0.034722in}{-0.009208in}}{\pgfqpoint{-0.031064in}{-0.018041in}}{\pgfqpoint{-0.024552in}{-0.024552in}}%
\pgfpathcurveto{\pgfqpoint{-0.018041in}{-0.031064in}}{\pgfqpoint{-0.009208in}{-0.034722in}}{\pgfqpoint{0.000000in}{-0.034722in}}%
\pgfusepath{stroke}%
\end{pgfscope}%
\begin{pgfscope}%
\pgfpathrectangle{\pgfqpoint{0.793996in}{0.443060in}}{\pgfqpoint{2.041610in}{2.577607in}}%
\pgfusepath{clip}%
\pgfsetbuttcap%
\pgfsetroundjoin%
\definecolor{currentfill}{rgb}{0.919097,0.862812,0.832112}%
\pgfsetfillcolor{currentfill}%
\pgfsetlinewidth{0.602250pt}%
\definecolor{currentstroke}{rgb}{0.296471,0.296471,0.296471}%
\pgfsetstrokecolor{currentstroke}%
\pgfsetdash{}{0pt}%
\pgfpathmoveto{\pgfqpoint{1.165352in}{2.395252in}}%
\pgfpathlineto{\pgfqpoint{1.189088in}{2.395252in}}%
\pgfpathlineto{\pgfqpoint{1.189088in}{2.394101in}}%
\pgfpathlineto{\pgfqpoint{1.165352in}{2.394101in}}%
\pgfpathlineto{\pgfqpoint{1.165352in}{2.395252in}}%
\pgfpathclose%
\pgfusepath{stroke,fill}%
\end{pgfscope}%
\begin{pgfscope}%
\pgfpathrectangle{\pgfqpoint{0.793996in}{0.443060in}}{\pgfqpoint{2.041610in}{2.577607in}}%
\pgfusepath{clip}%
\pgfsetbuttcap%
\pgfsetroundjoin%
\definecolor{currentfill}{rgb}{0.910863,0.840546,0.801899}%
\pgfsetfillcolor{currentfill}%
\pgfsetlinewidth{0.602250pt}%
\definecolor{currentstroke}{rgb}{0.296471,0.296471,0.296471}%
\pgfsetstrokecolor{currentstroke}%
\pgfsetdash{}{0pt}%
\pgfpathmoveto{\pgfqpoint{1.189088in}{2.395827in}}%
\pgfpathlineto{\pgfqpoint{1.196667in}{2.395827in}}%
\pgfpathlineto{\pgfqpoint{1.196667in}{2.393526in}}%
\pgfpathlineto{\pgfqpoint{1.189088in}{2.393526in}}%
\pgfpathlineto{\pgfqpoint{1.189088in}{2.395827in}}%
\pgfpathclose%
\pgfusepath{stroke,fill}%
\end{pgfscope}%
\begin{pgfscope}%
\pgfpathrectangle{\pgfqpoint{0.793996in}{0.443060in}}{\pgfqpoint{2.041610in}{2.577607in}}%
\pgfusepath{clip}%
\pgfsetbuttcap%
\pgfsetroundjoin%
\definecolor{currentfill}{rgb}{0.901453,0.815098,0.767370}%
\pgfsetfillcolor{currentfill}%
\pgfsetlinewidth{0.602250pt}%
\definecolor{currentstroke}{rgb}{0.296471,0.296471,0.296471}%
\pgfsetstrokecolor{currentstroke}%
\pgfsetdash{}{0pt}%
\pgfpathmoveto{\pgfqpoint{1.196667in}{2.396978in}}%
\pgfpathlineto{\pgfqpoint{1.219135in}{2.396978in}}%
\pgfpathlineto{\pgfqpoint{1.219135in}{2.392375in}}%
\pgfpathlineto{\pgfqpoint{1.196667in}{2.392375in}}%
\pgfpathlineto{\pgfqpoint{1.196667in}{2.396978in}}%
\pgfpathclose%
\pgfusepath{stroke,fill}%
\end{pgfscope}%
\begin{pgfscope}%
\pgfpathrectangle{\pgfqpoint{0.793996in}{0.443060in}}{\pgfqpoint{2.041610in}{2.577607in}}%
\pgfusepath{clip}%
\pgfsetbuttcap%
\pgfsetroundjoin%
\definecolor{currentfill}{rgb}{0.889102,0.781698,0.722050}%
\pgfsetfillcolor{currentfill}%
\pgfsetlinewidth{0.602250pt}%
\definecolor{currentstroke}{rgb}{0.296471,0.296471,0.296471}%
\pgfsetstrokecolor{currentstroke}%
\pgfsetdash{}{0pt}%
\pgfpathmoveto{\pgfqpoint{1.219135in}{2.399279in}}%
\pgfpathlineto{\pgfqpoint{1.276751in}{2.399279in}}%
\pgfpathlineto{\pgfqpoint{1.276751in}{2.390074in}}%
\pgfpathlineto{\pgfqpoint{1.219135in}{2.390074in}}%
\pgfpathlineto{\pgfqpoint{1.219135in}{2.399279in}}%
\pgfpathclose%
\pgfusepath{stroke,fill}%
\end{pgfscope}%
\begin{pgfscope}%
\pgfpathrectangle{\pgfqpoint{0.793996in}{0.443060in}}{\pgfqpoint{2.041610in}{2.577607in}}%
\pgfusepath{clip}%
\pgfsetbuttcap%
\pgfsetroundjoin%
\definecolor{currentfill}{rgb}{0.873223,0.738755,0.663782}%
\pgfsetfillcolor{currentfill}%
\pgfsetlinewidth{0.602250pt}%
\definecolor{currentstroke}{rgb}{0.296471,0.296471,0.296471}%
\pgfsetstrokecolor{currentstroke}%
\pgfsetdash{}{0pt}%
\pgfpathmoveto{\pgfqpoint{1.276751in}{2.403882in}}%
\pgfpathlineto{\pgfqpoint{1.356627in}{2.403882in}}%
\pgfpathlineto{\pgfqpoint{1.356627in}{2.385471in}}%
\pgfpathlineto{\pgfqpoint{1.276751in}{2.385471in}}%
\pgfpathlineto{\pgfqpoint{1.276751in}{2.403882in}}%
\pgfpathclose%
\pgfusepath{stroke,fill}%
\end{pgfscope}%
\begin{pgfscope}%
\pgfpathrectangle{\pgfqpoint{0.793996in}{0.443060in}}{\pgfqpoint{2.041610in}{2.577607in}}%
\pgfusepath{clip}%
\pgfsetbuttcap%
\pgfsetroundjoin%
\definecolor{currentfill}{rgb}{0.853814,0.686269,0.592565}%
\pgfsetfillcolor{currentfill}%
\pgfsetlinewidth{0.602250pt}%
\definecolor{currentstroke}{rgb}{0.296471,0.296471,0.296471}%
\pgfsetstrokecolor{currentstroke}%
\pgfsetdash{}{0pt}%
\pgfpathmoveto{\pgfqpoint{1.356627in}{2.413088in}}%
\pgfpathlineto{\pgfqpoint{1.449151in}{2.413088in}}%
\pgfpathlineto{\pgfqpoint{1.449151in}{2.376265in}}%
\pgfpathlineto{\pgfqpoint{1.356627in}{2.376265in}}%
\pgfpathlineto{\pgfqpoint{1.356627in}{2.413088in}}%
\pgfpathclose%
\pgfusepath{stroke,fill}%
\end{pgfscope}%
\begin{pgfscope}%
\pgfpathrectangle{\pgfqpoint{0.793996in}{0.443060in}}{\pgfqpoint{2.041610in}{2.577607in}}%
\pgfusepath{clip}%
\pgfsetbuttcap%
\pgfsetroundjoin%
\definecolor{currentfill}{rgb}{0.829112,0.619469,0.501926}%
\pgfsetfillcolor{currentfill}%
\pgfsetlinewidth{0.602250pt}%
\definecolor{currentstroke}{rgb}{0.296471,0.296471,0.296471}%
\pgfsetstrokecolor{currentstroke}%
\pgfsetdash{}{0pt}%
\pgfpathmoveto{\pgfqpoint{1.449151in}{2.431499in}}%
\pgfpathlineto{\pgfqpoint{1.545799in}{2.431499in}}%
\pgfpathlineto{\pgfqpoint{1.545799in}{2.357853in}}%
\pgfpathlineto{\pgfqpoint{1.449151in}{2.357853in}}%
\pgfpathlineto{\pgfqpoint{1.449151in}{2.431499in}}%
\pgfpathclose%
\pgfusepath{stroke,fill}%
\end{pgfscope}%
\begin{pgfscope}%
\pgfpathrectangle{\pgfqpoint{0.793996in}{0.443060in}}{\pgfqpoint{2.041610in}{2.577607in}}%
\pgfusepath{clip}%
\pgfsetbuttcap%
\pgfsetroundjoin%
\definecolor{currentfill}{rgb}{0.798529,0.536765,0.389706}%
\pgfsetfillcolor{currentfill}%
\pgfsetlinewidth{0.602250pt}%
\definecolor{currentstroke}{rgb}{0.296471,0.296471,0.296471}%
\pgfsetstrokecolor{currentstroke}%
\pgfsetdash{}{0pt}%
\pgfpathmoveto{\pgfqpoint{1.545799in}{2.468322in}}%
\pgfpathlineto{\pgfqpoint{2.655081in}{2.468322in}}%
\pgfpathlineto{\pgfqpoint{2.655081in}{2.321030in}}%
\pgfpathlineto{\pgfqpoint{1.545799in}{2.321030in}}%
\pgfpathlineto{\pgfqpoint{1.545799in}{2.468322in}}%
\pgfpathclose%
\pgfusepath{stroke,fill}%
\end{pgfscope}%
\begin{pgfscope}%
\pgfpathrectangle{\pgfqpoint{0.793996in}{0.443060in}}{\pgfqpoint{2.041610in}{2.577607in}}%
\pgfusepath{clip}%
\pgfsetbuttcap%
\pgfsetroundjoin%
\definecolor{currentfill}{rgb}{0.829112,0.619469,0.501926}%
\pgfsetfillcolor{currentfill}%
\pgfsetlinewidth{0.602250pt}%
\definecolor{currentstroke}{rgb}{0.296471,0.296471,0.296471}%
\pgfsetstrokecolor{currentstroke}%
\pgfsetdash{}{0pt}%
\pgfpathmoveto{\pgfqpoint{2.655081in}{2.431499in}}%
\pgfpathlineto{\pgfqpoint{2.790904in}{2.431499in}}%
\pgfpathlineto{\pgfqpoint{2.790904in}{2.357853in}}%
\pgfpathlineto{\pgfqpoint{2.655081in}{2.357853in}}%
\pgfpathlineto{\pgfqpoint{2.655081in}{2.431499in}}%
\pgfpathclose%
\pgfusepath{stroke,fill}%
\end{pgfscope}%
\begin{pgfscope}%
\pgfpathrectangle{\pgfqpoint{0.793996in}{0.443060in}}{\pgfqpoint{2.041610in}{2.577607in}}%
\pgfusepath{clip}%
\pgfsetbuttcap%
\pgfsetroundjoin%
\definecolor{currentfill}{rgb}{0.853814,0.686269,0.592565}%
\pgfsetfillcolor{currentfill}%
\pgfsetlinewidth{0.602250pt}%
\definecolor{currentstroke}{rgb}{0.296471,0.296471,0.296471}%
\pgfsetstrokecolor{currentstroke}%
\pgfsetdash{}{0pt}%
\pgfpathmoveto{\pgfqpoint{2.790904in}{2.413088in}}%
\pgfpathlineto{\pgfqpoint{2.835606in}{2.413088in}}%
\pgfpathlineto{\pgfqpoint{2.835606in}{2.376265in}}%
\pgfpathlineto{\pgfqpoint{2.790904in}{2.376265in}}%
\pgfpathlineto{\pgfqpoint{2.790904in}{2.413088in}}%
\pgfpathclose%
\pgfusepath{stroke,fill}%
\end{pgfscope}%
\begin{pgfscope}%
\pgfpathrectangle{\pgfqpoint{0.793996in}{0.443060in}}{\pgfqpoint{2.041610in}{2.577607in}}%
\pgfusepath{clip}%
\pgfsetbuttcap%
\pgfsetroundjoin%
\definecolor{currentfill}{rgb}{0.873223,0.738755,0.663782}%
\pgfsetfillcolor{currentfill}%
\pgfsetlinewidth{0.602250pt}%
\definecolor{currentstroke}{rgb}{0.296471,0.296471,0.296471}%
\pgfsetstrokecolor{currentstroke}%
\pgfsetdash{}{0pt}%
\pgfpathmoveto{\pgfqpoint{2.835606in}{2.403882in}}%
\pgfpathlineto{\pgfqpoint{2.835606in}{2.403882in}}%
\pgfpathlineto{\pgfqpoint{2.835606in}{2.385471in}}%
\pgfpathlineto{\pgfqpoint{2.835606in}{2.385471in}}%
\pgfpathlineto{\pgfqpoint{2.835606in}{2.403882in}}%
\pgfpathclose%
\pgfusepath{stroke,fill}%
\end{pgfscope}%
\begin{pgfscope}%
\pgfpathrectangle{\pgfqpoint{0.793996in}{0.443060in}}{\pgfqpoint{2.041610in}{2.577607in}}%
\pgfusepath{clip}%
\pgfsetbuttcap%
\pgfsetroundjoin%
\definecolor{currentfill}{rgb}{0.889102,0.781698,0.722050}%
\pgfsetfillcolor{currentfill}%
\pgfsetlinewidth{0.602250pt}%
\definecolor{currentstroke}{rgb}{0.296471,0.296471,0.296471}%
\pgfsetstrokecolor{currentstroke}%
\pgfsetdash{}{0pt}%
\pgfpathmoveto{\pgfqpoint{2.835606in}{2.399279in}}%
\pgfpathlineto{\pgfqpoint{2.835606in}{2.399279in}}%
\pgfpathlineto{\pgfqpoint{2.835606in}{2.390074in}}%
\pgfpathlineto{\pgfqpoint{2.835606in}{2.390074in}}%
\pgfpathlineto{\pgfqpoint{2.835606in}{2.399279in}}%
\pgfpathclose%
\pgfusepath{stroke,fill}%
\end{pgfscope}%
\begin{pgfscope}%
\pgfpathrectangle{\pgfqpoint{0.793996in}{0.443060in}}{\pgfqpoint{2.041610in}{2.577607in}}%
\pgfusepath{clip}%
\pgfsetbuttcap%
\pgfsetroundjoin%
\definecolor{currentfill}{rgb}{0.901453,0.815098,0.767370}%
\pgfsetfillcolor{currentfill}%
\pgfsetlinewidth{0.602250pt}%
\definecolor{currentstroke}{rgb}{0.296471,0.296471,0.296471}%
\pgfsetstrokecolor{currentstroke}%
\pgfsetdash{}{0pt}%
\pgfpathmoveto{\pgfqpoint{2.835606in}{2.396978in}}%
\pgfpathlineto{\pgfqpoint{2.835606in}{2.396978in}}%
\pgfpathlineto{\pgfqpoint{2.835606in}{2.392375in}}%
\pgfpathlineto{\pgfqpoint{2.835606in}{2.392375in}}%
\pgfpathlineto{\pgfqpoint{2.835606in}{2.396978in}}%
\pgfpathclose%
\pgfusepath{stroke,fill}%
\end{pgfscope}%
\begin{pgfscope}%
\pgfpathrectangle{\pgfqpoint{0.793996in}{0.443060in}}{\pgfqpoint{2.041610in}{2.577607in}}%
\pgfusepath{clip}%
\pgfsetbuttcap%
\pgfsetroundjoin%
\definecolor{currentfill}{rgb}{0.910863,0.840546,0.801899}%
\pgfsetfillcolor{currentfill}%
\pgfsetlinewidth{0.602250pt}%
\definecolor{currentstroke}{rgb}{0.296471,0.296471,0.296471}%
\pgfsetstrokecolor{currentstroke}%
\pgfsetdash{}{0pt}%
\pgfpathmoveto{\pgfqpoint{2.835606in}{2.395827in}}%
\pgfpathlineto{\pgfqpoint{2.835606in}{2.395827in}}%
\pgfpathlineto{\pgfqpoint{2.835606in}{2.393526in}}%
\pgfpathlineto{\pgfqpoint{2.835606in}{2.393526in}}%
\pgfpathlineto{\pgfqpoint{2.835606in}{2.395827in}}%
\pgfpathclose%
\pgfusepath{stroke,fill}%
\end{pgfscope}%
\begin{pgfscope}%
\pgfpathrectangle{\pgfqpoint{0.793996in}{0.443060in}}{\pgfqpoint{2.041610in}{2.577607in}}%
\pgfusepath{clip}%
\pgfsetbuttcap%
\pgfsetroundjoin%
\definecolor{currentfill}{rgb}{0.919097,0.862812,0.832112}%
\pgfsetfillcolor{currentfill}%
\pgfsetlinewidth{0.602250pt}%
\definecolor{currentstroke}{rgb}{0.296471,0.296471,0.296471}%
\pgfsetstrokecolor{currentstroke}%
\pgfsetdash{}{0pt}%
\pgfpathmoveto{\pgfqpoint{2.835606in}{2.395252in}}%
\pgfpathlineto{\pgfqpoint{2.835606in}{2.395252in}}%
\pgfpathlineto{\pgfqpoint{2.835606in}{2.394101in}}%
\pgfpathlineto{\pgfqpoint{2.835606in}{2.394101in}}%
\pgfpathlineto{\pgfqpoint{2.835606in}{2.395252in}}%
\pgfpathclose%
\pgfusepath{stroke,fill}%
\end{pgfscope}%
\begin{pgfscope}%
\pgfpathrectangle{\pgfqpoint{0.793996in}{0.443060in}}{\pgfqpoint{2.041610in}{2.577607in}}%
\pgfusepath{clip}%
\pgfsetbuttcap%
\pgfsetroundjoin%
\pgfsetlinewidth{0.803000pt}%
\definecolor{currentstroke}{rgb}{0.450000,0.450000,0.450000}%
\pgfsetstrokecolor{currentstroke}%
\pgfsetdash{}{0pt}%
\pgfpathmoveto{\pgfqpoint{0.000000in}{-0.034722in}}%
\pgfpathcurveto{\pgfqpoint{0.009208in}{-0.034722in}}{\pgfqpoint{0.018041in}{-0.031064in}}{\pgfqpoint{0.024552in}{-0.024552in}}%
\pgfpathcurveto{\pgfqpoint{0.031064in}{-0.018041in}}{\pgfqpoint{0.034722in}{-0.009208in}}{\pgfqpoint{0.034722in}{0.000000in}}%
\pgfpathcurveto{\pgfqpoint{0.034722in}{0.009208in}}{\pgfqpoint{0.031064in}{0.018041in}}{\pgfqpoint{0.024552in}{0.024552in}}%
\pgfpathcurveto{\pgfqpoint{0.018041in}{0.031064in}}{\pgfqpoint{0.009208in}{0.034722in}}{\pgfqpoint{0.000000in}{0.034722in}}%
\pgfpathcurveto{\pgfqpoint{-0.009208in}{0.034722in}}{\pgfqpoint{-0.018041in}{0.031064in}}{\pgfqpoint{-0.024552in}{0.024552in}}%
\pgfpathcurveto{\pgfqpoint{-0.031064in}{0.018041in}}{\pgfqpoint{-0.034722in}{0.009208in}}{\pgfqpoint{-0.034722in}{0.000000in}}%
\pgfpathcurveto{\pgfqpoint{-0.034722in}{-0.009208in}}{\pgfqpoint{-0.031064in}{-0.018041in}}{\pgfqpoint{-0.024552in}{-0.024552in}}%
\pgfpathcurveto{\pgfqpoint{-0.018041in}{-0.031064in}}{\pgfqpoint{-0.009208in}{-0.034722in}}{\pgfqpoint{0.000000in}{-0.034722in}}%
\pgfusepath{stroke}%
\end{pgfscope}%
\begin{pgfscope}%
\pgfpathrectangle{\pgfqpoint{0.793996in}{0.443060in}}{\pgfqpoint{2.041610in}{2.577607in}}%
\pgfusepath{clip}%
\pgfsetbuttcap%
\pgfsetroundjoin%
\definecolor{currentfill}{rgb}{0.848437,0.867532,0.899724}%
\pgfsetfillcolor{currentfill}%
\pgfsetlinewidth{0.602250pt}%
\definecolor{currentstroke}{rgb}{0.296471,0.296471,0.296471}%
\pgfsetstrokecolor{currentstroke}%
\pgfsetdash{}{0pt}%
\pgfpathmoveto{\pgfqpoint{0.871508in}{2.174026in}}%
\pgfpathlineto{\pgfqpoint{0.877292in}{2.174026in}}%
\pgfpathlineto{\pgfqpoint{0.877292in}{2.173451in}}%
\pgfpathlineto{\pgfqpoint{0.871508in}{2.173451in}}%
\pgfpathlineto{\pgfqpoint{0.871508in}{2.174026in}}%
\pgfpathclose%
\pgfusepath{stroke,fill}%
\end{pgfscope}%
\begin{pgfscope}%
\pgfpathrectangle{\pgfqpoint{0.793996in}{0.443060in}}{\pgfqpoint{2.041610in}{2.577607in}}%
\pgfusepath{clip}%
\pgfsetbuttcap%
\pgfsetroundjoin%
\definecolor{currentfill}{rgb}{0.825117,0.848522,0.887698}%
\pgfsetfillcolor{currentfill}%
\pgfsetlinewidth{0.602250pt}%
\definecolor{currentstroke}{rgb}{0.296471,0.296471,0.296471}%
\pgfsetstrokecolor{currentstroke}%
\pgfsetdash{}{0pt}%
\pgfpathmoveto{\pgfqpoint{0.877292in}{2.174314in}}%
\pgfpathlineto{\pgfqpoint{0.880052in}{2.174314in}}%
\pgfpathlineto{\pgfqpoint{0.880052in}{2.173163in}}%
\pgfpathlineto{\pgfqpoint{0.877292in}{2.173163in}}%
\pgfpathlineto{\pgfqpoint{0.877292in}{2.174314in}}%
\pgfpathclose%
\pgfusepath{stroke,fill}%
\end{pgfscope}%
\begin{pgfscope}%
\pgfpathrectangle{\pgfqpoint{0.793996in}{0.443060in}}{\pgfqpoint{2.041610in}{2.577607in}}%
\pgfusepath{clip}%
\pgfsetbuttcap%
\pgfsetroundjoin%
\definecolor{currentfill}{rgb}{0.792469,0.821908,0.870863}%
\pgfsetfillcolor{currentfill}%
\pgfsetlinewidth{0.602250pt}%
\definecolor{currentstroke}{rgb}{0.296471,0.296471,0.296471}%
\pgfsetstrokecolor{currentstroke}%
\pgfsetdash{}{0pt}%
\pgfpathmoveto{\pgfqpoint{0.880052in}{2.174889in}}%
\pgfpathlineto{\pgfqpoint{0.890647in}{2.174889in}}%
\pgfpathlineto{\pgfqpoint{0.890647in}{2.172588in}}%
\pgfpathlineto{\pgfqpoint{0.880052in}{2.172588in}}%
\pgfpathlineto{\pgfqpoint{0.880052in}{2.174889in}}%
\pgfpathclose%
\pgfusepath{stroke,fill}%
\end{pgfscope}%
\begin{pgfscope}%
\pgfpathrectangle{\pgfqpoint{0.793996in}{0.443060in}}{\pgfqpoint{2.041610in}{2.577607in}}%
\pgfusepath{clip}%
\pgfsetbuttcap%
\pgfsetroundjoin%
\definecolor{currentfill}{rgb}{0.755157,0.791493,0.851622}%
\pgfsetfillcolor{currentfill}%
\pgfsetlinewidth{0.602250pt}%
\definecolor{currentstroke}{rgb}{0.296471,0.296471,0.296471}%
\pgfsetstrokecolor{currentstroke}%
\pgfsetdash{}{0pt}%
\pgfpathmoveto{\pgfqpoint{0.890647in}{2.176040in}}%
\pgfpathlineto{\pgfqpoint{0.903778in}{2.176040in}}%
\pgfpathlineto{\pgfqpoint{0.903778in}{2.171437in}}%
\pgfpathlineto{\pgfqpoint{0.890647in}{2.171437in}}%
\pgfpathlineto{\pgfqpoint{0.890647in}{2.176040in}}%
\pgfpathclose%
\pgfusepath{stroke,fill}%
\end{pgfscope}%
\begin{pgfscope}%
\pgfpathrectangle{\pgfqpoint{0.793996in}{0.443060in}}{\pgfqpoint{2.041610in}{2.577607in}}%
\pgfusepath{clip}%
\pgfsetbuttcap%
\pgfsetroundjoin%
\definecolor{currentfill}{rgb}{0.706185,0.751573,0.826368}%
\pgfsetfillcolor{currentfill}%
\pgfsetlinewidth{0.602250pt}%
\definecolor{currentstroke}{rgb}{0.296471,0.296471,0.296471}%
\pgfsetstrokecolor{currentstroke}%
\pgfsetdash{}{0pt}%
\pgfpathmoveto{\pgfqpoint{0.903778in}{2.178341in}}%
\pgfpathlineto{\pgfqpoint{0.917160in}{2.178341in}}%
\pgfpathlineto{\pgfqpoint{0.917160in}{2.169136in}}%
\pgfpathlineto{\pgfqpoint{0.903778in}{2.169136in}}%
\pgfpathlineto{\pgfqpoint{0.903778in}{2.178341in}}%
\pgfpathclose%
\pgfusepath{stroke,fill}%
\end{pgfscope}%
\begin{pgfscope}%
\pgfpathrectangle{\pgfqpoint{0.793996in}{0.443060in}}{\pgfqpoint{2.041610in}{2.577607in}}%
\pgfusepath{clip}%
\pgfsetbuttcap%
\pgfsetroundjoin%
\definecolor{currentfill}{rgb}{0.643221,0.700246,0.793900}%
\pgfsetfillcolor{currentfill}%
\pgfsetlinewidth{0.602250pt}%
\definecolor{currentstroke}{rgb}{0.296471,0.296471,0.296471}%
\pgfsetstrokecolor{currentstroke}%
\pgfsetdash{}{0pt}%
\pgfpathmoveto{\pgfqpoint{0.917160in}{2.182944in}}%
\pgfpathlineto{\pgfqpoint{0.949335in}{2.182944in}}%
\pgfpathlineto{\pgfqpoint{0.949335in}{2.164533in}}%
\pgfpathlineto{\pgfqpoint{0.917160in}{2.164533in}}%
\pgfpathlineto{\pgfqpoint{0.917160in}{2.182944in}}%
\pgfpathclose%
\pgfusepath{stroke,fill}%
\end{pgfscope}%
\begin{pgfscope}%
\pgfpathrectangle{\pgfqpoint{0.793996in}{0.443060in}}{\pgfqpoint{2.041610in}{2.577607in}}%
\pgfusepath{clip}%
\pgfsetbuttcap%
\pgfsetroundjoin%
\definecolor{currentfill}{rgb}{0.566266,0.637515,0.754216}%
\pgfsetfillcolor{currentfill}%
\pgfsetlinewidth{0.602250pt}%
\definecolor{currentstroke}{rgb}{0.296471,0.296471,0.296471}%
\pgfsetstrokecolor{currentstroke}%
\pgfsetdash{}{0pt}%
\pgfpathmoveto{\pgfqpoint{0.949335in}{2.192150in}}%
\pgfpathlineto{\pgfqpoint{1.004237in}{2.192150in}}%
\pgfpathlineto{\pgfqpoint{1.004237in}{2.155327in}}%
\pgfpathlineto{\pgfqpoint{0.949335in}{2.155327in}}%
\pgfpathlineto{\pgfqpoint{0.949335in}{2.192150in}}%
\pgfpathclose%
\pgfusepath{stroke,fill}%
\end{pgfscope}%
\begin{pgfscope}%
\pgfpathrectangle{\pgfqpoint{0.793996in}{0.443060in}}{\pgfqpoint{2.041610in}{2.577607in}}%
\pgfusepath{clip}%
\pgfsetbuttcap%
\pgfsetroundjoin%
\definecolor{currentfill}{rgb}{0.468322,0.557674,0.703709}%
\pgfsetfillcolor{currentfill}%
\pgfsetlinewidth{0.602250pt}%
\definecolor{currentstroke}{rgb}{0.296471,0.296471,0.296471}%
\pgfsetstrokecolor{currentstroke}%
\pgfsetdash{}{0pt}%
\pgfpathmoveto{\pgfqpoint{1.004237in}{2.210562in}}%
\pgfpathlineto{\pgfqpoint{1.111407in}{2.210562in}}%
\pgfpathlineto{\pgfqpoint{1.111407in}{2.136916in}}%
\pgfpathlineto{\pgfqpoint{1.004237in}{2.136916in}}%
\pgfpathlineto{\pgfqpoint{1.004237in}{2.210562in}}%
\pgfpathclose%
\pgfusepath{stroke,fill}%
\end{pgfscope}%
\begin{pgfscope}%
\pgfpathrectangle{\pgfqpoint{0.793996in}{0.443060in}}{\pgfqpoint{2.041610in}{2.577607in}}%
\pgfusepath{clip}%
\pgfsetbuttcap%
\pgfsetroundjoin%
\definecolor{currentfill}{rgb}{0.347059,0.458824,0.641176}%
\pgfsetfillcolor{currentfill}%
\pgfsetlinewidth{0.602250pt}%
\definecolor{currentstroke}{rgb}{0.296471,0.296471,0.296471}%
\pgfsetstrokecolor{currentstroke}%
\pgfsetdash{}{0pt}%
\pgfpathmoveto{\pgfqpoint{1.111407in}{2.247385in}}%
\pgfpathlineto{\pgfqpoint{1.824204in}{2.247385in}}%
\pgfpathlineto{\pgfqpoint{1.824204in}{2.100093in}}%
\pgfpathlineto{\pgfqpoint{1.111407in}{2.100093in}}%
\pgfpathlineto{\pgfqpoint{1.111407in}{2.247385in}}%
\pgfpathclose%
\pgfusepath{stroke,fill}%
\end{pgfscope}%
\begin{pgfscope}%
\pgfpathrectangle{\pgfqpoint{0.793996in}{0.443060in}}{\pgfqpoint{2.041610in}{2.577607in}}%
\pgfusepath{clip}%
\pgfsetbuttcap%
\pgfsetroundjoin%
\definecolor{currentfill}{rgb}{0.468322,0.557674,0.703709}%
\pgfsetfillcolor{currentfill}%
\pgfsetlinewidth{0.602250pt}%
\definecolor{currentstroke}{rgb}{0.296471,0.296471,0.296471}%
\pgfsetstrokecolor{currentstroke}%
\pgfsetdash{}{0pt}%
\pgfpathmoveto{\pgfqpoint{1.824204in}{2.210562in}}%
\pgfpathlineto{\pgfqpoint{2.738537in}{2.210562in}}%
\pgfpathlineto{\pgfqpoint{2.738537in}{2.136916in}}%
\pgfpathlineto{\pgfqpoint{1.824204in}{2.136916in}}%
\pgfpathlineto{\pgfqpoint{1.824204in}{2.210562in}}%
\pgfpathclose%
\pgfusepath{stroke,fill}%
\end{pgfscope}%
\begin{pgfscope}%
\pgfpathrectangle{\pgfqpoint{0.793996in}{0.443060in}}{\pgfqpoint{2.041610in}{2.577607in}}%
\pgfusepath{clip}%
\pgfsetbuttcap%
\pgfsetroundjoin%
\definecolor{currentfill}{rgb}{0.566266,0.637515,0.754216}%
\pgfsetfillcolor{currentfill}%
\pgfsetlinewidth{0.602250pt}%
\definecolor{currentstroke}{rgb}{0.296471,0.296471,0.296471}%
\pgfsetstrokecolor{currentstroke}%
\pgfsetdash{}{0pt}%
\pgfpathmoveto{\pgfqpoint{2.738537in}{2.192150in}}%
\pgfpathlineto{\pgfqpoint{2.835606in}{2.192150in}}%
\pgfpathlineto{\pgfqpoint{2.835606in}{2.155327in}}%
\pgfpathlineto{\pgfqpoint{2.738537in}{2.155327in}}%
\pgfpathlineto{\pgfqpoint{2.738537in}{2.192150in}}%
\pgfpathclose%
\pgfusepath{stroke,fill}%
\end{pgfscope}%
\begin{pgfscope}%
\pgfpathrectangle{\pgfqpoint{0.793996in}{0.443060in}}{\pgfqpoint{2.041610in}{2.577607in}}%
\pgfusepath{clip}%
\pgfsetbuttcap%
\pgfsetroundjoin%
\definecolor{currentfill}{rgb}{0.643221,0.700246,0.793900}%
\pgfsetfillcolor{currentfill}%
\pgfsetlinewidth{0.602250pt}%
\definecolor{currentstroke}{rgb}{0.296471,0.296471,0.296471}%
\pgfsetstrokecolor{currentstroke}%
\pgfsetdash{}{0pt}%
\pgfpathmoveto{\pgfqpoint{2.835606in}{2.182944in}}%
\pgfpathlineto{\pgfqpoint{2.835606in}{2.182944in}}%
\pgfpathlineto{\pgfqpoint{2.835606in}{2.164533in}}%
\pgfpathlineto{\pgfqpoint{2.835606in}{2.164533in}}%
\pgfpathlineto{\pgfqpoint{2.835606in}{2.182944in}}%
\pgfpathclose%
\pgfusepath{stroke,fill}%
\end{pgfscope}%
\begin{pgfscope}%
\pgfpathrectangle{\pgfqpoint{0.793996in}{0.443060in}}{\pgfqpoint{2.041610in}{2.577607in}}%
\pgfusepath{clip}%
\pgfsetbuttcap%
\pgfsetroundjoin%
\definecolor{currentfill}{rgb}{0.706185,0.751573,0.826368}%
\pgfsetfillcolor{currentfill}%
\pgfsetlinewidth{0.602250pt}%
\definecolor{currentstroke}{rgb}{0.296471,0.296471,0.296471}%
\pgfsetstrokecolor{currentstroke}%
\pgfsetdash{}{0pt}%
\pgfpathmoveto{\pgfqpoint{2.835606in}{2.178341in}}%
\pgfpathlineto{\pgfqpoint{2.835606in}{2.178341in}}%
\pgfpathlineto{\pgfqpoint{2.835606in}{2.169136in}}%
\pgfpathlineto{\pgfqpoint{2.835606in}{2.169136in}}%
\pgfpathlineto{\pgfqpoint{2.835606in}{2.178341in}}%
\pgfpathclose%
\pgfusepath{stroke,fill}%
\end{pgfscope}%
\begin{pgfscope}%
\pgfpathrectangle{\pgfqpoint{0.793996in}{0.443060in}}{\pgfqpoint{2.041610in}{2.577607in}}%
\pgfusepath{clip}%
\pgfsetbuttcap%
\pgfsetroundjoin%
\definecolor{currentfill}{rgb}{0.755157,0.791493,0.851622}%
\pgfsetfillcolor{currentfill}%
\pgfsetlinewidth{0.602250pt}%
\definecolor{currentstroke}{rgb}{0.296471,0.296471,0.296471}%
\pgfsetstrokecolor{currentstroke}%
\pgfsetdash{}{0pt}%
\pgfpathmoveto{\pgfqpoint{2.835606in}{2.176040in}}%
\pgfpathlineto{\pgfqpoint{2.835606in}{2.176040in}}%
\pgfpathlineto{\pgfqpoint{2.835606in}{2.171437in}}%
\pgfpathlineto{\pgfqpoint{2.835606in}{2.171437in}}%
\pgfpathlineto{\pgfqpoint{2.835606in}{2.176040in}}%
\pgfpathclose%
\pgfusepath{stroke,fill}%
\end{pgfscope}%
\begin{pgfscope}%
\pgfpathrectangle{\pgfqpoint{0.793996in}{0.443060in}}{\pgfqpoint{2.041610in}{2.577607in}}%
\pgfusepath{clip}%
\pgfsetbuttcap%
\pgfsetroundjoin%
\definecolor{currentfill}{rgb}{0.792469,0.821908,0.870863}%
\pgfsetfillcolor{currentfill}%
\pgfsetlinewidth{0.602250pt}%
\definecolor{currentstroke}{rgb}{0.296471,0.296471,0.296471}%
\pgfsetstrokecolor{currentstroke}%
\pgfsetdash{}{0pt}%
\pgfpathmoveto{\pgfqpoint{2.835606in}{2.174889in}}%
\pgfpathlineto{\pgfqpoint{2.835606in}{2.174889in}}%
\pgfpathlineto{\pgfqpoint{2.835606in}{2.172588in}}%
\pgfpathlineto{\pgfqpoint{2.835606in}{2.172588in}}%
\pgfpathlineto{\pgfqpoint{2.835606in}{2.174889in}}%
\pgfpathclose%
\pgfusepath{stroke,fill}%
\end{pgfscope}%
\begin{pgfscope}%
\pgfpathrectangle{\pgfqpoint{0.793996in}{0.443060in}}{\pgfqpoint{2.041610in}{2.577607in}}%
\pgfusepath{clip}%
\pgfsetbuttcap%
\pgfsetroundjoin%
\definecolor{currentfill}{rgb}{0.825117,0.848522,0.887698}%
\pgfsetfillcolor{currentfill}%
\pgfsetlinewidth{0.602250pt}%
\definecolor{currentstroke}{rgb}{0.296471,0.296471,0.296471}%
\pgfsetstrokecolor{currentstroke}%
\pgfsetdash{}{0pt}%
\pgfpathmoveto{\pgfqpoint{2.835606in}{2.174314in}}%
\pgfpathlineto{\pgfqpoint{2.835606in}{2.174314in}}%
\pgfpathlineto{\pgfqpoint{2.835606in}{2.173163in}}%
\pgfpathlineto{\pgfqpoint{2.835606in}{2.173163in}}%
\pgfpathlineto{\pgfqpoint{2.835606in}{2.174314in}}%
\pgfpathclose%
\pgfusepath{stroke,fill}%
\end{pgfscope}%
\begin{pgfscope}%
\pgfpathrectangle{\pgfqpoint{0.793996in}{0.443060in}}{\pgfqpoint{2.041610in}{2.577607in}}%
\pgfusepath{clip}%
\pgfsetbuttcap%
\pgfsetroundjoin%
\definecolor{currentfill}{rgb}{0.848437,0.867532,0.899724}%
\pgfsetfillcolor{currentfill}%
\pgfsetlinewidth{0.602250pt}%
\definecolor{currentstroke}{rgb}{0.296471,0.296471,0.296471}%
\pgfsetstrokecolor{currentstroke}%
\pgfsetdash{}{0pt}%
\pgfpathmoveto{\pgfqpoint{2.835606in}{2.174026in}}%
\pgfpathlineto{\pgfqpoint{2.835606in}{2.174026in}}%
\pgfpathlineto{\pgfqpoint{2.835606in}{2.173451in}}%
\pgfpathlineto{\pgfqpoint{2.835606in}{2.173451in}}%
\pgfpathlineto{\pgfqpoint{2.835606in}{2.174026in}}%
\pgfpathclose%
\pgfusepath{stroke,fill}%
\end{pgfscope}%
\begin{pgfscope}%
\pgfpathrectangle{\pgfqpoint{0.793996in}{0.443060in}}{\pgfqpoint{2.041610in}{2.577607in}}%
\pgfusepath{clip}%
\pgfsetbuttcap%
\pgfsetroundjoin%
\pgfsetlinewidth{0.803000pt}%
\definecolor{currentstroke}{rgb}{0.450000,0.450000,0.450000}%
\pgfsetstrokecolor{currentstroke}%
\pgfsetdash{}{0pt}%
\pgfpathmoveto{\pgfqpoint{0.000000in}{-0.034722in}}%
\pgfpathcurveto{\pgfqpoint{0.009208in}{-0.034722in}}{\pgfqpoint{0.018041in}{-0.031064in}}{\pgfqpoint{0.024552in}{-0.024552in}}%
\pgfpathcurveto{\pgfqpoint{0.031064in}{-0.018041in}}{\pgfqpoint{0.034722in}{-0.009208in}}{\pgfqpoint{0.034722in}{0.000000in}}%
\pgfpathcurveto{\pgfqpoint{0.034722in}{0.009208in}}{\pgfqpoint{0.031064in}{0.018041in}}{\pgfqpoint{0.024552in}{0.024552in}}%
\pgfpathcurveto{\pgfqpoint{0.018041in}{0.031064in}}{\pgfqpoint{0.009208in}{0.034722in}}{\pgfqpoint{0.000000in}{0.034722in}}%
\pgfpathcurveto{\pgfqpoint{-0.009208in}{0.034722in}}{\pgfqpoint{-0.018041in}{0.031064in}}{\pgfqpoint{-0.024552in}{0.024552in}}%
\pgfpathcurveto{\pgfqpoint{-0.031064in}{0.018041in}}{\pgfqpoint{-0.034722in}{0.009208in}}{\pgfqpoint{-0.034722in}{0.000000in}}%
\pgfpathcurveto{\pgfqpoint{-0.034722in}{-0.009208in}}{\pgfqpoint{-0.031064in}{-0.018041in}}{\pgfqpoint{-0.024552in}{-0.024552in}}%
\pgfpathcurveto{\pgfqpoint{-0.018041in}{-0.031064in}}{\pgfqpoint{-0.009208in}{-0.034722in}}{\pgfqpoint{0.000000in}{-0.034722in}}%
\pgfusepath{stroke}%
\end{pgfscope}%
\begin{pgfscope}%
\pgfpathrectangle{\pgfqpoint{0.793996in}{0.443060in}}{\pgfqpoint{2.041610in}{2.577607in}}%
\pgfusepath{clip}%
\pgfsetbuttcap%
\pgfsetroundjoin%
\definecolor{currentfill}{rgb}{0.919097,0.862812,0.832112}%
\pgfsetfillcolor{currentfill}%
\pgfsetlinewidth{0.602250pt}%
\definecolor{currentstroke}{rgb}{0.296471,0.296471,0.296471}%
\pgfsetstrokecolor{currentstroke}%
\pgfsetdash{}{0pt}%
\pgfpathmoveto{\pgfqpoint{1.148191in}{2.027022in}}%
\pgfpathlineto{\pgfqpoint{1.184667in}{2.027022in}}%
\pgfpathlineto{\pgfqpoint{1.184667in}{2.025871in}}%
\pgfpathlineto{\pgfqpoint{1.148191in}{2.025871in}}%
\pgfpathlineto{\pgfqpoint{1.148191in}{2.027022in}}%
\pgfpathclose%
\pgfusepath{stroke,fill}%
\end{pgfscope}%
\begin{pgfscope}%
\pgfpathrectangle{\pgfqpoint{0.793996in}{0.443060in}}{\pgfqpoint{2.041610in}{2.577607in}}%
\pgfusepath{clip}%
\pgfsetbuttcap%
\pgfsetroundjoin%
\definecolor{currentfill}{rgb}{0.910863,0.840546,0.801899}%
\pgfsetfillcolor{currentfill}%
\pgfsetlinewidth{0.602250pt}%
\definecolor{currentstroke}{rgb}{0.296471,0.296471,0.296471}%
\pgfsetstrokecolor{currentstroke}%
\pgfsetdash{}{0pt}%
\pgfpathmoveto{\pgfqpoint{1.184667in}{2.027598in}}%
\pgfpathlineto{\pgfqpoint{1.196078in}{2.027598in}}%
\pgfpathlineto{\pgfqpoint{1.196078in}{2.025296in}}%
\pgfpathlineto{\pgfqpoint{1.184667in}{2.025296in}}%
\pgfpathlineto{\pgfqpoint{1.184667in}{2.027598in}}%
\pgfpathclose%
\pgfusepath{stroke,fill}%
\end{pgfscope}%
\begin{pgfscope}%
\pgfpathrectangle{\pgfqpoint{0.793996in}{0.443060in}}{\pgfqpoint{2.041610in}{2.577607in}}%
\pgfusepath{clip}%
\pgfsetbuttcap%
\pgfsetroundjoin%
\definecolor{currentfill}{rgb}{0.901453,0.815098,0.767370}%
\pgfsetfillcolor{currentfill}%
\pgfsetlinewidth{0.602250pt}%
\definecolor{currentstroke}{rgb}{0.296471,0.296471,0.296471}%
\pgfsetstrokecolor{currentstroke}%
\pgfsetdash{}{0pt}%
\pgfpathmoveto{\pgfqpoint{1.196078in}{2.028748in}}%
\pgfpathlineto{\pgfqpoint{1.220225in}{2.028748in}}%
\pgfpathlineto{\pgfqpoint{1.220225in}{2.024145in}}%
\pgfpathlineto{\pgfqpoint{1.196078in}{2.024145in}}%
\pgfpathlineto{\pgfqpoint{1.196078in}{2.028748in}}%
\pgfpathclose%
\pgfusepath{stroke,fill}%
\end{pgfscope}%
\begin{pgfscope}%
\pgfpathrectangle{\pgfqpoint{0.793996in}{0.443060in}}{\pgfqpoint{2.041610in}{2.577607in}}%
\pgfusepath{clip}%
\pgfsetbuttcap%
\pgfsetroundjoin%
\definecolor{currentfill}{rgb}{0.889102,0.781698,0.722050}%
\pgfsetfillcolor{currentfill}%
\pgfsetlinewidth{0.602250pt}%
\definecolor{currentstroke}{rgb}{0.296471,0.296471,0.296471}%
\pgfsetstrokecolor{currentstroke}%
\pgfsetdash{}{0pt}%
\pgfpathmoveto{\pgfqpoint{1.220225in}{2.031050in}}%
\pgfpathlineto{\pgfqpoint{1.260886in}{2.031050in}}%
\pgfpathlineto{\pgfqpoint{1.260886in}{2.021844in}}%
\pgfpathlineto{\pgfqpoint{1.220225in}{2.021844in}}%
\pgfpathlineto{\pgfqpoint{1.220225in}{2.031050in}}%
\pgfpathclose%
\pgfusepath{stroke,fill}%
\end{pgfscope}%
\begin{pgfscope}%
\pgfpathrectangle{\pgfqpoint{0.793996in}{0.443060in}}{\pgfqpoint{2.041610in}{2.577607in}}%
\pgfusepath{clip}%
\pgfsetbuttcap%
\pgfsetroundjoin%
\definecolor{currentfill}{rgb}{0.873223,0.738755,0.663782}%
\pgfsetfillcolor{currentfill}%
\pgfsetlinewidth{0.602250pt}%
\definecolor{currentstroke}{rgb}{0.296471,0.296471,0.296471}%
\pgfsetstrokecolor{currentstroke}%
\pgfsetdash{}{0pt}%
\pgfpathmoveto{\pgfqpoint{1.260886in}{2.035653in}}%
\pgfpathlineto{\pgfqpoint{1.322073in}{2.035653in}}%
\pgfpathlineto{\pgfqpoint{1.322073in}{2.017241in}}%
\pgfpathlineto{\pgfqpoint{1.260886in}{2.017241in}}%
\pgfpathlineto{\pgfqpoint{1.260886in}{2.035653in}}%
\pgfpathclose%
\pgfusepath{stroke,fill}%
\end{pgfscope}%
\begin{pgfscope}%
\pgfpathrectangle{\pgfqpoint{0.793996in}{0.443060in}}{\pgfqpoint{2.041610in}{2.577607in}}%
\pgfusepath{clip}%
\pgfsetbuttcap%
\pgfsetroundjoin%
\definecolor{currentfill}{rgb}{0.853814,0.686269,0.592565}%
\pgfsetfillcolor{currentfill}%
\pgfsetlinewidth{0.602250pt}%
\definecolor{currentstroke}{rgb}{0.296471,0.296471,0.296471}%
\pgfsetstrokecolor{currentstroke}%
\pgfsetdash{}{0pt}%
\pgfpathmoveto{\pgfqpoint{1.322073in}{2.044858in}}%
\pgfpathlineto{\pgfqpoint{1.401801in}{2.044858in}}%
\pgfpathlineto{\pgfqpoint{1.401801in}{2.008035in}}%
\pgfpathlineto{\pgfqpoint{1.322073in}{2.008035in}}%
\pgfpathlineto{\pgfqpoint{1.322073in}{2.044858in}}%
\pgfpathclose%
\pgfusepath{stroke,fill}%
\end{pgfscope}%
\begin{pgfscope}%
\pgfpathrectangle{\pgfqpoint{0.793996in}{0.443060in}}{\pgfqpoint{2.041610in}{2.577607in}}%
\pgfusepath{clip}%
\pgfsetbuttcap%
\pgfsetroundjoin%
\definecolor{currentfill}{rgb}{0.829112,0.619469,0.501926}%
\pgfsetfillcolor{currentfill}%
\pgfsetlinewidth{0.602250pt}%
\definecolor{currentstroke}{rgb}{0.296471,0.296471,0.296471}%
\pgfsetstrokecolor{currentstroke}%
\pgfsetdash{}{0pt}%
\pgfpathmoveto{\pgfqpoint{1.401801in}{2.063270in}}%
\pgfpathlineto{\pgfqpoint{1.482964in}{2.063270in}}%
\pgfpathlineto{\pgfqpoint{1.482964in}{1.989624in}}%
\pgfpathlineto{\pgfqpoint{1.401801in}{1.989624in}}%
\pgfpathlineto{\pgfqpoint{1.401801in}{2.063270in}}%
\pgfpathclose%
\pgfusepath{stroke,fill}%
\end{pgfscope}%
\begin{pgfscope}%
\pgfpathrectangle{\pgfqpoint{0.793996in}{0.443060in}}{\pgfqpoint{2.041610in}{2.577607in}}%
\pgfusepath{clip}%
\pgfsetbuttcap%
\pgfsetroundjoin%
\definecolor{currentfill}{rgb}{0.798529,0.536765,0.389706}%
\pgfsetfillcolor{currentfill}%
\pgfsetlinewidth{0.602250pt}%
\definecolor{currentstroke}{rgb}{0.296471,0.296471,0.296471}%
\pgfsetstrokecolor{currentstroke}%
\pgfsetdash{}{0pt}%
\pgfpathmoveto{\pgfqpoint{1.482964in}{2.100093in}}%
\pgfpathlineto{\pgfqpoint{2.835606in}{2.100093in}}%
\pgfpathlineto{\pgfqpoint{2.835606in}{1.952801in}}%
\pgfpathlineto{\pgfqpoint{1.482964in}{1.952801in}}%
\pgfpathlineto{\pgfqpoint{1.482964in}{2.100093in}}%
\pgfpathclose%
\pgfusepath{stroke,fill}%
\end{pgfscope}%
\begin{pgfscope}%
\pgfpathrectangle{\pgfqpoint{0.793996in}{0.443060in}}{\pgfqpoint{2.041610in}{2.577607in}}%
\pgfusepath{clip}%
\pgfsetbuttcap%
\pgfsetroundjoin%
\definecolor{currentfill}{rgb}{0.829112,0.619469,0.501926}%
\pgfsetfillcolor{currentfill}%
\pgfsetlinewidth{0.602250pt}%
\definecolor{currentstroke}{rgb}{0.296471,0.296471,0.296471}%
\pgfsetstrokecolor{currentstroke}%
\pgfsetdash{}{0pt}%
\pgfpathmoveto{\pgfqpoint{2.835606in}{2.063270in}}%
\pgfpathlineto{\pgfqpoint{2.835606in}{2.063270in}}%
\pgfpathlineto{\pgfqpoint{2.835606in}{1.989624in}}%
\pgfpathlineto{\pgfqpoint{2.835606in}{1.989624in}}%
\pgfpathlineto{\pgfqpoint{2.835606in}{2.063270in}}%
\pgfpathclose%
\pgfusepath{stroke,fill}%
\end{pgfscope}%
\begin{pgfscope}%
\pgfpathrectangle{\pgfqpoint{0.793996in}{0.443060in}}{\pgfqpoint{2.041610in}{2.577607in}}%
\pgfusepath{clip}%
\pgfsetbuttcap%
\pgfsetroundjoin%
\definecolor{currentfill}{rgb}{0.853814,0.686269,0.592565}%
\pgfsetfillcolor{currentfill}%
\pgfsetlinewidth{0.602250pt}%
\definecolor{currentstroke}{rgb}{0.296471,0.296471,0.296471}%
\pgfsetstrokecolor{currentstroke}%
\pgfsetdash{}{0pt}%
\pgfpathmoveto{\pgfqpoint{2.835606in}{2.044858in}}%
\pgfpathlineto{\pgfqpoint{2.835606in}{2.044858in}}%
\pgfpathlineto{\pgfqpoint{2.835606in}{2.008035in}}%
\pgfpathlineto{\pgfqpoint{2.835606in}{2.008035in}}%
\pgfpathlineto{\pgfqpoint{2.835606in}{2.044858in}}%
\pgfpathclose%
\pgfusepath{stroke,fill}%
\end{pgfscope}%
\begin{pgfscope}%
\pgfpathrectangle{\pgfqpoint{0.793996in}{0.443060in}}{\pgfqpoint{2.041610in}{2.577607in}}%
\pgfusepath{clip}%
\pgfsetbuttcap%
\pgfsetroundjoin%
\definecolor{currentfill}{rgb}{0.873223,0.738755,0.663782}%
\pgfsetfillcolor{currentfill}%
\pgfsetlinewidth{0.602250pt}%
\definecolor{currentstroke}{rgb}{0.296471,0.296471,0.296471}%
\pgfsetstrokecolor{currentstroke}%
\pgfsetdash{}{0pt}%
\pgfpathmoveto{\pgfqpoint{2.835606in}{2.035653in}}%
\pgfpathlineto{\pgfqpoint{2.835606in}{2.035653in}}%
\pgfpathlineto{\pgfqpoint{2.835606in}{2.017241in}}%
\pgfpathlineto{\pgfqpoint{2.835606in}{2.017241in}}%
\pgfpathlineto{\pgfqpoint{2.835606in}{2.035653in}}%
\pgfpathclose%
\pgfusepath{stroke,fill}%
\end{pgfscope}%
\begin{pgfscope}%
\pgfpathrectangle{\pgfqpoint{0.793996in}{0.443060in}}{\pgfqpoint{2.041610in}{2.577607in}}%
\pgfusepath{clip}%
\pgfsetbuttcap%
\pgfsetroundjoin%
\definecolor{currentfill}{rgb}{0.889102,0.781698,0.722050}%
\pgfsetfillcolor{currentfill}%
\pgfsetlinewidth{0.602250pt}%
\definecolor{currentstroke}{rgb}{0.296471,0.296471,0.296471}%
\pgfsetstrokecolor{currentstroke}%
\pgfsetdash{}{0pt}%
\pgfpathmoveto{\pgfqpoint{2.835606in}{2.031050in}}%
\pgfpathlineto{\pgfqpoint{2.835606in}{2.031050in}}%
\pgfpathlineto{\pgfqpoint{2.835606in}{2.021844in}}%
\pgfpathlineto{\pgfqpoint{2.835606in}{2.021844in}}%
\pgfpathlineto{\pgfqpoint{2.835606in}{2.031050in}}%
\pgfpathclose%
\pgfusepath{stroke,fill}%
\end{pgfscope}%
\begin{pgfscope}%
\pgfpathrectangle{\pgfqpoint{0.793996in}{0.443060in}}{\pgfqpoint{2.041610in}{2.577607in}}%
\pgfusepath{clip}%
\pgfsetbuttcap%
\pgfsetroundjoin%
\definecolor{currentfill}{rgb}{0.901453,0.815098,0.767370}%
\pgfsetfillcolor{currentfill}%
\pgfsetlinewidth{0.602250pt}%
\definecolor{currentstroke}{rgb}{0.296471,0.296471,0.296471}%
\pgfsetstrokecolor{currentstroke}%
\pgfsetdash{}{0pt}%
\pgfpathmoveto{\pgfqpoint{2.835606in}{2.028748in}}%
\pgfpathlineto{\pgfqpoint{2.835606in}{2.028748in}}%
\pgfpathlineto{\pgfqpoint{2.835606in}{2.024145in}}%
\pgfpathlineto{\pgfqpoint{2.835606in}{2.024145in}}%
\pgfpathlineto{\pgfqpoint{2.835606in}{2.028748in}}%
\pgfpathclose%
\pgfusepath{stroke,fill}%
\end{pgfscope}%
\begin{pgfscope}%
\pgfpathrectangle{\pgfqpoint{0.793996in}{0.443060in}}{\pgfqpoint{2.041610in}{2.577607in}}%
\pgfusepath{clip}%
\pgfsetbuttcap%
\pgfsetroundjoin%
\definecolor{currentfill}{rgb}{0.910863,0.840546,0.801899}%
\pgfsetfillcolor{currentfill}%
\pgfsetlinewidth{0.602250pt}%
\definecolor{currentstroke}{rgb}{0.296471,0.296471,0.296471}%
\pgfsetstrokecolor{currentstroke}%
\pgfsetdash{}{0pt}%
\pgfpathmoveto{\pgfqpoint{2.835606in}{2.027598in}}%
\pgfpathlineto{\pgfqpoint{2.835606in}{2.027598in}}%
\pgfpathlineto{\pgfqpoint{2.835606in}{2.025296in}}%
\pgfpathlineto{\pgfqpoint{2.835606in}{2.025296in}}%
\pgfpathlineto{\pgfqpoint{2.835606in}{2.027598in}}%
\pgfpathclose%
\pgfusepath{stroke,fill}%
\end{pgfscope}%
\begin{pgfscope}%
\pgfpathrectangle{\pgfqpoint{0.793996in}{0.443060in}}{\pgfqpoint{2.041610in}{2.577607in}}%
\pgfusepath{clip}%
\pgfsetbuttcap%
\pgfsetroundjoin%
\definecolor{currentfill}{rgb}{0.919097,0.862812,0.832112}%
\pgfsetfillcolor{currentfill}%
\pgfsetlinewidth{0.602250pt}%
\definecolor{currentstroke}{rgb}{0.296471,0.296471,0.296471}%
\pgfsetstrokecolor{currentstroke}%
\pgfsetdash{}{0pt}%
\pgfpathmoveto{\pgfqpoint{2.835606in}{2.027022in}}%
\pgfpathlineto{\pgfqpoint{2.835606in}{2.027022in}}%
\pgfpathlineto{\pgfqpoint{2.835606in}{2.025871in}}%
\pgfpathlineto{\pgfqpoint{2.835606in}{2.025871in}}%
\pgfpathlineto{\pgfqpoint{2.835606in}{2.027022in}}%
\pgfpathclose%
\pgfusepath{stroke,fill}%
\end{pgfscope}%
\begin{pgfscope}%
\pgfpathrectangle{\pgfqpoint{0.793996in}{0.443060in}}{\pgfqpoint{2.041610in}{2.577607in}}%
\pgfusepath{clip}%
\pgfsetbuttcap%
\pgfsetroundjoin%
\pgfsetlinewidth{0.803000pt}%
\definecolor{currentstroke}{rgb}{0.450000,0.450000,0.450000}%
\pgfsetstrokecolor{currentstroke}%
\pgfsetdash{}{0pt}%
\pgfpathmoveto{\pgfqpoint{0.000000in}{-0.034722in}}%
\pgfpathcurveto{\pgfqpoint{0.009208in}{-0.034722in}}{\pgfqpoint{0.018041in}{-0.031064in}}{\pgfqpoint{0.024552in}{-0.024552in}}%
\pgfpathcurveto{\pgfqpoint{0.031064in}{-0.018041in}}{\pgfqpoint{0.034722in}{-0.009208in}}{\pgfqpoint{0.034722in}{0.000000in}}%
\pgfpathcurveto{\pgfqpoint{0.034722in}{0.009208in}}{\pgfqpoint{0.031064in}{0.018041in}}{\pgfqpoint{0.024552in}{0.024552in}}%
\pgfpathcurveto{\pgfqpoint{0.018041in}{0.031064in}}{\pgfqpoint{0.009208in}{0.034722in}}{\pgfqpoint{0.000000in}{0.034722in}}%
\pgfpathcurveto{\pgfqpoint{-0.009208in}{0.034722in}}{\pgfqpoint{-0.018041in}{0.031064in}}{\pgfqpoint{-0.024552in}{0.024552in}}%
\pgfpathcurveto{\pgfqpoint{-0.031064in}{0.018041in}}{\pgfqpoint{-0.034722in}{0.009208in}}{\pgfqpoint{-0.034722in}{0.000000in}}%
\pgfpathcurveto{\pgfqpoint{-0.034722in}{-0.009208in}}{\pgfqpoint{-0.031064in}{-0.018041in}}{\pgfqpoint{-0.024552in}{-0.024552in}}%
\pgfpathcurveto{\pgfqpoint{-0.018041in}{-0.031064in}}{\pgfqpoint{-0.009208in}{-0.034722in}}{\pgfqpoint{0.000000in}{-0.034722in}}%
\pgfusepath{stroke}%
\end{pgfscope}%
\begin{pgfscope}%
\pgfpathrectangle{\pgfqpoint{0.793996in}{0.443060in}}{\pgfqpoint{2.041610in}{2.577607in}}%
\pgfusepath{clip}%
\pgfsetbuttcap%
\pgfsetroundjoin%
\definecolor{currentfill}{rgb}{0.848437,0.867532,0.899724}%
\pgfsetfillcolor{currentfill}%
\pgfsetlinewidth{0.602250pt}%
\definecolor{currentstroke}{rgb}{0.296471,0.296471,0.296471}%
\pgfsetstrokecolor{currentstroke}%
\pgfsetdash{}{0pt}%
\pgfpathmoveto{\pgfqpoint{0.793996in}{1.805797in}}%
\pgfpathlineto{\pgfqpoint{0.793996in}{1.805797in}}%
\pgfpathlineto{\pgfqpoint{0.793996in}{1.805221in}}%
\pgfpathlineto{\pgfqpoint{0.793996in}{1.805221in}}%
\pgfpathlineto{\pgfqpoint{0.793996in}{1.805797in}}%
\pgfpathclose%
\pgfusepath{stroke,fill}%
\end{pgfscope}%
\begin{pgfscope}%
\pgfpathrectangle{\pgfqpoint{0.793996in}{0.443060in}}{\pgfqpoint{2.041610in}{2.577607in}}%
\pgfusepath{clip}%
\pgfsetbuttcap%
\pgfsetroundjoin%
\definecolor{currentfill}{rgb}{0.825117,0.848522,0.887698}%
\pgfsetfillcolor{currentfill}%
\pgfsetlinewidth{0.602250pt}%
\definecolor{currentstroke}{rgb}{0.296471,0.296471,0.296471}%
\pgfsetstrokecolor{currentstroke}%
\pgfsetdash{}{0pt}%
\pgfpathmoveto{\pgfqpoint{0.793996in}{1.806084in}}%
\pgfpathlineto{\pgfqpoint{0.793996in}{1.806084in}}%
\pgfpathlineto{\pgfqpoint{0.793996in}{1.804934in}}%
\pgfpathlineto{\pgfqpoint{0.793996in}{1.804934in}}%
\pgfpathlineto{\pgfqpoint{0.793996in}{1.806084in}}%
\pgfpathclose%
\pgfusepath{stroke,fill}%
\end{pgfscope}%
\begin{pgfscope}%
\pgfpathrectangle{\pgfqpoint{0.793996in}{0.443060in}}{\pgfqpoint{2.041610in}{2.577607in}}%
\pgfusepath{clip}%
\pgfsetbuttcap%
\pgfsetroundjoin%
\definecolor{currentfill}{rgb}{0.792469,0.821908,0.870863}%
\pgfsetfillcolor{currentfill}%
\pgfsetlinewidth{0.602250pt}%
\definecolor{currentstroke}{rgb}{0.296471,0.296471,0.296471}%
\pgfsetstrokecolor{currentstroke}%
\pgfsetdash{}{0pt}%
\pgfpathmoveto{\pgfqpoint{0.793996in}{1.806660in}}%
\pgfpathlineto{\pgfqpoint{0.793996in}{1.806660in}}%
\pgfpathlineto{\pgfqpoint{0.793996in}{1.804358in}}%
\pgfpathlineto{\pgfqpoint{0.793996in}{1.804358in}}%
\pgfpathlineto{\pgfqpoint{0.793996in}{1.806660in}}%
\pgfpathclose%
\pgfusepath{stroke,fill}%
\end{pgfscope}%
\begin{pgfscope}%
\pgfpathrectangle{\pgfqpoint{0.793996in}{0.443060in}}{\pgfqpoint{2.041610in}{2.577607in}}%
\pgfusepath{clip}%
\pgfsetbuttcap%
\pgfsetroundjoin%
\definecolor{currentfill}{rgb}{0.755157,0.791493,0.851622}%
\pgfsetfillcolor{currentfill}%
\pgfsetlinewidth{0.602250pt}%
\definecolor{currentstroke}{rgb}{0.296471,0.296471,0.296471}%
\pgfsetstrokecolor{currentstroke}%
\pgfsetdash{}{0pt}%
\pgfpathmoveto{\pgfqpoint{0.793996in}{1.807810in}}%
\pgfpathlineto{\pgfqpoint{0.793996in}{1.807810in}}%
\pgfpathlineto{\pgfqpoint{0.793996in}{1.803208in}}%
\pgfpathlineto{\pgfqpoint{0.793996in}{1.803208in}}%
\pgfpathlineto{\pgfqpoint{0.793996in}{1.807810in}}%
\pgfpathclose%
\pgfusepath{stroke,fill}%
\end{pgfscope}%
\begin{pgfscope}%
\pgfpathrectangle{\pgfqpoint{0.793996in}{0.443060in}}{\pgfqpoint{2.041610in}{2.577607in}}%
\pgfusepath{clip}%
\pgfsetbuttcap%
\pgfsetroundjoin%
\definecolor{currentfill}{rgb}{0.706185,0.751573,0.826368}%
\pgfsetfillcolor{currentfill}%
\pgfsetlinewidth{0.602250pt}%
\definecolor{currentstroke}{rgb}{0.296471,0.296471,0.296471}%
\pgfsetstrokecolor{currentstroke}%
\pgfsetdash{}{0pt}%
\pgfpathmoveto{\pgfqpoint{0.793996in}{1.810112in}}%
\pgfpathlineto{\pgfqpoint{0.793996in}{1.810112in}}%
\pgfpathlineto{\pgfqpoint{0.793996in}{1.800906in}}%
\pgfpathlineto{\pgfqpoint{0.793996in}{1.800906in}}%
\pgfpathlineto{\pgfqpoint{0.793996in}{1.810112in}}%
\pgfpathclose%
\pgfusepath{stroke,fill}%
\end{pgfscope}%
\begin{pgfscope}%
\pgfpathrectangle{\pgfqpoint{0.793996in}{0.443060in}}{\pgfqpoint{2.041610in}{2.577607in}}%
\pgfusepath{clip}%
\pgfsetbuttcap%
\pgfsetroundjoin%
\definecolor{currentfill}{rgb}{0.643221,0.700246,0.793900}%
\pgfsetfillcolor{currentfill}%
\pgfsetlinewidth{0.602250pt}%
\definecolor{currentstroke}{rgb}{0.296471,0.296471,0.296471}%
\pgfsetstrokecolor{currentstroke}%
\pgfsetdash{}{0pt}%
\pgfpathmoveto{\pgfqpoint{0.793996in}{1.814715in}}%
\pgfpathlineto{\pgfqpoint{0.793996in}{1.814715in}}%
\pgfpathlineto{\pgfqpoint{0.793996in}{1.796303in}}%
\pgfpathlineto{\pgfqpoint{0.793996in}{1.796303in}}%
\pgfpathlineto{\pgfqpoint{0.793996in}{1.814715in}}%
\pgfpathclose%
\pgfusepath{stroke,fill}%
\end{pgfscope}%
\begin{pgfscope}%
\pgfpathrectangle{\pgfqpoint{0.793996in}{0.443060in}}{\pgfqpoint{2.041610in}{2.577607in}}%
\pgfusepath{clip}%
\pgfsetbuttcap%
\pgfsetroundjoin%
\definecolor{currentfill}{rgb}{0.566266,0.637515,0.754216}%
\pgfsetfillcolor{currentfill}%
\pgfsetlinewidth{0.602250pt}%
\definecolor{currentstroke}{rgb}{0.296471,0.296471,0.296471}%
\pgfsetstrokecolor{currentstroke}%
\pgfsetdash{}{0pt}%
\pgfpathmoveto{\pgfqpoint{0.793996in}{1.823921in}}%
\pgfpathlineto{\pgfqpoint{0.793996in}{1.823921in}}%
\pgfpathlineto{\pgfqpoint{0.793996in}{1.787098in}}%
\pgfpathlineto{\pgfqpoint{0.793996in}{1.787098in}}%
\pgfpathlineto{\pgfqpoint{0.793996in}{1.823921in}}%
\pgfpathclose%
\pgfusepath{stroke,fill}%
\end{pgfscope}%
\begin{pgfscope}%
\pgfpathrectangle{\pgfqpoint{0.793996in}{0.443060in}}{\pgfqpoint{2.041610in}{2.577607in}}%
\pgfusepath{clip}%
\pgfsetbuttcap%
\pgfsetroundjoin%
\definecolor{currentfill}{rgb}{0.468322,0.557674,0.703709}%
\pgfsetfillcolor{currentfill}%
\pgfsetlinewidth{0.602250pt}%
\definecolor{currentstroke}{rgb}{0.296471,0.296471,0.296471}%
\pgfsetstrokecolor{currentstroke}%
\pgfsetdash{}{0pt}%
\pgfpathmoveto{\pgfqpoint{0.793996in}{1.842332in}}%
\pgfpathlineto{\pgfqpoint{0.793996in}{1.842332in}}%
\pgfpathlineto{\pgfqpoint{0.793996in}{1.768686in}}%
\pgfpathlineto{\pgfqpoint{0.793996in}{1.768686in}}%
\pgfpathlineto{\pgfqpoint{0.793996in}{1.842332in}}%
\pgfpathclose%
\pgfusepath{stroke,fill}%
\end{pgfscope}%
\begin{pgfscope}%
\pgfpathrectangle{\pgfqpoint{0.793996in}{0.443060in}}{\pgfqpoint{2.041610in}{2.577607in}}%
\pgfusepath{clip}%
\pgfsetbuttcap%
\pgfsetroundjoin%
\definecolor{currentfill}{rgb}{0.347059,0.458824,0.641176}%
\pgfsetfillcolor{currentfill}%
\pgfsetlinewidth{0.602250pt}%
\definecolor{currentstroke}{rgb}{0.296471,0.296471,0.296471}%
\pgfsetstrokecolor{currentstroke}%
\pgfsetdash{}{0pt}%
\pgfpathmoveto{\pgfqpoint{0.793996in}{1.879155in}}%
\pgfpathlineto{\pgfqpoint{0.793996in}{1.879155in}}%
\pgfpathlineto{\pgfqpoint{0.793996in}{1.731863in}}%
\pgfpathlineto{\pgfqpoint{0.793996in}{1.731863in}}%
\pgfpathlineto{\pgfqpoint{0.793996in}{1.879155in}}%
\pgfpathclose%
\pgfusepath{stroke,fill}%
\end{pgfscope}%
\begin{pgfscope}%
\pgfpathrectangle{\pgfqpoint{0.793996in}{0.443060in}}{\pgfqpoint{2.041610in}{2.577607in}}%
\pgfusepath{clip}%
\pgfsetbuttcap%
\pgfsetroundjoin%
\definecolor{currentfill}{rgb}{0.468322,0.557674,0.703709}%
\pgfsetfillcolor{currentfill}%
\pgfsetlinewidth{0.602250pt}%
\definecolor{currentstroke}{rgb}{0.296471,0.296471,0.296471}%
\pgfsetstrokecolor{currentstroke}%
\pgfsetdash{}{0pt}%
\pgfpathmoveto{\pgfqpoint{0.793996in}{1.842332in}}%
\pgfpathlineto{\pgfqpoint{0.793996in}{1.842332in}}%
\pgfpathlineto{\pgfqpoint{0.793996in}{1.768686in}}%
\pgfpathlineto{\pgfqpoint{0.793996in}{1.768686in}}%
\pgfpathlineto{\pgfqpoint{0.793996in}{1.842332in}}%
\pgfpathclose%
\pgfusepath{stroke,fill}%
\end{pgfscope}%
\begin{pgfscope}%
\pgfpathrectangle{\pgfqpoint{0.793996in}{0.443060in}}{\pgfqpoint{2.041610in}{2.577607in}}%
\pgfusepath{clip}%
\pgfsetbuttcap%
\pgfsetroundjoin%
\definecolor{currentfill}{rgb}{0.566266,0.637515,0.754216}%
\pgfsetfillcolor{currentfill}%
\pgfsetlinewidth{0.602250pt}%
\definecolor{currentstroke}{rgb}{0.296471,0.296471,0.296471}%
\pgfsetstrokecolor{currentstroke}%
\pgfsetdash{}{0pt}%
\pgfpathmoveto{\pgfqpoint{0.793996in}{1.823921in}}%
\pgfpathlineto{\pgfqpoint{0.793996in}{1.823921in}}%
\pgfpathlineto{\pgfqpoint{0.793996in}{1.787098in}}%
\pgfpathlineto{\pgfqpoint{0.793996in}{1.787098in}}%
\pgfpathlineto{\pgfqpoint{0.793996in}{1.823921in}}%
\pgfpathclose%
\pgfusepath{stroke,fill}%
\end{pgfscope}%
\begin{pgfscope}%
\pgfpathrectangle{\pgfqpoint{0.793996in}{0.443060in}}{\pgfqpoint{2.041610in}{2.577607in}}%
\pgfusepath{clip}%
\pgfsetbuttcap%
\pgfsetroundjoin%
\definecolor{currentfill}{rgb}{0.643221,0.700246,0.793900}%
\pgfsetfillcolor{currentfill}%
\pgfsetlinewidth{0.602250pt}%
\definecolor{currentstroke}{rgb}{0.296471,0.296471,0.296471}%
\pgfsetstrokecolor{currentstroke}%
\pgfsetdash{}{0pt}%
\pgfpathmoveto{\pgfqpoint{0.793996in}{1.814715in}}%
\pgfpathlineto{\pgfqpoint{0.793996in}{1.814715in}}%
\pgfpathlineto{\pgfqpoint{0.793996in}{1.796303in}}%
\pgfpathlineto{\pgfqpoint{0.793996in}{1.796303in}}%
\pgfpathlineto{\pgfqpoint{0.793996in}{1.814715in}}%
\pgfpathclose%
\pgfusepath{stroke,fill}%
\end{pgfscope}%
\begin{pgfscope}%
\pgfpathrectangle{\pgfqpoint{0.793996in}{0.443060in}}{\pgfqpoint{2.041610in}{2.577607in}}%
\pgfusepath{clip}%
\pgfsetbuttcap%
\pgfsetroundjoin%
\definecolor{currentfill}{rgb}{0.706185,0.751573,0.826368}%
\pgfsetfillcolor{currentfill}%
\pgfsetlinewidth{0.602250pt}%
\definecolor{currentstroke}{rgb}{0.296471,0.296471,0.296471}%
\pgfsetstrokecolor{currentstroke}%
\pgfsetdash{}{0pt}%
\pgfpathmoveto{\pgfqpoint{0.793996in}{1.810112in}}%
\pgfpathlineto{\pgfqpoint{0.793996in}{1.810112in}}%
\pgfpathlineto{\pgfqpoint{0.793996in}{1.800906in}}%
\pgfpathlineto{\pgfqpoint{0.793996in}{1.800906in}}%
\pgfpathlineto{\pgfqpoint{0.793996in}{1.810112in}}%
\pgfpathclose%
\pgfusepath{stroke,fill}%
\end{pgfscope}%
\begin{pgfscope}%
\pgfpathrectangle{\pgfqpoint{0.793996in}{0.443060in}}{\pgfqpoint{2.041610in}{2.577607in}}%
\pgfusepath{clip}%
\pgfsetbuttcap%
\pgfsetroundjoin%
\definecolor{currentfill}{rgb}{0.755157,0.791493,0.851622}%
\pgfsetfillcolor{currentfill}%
\pgfsetlinewidth{0.602250pt}%
\definecolor{currentstroke}{rgb}{0.296471,0.296471,0.296471}%
\pgfsetstrokecolor{currentstroke}%
\pgfsetdash{}{0pt}%
\pgfpathmoveto{\pgfqpoint{0.793996in}{1.807810in}}%
\pgfpathlineto{\pgfqpoint{0.793996in}{1.807810in}}%
\pgfpathlineto{\pgfqpoint{0.793996in}{1.803208in}}%
\pgfpathlineto{\pgfqpoint{0.793996in}{1.803208in}}%
\pgfpathlineto{\pgfqpoint{0.793996in}{1.807810in}}%
\pgfpathclose%
\pgfusepath{stroke,fill}%
\end{pgfscope}%
\begin{pgfscope}%
\pgfpathrectangle{\pgfqpoint{0.793996in}{0.443060in}}{\pgfqpoint{2.041610in}{2.577607in}}%
\pgfusepath{clip}%
\pgfsetbuttcap%
\pgfsetroundjoin%
\definecolor{currentfill}{rgb}{0.792469,0.821908,0.870863}%
\pgfsetfillcolor{currentfill}%
\pgfsetlinewidth{0.602250pt}%
\definecolor{currentstroke}{rgb}{0.296471,0.296471,0.296471}%
\pgfsetstrokecolor{currentstroke}%
\pgfsetdash{}{0pt}%
\pgfpathmoveto{\pgfqpoint{0.793996in}{1.806660in}}%
\pgfpathlineto{\pgfqpoint{0.793996in}{1.806660in}}%
\pgfpathlineto{\pgfqpoint{0.793996in}{1.804358in}}%
\pgfpathlineto{\pgfqpoint{0.793996in}{1.804358in}}%
\pgfpathlineto{\pgfqpoint{0.793996in}{1.806660in}}%
\pgfpathclose%
\pgfusepath{stroke,fill}%
\end{pgfscope}%
\begin{pgfscope}%
\pgfpathrectangle{\pgfqpoint{0.793996in}{0.443060in}}{\pgfqpoint{2.041610in}{2.577607in}}%
\pgfusepath{clip}%
\pgfsetbuttcap%
\pgfsetroundjoin%
\definecolor{currentfill}{rgb}{0.825117,0.848522,0.887698}%
\pgfsetfillcolor{currentfill}%
\pgfsetlinewidth{0.602250pt}%
\definecolor{currentstroke}{rgb}{0.296471,0.296471,0.296471}%
\pgfsetstrokecolor{currentstroke}%
\pgfsetdash{}{0pt}%
\pgfpathmoveto{\pgfqpoint{0.793996in}{1.806084in}}%
\pgfpathlineto{\pgfqpoint{0.793996in}{1.806084in}}%
\pgfpathlineto{\pgfqpoint{0.793996in}{1.804934in}}%
\pgfpathlineto{\pgfqpoint{0.793996in}{1.804934in}}%
\pgfpathlineto{\pgfqpoint{0.793996in}{1.806084in}}%
\pgfpathclose%
\pgfusepath{stroke,fill}%
\end{pgfscope}%
\begin{pgfscope}%
\pgfpathrectangle{\pgfqpoint{0.793996in}{0.443060in}}{\pgfqpoint{2.041610in}{2.577607in}}%
\pgfusepath{clip}%
\pgfsetbuttcap%
\pgfsetroundjoin%
\definecolor{currentfill}{rgb}{0.848437,0.867532,0.899724}%
\pgfsetfillcolor{currentfill}%
\pgfsetlinewidth{0.602250pt}%
\definecolor{currentstroke}{rgb}{0.296471,0.296471,0.296471}%
\pgfsetstrokecolor{currentstroke}%
\pgfsetdash{}{0pt}%
\pgfpathmoveto{\pgfqpoint{0.793996in}{1.805797in}}%
\pgfpathlineto{\pgfqpoint{0.793996in}{1.805797in}}%
\pgfpathlineto{\pgfqpoint{0.793996in}{1.805221in}}%
\pgfpathlineto{\pgfqpoint{0.793996in}{1.805221in}}%
\pgfpathlineto{\pgfqpoint{0.793996in}{1.805797in}}%
\pgfpathclose%
\pgfusepath{stroke,fill}%
\end{pgfscope}%
\begin{pgfscope}%
\pgfpathrectangle{\pgfqpoint{0.793996in}{0.443060in}}{\pgfqpoint{2.041610in}{2.577607in}}%
\pgfusepath{clip}%
\pgfsetbuttcap%
\pgfsetroundjoin%
\pgfsetlinewidth{0.803000pt}%
\definecolor{currentstroke}{rgb}{0.450000,0.450000,0.450000}%
\pgfsetstrokecolor{currentstroke}%
\pgfsetdash{}{0pt}%
\pgfpathmoveto{\pgfqpoint{0.000000in}{-0.034722in}}%
\pgfpathcurveto{\pgfqpoint{0.009208in}{-0.034722in}}{\pgfqpoint{0.018041in}{-0.031064in}}{\pgfqpoint{0.024552in}{-0.024552in}}%
\pgfpathcurveto{\pgfqpoint{0.031064in}{-0.018041in}}{\pgfqpoint{0.034722in}{-0.009208in}}{\pgfqpoint{0.034722in}{0.000000in}}%
\pgfpathcurveto{\pgfqpoint{0.034722in}{0.009208in}}{\pgfqpoint{0.031064in}{0.018041in}}{\pgfqpoint{0.024552in}{0.024552in}}%
\pgfpathcurveto{\pgfqpoint{0.018041in}{0.031064in}}{\pgfqpoint{0.009208in}{0.034722in}}{\pgfqpoint{0.000000in}{0.034722in}}%
\pgfpathcurveto{\pgfqpoint{-0.009208in}{0.034722in}}{\pgfqpoint{-0.018041in}{0.031064in}}{\pgfqpoint{-0.024552in}{0.024552in}}%
\pgfpathcurveto{\pgfqpoint{-0.031064in}{0.018041in}}{\pgfqpoint{-0.034722in}{0.009208in}}{\pgfqpoint{-0.034722in}{0.000000in}}%
\pgfpathcurveto{\pgfqpoint{-0.034722in}{-0.009208in}}{\pgfqpoint{-0.031064in}{-0.018041in}}{\pgfqpoint{-0.024552in}{-0.024552in}}%
\pgfpathcurveto{\pgfqpoint{-0.018041in}{-0.031064in}}{\pgfqpoint{-0.009208in}{-0.034722in}}{\pgfqpoint{0.000000in}{-0.034722in}}%
\pgfusepath{stroke}%
\end{pgfscope}%
\begin{pgfscope}%
\pgfpathrectangle{\pgfqpoint{0.793996in}{0.443060in}}{\pgfqpoint{2.041610in}{2.577607in}}%
\pgfusepath{clip}%
\pgfsetbuttcap%
\pgfsetroundjoin%
\definecolor{currentfill}{rgb}{0.919097,0.862812,0.832112}%
\pgfsetfillcolor{currentfill}%
\pgfsetlinewidth{0.602250pt}%
\definecolor{currentstroke}{rgb}{0.296471,0.296471,0.296471}%
\pgfsetstrokecolor{currentstroke}%
\pgfsetdash{}{0pt}%
\pgfpathmoveto{\pgfqpoint{1.022173in}{1.658793in}}%
\pgfpathlineto{\pgfqpoint{1.084883in}{1.658793in}}%
\pgfpathlineto{\pgfqpoint{1.084883in}{1.657642in}}%
\pgfpathlineto{\pgfqpoint{1.022173in}{1.657642in}}%
\pgfpathlineto{\pgfqpoint{1.022173in}{1.658793in}}%
\pgfpathclose%
\pgfusepath{stroke,fill}%
\end{pgfscope}%
\begin{pgfscope}%
\pgfpathrectangle{\pgfqpoint{0.793996in}{0.443060in}}{\pgfqpoint{2.041610in}{2.577607in}}%
\pgfusepath{clip}%
\pgfsetbuttcap%
\pgfsetroundjoin%
\definecolor{currentfill}{rgb}{0.910863,0.840546,0.801899}%
\pgfsetfillcolor{currentfill}%
\pgfsetlinewidth{0.602250pt}%
\definecolor{currentstroke}{rgb}{0.296471,0.296471,0.296471}%
\pgfsetstrokecolor{currentstroke}%
\pgfsetdash{}{0pt}%
\pgfpathmoveto{\pgfqpoint{1.084883in}{1.659368in}}%
\pgfpathlineto{\pgfqpoint{1.124605in}{1.659368in}}%
\pgfpathlineto{\pgfqpoint{1.124605in}{1.657066in}}%
\pgfpathlineto{\pgfqpoint{1.084883in}{1.657066in}}%
\pgfpathlineto{\pgfqpoint{1.084883in}{1.659368in}}%
\pgfpathclose%
\pgfusepath{stroke,fill}%
\end{pgfscope}%
\begin{pgfscope}%
\pgfpathrectangle{\pgfqpoint{0.793996in}{0.443060in}}{\pgfqpoint{2.041610in}{2.577607in}}%
\pgfusepath{clip}%
\pgfsetbuttcap%
\pgfsetroundjoin%
\definecolor{currentfill}{rgb}{0.901453,0.815098,0.767370}%
\pgfsetfillcolor{currentfill}%
\pgfsetlinewidth{0.602250pt}%
\definecolor{currentstroke}{rgb}{0.296471,0.296471,0.296471}%
\pgfsetstrokecolor{currentstroke}%
\pgfsetdash{}{0pt}%
\pgfpathmoveto{\pgfqpoint{1.124605in}{1.660519in}}%
\pgfpathlineto{\pgfqpoint{1.203835in}{1.660519in}}%
\pgfpathlineto{\pgfqpoint{1.203835in}{1.655916in}}%
\pgfpathlineto{\pgfqpoint{1.124605in}{1.655916in}}%
\pgfpathlineto{\pgfqpoint{1.124605in}{1.660519in}}%
\pgfpathclose%
\pgfusepath{stroke,fill}%
\end{pgfscope}%
\begin{pgfscope}%
\pgfpathrectangle{\pgfqpoint{0.793996in}{0.443060in}}{\pgfqpoint{2.041610in}{2.577607in}}%
\pgfusepath{clip}%
\pgfsetbuttcap%
\pgfsetroundjoin%
\definecolor{currentfill}{rgb}{0.889102,0.781698,0.722050}%
\pgfsetfillcolor{currentfill}%
\pgfsetlinewidth{0.602250pt}%
\definecolor{currentstroke}{rgb}{0.296471,0.296471,0.296471}%
\pgfsetstrokecolor{currentstroke}%
\pgfsetdash{}{0pt}%
\pgfpathmoveto{\pgfqpoint{1.203835in}{1.662820in}}%
\pgfpathlineto{\pgfqpoint{1.346808in}{1.662820in}}%
\pgfpathlineto{\pgfqpoint{1.346808in}{1.653614in}}%
\pgfpathlineto{\pgfqpoint{1.203835in}{1.653614in}}%
\pgfpathlineto{\pgfqpoint{1.203835in}{1.662820in}}%
\pgfpathclose%
\pgfusepath{stroke,fill}%
\end{pgfscope}%
\begin{pgfscope}%
\pgfpathrectangle{\pgfqpoint{0.793996in}{0.443060in}}{\pgfqpoint{2.041610in}{2.577607in}}%
\pgfusepath{clip}%
\pgfsetbuttcap%
\pgfsetroundjoin%
\definecolor{currentfill}{rgb}{0.873223,0.738755,0.663782}%
\pgfsetfillcolor{currentfill}%
\pgfsetlinewidth{0.602250pt}%
\definecolor{currentstroke}{rgb}{0.296471,0.296471,0.296471}%
\pgfsetstrokecolor{currentstroke}%
\pgfsetdash{}{0pt}%
\pgfpathmoveto{\pgfqpoint{1.346808in}{1.667423in}}%
\pgfpathlineto{\pgfqpoint{1.601904in}{1.667423in}}%
\pgfpathlineto{\pgfqpoint{1.601904in}{1.649011in}}%
\pgfpathlineto{\pgfqpoint{1.346808in}{1.649011in}}%
\pgfpathlineto{\pgfqpoint{1.346808in}{1.667423in}}%
\pgfpathclose%
\pgfusepath{stroke,fill}%
\end{pgfscope}%
\begin{pgfscope}%
\pgfpathrectangle{\pgfqpoint{0.793996in}{0.443060in}}{\pgfqpoint{2.041610in}{2.577607in}}%
\pgfusepath{clip}%
\pgfsetbuttcap%
\pgfsetroundjoin%
\definecolor{currentfill}{rgb}{0.853814,0.686269,0.592565}%
\pgfsetfillcolor{currentfill}%
\pgfsetlinewidth{0.602250pt}%
\definecolor{currentstroke}{rgb}{0.296471,0.296471,0.296471}%
\pgfsetstrokecolor{currentstroke}%
\pgfsetdash{}{0pt}%
\pgfpathmoveto{\pgfqpoint{1.601904in}{1.676629in}}%
\pgfpathlineto{\pgfqpoint{1.946517in}{1.676629in}}%
\pgfpathlineto{\pgfqpoint{1.946517in}{1.639806in}}%
\pgfpathlineto{\pgfqpoint{1.601904in}{1.639806in}}%
\pgfpathlineto{\pgfqpoint{1.601904in}{1.676629in}}%
\pgfpathclose%
\pgfusepath{stroke,fill}%
\end{pgfscope}%
\begin{pgfscope}%
\pgfpathrectangle{\pgfqpoint{0.793996in}{0.443060in}}{\pgfqpoint{2.041610in}{2.577607in}}%
\pgfusepath{clip}%
\pgfsetbuttcap%
\pgfsetroundjoin%
\definecolor{currentfill}{rgb}{0.829112,0.619469,0.501926}%
\pgfsetfillcolor{currentfill}%
\pgfsetlinewidth{0.602250pt}%
\definecolor{currentstroke}{rgb}{0.296471,0.296471,0.296471}%
\pgfsetstrokecolor{currentstroke}%
\pgfsetdash{}{0pt}%
\pgfpathmoveto{\pgfqpoint{1.946517in}{1.695040in}}%
\pgfpathlineto{\pgfqpoint{2.322910in}{1.695040in}}%
\pgfpathlineto{\pgfqpoint{2.322910in}{1.621394in}}%
\pgfpathlineto{\pgfqpoint{1.946517in}{1.621394in}}%
\pgfpathlineto{\pgfqpoint{1.946517in}{1.695040in}}%
\pgfpathclose%
\pgfusepath{stroke,fill}%
\end{pgfscope}%
\begin{pgfscope}%
\pgfpathrectangle{\pgfqpoint{0.793996in}{0.443060in}}{\pgfqpoint{2.041610in}{2.577607in}}%
\pgfusepath{clip}%
\pgfsetbuttcap%
\pgfsetroundjoin%
\definecolor{currentfill}{rgb}{0.798529,0.536765,0.389706}%
\pgfsetfillcolor{currentfill}%
\pgfsetlinewidth{0.602250pt}%
\definecolor{currentstroke}{rgb}{0.296471,0.296471,0.296471}%
\pgfsetstrokecolor{currentstroke}%
\pgfsetdash{}{0pt}%
\pgfpathmoveto{\pgfqpoint{2.322910in}{1.731863in}}%
\pgfpathlineto{\pgfqpoint{2.790237in}{1.731863in}}%
\pgfpathlineto{\pgfqpoint{2.790237in}{1.584571in}}%
\pgfpathlineto{\pgfqpoint{2.322910in}{1.584571in}}%
\pgfpathlineto{\pgfqpoint{2.322910in}{1.731863in}}%
\pgfpathclose%
\pgfusepath{stroke,fill}%
\end{pgfscope}%
\begin{pgfscope}%
\pgfpathrectangle{\pgfqpoint{0.793996in}{0.443060in}}{\pgfqpoint{2.041610in}{2.577607in}}%
\pgfusepath{clip}%
\pgfsetbuttcap%
\pgfsetroundjoin%
\definecolor{currentfill}{rgb}{0.829112,0.619469,0.501926}%
\pgfsetfillcolor{currentfill}%
\pgfsetlinewidth{0.602250pt}%
\definecolor{currentstroke}{rgb}{0.296471,0.296471,0.296471}%
\pgfsetstrokecolor{currentstroke}%
\pgfsetdash{}{0pt}%
\pgfpathmoveto{\pgfqpoint{2.790237in}{1.695040in}}%
\pgfpathlineto{\pgfqpoint{2.835606in}{1.695040in}}%
\pgfpathlineto{\pgfqpoint{2.835606in}{1.621394in}}%
\pgfpathlineto{\pgfqpoint{2.790237in}{1.621394in}}%
\pgfpathlineto{\pgfqpoint{2.790237in}{1.695040in}}%
\pgfpathclose%
\pgfusepath{stroke,fill}%
\end{pgfscope}%
\begin{pgfscope}%
\pgfpathrectangle{\pgfqpoint{0.793996in}{0.443060in}}{\pgfqpoint{2.041610in}{2.577607in}}%
\pgfusepath{clip}%
\pgfsetbuttcap%
\pgfsetroundjoin%
\definecolor{currentfill}{rgb}{0.853814,0.686269,0.592565}%
\pgfsetfillcolor{currentfill}%
\pgfsetlinewidth{0.602250pt}%
\definecolor{currentstroke}{rgb}{0.296471,0.296471,0.296471}%
\pgfsetstrokecolor{currentstroke}%
\pgfsetdash{}{0pt}%
\pgfpathmoveto{\pgfqpoint{2.835606in}{1.676629in}}%
\pgfpathlineto{\pgfqpoint{2.835606in}{1.676629in}}%
\pgfpathlineto{\pgfqpoint{2.835606in}{1.639806in}}%
\pgfpathlineto{\pgfqpoint{2.835606in}{1.639806in}}%
\pgfpathlineto{\pgfqpoint{2.835606in}{1.676629in}}%
\pgfpathclose%
\pgfusepath{stroke,fill}%
\end{pgfscope}%
\begin{pgfscope}%
\pgfpathrectangle{\pgfqpoint{0.793996in}{0.443060in}}{\pgfqpoint{2.041610in}{2.577607in}}%
\pgfusepath{clip}%
\pgfsetbuttcap%
\pgfsetroundjoin%
\definecolor{currentfill}{rgb}{0.873223,0.738755,0.663782}%
\pgfsetfillcolor{currentfill}%
\pgfsetlinewidth{0.602250pt}%
\definecolor{currentstroke}{rgb}{0.296471,0.296471,0.296471}%
\pgfsetstrokecolor{currentstroke}%
\pgfsetdash{}{0pt}%
\pgfpathmoveto{\pgfqpoint{2.835606in}{1.667423in}}%
\pgfpathlineto{\pgfqpoint{2.835606in}{1.667423in}}%
\pgfpathlineto{\pgfqpoint{2.835606in}{1.649011in}}%
\pgfpathlineto{\pgfqpoint{2.835606in}{1.649011in}}%
\pgfpathlineto{\pgfqpoint{2.835606in}{1.667423in}}%
\pgfpathclose%
\pgfusepath{stroke,fill}%
\end{pgfscope}%
\begin{pgfscope}%
\pgfpathrectangle{\pgfqpoint{0.793996in}{0.443060in}}{\pgfqpoint{2.041610in}{2.577607in}}%
\pgfusepath{clip}%
\pgfsetbuttcap%
\pgfsetroundjoin%
\definecolor{currentfill}{rgb}{0.889102,0.781698,0.722050}%
\pgfsetfillcolor{currentfill}%
\pgfsetlinewidth{0.602250pt}%
\definecolor{currentstroke}{rgb}{0.296471,0.296471,0.296471}%
\pgfsetstrokecolor{currentstroke}%
\pgfsetdash{}{0pt}%
\pgfpathmoveto{\pgfqpoint{2.835606in}{1.662820in}}%
\pgfpathlineto{\pgfqpoint{2.835606in}{1.662820in}}%
\pgfpathlineto{\pgfqpoint{2.835606in}{1.653614in}}%
\pgfpathlineto{\pgfqpoint{2.835606in}{1.653614in}}%
\pgfpathlineto{\pgfqpoint{2.835606in}{1.662820in}}%
\pgfpathclose%
\pgfusepath{stroke,fill}%
\end{pgfscope}%
\begin{pgfscope}%
\pgfpathrectangle{\pgfqpoint{0.793996in}{0.443060in}}{\pgfqpoint{2.041610in}{2.577607in}}%
\pgfusepath{clip}%
\pgfsetbuttcap%
\pgfsetroundjoin%
\definecolor{currentfill}{rgb}{0.901453,0.815098,0.767370}%
\pgfsetfillcolor{currentfill}%
\pgfsetlinewidth{0.602250pt}%
\definecolor{currentstroke}{rgb}{0.296471,0.296471,0.296471}%
\pgfsetstrokecolor{currentstroke}%
\pgfsetdash{}{0pt}%
\pgfpathmoveto{\pgfqpoint{2.835606in}{1.660519in}}%
\pgfpathlineto{\pgfqpoint{2.835606in}{1.660519in}}%
\pgfpathlineto{\pgfqpoint{2.835606in}{1.655916in}}%
\pgfpathlineto{\pgfqpoint{2.835606in}{1.655916in}}%
\pgfpathlineto{\pgfqpoint{2.835606in}{1.660519in}}%
\pgfpathclose%
\pgfusepath{stroke,fill}%
\end{pgfscope}%
\begin{pgfscope}%
\pgfpathrectangle{\pgfqpoint{0.793996in}{0.443060in}}{\pgfqpoint{2.041610in}{2.577607in}}%
\pgfusepath{clip}%
\pgfsetbuttcap%
\pgfsetroundjoin%
\definecolor{currentfill}{rgb}{0.910863,0.840546,0.801899}%
\pgfsetfillcolor{currentfill}%
\pgfsetlinewidth{0.602250pt}%
\definecolor{currentstroke}{rgb}{0.296471,0.296471,0.296471}%
\pgfsetstrokecolor{currentstroke}%
\pgfsetdash{}{0pt}%
\pgfpathmoveto{\pgfqpoint{2.835606in}{1.659368in}}%
\pgfpathlineto{\pgfqpoint{2.835606in}{1.659368in}}%
\pgfpathlineto{\pgfqpoint{2.835606in}{1.657066in}}%
\pgfpathlineto{\pgfqpoint{2.835606in}{1.657066in}}%
\pgfpathlineto{\pgfqpoint{2.835606in}{1.659368in}}%
\pgfpathclose%
\pgfusepath{stroke,fill}%
\end{pgfscope}%
\begin{pgfscope}%
\pgfpathrectangle{\pgfqpoint{0.793996in}{0.443060in}}{\pgfqpoint{2.041610in}{2.577607in}}%
\pgfusepath{clip}%
\pgfsetbuttcap%
\pgfsetroundjoin%
\definecolor{currentfill}{rgb}{0.919097,0.862812,0.832112}%
\pgfsetfillcolor{currentfill}%
\pgfsetlinewidth{0.602250pt}%
\definecolor{currentstroke}{rgb}{0.296471,0.296471,0.296471}%
\pgfsetstrokecolor{currentstroke}%
\pgfsetdash{}{0pt}%
\pgfpathmoveto{\pgfqpoint{2.835606in}{1.658793in}}%
\pgfpathlineto{\pgfqpoint{2.835606in}{1.658793in}}%
\pgfpathlineto{\pgfqpoint{2.835606in}{1.657642in}}%
\pgfpathlineto{\pgfqpoint{2.835606in}{1.657642in}}%
\pgfpathlineto{\pgfqpoint{2.835606in}{1.658793in}}%
\pgfpathclose%
\pgfusepath{stroke,fill}%
\end{pgfscope}%
\begin{pgfscope}%
\pgfpathrectangle{\pgfqpoint{0.793996in}{0.443060in}}{\pgfqpoint{2.041610in}{2.577607in}}%
\pgfusepath{clip}%
\pgfsetbuttcap%
\pgfsetroundjoin%
\pgfsetlinewidth{0.803000pt}%
\definecolor{currentstroke}{rgb}{0.450000,0.450000,0.450000}%
\pgfsetstrokecolor{currentstroke}%
\pgfsetdash{}{0pt}%
\pgfpathmoveto{\pgfqpoint{0.000000in}{-0.034722in}}%
\pgfpathcurveto{\pgfqpoint{0.009208in}{-0.034722in}}{\pgfqpoint{0.018041in}{-0.031064in}}{\pgfqpoint{0.024552in}{-0.024552in}}%
\pgfpathcurveto{\pgfqpoint{0.031064in}{-0.018041in}}{\pgfqpoint{0.034722in}{-0.009208in}}{\pgfqpoint{0.034722in}{0.000000in}}%
\pgfpathcurveto{\pgfqpoint{0.034722in}{0.009208in}}{\pgfqpoint{0.031064in}{0.018041in}}{\pgfqpoint{0.024552in}{0.024552in}}%
\pgfpathcurveto{\pgfqpoint{0.018041in}{0.031064in}}{\pgfqpoint{0.009208in}{0.034722in}}{\pgfqpoint{0.000000in}{0.034722in}}%
\pgfpathcurveto{\pgfqpoint{-0.009208in}{0.034722in}}{\pgfqpoint{-0.018041in}{0.031064in}}{\pgfqpoint{-0.024552in}{0.024552in}}%
\pgfpathcurveto{\pgfqpoint{-0.031064in}{0.018041in}}{\pgfqpoint{-0.034722in}{0.009208in}}{\pgfqpoint{-0.034722in}{0.000000in}}%
\pgfpathcurveto{\pgfqpoint{-0.034722in}{-0.009208in}}{\pgfqpoint{-0.031064in}{-0.018041in}}{\pgfqpoint{-0.024552in}{-0.024552in}}%
\pgfpathcurveto{\pgfqpoint{-0.018041in}{-0.031064in}}{\pgfqpoint{-0.009208in}{-0.034722in}}{\pgfqpoint{0.000000in}{-0.034722in}}%
\pgfusepath{stroke}%
\end{pgfscope}%
\begin{pgfscope}%
\pgfpathrectangle{\pgfqpoint{0.793996in}{0.443060in}}{\pgfqpoint{2.041610in}{2.577607in}}%
\pgfusepath{clip}%
\pgfsetbuttcap%
\pgfsetroundjoin%
\definecolor{currentfill}{rgb}{0.848437,0.867532,0.899724}%
\pgfsetfillcolor{currentfill}%
\pgfsetlinewidth{0.602250pt}%
\definecolor{currentstroke}{rgb}{0.296471,0.296471,0.296471}%
\pgfsetstrokecolor{currentstroke}%
\pgfsetdash{}{0pt}%
\pgfpathmoveto{\pgfqpoint{0.793996in}{1.437567in}}%
\pgfpathlineto{\pgfqpoint{0.793996in}{1.437567in}}%
\pgfpathlineto{\pgfqpoint{0.793996in}{1.436992in}}%
\pgfpathlineto{\pgfqpoint{0.793996in}{1.436992in}}%
\pgfpathlineto{\pgfqpoint{0.793996in}{1.437567in}}%
\pgfpathclose%
\pgfusepath{stroke,fill}%
\end{pgfscope}%
\begin{pgfscope}%
\pgfpathrectangle{\pgfqpoint{0.793996in}{0.443060in}}{\pgfqpoint{2.041610in}{2.577607in}}%
\pgfusepath{clip}%
\pgfsetbuttcap%
\pgfsetroundjoin%
\definecolor{currentfill}{rgb}{0.825117,0.848522,0.887698}%
\pgfsetfillcolor{currentfill}%
\pgfsetlinewidth{0.602250pt}%
\definecolor{currentstroke}{rgb}{0.296471,0.296471,0.296471}%
\pgfsetstrokecolor{currentstroke}%
\pgfsetdash{}{0pt}%
\pgfpathmoveto{\pgfqpoint{0.793996in}{1.437855in}}%
\pgfpathlineto{\pgfqpoint{0.793996in}{1.437855in}}%
\pgfpathlineto{\pgfqpoint{0.793996in}{1.436704in}}%
\pgfpathlineto{\pgfqpoint{0.793996in}{1.436704in}}%
\pgfpathlineto{\pgfqpoint{0.793996in}{1.437855in}}%
\pgfpathclose%
\pgfusepath{stroke,fill}%
\end{pgfscope}%
\begin{pgfscope}%
\pgfpathrectangle{\pgfqpoint{0.793996in}{0.443060in}}{\pgfqpoint{2.041610in}{2.577607in}}%
\pgfusepath{clip}%
\pgfsetbuttcap%
\pgfsetroundjoin%
\definecolor{currentfill}{rgb}{0.792469,0.821908,0.870863}%
\pgfsetfillcolor{currentfill}%
\pgfsetlinewidth{0.602250pt}%
\definecolor{currentstroke}{rgb}{0.296471,0.296471,0.296471}%
\pgfsetstrokecolor{currentstroke}%
\pgfsetdash{}{0pt}%
\pgfpathmoveto{\pgfqpoint{0.793996in}{1.438430in}}%
\pgfpathlineto{\pgfqpoint{0.793996in}{1.438430in}}%
\pgfpathlineto{\pgfqpoint{0.793996in}{1.436129in}}%
\pgfpathlineto{\pgfqpoint{0.793996in}{1.436129in}}%
\pgfpathlineto{\pgfqpoint{0.793996in}{1.438430in}}%
\pgfpathclose%
\pgfusepath{stroke,fill}%
\end{pgfscope}%
\begin{pgfscope}%
\pgfpathrectangle{\pgfqpoint{0.793996in}{0.443060in}}{\pgfqpoint{2.041610in}{2.577607in}}%
\pgfusepath{clip}%
\pgfsetbuttcap%
\pgfsetroundjoin%
\definecolor{currentfill}{rgb}{0.755157,0.791493,0.851622}%
\pgfsetfillcolor{currentfill}%
\pgfsetlinewidth{0.602250pt}%
\definecolor{currentstroke}{rgb}{0.296471,0.296471,0.296471}%
\pgfsetstrokecolor{currentstroke}%
\pgfsetdash{}{0pt}%
\pgfpathmoveto{\pgfqpoint{0.793996in}{1.439581in}}%
\pgfpathlineto{\pgfqpoint{0.793996in}{1.439581in}}%
\pgfpathlineto{\pgfqpoint{0.793996in}{1.434978in}}%
\pgfpathlineto{\pgfqpoint{0.793996in}{1.434978in}}%
\pgfpathlineto{\pgfqpoint{0.793996in}{1.439581in}}%
\pgfpathclose%
\pgfusepath{stroke,fill}%
\end{pgfscope}%
\begin{pgfscope}%
\pgfpathrectangle{\pgfqpoint{0.793996in}{0.443060in}}{\pgfqpoint{2.041610in}{2.577607in}}%
\pgfusepath{clip}%
\pgfsetbuttcap%
\pgfsetroundjoin%
\definecolor{currentfill}{rgb}{0.706185,0.751573,0.826368}%
\pgfsetfillcolor{currentfill}%
\pgfsetlinewidth{0.602250pt}%
\definecolor{currentstroke}{rgb}{0.296471,0.296471,0.296471}%
\pgfsetstrokecolor{currentstroke}%
\pgfsetdash{}{0pt}%
\pgfpathmoveto{\pgfqpoint{0.793996in}{1.441882in}}%
\pgfpathlineto{\pgfqpoint{0.793996in}{1.441882in}}%
\pgfpathlineto{\pgfqpoint{0.793996in}{1.432677in}}%
\pgfpathlineto{\pgfqpoint{0.793996in}{1.432677in}}%
\pgfpathlineto{\pgfqpoint{0.793996in}{1.441882in}}%
\pgfpathclose%
\pgfusepath{stroke,fill}%
\end{pgfscope}%
\begin{pgfscope}%
\pgfpathrectangle{\pgfqpoint{0.793996in}{0.443060in}}{\pgfqpoint{2.041610in}{2.577607in}}%
\pgfusepath{clip}%
\pgfsetbuttcap%
\pgfsetroundjoin%
\definecolor{currentfill}{rgb}{0.643221,0.700246,0.793900}%
\pgfsetfillcolor{currentfill}%
\pgfsetlinewidth{0.602250pt}%
\definecolor{currentstroke}{rgb}{0.296471,0.296471,0.296471}%
\pgfsetstrokecolor{currentstroke}%
\pgfsetdash{}{0pt}%
\pgfpathmoveto{\pgfqpoint{0.793996in}{1.446485in}}%
\pgfpathlineto{\pgfqpoint{0.793996in}{1.446485in}}%
\pgfpathlineto{\pgfqpoint{0.793996in}{1.428074in}}%
\pgfpathlineto{\pgfqpoint{0.793996in}{1.428074in}}%
\pgfpathlineto{\pgfqpoint{0.793996in}{1.446485in}}%
\pgfpathclose%
\pgfusepath{stroke,fill}%
\end{pgfscope}%
\begin{pgfscope}%
\pgfpathrectangle{\pgfqpoint{0.793996in}{0.443060in}}{\pgfqpoint{2.041610in}{2.577607in}}%
\pgfusepath{clip}%
\pgfsetbuttcap%
\pgfsetroundjoin%
\definecolor{currentfill}{rgb}{0.566266,0.637515,0.754216}%
\pgfsetfillcolor{currentfill}%
\pgfsetlinewidth{0.602250pt}%
\definecolor{currentstroke}{rgb}{0.296471,0.296471,0.296471}%
\pgfsetstrokecolor{currentstroke}%
\pgfsetdash{}{0pt}%
\pgfpathmoveto{\pgfqpoint{0.793996in}{1.455691in}}%
\pgfpathlineto{\pgfqpoint{0.793996in}{1.455691in}}%
\pgfpathlineto{\pgfqpoint{0.793996in}{1.418868in}}%
\pgfpathlineto{\pgfqpoint{0.793996in}{1.418868in}}%
\pgfpathlineto{\pgfqpoint{0.793996in}{1.455691in}}%
\pgfpathclose%
\pgfusepath{stroke,fill}%
\end{pgfscope}%
\begin{pgfscope}%
\pgfpathrectangle{\pgfqpoint{0.793996in}{0.443060in}}{\pgfqpoint{2.041610in}{2.577607in}}%
\pgfusepath{clip}%
\pgfsetbuttcap%
\pgfsetroundjoin%
\definecolor{currentfill}{rgb}{0.468322,0.557674,0.703709}%
\pgfsetfillcolor{currentfill}%
\pgfsetlinewidth{0.602250pt}%
\definecolor{currentstroke}{rgb}{0.296471,0.296471,0.296471}%
\pgfsetstrokecolor{currentstroke}%
\pgfsetdash{}{0pt}%
\pgfpathmoveto{\pgfqpoint{0.793996in}{1.474102in}}%
\pgfpathlineto{\pgfqpoint{0.793996in}{1.474102in}}%
\pgfpathlineto{\pgfqpoint{0.793996in}{1.400457in}}%
\pgfpathlineto{\pgfqpoint{0.793996in}{1.400457in}}%
\pgfpathlineto{\pgfqpoint{0.793996in}{1.474102in}}%
\pgfpathclose%
\pgfusepath{stroke,fill}%
\end{pgfscope}%
\begin{pgfscope}%
\pgfpathrectangle{\pgfqpoint{0.793996in}{0.443060in}}{\pgfqpoint{2.041610in}{2.577607in}}%
\pgfusepath{clip}%
\pgfsetbuttcap%
\pgfsetroundjoin%
\definecolor{currentfill}{rgb}{0.347059,0.458824,0.641176}%
\pgfsetfillcolor{currentfill}%
\pgfsetlinewidth{0.602250pt}%
\definecolor{currentstroke}{rgb}{0.296471,0.296471,0.296471}%
\pgfsetstrokecolor{currentstroke}%
\pgfsetdash{}{0pt}%
\pgfpathmoveto{\pgfqpoint{0.793996in}{1.510925in}}%
\pgfpathlineto{\pgfqpoint{0.793996in}{1.510925in}}%
\pgfpathlineto{\pgfqpoint{0.793996in}{1.363634in}}%
\pgfpathlineto{\pgfqpoint{0.793996in}{1.363634in}}%
\pgfpathlineto{\pgfqpoint{0.793996in}{1.510925in}}%
\pgfpathclose%
\pgfusepath{stroke,fill}%
\end{pgfscope}%
\begin{pgfscope}%
\pgfpathrectangle{\pgfqpoint{0.793996in}{0.443060in}}{\pgfqpoint{2.041610in}{2.577607in}}%
\pgfusepath{clip}%
\pgfsetbuttcap%
\pgfsetroundjoin%
\definecolor{currentfill}{rgb}{0.468322,0.557674,0.703709}%
\pgfsetfillcolor{currentfill}%
\pgfsetlinewidth{0.602250pt}%
\definecolor{currentstroke}{rgb}{0.296471,0.296471,0.296471}%
\pgfsetstrokecolor{currentstroke}%
\pgfsetdash{}{0pt}%
\pgfpathmoveto{\pgfqpoint{0.793996in}{1.474102in}}%
\pgfpathlineto{\pgfqpoint{0.793996in}{1.474102in}}%
\pgfpathlineto{\pgfqpoint{0.793996in}{1.400457in}}%
\pgfpathlineto{\pgfqpoint{0.793996in}{1.400457in}}%
\pgfpathlineto{\pgfqpoint{0.793996in}{1.474102in}}%
\pgfpathclose%
\pgfusepath{stroke,fill}%
\end{pgfscope}%
\begin{pgfscope}%
\pgfpathrectangle{\pgfqpoint{0.793996in}{0.443060in}}{\pgfqpoint{2.041610in}{2.577607in}}%
\pgfusepath{clip}%
\pgfsetbuttcap%
\pgfsetroundjoin%
\definecolor{currentfill}{rgb}{0.566266,0.637515,0.754216}%
\pgfsetfillcolor{currentfill}%
\pgfsetlinewidth{0.602250pt}%
\definecolor{currentstroke}{rgb}{0.296471,0.296471,0.296471}%
\pgfsetstrokecolor{currentstroke}%
\pgfsetdash{}{0pt}%
\pgfpathmoveto{\pgfqpoint{0.793996in}{1.455691in}}%
\pgfpathlineto{\pgfqpoint{0.793996in}{1.455691in}}%
\pgfpathlineto{\pgfqpoint{0.793996in}{1.418868in}}%
\pgfpathlineto{\pgfqpoint{0.793996in}{1.418868in}}%
\pgfpathlineto{\pgfqpoint{0.793996in}{1.455691in}}%
\pgfpathclose%
\pgfusepath{stroke,fill}%
\end{pgfscope}%
\begin{pgfscope}%
\pgfpathrectangle{\pgfqpoint{0.793996in}{0.443060in}}{\pgfqpoint{2.041610in}{2.577607in}}%
\pgfusepath{clip}%
\pgfsetbuttcap%
\pgfsetroundjoin%
\definecolor{currentfill}{rgb}{0.643221,0.700246,0.793900}%
\pgfsetfillcolor{currentfill}%
\pgfsetlinewidth{0.602250pt}%
\definecolor{currentstroke}{rgb}{0.296471,0.296471,0.296471}%
\pgfsetstrokecolor{currentstroke}%
\pgfsetdash{}{0pt}%
\pgfpathmoveto{\pgfqpoint{0.793996in}{1.446485in}}%
\pgfpathlineto{\pgfqpoint{0.793996in}{1.446485in}}%
\pgfpathlineto{\pgfqpoint{0.793996in}{1.428074in}}%
\pgfpathlineto{\pgfqpoint{0.793996in}{1.428074in}}%
\pgfpathlineto{\pgfqpoint{0.793996in}{1.446485in}}%
\pgfpathclose%
\pgfusepath{stroke,fill}%
\end{pgfscope}%
\begin{pgfscope}%
\pgfpathrectangle{\pgfqpoint{0.793996in}{0.443060in}}{\pgfqpoint{2.041610in}{2.577607in}}%
\pgfusepath{clip}%
\pgfsetbuttcap%
\pgfsetroundjoin%
\definecolor{currentfill}{rgb}{0.706185,0.751573,0.826368}%
\pgfsetfillcolor{currentfill}%
\pgfsetlinewidth{0.602250pt}%
\definecolor{currentstroke}{rgb}{0.296471,0.296471,0.296471}%
\pgfsetstrokecolor{currentstroke}%
\pgfsetdash{}{0pt}%
\pgfpathmoveto{\pgfqpoint{0.793996in}{1.441882in}}%
\pgfpathlineto{\pgfqpoint{0.793996in}{1.441882in}}%
\pgfpathlineto{\pgfqpoint{0.793996in}{1.432677in}}%
\pgfpathlineto{\pgfqpoint{0.793996in}{1.432677in}}%
\pgfpathlineto{\pgfqpoint{0.793996in}{1.441882in}}%
\pgfpathclose%
\pgfusepath{stroke,fill}%
\end{pgfscope}%
\begin{pgfscope}%
\pgfpathrectangle{\pgfqpoint{0.793996in}{0.443060in}}{\pgfqpoint{2.041610in}{2.577607in}}%
\pgfusepath{clip}%
\pgfsetbuttcap%
\pgfsetroundjoin%
\definecolor{currentfill}{rgb}{0.755157,0.791493,0.851622}%
\pgfsetfillcolor{currentfill}%
\pgfsetlinewidth{0.602250pt}%
\definecolor{currentstroke}{rgb}{0.296471,0.296471,0.296471}%
\pgfsetstrokecolor{currentstroke}%
\pgfsetdash{}{0pt}%
\pgfpathmoveto{\pgfqpoint{0.793996in}{1.439581in}}%
\pgfpathlineto{\pgfqpoint{0.793996in}{1.439581in}}%
\pgfpathlineto{\pgfqpoint{0.793996in}{1.434978in}}%
\pgfpathlineto{\pgfqpoint{0.793996in}{1.434978in}}%
\pgfpathlineto{\pgfqpoint{0.793996in}{1.439581in}}%
\pgfpathclose%
\pgfusepath{stroke,fill}%
\end{pgfscope}%
\begin{pgfscope}%
\pgfpathrectangle{\pgfqpoint{0.793996in}{0.443060in}}{\pgfqpoint{2.041610in}{2.577607in}}%
\pgfusepath{clip}%
\pgfsetbuttcap%
\pgfsetroundjoin%
\definecolor{currentfill}{rgb}{0.792469,0.821908,0.870863}%
\pgfsetfillcolor{currentfill}%
\pgfsetlinewidth{0.602250pt}%
\definecolor{currentstroke}{rgb}{0.296471,0.296471,0.296471}%
\pgfsetstrokecolor{currentstroke}%
\pgfsetdash{}{0pt}%
\pgfpathmoveto{\pgfqpoint{0.793996in}{1.438430in}}%
\pgfpathlineto{\pgfqpoint{0.793996in}{1.438430in}}%
\pgfpathlineto{\pgfqpoint{0.793996in}{1.436129in}}%
\pgfpathlineto{\pgfqpoint{0.793996in}{1.436129in}}%
\pgfpathlineto{\pgfqpoint{0.793996in}{1.438430in}}%
\pgfpathclose%
\pgfusepath{stroke,fill}%
\end{pgfscope}%
\begin{pgfscope}%
\pgfpathrectangle{\pgfqpoint{0.793996in}{0.443060in}}{\pgfqpoint{2.041610in}{2.577607in}}%
\pgfusepath{clip}%
\pgfsetbuttcap%
\pgfsetroundjoin%
\definecolor{currentfill}{rgb}{0.825117,0.848522,0.887698}%
\pgfsetfillcolor{currentfill}%
\pgfsetlinewidth{0.602250pt}%
\definecolor{currentstroke}{rgb}{0.296471,0.296471,0.296471}%
\pgfsetstrokecolor{currentstroke}%
\pgfsetdash{}{0pt}%
\pgfpathmoveto{\pgfqpoint{0.793996in}{1.437855in}}%
\pgfpathlineto{\pgfqpoint{0.793996in}{1.437855in}}%
\pgfpathlineto{\pgfqpoint{0.793996in}{1.436704in}}%
\pgfpathlineto{\pgfqpoint{0.793996in}{1.436704in}}%
\pgfpathlineto{\pgfqpoint{0.793996in}{1.437855in}}%
\pgfpathclose%
\pgfusepath{stroke,fill}%
\end{pgfscope}%
\begin{pgfscope}%
\pgfpathrectangle{\pgfqpoint{0.793996in}{0.443060in}}{\pgfqpoint{2.041610in}{2.577607in}}%
\pgfusepath{clip}%
\pgfsetbuttcap%
\pgfsetroundjoin%
\definecolor{currentfill}{rgb}{0.848437,0.867532,0.899724}%
\pgfsetfillcolor{currentfill}%
\pgfsetlinewidth{0.602250pt}%
\definecolor{currentstroke}{rgb}{0.296471,0.296471,0.296471}%
\pgfsetstrokecolor{currentstroke}%
\pgfsetdash{}{0pt}%
\pgfpathmoveto{\pgfqpoint{0.793996in}{1.437567in}}%
\pgfpathlineto{\pgfqpoint{0.793996in}{1.437567in}}%
\pgfpathlineto{\pgfqpoint{0.793996in}{1.436992in}}%
\pgfpathlineto{\pgfqpoint{0.793996in}{1.436992in}}%
\pgfpathlineto{\pgfqpoint{0.793996in}{1.437567in}}%
\pgfpathclose%
\pgfusepath{stroke,fill}%
\end{pgfscope}%
\begin{pgfscope}%
\pgfpathrectangle{\pgfqpoint{0.793996in}{0.443060in}}{\pgfqpoint{2.041610in}{2.577607in}}%
\pgfusepath{clip}%
\pgfsetbuttcap%
\pgfsetroundjoin%
\pgfsetlinewidth{0.803000pt}%
\definecolor{currentstroke}{rgb}{0.450000,0.450000,0.450000}%
\pgfsetstrokecolor{currentstroke}%
\pgfsetdash{}{0pt}%
\pgfpathmoveto{\pgfqpoint{0.000000in}{-0.034722in}}%
\pgfpathcurveto{\pgfqpoint{0.009208in}{-0.034722in}}{\pgfqpoint{0.018041in}{-0.031064in}}{\pgfqpoint{0.024552in}{-0.024552in}}%
\pgfpathcurveto{\pgfqpoint{0.031064in}{-0.018041in}}{\pgfqpoint{0.034722in}{-0.009208in}}{\pgfqpoint{0.034722in}{0.000000in}}%
\pgfpathcurveto{\pgfqpoint{0.034722in}{0.009208in}}{\pgfqpoint{0.031064in}{0.018041in}}{\pgfqpoint{0.024552in}{0.024552in}}%
\pgfpathcurveto{\pgfqpoint{0.018041in}{0.031064in}}{\pgfqpoint{0.009208in}{0.034722in}}{\pgfqpoint{0.000000in}{0.034722in}}%
\pgfpathcurveto{\pgfqpoint{-0.009208in}{0.034722in}}{\pgfqpoint{-0.018041in}{0.031064in}}{\pgfqpoint{-0.024552in}{0.024552in}}%
\pgfpathcurveto{\pgfqpoint{-0.031064in}{0.018041in}}{\pgfqpoint{-0.034722in}{0.009208in}}{\pgfqpoint{-0.034722in}{0.000000in}}%
\pgfpathcurveto{\pgfqpoint{-0.034722in}{-0.009208in}}{\pgfqpoint{-0.031064in}{-0.018041in}}{\pgfqpoint{-0.024552in}{-0.024552in}}%
\pgfpathcurveto{\pgfqpoint{-0.018041in}{-0.031064in}}{\pgfqpoint{-0.009208in}{-0.034722in}}{\pgfqpoint{0.000000in}{-0.034722in}}%
\pgfusepath{stroke}%
\end{pgfscope}%
\begin{pgfscope}%
\pgfpathrectangle{\pgfqpoint{0.793996in}{0.443060in}}{\pgfqpoint{2.041610in}{2.577607in}}%
\pgfusepath{clip}%
\pgfsetbuttcap%
\pgfsetroundjoin%
\definecolor{currentfill}{rgb}{0.919097,0.862812,0.832112}%
\pgfsetfillcolor{currentfill}%
\pgfsetlinewidth{0.602250pt}%
\definecolor{currentstroke}{rgb}{0.296471,0.296471,0.296471}%
\pgfsetstrokecolor{currentstroke}%
\pgfsetdash{}{0pt}%
\pgfpathmoveto{\pgfqpoint{1.381942in}{1.290563in}}%
\pgfpathlineto{\pgfqpoint{1.545389in}{1.290563in}}%
\pgfpathlineto{\pgfqpoint{1.545389in}{1.289412in}}%
\pgfpathlineto{\pgfqpoint{1.381942in}{1.289412in}}%
\pgfpathlineto{\pgfqpoint{1.381942in}{1.290563in}}%
\pgfpathclose%
\pgfusepath{stroke,fill}%
\end{pgfscope}%
\begin{pgfscope}%
\pgfpathrectangle{\pgfqpoint{0.793996in}{0.443060in}}{\pgfqpoint{2.041610in}{2.577607in}}%
\pgfusepath{clip}%
\pgfsetbuttcap%
\pgfsetroundjoin%
\definecolor{currentfill}{rgb}{0.910863,0.840546,0.801899}%
\pgfsetfillcolor{currentfill}%
\pgfsetlinewidth{0.602250pt}%
\definecolor{currentstroke}{rgb}{0.296471,0.296471,0.296471}%
\pgfsetstrokecolor{currentstroke}%
\pgfsetdash{}{0pt}%
\pgfpathmoveto{\pgfqpoint{1.545389in}{1.291138in}}%
\pgfpathlineto{\pgfqpoint{1.595318in}{1.291138in}}%
\pgfpathlineto{\pgfqpoint{1.595318in}{1.288837in}}%
\pgfpathlineto{\pgfqpoint{1.545389in}{1.288837in}}%
\pgfpathlineto{\pgfqpoint{1.545389in}{1.291138in}}%
\pgfpathclose%
\pgfusepath{stroke,fill}%
\end{pgfscope}%
\begin{pgfscope}%
\pgfpathrectangle{\pgfqpoint{0.793996in}{0.443060in}}{\pgfqpoint{2.041610in}{2.577607in}}%
\pgfusepath{clip}%
\pgfsetbuttcap%
\pgfsetroundjoin%
\definecolor{currentfill}{rgb}{0.901453,0.815098,0.767370}%
\pgfsetfillcolor{currentfill}%
\pgfsetlinewidth{0.602250pt}%
\definecolor{currentstroke}{rgb}{0.296471,0.296471,0.296471}%
\pgfsetstrokecolor{currentstroke}%
\pgfsetdash{}{0pt}%
\pgfpathmoveto{\pgfqpoint{1.595318in}{1.292289in}}%
\pgfpathlineto{\pgfqpoint{1.614615in}{1.292289in}}%
\pgfpathlineto{\pgfqpoint{1.614615in}{1.287686in}}%
\pgfpathlineto{\pgfqpoint{1.595318in}{1.287686in}}%
\pgfpathlineto{\pgfqpoint{1.595318in}{1.292289in}}%
\pgfpathclose%
\pgfusepath{stroke,fill}%
\end{pgfscope}%
\begin{pgfscope}%
\pgfpathrectangle{\pgfqpoint{0.793996in}{0.443060in}}{\pgfqpoint{2.041610in}{2.577607in}}%
\pgfusepath{clip}%
\pgfsetbuttcap%
\pgfsetroundjoin%
\definecolor{currentfill}{rgb}{0.889102,0.781698,0.722050}%
\pgfsetfillcolor{currentfill}%
\pgfsetlinewidth{0.602250pt}%
\definecolor{currentstroke}{rgb}{0.296471,0.296471,0.296471}%
\pgfsetstrokecolor{currentstroke}%
\pgfsetdash{}{0pt}%
\pgfpathmoveto{\pgfqpoint{1.614615in}{1.294590in}}%
\pgfpathlineto{\pgfqpoint{1.630532in}{1.294590in}}%
\pgfpathlineto{\pgfqpoint{1.630532in}{1.285385in}}%
\pgfpathlineto{\pgfqpoint{1.614615in}{1.285385in}}%
\pgfpathlineto{\pgfqpoint{1.614615in}{1.294590in}}%
\pgfpathclose%
\pgfusepath{stroke,fill}%
\end{pgfscope}%
\begin{pgfscope}%
\pgfpathrectangle{\pgfqpoint{0.793996in}{0.443060in}}{\pgfqpoint{2.041610in}{2.577607in}}%
\pgfusepath{clip}%
\pgfsetbuttcap%
\pgfsetroundjoin%
\definecolor{currentfill}{rgb}{0.873223,0.738755,0.663782}%
\pgfsetfillcolor{currentfill}%
\pgfsetlinewidth{0.602250pt}%
\definecolor{currentstroke}{rgb}{0.296471,0.296471,0.296471}%
\pgfsetstrokecolor{currentstroke}%
\pgfsetdash{}{0pt}%
\pgfpathmoveto{\pgfqpoint{1.630532in}{1.299193in}}%
\pgfpathlineto{\pgfqpoint{1.671531in}{1.299193in}}%
\pgfpathlineto{\pgfqpoint{1.671531in}{1.280782in}}%
\pgfpathlineto{\pgfqpoint{1.630532in}{1.280782in}}%
\pgfpathlineto{\pgfqpoint{1.630532in}{1.299193in}}%
\pgfpathclose%
\pgfusepath{stroke,fill}%
\end{pgfscope}%
\begin{pgfscope}%
\pgfpathrectangle{\pgfqpoint{0.793996in}{0.443060in}}{\pgfqpoint{2.041610in}{2.577607in}}%
\pgfusepath{clip}%
\pgfsetbuttcap%
\pgfsetroundjoin%
\definecolor{currentfill}{rgb}{0.853814,0.686269,0.592565}%
\pgfsetfillcolor{currentfill}%
\pgfsetlinewidth{0.602250pt}%
\definecolor{currentstroke}{rgb}{0.296471,0.296471,0.296471}%
\pgfsetstrokecolor{currentstroke}%
\pgfsetdash{}{0pt}%
\pgfpathmoveto{\pgfqpoint{1.671531in}{1.308399in}}%
\pgfpathlineto{\pgfqpoint{1.782559in}{1.308399in}}%
\pgfpathlineto{\pgfqpoint{1.782559in}{1.271576in}}%
\pgfpathlineto{\pgfqpoint{1.671531in}{1.271576in}}%
\pgfpathlineto{\pgfqpoint{1.671531in}{1.308399in}}%
\pgfpathclose%
\pgfusepath{stroke,fill}%
\end{pgfscope}%
\begin{pgfscope}%
\pgfpathrectangle{\pgfqpoint{0.793996in}{0.443060in}}{\pgfqpoint{2.041610in}{2.577607in}}%
\pgfusepath{clip}%
\pgfsetbuttcap%
\pgfsetroundjoin%
\definecolor{currentfill}{rgb}{0.829112,0.619469,0.501926}%
\pgfsetfillcolor{currentfill}%
\pgfsetlinewidth{0.602250pt}%
\definecolor{currentstroke}{rgb}{0.296471,0.296471,0.296471}%
\pgfsetstrokecolor{currentstroke}%
\pgfsetdash{}{0pt}%
\pgfpathmoveto{\pgfqpoint{1.782559in}{1.326811in}}%
\pgfpathlineto{\pgfqpoint{1.881539in}{1.326811in}}%
\pgfpathlineto{\pgfqpoint{1.881539in}{1.253165in}}%
\pgfpathlineto{\pgfqpoint{1.782559in}{1.253165in}}%
\pgfpathlineto{\pgfqpoint{1.782559in}{1.326811in}}%
\pgfpathclose%
\pgfusepath{stroke,fill}%
\end{pgfscope}%
\begin{pgfscope}%
\pgfpathrectangle{\pgfqpoint{0.793996in}{0.443060in}}{\pgfqpoint{2.041610in}{2.577607in}}%
\pgfusepath{clip}%
\pgfsetbuttcap%
\pgfsetroundjoin%
\definecolor{currentfill}{rgb}{0.798529,0.536765,0.389706}%
\pgfsetfillcolor{currentfill}%
\pgfsetlinewidth{0.602250pt}%
\definecolor{currentstroke}{rgb}{0.296471,0.296471,0.296471}%
\pgfsetstrokecolor{currentstroke}%
\pgfsetdash{}{0pt}%
\pgfpathmoveto{\pgfqpoint{1.881539in}{1.363634in}}%
\pgfpathlineto{\pgfqpoint{2.155464in}{1.363634in}}%
\pgfpathlineto{\pgfqpoint{2.155464in}{1.216342in}}%
\pgfpathlineto{\pgfqpoint{1.881539in}{1.216342in}}%
\pgfpathlineto{\pgfqpoint{1.881539in}{1.363634in}}%
\pgfpathclose%
\pgfusepath{stroke,fill}%
\end{pgfscope}%
\begin{pgfscope}%
\pgfpathrectangle{\pgfqpoint{0.793996in}{0.443060in}}{\pgfqpoint{2.041610in}{2.577607in}}%
\pgfusepath{clip}%
\pgfsetbuttcap%
\pgfsetroundjoin%
\definecolor{currentfill}{rgb}{0.829112,0.619469,0.501926}%
\pgfsetfillcolor{currentfill}%
\pgfsetlinewidth{0.602250pt}%
\definecolor{currentstroke}{rgb}{0.296471,0.296471,0.296471}%
\pgfsetstrokecolor{currentstroke}%
\pgfsetdash{}{0pt}%
\pgfpathmoveto{\pgfqpoint{2.155464in}{1.326811in}}%
\pgfpathlineto{\pgfqpoint{2.508948in}{1.326811in}}%
\pgfpathlineto{\pgfqpoint{2.508948in}{1.253165in}}%
\pgfpathlineto{\pgfqpoint{2.155464in}{1.253165in}}%
\pgfpathlineto{\pgfqpoint{2.155464in}{1.326811in}}%
\pgfpathclose%
\pgfusepath{stroke,fill}%
\end{pgfscope}%
\begin{pgfscope}%
\pgfpathrectangle{\pgfqpoint{0.793996in}{0.443060in}}{\pgfqpoint{2.041610in}{2.577607in}}%
\pgfusepath{clip}%
\pgfsetbuttcap%
\pgfsetroundjoin%
\definecolor{currentfill}{rgb}{0.853814,0.686269,0.592565}%
\pgfsetfillcolor{currentfill}%
\pgfsetlinewidth{0.602250pt}%
\definecolor{currentstroke}{rgb}{0.296471,0.296471,0.296471}%
\pgfsetstrokecolor{currentstroke}%
\pgfsetdash{}{0pt}%
\pgfpathmoveto{\pgfqpoint{2.508948in}{1.308399in}}%
\pgfpathlineto{\pgfqpoint{2.543947in}{1.308399in}}%
\pgfpathlineto{\pgfqpoint{2.543947in}{1.271576in}}%
\pgfpathlineto{\pgfqpoint{2.508948in}{1.271576in}}%
\pgfpathlineto{\pgfqpoint{2.508948in}{1.308399in}}%
\pgfpathclose%
\pgfusepath{stroke,fill}%
\end{pgfscope}%
\begin{pgfscope}%
\pgfpathrectangle{\pgfqpoint{0.793996in}{0.443060in}}{\pgfqpoint{2.041610in}{2.577607in}}%
\pgfusepath{clip}%
\pgfsetbuttcap%
\pgfsetroundjoin%
\definecolor{currentfill}{rgb}{0.873223,0.738755,0.663782}%
\pgfsetfillcolor{currentfill}%
\pgfsetlinewidth{0.602250pt}%
\definecolor{currentstroke}{rgb}{0.296471,0.296471,0.296471}%
\pgfsetstrokecolor{currentstroke}%
\pgfsetdash{}{0pt}%
\pgfpathmoveto{\pgfqpoint{2.543947in}{1.299193in}}%
\pgfpathlineto{\pgfqpoint{2.589502in}{1.299193in}}%
\pgfpathlineto{\pgfqpoint{2.589502in}{1.280782in}}%
\pgfpathlineto{\pgfqpoint{2.543947in}{1.280782in}}%
\pgfpathlineto{\pgfqpoint{2.543947in}{1.299193in}}%
\pgfpathclose%
\pgfusepath{stroke,fill}%
\end{pgfscope}%
\begin{pgfscope}%
\pgfpathrectangle{\pgfqpoint{0.793996in}{0.443060in}}{\pgfqpoint{2.041610in}{2.577607in}}%
\pgfusepath{clip}%
\pgfsetbuttcap%
\pgfsetroundjoin%
\definecolor{currentfill}{rgb}{0.889102,0.781698,0.722050}%
\pgfsetfillcolor{currentfill}%
\pgfsetlinewidth{0.602250pt}%
\definecolor{currentstroke}{rgb}{0.296471,0.296471,0.296471}%
\pgfsetstrokecolor{currentstroke}%
\pgfsetdash{}{0pt}%
\pgfpathmoveto{\pgfqpoint{2.589502in}{1.294590in}}%
\pgfpathlineto{\pgfqpoint{2.717508in}{1.294590in}}%
\pgfpathlineto{\pgfqpoint{2.717508in}{1.285385in}}%
\pgfpathlineto{\pgfqpoint{2.589502in}{1.285385in}}%
\pgfpathlineto{\pgfqpoint{2.589502in}{1.294590in}}%
\pgfpathclose%
\pgfusepath{stroke,fill}%
\end{pgfscope}%
\begin{pgfscope}%
\pgfpathrectangle{\pgfqpoint{0.793996in}{0.443060in}}{\pgfqpoint{2.041610in}{2.577607in}}%
\pgfusepath{clip}%
\pgfsetbuttcap%
\pgfsetroundjoin%
\definecolor{currentfill}{rgb}{0.901453,0.815098,0.767370}%
\pgfsetfillcolor{currentfill}%
\pgfsetlinewidth{0.602250pt}%
\definecolor{currentstroke}{rgb}{0.296471,0.296471,0.296471}%
\pgfsetstrokecolor{currentstroke}%
\pgfsetdash{}{0pt}%
\pgfpathmoveto{\pgfqpoint{2.717508in}{1.292289in}}%
\pgfpathlineto{\pgfqpoint{2.721595in}{1.292289in}}%
\pgfpathlineto{\pgfqpoint{2.721595in}{1.287686in}}%
\pgfpathlineto{\pgfqpoint{2.717508in}{1.287686in}}%
\pgfpathlineto{\pgfqpoint{2.717508in}{1.292289in}}%
\pgfpathclose%
\pgfusepath{stroke,fill}%
\end{pgfscope}%
\begin{pgfscope}%
\pgfpathrectangle{\pgfqpoint{0.793996in}{0.443060in}}{\pgfqpoint{2.041610in}{2.577607in}}%
\pgfusepath{clip}%
\pgfsetbuttcap%
\pgfsetroundjoin%
\definecolor{currentfill}{rgb}{0.910863,0.840546,0.801899}%
\pgfsetfillcolor{currentfill}%
\pgfsetlinewidth{0.602250pt}%
\definecolor{currentstroke}{rgb}{0.296471,0.296471,0.296471}%
\pgfsetstrokecolor{currentstroke}%
\pgfsetdash{}{0pt}%
\pgfpathmoveto{\pgfqpoint{2.721595in}{1.291138in}}%
\pgfpathlineto{\pgfqpoint{2.732326in}{1.291138in}}%
\pgfpathlineto{\pgfqpoint{2.732326in}{1.288837in}}%
\pgfpathlineto{\pgfqpoint{2.721595in}{1.288837in}}%
\pgfpathlineto{\pgfqpoint{2.721595in}{1.291138in}}%
\pgfpathclose%
\pgfusepath{stroke,fill}%
\end{pgfscope}%
\begin{pgfscope}%
\pgfpathrectangle{\pgfqpoint{0.793996in}{0.443060in}}{\pgfqpoint{2.041610in}{2.577607in}}%
\pgfusepath{clip}%
\pgfsetbuttcap%
\pgfsetroundjoin%
\definecolor{currentfill}{rgb}{0.919097,0.862812,0.832112}%
\pgfsetfillcolor{currentfill}%
\pgfsetlinewidth{0.602250pt}%
\definecolor{currentstroke}{rgb}{0.296471,0.296471,0.296471}%
\pgfsetstrokecolor{currentstroke}%
\pgfsetdash{}{0pt}%
\pgfpathmoveto{\pgfqpoint{2.732326in}{1.290563in}}%
\pgfpathlineto{\pgfqpoint{2.733050in}{1.290563in}}%
\pgfpathlineto{\pgfqpoint{2.733050in}{1.289412in}}%
\pgfpathlineto{\pgfqpoint{2.732326in}{1.289412in}}%
\pgfpathlineto{\pgfqpoint{2.732326in}{1.290563in}}%
\pgfpathclose%
\pgfusepath{stroke,fill}%
\end{pgfscope}%
\begin{pgfscope}%
\pgfpathrectangle{\pgfqpoint{0.793996in}{0.443060in}}{\pgfqpoint{2.041610in}{2.577607in}}%
\pgfusepath{clip}%
\pgfsetbuttcap%
\pgfsetroundjoin%
\pgfsetlinewidth{0.803000pt}%
\definecolor{currentstroke}{rgb}{0.450000,0.450000,0.450000}%
\pgfsetstrokecolor{currentstroke}%
\pgfsetdash{}{0pt}%
\pgfpathmoveto{\pgfqpoint{0.000000in}{-0.034722in}}%
\pgfpathcurveto{\pgfqpoint{0.009208in}{-0.034722in}}{\pgfqpoint{0.018041in}{-0.031064in}}{\pgfqpoint{0.024552in}{-0.024552in}}%
\pgfpathcurveto{\pgfqpoint{0.031064in}{-0.018041in}}{\pgfqpoint{0.034722in}{-0.009208in}}{\pgfqpoint{0.034722in}{0.000000in}}%
\pgfpathcurveto{\pgfqpoint{0.034722in}{0.009208in}}{\pgfqpoint{0.031064in}{0.018041in}}{\pgfqpoint{0.024552in}{0.024552in}}%
\pgfpathcurveto{\pgfqpoint{0.018041in}{0.031064in}}{\pgfqpoint{0.009208in}{0.034722in}}{\pgfqpoint{0.000000in}{0.034722in}}%
\pgfpathcurveto{\pgfqpoint{-0.009208in}{0.034722in}}{\pgfqpoint{-0.018041in}{0.031064in}}{\pgfqpoint{-0.024552in}{0.024552in}}%
\pgfpathcurveto{\pgfqpoint{-0.031064in}{0.018041in}}{\pgfqpoint{-0.034722in}{0.009208in}}{\pgfqpoint{-0.034722in}{0.000000in}}%
\pgfpathcurveto{\pgfqpoint{-0.034722in}{-0.009208in}}{\pgfqpoint{-0.031064in}{-0.018041in}}{\pgfqpoint{-0.024552in}{-0.024552in}}%
\pgfpathcurveto{\pgfqpoint{-0.018041in}{-0.031064in}}{\pgfqpoint{-0.009208in}{-0.034722in}}{\pgfqpoint{0.000000in}{-0.034722in}}%
\pgfusepath{stroke}%
\end{pgfscope}%
\begin{pgfscope}%
\pgfpathrectangle{\pgfqpoint{0.793996in}{0.443060in}}{\pgfqpoint{2.041610in}{2.577607in}}%
\pgfusepath{clip}%
\pgfsetbuttcap%
\pgfsetroundjoin%
\definecolor{currentfill}{rgb}{0.848437,0.867532,0.899724}%
\pgfsetfillcolor{currentfill}%
\pgfsetlinewidth{0.602250pt}%
\definecolor{currentstroke}{rgb}{0.296471,0.296471,0.296471}%
\pgfsetstrokecolor{currentstroke}%
\pgfsetdash{}{0pt}%
\pgfpathmoveto{\pgfqpoint{0.793996in}{1.069338in}}%
\pgfpathlineto{\pgfqpoint{0.793996in}{1.069338in}}%
\pgfpathlineto{\pgfqpoint{0.793996in}{1.068762in}}%
\pgfpathlineto{\pgfqpoint{0.793996in}{1.068762in}}%
\pgfpathlineto{\pgfqpoint{0.793996in}{1.069338in}}%
\pgfpathclose%
\pgfusepath{stroke,fill}%
\end{pgfscope}%
\begin{pgfscope}%
\pgfpathrectangle{\pgfqpoint{0.793996in}{0.443060in}}{\pgfqpoint{2.041610in}{2.577607in}}%
\pgfusepath{clip}%
\pgfsetbuttcap%
\pgfsetroundjoin%
\definecolor{currentfill}{rgb}{0.825117,0.848522,0.887698}%
\pgfsetfillcolor{currentfill}%
\pgfsetlinewidth{0.602250pt}%
\definecolor{currentstroke}{rgb}{0.296471,0.296471,0.296471}%
\pgfsetstrokecolor{currentstroke}%
\pgfsetdash{}{0pt}%
\pgfpathmoveto{\pgfqpoint{0.793996in}{1.069625in}}%
\pgfpathlineto{\pgfqpoint{0.793996in}{1.069625in}}%
\pgfpathlineto{\pgfqpoint{0.793996in}{1.068475in}}%
\pgfpathlineto{\pgfqpoint{0.793996in}{1.068475in}}%
\pgfpathlineto{\pgfqpoint{0.793996in}{1.069625in}}%
\pgfpathclose%
\pgfusepath{stroke,fill}%
\end{pgfscope}%
\begin{pgfscope}%
\pgfpathrectangle{\pgfqpoint{0.793996in}{0.443060in}}{\pgfqpoint{2.041610in}{2.577607in}}%
\pgfusepath{clip}%
\pgfsetbuttcap%
\pgfsetroundjoin%
\definecolor{currentfill}{rgb}{0.792469,0.821908,0.870863}%
\pgfsetfillcolor{currentfill}%
\pgfsetlinewidth{0.602250pt}%
\definecolor{currentstroke}{rgb}{0.296471,0.296471,0.296471}%
\pgfsetstrokecolor{currentstroke}%
\pgfsetdash{}{0pt}%
\pgfpathmoveto{\pgfqpoint{0.793996in}{1.070201in}}%
\pgfpathlineto{\pgfqpoint{0.793996in}{1.070201in}}%
\pgfpathlineto{\pgfqpoint{0.793996in}{1.067899in}}%
\pgfpathlineto{\pgfqpoint{0.793996in}{1.067899in}}%
\pgfpathlineto{\pgfqpoint{0.793996in}{1.070201in}}%
\pgfpathclose%
\pgfusepath{stroke,fill}%
\end{pgfscope}%
\begin{pgfscope}%
\pgfpathrectangle{\pgfqpoint{0.793996in}{0.443060in}}{\pgfqpoint{2.041610in}{2.577607in}}%
\pgfusepath{clip}%
\pgfsetbuttcap%
\pgfsetroundjoin%
\definecolor{currentfill}{rgb}{0.755157,0.791493,0.851622}%
\pgfsetfillcolor{currentfill}%
\pgfsetlinewidth{0.602250pt}%
\definecolor{currentstroke}{rgb}{0.296471,0.296471,0.296471}%
\pgfsetstrokecolor{currentstroke}%
\pgfsetdash{}{0pt}%
\pgfpathmoveto{\pgfqpoint{0.793996in}{1.071351in}}%
\pgfpathlineto{\pgfqpoint{0.793996in}{1.071351in}}%
\pgfpathlineto{\pgfqpoint{0.793996in}{1.066748in}}%
\pgfpathlineto{\pgfqpoint{0.793996in}{1.066748in}}%
\pgfpathlineto{\pgfqpoint{0.793996in}{1.071351in}}%
\pgfpathclose%
\pgfusepath{stroke,fill}%
\end{pgfscope}%
\begin{pgfscope}%
\pgfpathrectangle{\pgfqpoint{0.793996in}{0.443060in}}{\pgfqpoint{2.041610in}{2.577607in}}%
\pgfusepath{clip}%
\pgfsetbuttcap%
\pgfsetroundjoin%
\definecolor{currentfill}{rgb}{0.706185,0.751573,0.826368}%
\pgfsetfillcolor{currentfill}%
\pgfsetlinewidth{0.602250pt}%
\definecolor{currentstroke}{rgb}{0.296471,0.296471,0.296471}%
\pgfsetstrokecolor{currentstroke}%
\pgfsetdash{}{0pt}%
\pgfpathmoveto{\pgfqpoint{0.793996in}{1.073653in}}%
\pgfpathlineto{\pgfqpoint{0.793996in}{1.073653in}}%
\pgfpathlineto{\pgfqpoint{0.793996in}{1.064447in}}%
\pgfpathlineto{\pgfqpoint{0.793996in}{1.064447in}}%
\pgfpathlineto{\pgfqpoint{0.793996in}{1.073653in}}%
\pgfpathclose%
\pgfusepath{stroke,fill}%
\end{pgfscope}%
\begin{pgfscope}%
\pgfpathrectangle{\pgfqpoint{0.793996in}{0.443060in}}{\pgfqpoint{2.041610in}{2.577607in}}%
\pgfusepath{clip}%
\pgfsetbuttcap%
\pgfsetroundjoin%
\definecolor{currentfill}{rgb}{0.643221,0.700246,0.793900}%
\pgfsetfillcolor{currentfill}%
\pgfsetlinewidth{0.602250pt}%
\definecolor{currentstroke}{rgb}{0.296471,0.296471,0.296471}%
\pgfsetstrokecolor{currentstroke}%
\pgfsetdash{}{0pt}%
\pgfpathmoveto{\pgfqpoint{0.793996in}{1.078256in}}%
\pgfpathlineto{\pgfqpoint{0.793996in}{1.078256in}}%
\pgfpathlineto{\pgfqpoint{0.793996in}{1.059844in}}%
\pgfpathlineto{\pgfqpoint{0.793996in}{1.059844in}}%
\pgfpathlineto{\pgfqpoint{0.793996in}{1.078256in}}%
\pgfpathclose%
\pgfusepath{stroke,fill}%
\end{pgfscope}%
\begin{pgfscope}%
\pgfpathrectangle{\pgfqpoint{0.793996in}{0.443060in}}{\pgfqpoint{2.041610in}{2.577607in}}%
\pgfusepath{clip}%
\pgfsetbuttcap%
\pgfsetroundjoin%
\definecolor{currentfill}{rgb}{0.566266,0.637515,0.754216}%
\pgfsetfillcolor{currentfill}%
\pgfsetlinewidth{0.602250pt}%
\definecolor{currentstroke}{rgb}{0.296471,0.296471,0.296471}%
\pgfsetstrokecolor{currentstroke}%
\pgfsetdash{}{0pt}%
\pgfpathmoveto{\pgfqpoint{0.793996in}{1.087461in}}%
\pgfpathlineto{\pgfqpoint{0.793996in}{1.087461in}}%
\pgfpathlineto{\pgfqpoint{0.793996in}{1.050638in}}%
\pgfpathlineto{\pgfqpoint{0.793996in}{1.050638in}}%
\pgfpathlineto{\pgfqpoint{0.793996in}{1.087461in}}%
\pgfpathclose%
\pgfusepath{stroke,fill}%
\end{pgfscope}%
\begin{pgfscope}%
\pgfpathrectangle{\pgfqpoint{0.793996in}{0.443060in}}{\pgfqpoint{2.041610in}{2.577607in}}%
\pgfusepath{clip}%
\pgfsetbuttcap%
\pgfsetroundjoin%
\definecolor{currentfill}{rgb}{0.468322,0.557674,0.703709}%
\pgfsetfillcolor{currentfill}%
\pgfsetlinewidth{0.602250pt}%
\definecolor{currentstroke}{rgb}{0.296471,0.296471,0.296471}%
\pgfsetstrokecolor{currentstroke}%
\pgfsetdash{}{0pt}%
\pgfpathmoveto{\pgfqpoint{0.793996in}{1.105873in}}%
\pgfpathlineto{\pgfqpoint{0.793996in}{1.105873in}}%
\pgfpathlineto{\pgfqpoint{0.793996in}{1.032227in}}%
\pgfpathlineto{\pgfqpoint{0.793996in}{1.032227in}}%
\pgfpathlineto{\pgfqpoint{0.793996in}{1.105873in}}%
\pgfpathclose%
\pgfusepath{stroke,fill}%
\end{pgfscope}%
\begin{pgfscope}%
\pgfpathrectangle{\pgfqpoint{0.793996in}{0.443060in}}{\pgfqpoint{2.041610in}{2.577607in}}%
\pgfusepath{clip}%
\pgfsetbuttcap%
\pgfsetroundjoin%
\definecolor{currentfill}{rgb}{0.347059,0.458824,0.641176}%
\pgfsetfillcolor{currentfill}%
\pgfsetlinewidth{0.602250pt}%
\definecolor{currentstroke}{rgb}{0.296471,0.296471,0.296471}%
\pgfsetstrokecolor{currentstroke}%
\pgfsetdash{}{0pt}%
\pgfpathmoveto{\pgfqpoint{0.793996in}{1.142696in}}%
\pgfpathlineto{\pgfqpoint{0.793996in}{1.142696in}}%
\pgfpathlineto{\pgfqpoint{0.793996in}{0.995404in}}%
\pgfpathlineto{\pgfqpoint{0.793996in}{0.995404in}}%
\pgfpathlineto{\pgfqpoint{0.793996in}{1.142696in}}%
\pgfpathclose%
\pgfusepath{stroke,fill}%
\end{pgfscope}%
\begin{pgfscope}%
\pgfpathrectangle{\pgfqpoint{0.793996in}{0.443060in}}{\pgfqpoint{2.041610in}{2.577607in}}%
\pgfusepath{clip}%
\pgfsetbuttcap%
\pgfsetroundjoin%
\definecolor{currentfill}{rgb}{0.468322,0.557674,0.703709}%
\pgfsetfillcolor{currentfill}%
\pgfsetlinewidth{0.602250pt}%
\definecolor{currentstroke}{rgb}{0.296471,0.296471,0.296471}%
\pgfsetstrokecolor{currentstroke}%
\pgfsetdash{}{0pt}%
\pgfpathmoveto{\pgfqpoint{0.793996in}{1.105873in}}%
\pgfpathlineto{\pgfqpoint{0.793996in}{1.105873in}}%
\pgfpathlineto{\pgfqpoint{0.793996in}{1.032227in}}%
\pgfpathlineto{\pgfqpoint{0.793996in}{1.032227in}}%
\pgfpathlineto{\pgfqpoint{0.793996in}{1.105873in}}%
\pgfpathclose%
\pgfusepath{stroke,fill}%
\end{pgfscope}%
\begin{pgfscope}%
\pgfpathrectangle{\pgfqpoint{0.793996in}{0.443060in}}{\pgfqpoint{2.041610in}{2.577607in}}%
\pgfusepath{clip}%
\pgfsetbuttcap%
\pgfsetroundjoin%
\definecolor{currentfill}{rgb}{0.566266,0.637515,0.754216}%
\pgfsetfillcolor{currentfill}%
\pgfsetlinewidth{0.602250pt}%
\definecolor{currentstroke}{rgb}{0.296471,0.296471,0.296471}%
\pgfsetstrokecolor{currentstroke}%
\pgfsetdash{}{0pt}%
\pgfpathmoveto{\pgfqpoint{0.793996in}{1.087461in}}%
\pgfpathlineto{\pgfqpoint{0.793996in}{1.087461in}}%
\pgfpathlineto{\pgfqpoint{0.793996in}{1.050638in}}%
\pgfpathlineto{\pgfqpoint{0.793996in}{1.050638in}}%
\pgfpathlineto{\pgfqpoint{0.793996in}{1.087461in}}%
\pgfpathclose%
\pgfusepath{stroke,fill}%
\end{pgfscope}%
\begin{pgfscope}%
\pgfpathrectangle{\pgfqpoint{0.793996in}{0.443060in}}{\pgfqpoint{2.041610in}{2.577607in}}%
\pgfusepath{clip}%
\pgfsetbuttcap%
\pgfsetroundjoin%
\definecolor{currentfill}{rgb}{0.643221,0.700246,0.793900}%
\pgfsetfillcolor{currentfill}%
\pgfsetlinewidth{0.602250pt}%
\definecolor{currentstroke}{rgb}{0.296471,0.296471,0.296471}%
\pgfsetstrokecolor{currentstroke}%
\pgfsetdash{}{0pt}%
\pgfpathmoveto{\pgfqpoint{0.793996in}{1.078256in}}%
\pgfpathlineto{\pgfqpoint{0.793996in}{1.078256in}}%
\pgfpathlineto{\pgfqpoint{0.793996in}{1.059844in}}%
\pgfpathlineto{\pgfqpoint{0.793996in}{1.059844in}}%
\pgfpathlineto{\pgfqpoint{0.793996in}{1.078256in}}%
\pgfpathclose%
\pgfusepath{stroke,fill}%
\end{pgfscope}%
\begin{pgfscope}%
\pgfpathrectangle{\pgfqpoint{0.793996in}{0.443060in}}{\pgfqpoint{2.041610in}{2.577607in}}%
\pgfusepath{clip}%
\pgfsetbuttcap%
\pgfsetroundjoin%
\definecolor{currentfill}{rgb}{0.706185,0.751573,0.826368}%
\pgfsetfillcolor{currentfill}%
\pgfsetlinewidth{0.602250pt}%
\definecolor{currentstroke}{rgb}{0.296471,0.296471,0.296471}%
\pgfsetstrokecolor{currentstroke}%
\pgfsetdash{}{0pt}%
\pgfpathmoveto{\pgfqpoint{0.793996in}{1.073653in}}%
\pgfpathlineto{\pgfqpoint{0.793996in}{1.073653in}}%
\pgfpathlineto{\pgfqpoint{0.793996in}{1.064447in}}%
\pgfpathlineto{\pgfqpoint{0.793996in}{1.064447in}}%
\pgfpathlineto{\pgfqpoint{0.793996in}{1.073653in}}%
\pgfpathclose%
\pgfusepath{stroke,fill}%
\end{pgfscope}%
\begin{pgfscope}%
\pgfpathrectangle{\pgfqpoint{0.793996in}{0.443060in}}{\pgfqpoint{2.041610in}{2.577607in}}%
\pgfusepath{clip}%
\pgfsetbuttcap%
\pgfsetroundjoin%
\definecolor{currentfill}{rgb}{0.755157,0.791493,0.851622}%
\pgfsetfillcolor{currentfill}%
\pgfsetlinewidth{0.602250pt}%
\definecolor{currentstroke}{rgb}{0.296471,0.296471,0.296471}%
\pgfsetstrokecolor{currentstroke}%
\pgfsetdash{}{0pt}%
\pgfpathmoveto{\pgfqpoint{0.793996in}{1.071351in}}%
\pgfpathlineto{\pgfqpoint{0.793996in}{1.071351in}}%
\pgfpathlineto{\pgfqpoint{0.793996in}{1.066748in}}%
\pgfpathlineto{\pgfqpoint{0.793996in}{1.066748in}}%
\pgfpathlineto{\pgfqpoint{0.793996in}{1.071351in}}%
\pgfpathclose%
\pgfusepath{stroke,fill}%
\end{pgfscope}%
\begin{pgfscope}%
\pgfpathrectangle{\pgfqpoint{0.793996in}{0.443060in}}{\pgfqpoint{2.041610in}{2.577607in}}%
\pgfusepath{clip}%
\pgfsetbuttcap%
\pgfsetroundjoin%
\definecolor{currentfill}{rgb}{0.792469,0.821908,0.870863}%
\pgfsetfillcolor{currentfill}%
\pgfsetlinewidth{0.602250pt}%
\definecolor{currentstroke}{rgb}{0.296471,0.296471,0.296471}%
\pgfsetstrokecolor{currentstroke}%
\pgfsetdash{}{0pt}%
\pgfpathmoveto{\pgfqpoint{0.793996in}{1.070201in}}%
\pgfpathlineto{\pgfqpoint{0.793996in}{1.070201in}}%
\pgfpathlineto{\pgfqpoint{0.793996in}{1.067899in}}%
\pgfpathlineto{\pgfqpoint{0.793996in}{1.067899in}}%
\pgfpathlineto{\pgfqpoint{0.793996in}{1.070201in}}%
\pgfpathclose%
\pgfusepath{stroke,fill}%
\end{pgfscope}%
\begin{pgfscope}%
\pgfpathrectangle{\pgfqpoint{0.793996in}{0.443060in}}{\pgfqpoint{2.041610in}{2.577607in}}%
\pgfusepath{clip}%
\pgfsetbuttcap%
\pgfsetroundjoin%
\definecolor{currentfill}{rgb}{0.825117,0.848522,0.887698}%
\pgfsetfillcolor{currentfill}%
\pgfsetlinewidth{0.602250pt}%
\definecolor{currentstroke}{rgb}{0.296471,0.296471,0.296471}%
\pgfsetstrokecolor{currentstroke}%
\pgfsetdash{}{0pt}%
\pgfpathmoveto{\pgfqpoint{0.793996in}{1.069625in}}%
\pgfpathlineto{\pgfqpoint{0.793996in}{1.069625in}}%
\pgfpathlineto{\pgfqpoint{0.793996in}{1.068475in}}%
\pgfpathlineto{\pgfqpoint{0.793996in}{1.068475in}}%
\pgfpathlineto{\pgfqpoint{0.793996in}{1.069625in}}%
\pgfpathclose%
\pgfusepath{stroke,fill}%
\end{pgfscope}%
\begin{pgfscope}%
\pgfpathrectangle{\pgfqpoint{0.793996in}{0.443060in}}{\pgfqpoint{2.041610in}{2.577607in}}%
\pgfusepath{clip}%
\pgfsetbuttcap%
\pgfsetroundjoin%
\definecolor{currentfill}{rgb}{0.848437,0.867532,0.899724}%
\pgfsetfillcolor{currentfill}%
\pgfsetlinewidth{0.602250pt}%
\definecolor{currentstroke}{rgb}{0.296471,0.296471,0.296471}%
\pgfsetstrokecolor{currentstroke}%
\pgfsetdash{}{0pt}%
\pgfpathmoveto{\pgfqpoint{0.793996in}{1.069338in}}%
\pgfpathlineto{\pgfqpoint{0.793996in}{1.069338in}}%
\pgfpathlineto{\pgfqpoint{0.793996in}{1.068762in}}%
\pgfpathlineto{\pgfqpoint{0.793996in}{1.068762in}}%
\pgfpathlineto{\pgfqpoint{0.793996in}{1.069338in}}%
\pgfpathclose%
\pgfusepath{stroke,fill}%
\end{pgfscope}%
\begin{pgfscope}%
\pgfpathrectangle{\pgfqpoint{0.793996in}{0.443060in}}{\pgfqpoint{2.041610in}{2.577607in}}%
\pgfusepath{clip}%
\pgfsetbuttcap%
\pgfsetroundjoin%
\pgfsetlinewidth{0.803000pt}%
\definecolor{currentstroke}{rgb}{0.450000,0.450000,0.450000}%
\pgfsetstrokecolor{currentstroke}%
\pgfsetdash{}{0pt}%
\pgfpathmoveto{\pgfqpoint{0.000000in}{-0.034722in}}%
\pgfpathcurveto{\pgfqpoint{0.009208in}{-0.034722in}}{\pgfqpoint{0.018041in}{-0.031064in}}{\pgfqpoint{0.024552in}{-0.024552in}}%
\pgfpathcurveto{\pgfqpoint{0.031064in}{-0.018041in}}{\pgfqpoint{0.034722in}{-0.009208in}}{\pgfqpoint{0.034722in}{0.000000in}}%
\pgfpathcurveto{\pgfqpoint{0.034722in}{0.009208in}}{\pgfqpoint{0.031064in}{0.018041in}}{\pgfqpoint{0.024552in}{0.024552in}}%
\pgfpathcurveto{\pgfqpoint{0.018041in}{0.031064in}}{\pgfqpoint{0.009208in}{0.034722in}}{\pgfqpoint{0.000000in}{0.034722in}}%
\pgfpathcurveto{\pgfqpoint{-0.009208in}{0.034722in}}{\pgfqpoint{-0.018041in}{0.031064in}}{\pgfqpoint{-0.024552in}{0.024552in}}%
\pgfpathcurveto{\pgfqpoint{-0.031064in}{0.018041in}}{\pgfqpoint{-0.034722in}{0.009208in}}{\pgfqpoint{-0.034722in}{0.000000in}}%
\pgfpathcurveto{\pgfqpoint{-0.034722in}{-0.009208in}}{\pgfqpoint{-0.031064in}{-0.018041in}}{\pgfqpoint{-0.024552in}{-0.024552in}}%
\pgfpathcurveto{\pgfqpoint{-0.018041in}{-0.031064in}}{\pgfqpoint{-0.009208in}{-0.034722in}}{\pgfqpoint{0.000000in}{-0.034722in}}%
\pgfusepath{stroke}%
\end{pgfscope}%
\begin{pgfscope}%
\pgfpathrectangle{\pgfqpoint{0.793996in}{0.443060in}}{\pgfqpoint{2.041610in}{2.577607in}}%
\pgfusepath{clip}%
\pgfsetbuttcap%
\pgfsetroundjoin%
\definecolor{currentfill}{rgb}{0.919097,0.862812,0.832112}%
\pgfsetfillcolor{currentfill}%
\pgfsetlinewidth{0.602250pt}%
\definecolor{currentstroke}{rgb}{0.296471,0.296471,0.296471}%
\pgfsetstrokecolor{currentstroke}%
\pgfsetdash{}{0pt}%
\pgfpathmoveto{\pgfqpoint{0.946518in}{0.922333in}}%
\pgfpathlineto{\pgfqpoint{0.953100in}{0.922333in}}%
\pgfpathlineto{\pgfqpoint{0.953100in}{0.921183in}}%
\pgfpathlineto{\pgfqpoint{0.946518in}{0.921183in}}%
\pgfpathlineto{\pgfqpoint{0.946518in}{0.922333in}}%
\pgfpathclose%
\pgfusepath{stroke,fill}%
\end{pgfscope}%
\begin{pgfscope}%
\pgfpathrectangle{\pgfqpoint{0.793996in}{0.443060in}}{\pgfqpoint{2.041610in}{2.577607in}}%
\pgfusepath{clip}%
\pgfsetbuttcap%
\pgfsetroundjoin%
\definecolor{currentfill}{rgb}{0.910863,0.840546,0.801899}%
\pgfsetfillcolor{currentfill}%
\pgfsetlinewidth{0.602250pt}%
\definecolor{currentstroke}{rgb}{0.296471,0.296471,0.296471}%
\pgfsetstrokecolor{currentstroke}%
\pgfsetdash{}{0pt}%
\pgfpathmoveto{\pgfqpoint{0.953100in}{0.922909in}}%
\pgfpathlineto{\pgfqpoint{0.990421in}{0.922909in}}%
\pgfpathlineto{\pgfqpoint{0.990421in}{0.920607in}}%
\pgfpathlineto{\pgfqpoint{0.953100in}{0.920607in}}%
\pgfpathlineto{\pgfqpoint{0.953100in}{0.922909in}}%
\pgfpathclose%
\pgfusepath{stroke,fill}%
\end{pgfscope}%
\begin{pgfscope}%
\pgfpathrectangle{\pgfqpoint{0.793996in}{0.443060in}}{\pgfqpoint{2.041610in}{2.577607in}}%
\pgfusepath{clip}%
\pgfsetbuttcap%
\pgfsetroundjoin%
\definecolor{currentfill}{rgb}{0.901453,0.815098,0.767370}%
\pgfsetfillcolor{currentfill}%
\pgfsetlinewidth{0.602250pt}%
\definecolor{currentstroke}{rgb}{0.296471,0.296471,0.296471}%
\pgfsetstrokecolor{currentstroke}%
\pgfsetdash{}{0pt}%
\pgfpathmoveto{\pgfqpoint{0.990421in}{0.924059in}}%
\pgfpathlineto{\pgfqpoint{1.061434in}{0.924059in}}%
\pgfpathlineto{\pgfqpoint{1.061434in}{0.919457in}}%
\pgfpathlineto{\pgfqpoint{0.990421in}{0.919457in}}%
\pgfpathlineto{\pgfqpoint{0.990421in}{0.924059in}}%
\pgfpathclose%
\pgfusepath{stroke,fill}%
\end{pgfscope}%
\begin{pgfscope}%
\pgfpathrectangle{\pgfqpoint{0.793996in}{0.443060in}}{\pgfqpoint{2.041610in}{2.577607in}}%
\pgfusepath{clip}%
\pgfsetbuttcap%
\pgfsetroundjoin%
\definecolor{currentfill}{rgb}{0.889102,0.781698,0.722050}%
\pgfsetfillcolor{currentfill}%
\pgfsetlinewidth{0.602250pt}%
\definecolor{currentstroke}{rgb}{0.296471,0.296471,0.296471}%
\pgfsetstrokecolor{currentstroke}%
\pgfsetdash{}{0pt}%
\pgfpathmoveto{\pgfqpoint{1.061434in}{0.926361in}}%
\pgfpathlineto{\pgfqpoint{1.216118in}{0.926361in}}%
\pgfpathlineto{\pgfqpoint{1.216118in}{0.917155in}}%
\pgfpathlineto{\pgfqpoint{1.061434in}{0.917155in}}%
\pgfpathlineto{\pgfqpoint{1.061434in}{0.926361in}}%
\pgfpathclose%
\pgfusepath{stroke,fill}%
\end{pgfscope}%
\begin{pgfscope}%
\pgfpathrectangle{\pgfqpoint{0.793996in}{0.443060in}}{\pgfqpoint{2.041610in}{2.577607in}}%
\pgfusepath{clip}%
\pgfsetbuttcap%
\pgfsetroundjoin%
\definecolor{currentfill}{rgb}{0.873223,0.738755,0.663782}%
\pgfsetfillcolor{currentfill}%
\pgfsetlinewidth{0.602250pt}%
\definecolor{currentstroke}{rgb}{0.296471,0.296471,0.296471}%
\pgfsetstrokecolor{currentstroke}%
\pgfsetdash{}{0pt}%
\pgfpathmoveto{\pgfqpoint{1.216118in}{0.930964in}}%
\pgfpathlineto{\pgfqpoint{1.418345in}{0.930964in}}%
\pgfpathlineto{\pgfqpoint{1.418345in}{0.912552in}}%
\pgfpathlineto{\pgfqpoint{1.216118in}{0.912552in}}%
\pgfpathlineto{\pgfqpoint{1.216118in}{0.930964in}}%
\pgfpathclose%
\pgfusepath{stroke,fill}%
\end{pgfscope}%
\begin{pgfscope}%
\pgfpathrectangle{\pgfqpoint{0.793996in}{0.443060in}}{\pgfqpoint{2.041610in}{2.577607in}}%
\pgfusepath{clip}%
\pgfsetbuttcap%
\pgfsetroundjoin%
\definecolor{currentfill}{rgb}{0.853814,0.686269,0.592565}%
\pgfsetfillcolor{currentfill}%
\pgfsetlinewidth{0.602250pt}%
\definecolor{currentstroke}{rgb}{0.296471,0.296471,0.296471}%
\pgfsetstrokecolor{currentstroke}%
\pgfsetdash{}{0pt}%
\pgfpathmoveto{\pgfqpoint{1.418345in}{0.940170in}}%
\pgfpathlineto{\pgfqpoint{1.662173in}{0.940170in}}%
\pgfpathlineto{\pgfqpoint{1.662173in}{0.903347in}}%
\pgfpathlineto{\pgfqpoint{1.418345in}{0.903347in}}%
\pgfpathlineto{\pgfqpoint{1.418345in}{0.940170in}}%
\pgfpathclose%
\pgfusepath{stroke,fill}%
\end{pgfscope}%
\begin{pgfscope}%
\pgfpathrectangle{\pgfqpoint{0.793996in}{0.443060in}}{\pgfqpoint{2.041610in}{2.577607in}}%
\pgfusepath{clip}%
\pgfsetbuttcap%
\pgfsetroundjoin%
\definecolor{currentfill}{rgb}{0.829112,0.619469,0.501926}%
\pgfsetfillcolor{currentfill}%
\pgfsetlinewidth{0.602250pt}%
\definecolor{currentstroke}{rgb}{0.296471,0.296471,0.296471}%
\pgfsetstrokecolor{currentstroke}%
\pgfsetdash{}{0pt}%
\pgfpathmoveto{\pgfqpoint{1.662173in}{0.958581in}}%
\pgfpathlineto{\pgfqpoint{1.856127in}{0.958581in}}%
\pgfpathlineto{\pgfqpoint{1.856127in}{0.884935in}}%
\pgfpathlineto{\pgfqpoint{1.662173in}{0.884935in}}%
\pgfpathlineto{\pgfqpoint{1.662173in}{0.958581in}}%
\pgfpathclose%
\pgfusepath{stroke,fill}%
\end{pgfscope}%
\begin{pgfscope}%
\pgfpathrectangle{\pgfqpoint{0.793996in}{0.443060in}}{\pgfqpoint{2.041610in}{2.577607in}}%
\pgfusepath{clip}%
\pgfsetbuttcap%
\pgfsetroundjoin%
\definecolor{currentfill}{rgb}{0.798529,0.536765,0.389706}%
\pgfsetfillcolor{currentfill}%
\pgfsetlinewidth{0.602250pt}%
\definecolor{currentstroke}{rgb}{0.296471,0.296471,0.296471}%
\pgfsetstrokecolor{currentstroke}%
\pgfsetdash{}{0pt}%
\pgfpathmoveto{\pgfqpoint{1.856127in}{0.995404in}}%
\pgfpathlineto{\pgfqpoint{2.546787in}{0.995404in}}%
\pgfpathlineto{\pgfqpoint{2.546787in}{0.848112in}}%
\pgfpathlineto{\pgfqpoint{1.856127in}{0.848112in}}%
\pgfpathlineto{\pgfqpoint{1.856127in}{0.995404in}}%
\pgfpathclose%
\pgfusepath{stroke,fill}%
\end{pgfscope}%
\begin{pgfscope}%
\pgfpathrectangle{\pgfqpoint{0.793996in}{0.443060in}}{\pgfqpoint{2.041610in}{2.577607in}}%
\pgfusepath{clip}%
\pgfsetbuttcap%
\pgfsetroundjoin%
\definecolor{currentfill}{rgb}{0.829112,0.619469,0.501926}%
\pgfsetfillcolor{currentfill}%
\pgfsetlinewidth{0.602250pt}%
\definecolor{currentstroke}{rgb}{0.296471,0.296471,0.296471}%
\pgfsetstrokecolor{currentstroke}%
\pgfsetdash{}{0pt}%
\pgfpathmoveto{\pgfqpoint{2.546787in}{0.958581in}}%
\pgfpathlineto{\pgfqpoint{2.727691in}{0.958581in}}%
\pgfpathlineto{\pgfqpoint{2.727691in}{0.884935in}}%
\pgfpathlineto{\pgfqpoint{2.546787in}{0.884935in}}%
\pgfpathlineto{\pgfqpoint{2.546787in}{0.958581in}}%
\pgfpathclose%
\pgfusepath{stroke,fill}%
\end{pgfscope}%
\begin{pgfscope}%
\pgfpathrectangle{\pgfqpoint{0.793996in}{0.443060in}}{\pgfqpoint{2.041610in}{2.577607in}}%
\pgfusepath{clip}%
\pgfsetbuttcap%
\pgfsetroundjoin%
\definecolor{currentfill}{rgb}{0.853814,0.686269,0.592565}%
\pgfsetfillcolor{currentfill}%
\pgfsetlinewidth{0.602250pt}%
\definecolor{currentstroke}{rgb}{0.296471,0.296471,0.296471}%
\pgfsetstrokecolor{currentstroke}%
\pgfsetdash{}{0pt}%
\pgfpathmoveto{\pgfqpoint{2.727691in}{0.940170in}}%
\pgfpathlineto{\pgfqpoint{2.783606in}{0.940170in}}%
\pgfpathlineto{\pgfqpoint{2.783606in}{0.903347in}}%
\pgfpathlineto{\pgfqpoint{2.727691in}{0.903347in}}%
\pgfpathlineto{\pgfqpoint{2.727691in}{0.940170in}}%
\pgfpathclose%
\pgfusepath{stroke,fill}%
\end{pgfscope}%
\begin{pgfscope}%
\pgfpathrectangle{\pgfqpoint{0.793996in}{0.443060in}}{\pgfqpoint{2.041610in}{2.577607in}}%
\pgfusepath{clip}%
\pgfsetbuttcap%
\pgfsetroundjoin%
\definecolor{currentfill}{rgb}{0.873223,0.738755,0.663782}%
\pgfsetfillcolor{currentfill}%
\pgfsetlinewidth{0.602250pt}%
\definecolor{currentstroke}{rgb}{0.296471,0.296471,0.296471}%
\pgfsetstrokecolor{currentstroke}%
\pgfsetdash{}{0pt}%
\pgfpathmoveto{\pgfqpoint{2.783606in}{0.930964in}}%
\pgfpathlineto{\pgfqpoint{2.835606in}{0.930964in}}%
\pgfpathlineto{\pgfqpoint{2.835606in}{0.912552in}}%
\pgfpathlineto{\pgfqpoint{2.783606in}{0.912552in}}%
\pgfpathlineto{\pgfqpoint{2.783606in}{0.930964in}}%
\pgfpathclose%
\pgfusepath{stroke,fill}%
\end{pgfscope}%
\begin{pgfscope}%
\pgfpathrectangle{\pgfqpoint{0.793996in}{0.443060in}}{\pgfqpoint{2.041610in}{2.577607in}}%
\pgfusepath{clip}%
\pgfsetbuttcap%
\pgfsetroundjoin%
\definecolor{currentfill}{rgb}{0.889102,0.781698,0.722050}%
\pgfsetfillcolor{currentfill}%
\pgfsetlinewidth{0.602250pt}%
\definecolor{currentstroke}{rgb}{0.296471,0.296471,0.296471}%
\pgfsetstrokecolor{currentstroke}%
\pgfsetdash{}{0pt}%
\pgfpathmoveto{\pgfqpoint{2.835606in}{0.926361in}}%
\pgfpathlineto{\pgfqpoint{2.835606in}{0.926361in}}%
\pgfpathlineto{\pgfqpoint{2.835606in}{0.917155in}}%
\pgfpathlineto{\pgfqpoint{2.835606in}{0.917155in}}%
\pgfpathlineto{\pgfqpoint{2.835606in}{0.926361in}}%
\pgfpathclose%
\pgfusepath{stroke,fill}%
\end{pgfscope}%
\begin{pgfscope}%
\pgfpathrectangle{\pgfqpoint{0.793996in}{0.443060in}}{\pgfqpoint{2.041610in}{2.577607in}}%
\pgfusepath{clip}%
\pgfsetbuttcap%
\pgfsetroundjoin%
\definecolor{currentfill}{rgb}{0.901453,0.815098,0.767370}%
\pgfsetfillcolor{currentfill}%
\pgfsetlinewidth{0.602250pt}%
\definecolor{currentstroke}{rgb}{0.296471,0.296471,0.296471}%
\pgfsetstrokecolor{currentstroke}%
\pgfsetdash{}{0pt}%
\pgfpathmoveto{\pgfqpoint{2.835606in}{0.924059in}}%
\pgfpathlineto{\pgfqpoint{2.835606in}{0.924059in}}%
\pgfpathlineto{\pgfqpoint{2.835606in}{0.919457in}}%
\pgfpathlineto{\pgfqpoint{2.835606in}{0.919457in}}%
\pgfpathlineto{\pgfqpoint{2.835606in}{0.924059in}}%
\pgfpathclose%
\pgfusepath{stroke,fill}%
\end{pgfscope}%
\begin{pgfscope}%
\pgfpathrectangle{\pgfqpoint{0.793996in}{0.443060in}}{\pgfqpoint{2.041610in}{2.577607in}}%
\pgfusepath{clip}%
\pgfsetbuttcap%
\pgfsetroundjoin%
\definecolor{currentfill}{rgb}{0.910863,0.840546,0.801899}%
\pgfsetfillcolor{currentfill}%
\pgfsetlinewidth{0.602250pt}%
\definecolor{currentstroke}{rgb}{0.296471,0.296471,0.296471}%
\pgfsetstrokecolor{currentstroke}%
\pgfsetdash{}{0pt}%
\pgfpathmoveto{\pgfqpoint{2.835606in}{0.922909in}}%
\pgfpathlineto{\pgfqpoint{2.835606in}{0.922909in}}%
\pgfpathlineto{\pgfqpoint{2.835606in}{0.920607in}}%
\pgfpathlineto{\pgfqpoint{2.835606in}{0.920607in}}%
\pgfpathlineto{\pgfqpoint{2.835606in}{0.922909in}}%
\pgfpathclose%
\pgfusepath{stroke,fill}%
\end{pgfscope}%
\begin{pgfscope}%
\pgfpathrectangle{\pgfqpoint{0.793996in}{0.443060in}}{\pgfqpoint{2.041610in}{2.577607in}}%
\pgfusepath{clip}%
\pgfsetbuttcap%
\pgfsetroundjoin%
\definecolor{currentfill}{rgb}{0.919097,0.862812,0.832112}%
\pgfsetfillcolor{currentfill}%
\pgfsetlinewidth{0.602250pt}%
\definecolor{currentstroke}{rgb}{0.296471,0.296471,0.296471}%
\pgfsetstrokecolor{currentstroke}%
\pgfsetdash{}{0pt}%
\pgfpathmoveto{\pgfqpoint{2.835606in}{0.922333in}}%
\pgfpathlineto{\pgfqpoint{2.835606in}{0.922333in}}%
\pgfpathlineto{\pgfqpoint{2.835606in}{0.921183in}}%
\pgfpathlineto{\pgfqpoint{2.835606in}{0.921183in}}%
\pgfpathlineto{\pgfqpoint{2.835606in}{0.922333in}}%
\pgfpathclose%
\pgfusepath{stroke,fill}%
\end{pgfscope}%
\begin{pgfscope}%
\pgfpathrectangle{\pgfqpoint{0.793996in}{0.443060in}}{\pgfqpoint{2.041610in}{2.577607in}}%
\pgfusepath{clip}%
\pgfsetbuttcap%
\pgfsetroundjoin%
\pgfsetlinewidth{0.803000pt}%
\definecolor{currentstroke}{rgb}{0.450000,0.450000,0.450000}%
\pgfsetstrokecolor{currentstroke}%
\pgfsetdash{}{0pt}%
\pgfpathmoveto{\pgfqpoint{0.000000in}{-0.034722in}}%
\pgfpathcurveto{\pgfqpoint{0.009208in}{-0.034722in}}{\pgfqpoint{0.018041in}{-0.031064in}}{\pgfqpoint{0.024552in}{-0.024552in}}%
\pgfpathcurveto{\pgfqpoint{0.031064in}{-0.018041in}}{\pgfqpoint{0.034722in}{-0.009208in}}{\pgfqpoint{0.034722in}{0.000000in}}%
\pgfpathcurveto{\pgfqpoint{0.034722in}{0.009208in}}{\pgfqpoint{0.031064in}{0.018041in}}{\pgfqpoint{0.024552in}{0.024552in}}%
\pgfpathcurveto{\pgfqpoint{0.018041in}{0.031064in}}{\pgfqpoint{0.009208in}{0.034722in}}{\pgfqpoint{0.000000in}{0.034722in}}%
\pgfpathcurveto{\pgfqpoint{-0.009208in}{0.034722in}}{\pgfqpoint{-0.018041in}{0.031064in}}{\pgfqpoint{-0.024552in}{0.024552in}}%
\pgfpathcurveto{\pgfqpoint{-0.031064in}{0.018041in}}{\pgfqpoint{-0.034722in}{0.009208in}}{\pgfqpoint{-0.034722in}{0.000000in}}%
\pgfpathcurveto{\pgfqpoint{-0.034722in}{-0.009208in}}{\pgfqpoint{-0.031064in}{-0.018041in}}{\pgfqpoint{-0.024552in}{-0.024552in}}%
\pgfpathcurveto{\pgfqpoint{-0.018041in}{-0.031064in}}{\pgfqpoint{-0.009208in}{-0.034722in}}{\pgfqpoint{0.000000in}{-0.034722in}}%
\pgfusepath{stroke}%
\end{pgfscope}%
\begin{pgfscope}%
\pgfpathrectangle{\pgfqpoint{0.793996in}{0.443060in}}{\pgfqpoint{2.041610in}{2.577607in}}%
\pgfusepath{clip}%
\pgfsetbuttcap%
\pgfsetroundjoin%
\definecolor{currentfill}{rgb}{0.848437,0.867532,0.899724}%
\pgfsetfillcolor{currentfill}%
\pgfsetlinewidth{0.602250pt}%
\definecolor{currentstroke}{rgb}{0.296471,0.296471,0.296471}%
\pgfsetstrokecolor{currentstroke}%
\pgfsetdash{}{0pt}%
\pgfpathmoveto{\pgfqpoint{0.793996in}{0.701108in}}%
\pgfpathlineto{\pgfqpoint{0.793996in}{0.701108in}}%
\pgfpathlineto{\pgfqpoint{0.793996in}{0.700533in}}%
\pgfpathlineto{\pgfqpoint{0.793996in}{0.700533in}}%
\pgfpathlineto{\pgfqpoint{0.793996in}{0.701108in}}%
\pgfpathclose%
\pgfusepath{stroke,fill}%
\end{pgfscope}%
\begin{pgfscope}%
\pgfpathrectangle{\pgfqpoint{0.793996in}{0.443060in}}{\pgfqpoint{2.041610in}{2.577607in}}%
\pgfusepath{clip}%
\pgfsetbuttcap%
\pgfsetroundjoin%
\definecolor{currentfill}{rgb}{0.825117,0.848522,0.887698}%
\pgfsetfillcolor{currentfill}%
\pgfsetlinewidth{0.602250pt}%
\definecolor{currentstroke}{rgb}{0.296471,0.296471,0.296471}%
\pgfsetstrokecolor{currentstroke}%
\pgfsetdash{}{0pt}%
\pgfpathmoveto{\pgfqpoint{0.793996in}{0.701396in}}%
\pgfpathlineto{\pgfqpoint{0.793996in}{0.701396in}}%
\pgfpathlineto{\pgfqpoint{0.793996in}{0.700245in}}%
\pgfpathlineto{\pgfqpoint{0.793996in}{0.700245in}}%
\pgfpathlineto{\pgfqpoint{0.793996in}{0.701396in}}%
\pgfpathclose%
\pgfusepath{stroke,fill}%
\end{pgfscope}%
\begin{pgfscope}%
\pgfpathrectangle{\pgfqpoint{0.793996in}{0.443060in}}{\pgfqpoint{2.041610in}{2.577607in}}%
\pgfusepath{clip}%
\pgfsetbuttcap%
\pgfsetroundjoin%
\definecolor{currentfill}{rgb}{0.792469,0.821908,0.870863}%
\pgfsetfillcolor{currentfill}%
\pgfsetlinewidth{0.602250pt}%
\definecolor{currentstroke}{rgb}{0.296471,0.296471,0.296471}%
\pgfsetstrokecolor{currentstroke}%
\pgfsetdash{}{0pt}%
\pgfpathmoveto{\pgfqpoint{0.793996in}{0.701971in}}%
\pgfpathlineto{\pgfqpoint{0.793996in}{0.701971in}}%
\pgfpathlineto{\pgfqpoint{0.793996in}{0.699670in}}%
\pgfpathlineto{\pgfqpoint{0.793996in}{0.699670in}}%
\pgfpathlineto{\pgfqpoint{0.793996in}{0.701971in}}%
\pgfpathclose%
\pgfusepath{stroke,fill}%
\end{pgfscope}%
\begin{pgfscope}%
\pgfpathrectangle{\pgfqpoint{0.793996in}{0.443060in}}{\pgfqpoint{2.041610in}{2.577607in}}%
\pgfusepath{clip}%
\pgfsetbuttcap%
\pgfsetroundjoin%
\definecolor{currentfill}{rgb}{0.755157,0.791493,0.851622}%
\pgfsetfillcolor{currentfill}%
\pgfsetlinewidth{0.602250pt}%
\definecolor{currentstroke}{rgb}{0.296471,0.296471,0.296471}%
\pgfsetstrokecolor{currentstroke}%
\pgfsetdash{}{0pt}%
\pgfpathmoveto{\pgfqpoint{0.793996in}{0.703122in}}%
\pgfpathlineto{\pgfqpoint{0.793996in}{0.703122in}}%
\pgfpathlineto{\pgfqpoint{0.793996in}{0.698519in}}%
\pgfpathlineto{\pgfqpoint{0.793996in}{0.698519in}}%
\pgfpathlineto{\pgfqpoint{0.793996in}{0.703122in}}%
\pgfpathclose%
\pgfusepath{stroke,fill}%
\end{pgfscope}%
\begin{pgfscope}%
\pgfpathrectangle{\pgfqpoint{0.793996in}{0.443060in}}{\pgfqpoint{2.041610in}{2.577607in}}%
\pgfusepath{clip}%
\pgfsetbuttcap%
\pgfsetroundjoin%
\definecolor{currentfill}{rgb}{0.706185,0.751573,0.826368}%
\pgfsetfillcolor{currentfill}%
\pgfsetlinewidth{0.602250pt}%
\definecolor{currentstroke}{rgb}{0.296471,0.296471,0.296471}%
\pgfsetstrokecolor{currentstroke}%
\pgfsetdash{}{0pt}%
\pgfpathmoveto{\pgfqpoint{0.793996in}{0.705423in}}%
\pgfpathlineto{\pgfqpoint{0.793996in}{0.705423in}}%
\pgfpathlineto{\pgfqpoint{0.793996in}{0.696217in}}%
\pgfpathlineto{\pgfqpoint{0.793996in}{0.696217in}}%
\pgfpathlineto{\pgfqpoint{0.793996in}{0.705423in}}%
\pgfpathclose%
\pgfusepath{stroke,fill}%
\end{pgfscope}%
\begin{pgfscope}%
\pgfpathrectangle{\pgfqpoint{0.793996in}{0.443060in}}{\pgfqpoint{2.041610in}{2.577607in}}%
\pgfusepath{clip}%
\pgfsetbuttcap%
\pgfsetroundjoin%
\definecolor{currentfill}{rgb}{0.643221,0.700246,0.793900}%
\pgfsetfillcolor{currentfill}%
\pgfsetlinewidth{0.602250pt}%
\definecolor{currentstroke}{rgb}{0.296471,0.296471,0.296471}%
\pgfsetstrokecolor{currentstroke}%
\pgfsetdash{}{0pt}%
\pgfpathmoveto{\pgfqpoint{0.793996in}{0.710026in}}%
\pgfpathlineto{\pgfqpoint{0.793996in}{0.710026in}}%
\pgfpathlineto{\pgfqpoint{0.793996in}{0.691615in}}%
\pgfpathlineto{\pgfqpoint{0.793996in}{0.691615in}}%
\pgfpathlineto{\pgfqpoint{0.793996in}{0.710026in}}%
\pgfpathclose%
\pgfusepath{stroke,fill}%
\end{pgfscope}%
\begin{pgfscope}%
\pgfpathrectangle{\pgfqpoint{0.793996in}{0.443060in}}{\pgfqpoint{2.041610in}{2.577607in}}%
\pgfusepath{clip}%
\pgfsetbuttcap%
\pgfsetroundjoin%
\definecolor{currentfill}{rgb}{0.566266,0.637515,0.754216}%
\pgfsetfillcolor{currentfill}%
\pgfsetlinewidth{0.602250pt}%
\definecolor{currentstroke}{rgb}{0.296471,0.296471,0.296471}%
\pgfsetstrokecolor{currentstroke}%
\pgfsetdash{}{0pt}%
\pgfpathmoveto{\pgfqpoint{0.793996in}{0.719232in}}%
\pgfpathlineto{\pgfqpoint{0.793996in}{0.719232in}}%
\pgfpathlineto{\pgfqpoint{0.793996in}{0.682409in}}%
\pgfpathlineto{\pgfqpoint{0.793996in}{0.682409in}}%
\pgfpathlineto{\pgfqpoint{0.793996in}{0.719232in}}%
\pgfpathclose%
\pgfusepath{stroke,fill}%
\end{pgfscope}%
\begin{pgfscope}%
\pgfpathrectangle{\pgfqpoint{0.793996in}{0.443060in}}{\pgfqpoint{2.041610in}{2.577607in}}%
\pgfusepath{clip}%
\pgfsetbuttcap%
\pgfsetroundjoin%
\definecolor{currentfill}{rgb}{0.468322,0.557674,0.703709}%
\pgfsetfillcolor{currentfill}%
\pgfsetlinewidth{0.602250pt}%
\definecolor{currentstroke}{rgb}{0.296471,0.296471,0.296471}%
\pgfsetstrokecolor{currentstroke}%
\pgfsetdash{}{0pt}%
\pgfpathmoveto{\pgfqpoint{0.793996in}{0.737643in}}%
\pgfpathlineto{\pgfqpoint{0.793996in}{0.737643in}}%
\pgfpathlineto{\pgfqpoint{0.793996in}{0.663997in}}%
\pgfpathlineto{\pgfqpoint{0.793996in}{0.663997in}}%
\pgfpathlineto{\pgfqpoint{0.793996in}{0.737643in}}%
\pgfpathclose%
\pgfusepath{stroke,fill}%
\end{pgfscope}%
\begin{pgfscope}%
\pgfpathrectangle{\pgfqpoint{0.793996in}{0.443060in}}{\pgfqpoint{2.041610in}{2.577607in}}%
\pgfusepath{clip}%
\pgfsetbuttcap%
\pgfsetroundjoin%
\definecolor{currentfill}{rgb}{0.347059,0.458824,0.641176}%
\pgfsetfillcolor{currentfill}%
\pgfsetlinewidth{0.602250pt}%
\definecolor{currentstroke}{rgb}{0.296471,0.296471,0.296471}%
\pgfsetstrokecolor{currentstroke}%
\pgfsetdash{}{0pt}%
\pgfpathmoveto{\pgfqpoint{0.793996in}{0.774466in}}%
\pgfpathlineto{\pgfqpoint{1.035614in}{0.774466in}}%
\pgfpathlineto{\pgfqpoint{1.035614in}{0.627174in}}%
\pgfpathlineto{\pgfqpoint{0.793996in}{0.627174in}}%
\pgfpathlineto{\pgfqpoint{0.793996in}{0.774466in}}%
\pgfpathclose%
\pgfusepath{stroke,fill}%
\end{pgfscope}%
\begin{pgfscope}%
\pgfpathrectangle{\pgfqpoint{0.793996in}{0.443060in}}{\pgfqpoint{2.041610in}{2.577607in}}%
\pgfusepath{clip}%
\pgfsetbuttcap%
\pgfsetroundjoin%
\definecolor{currentfill}{rgb}{0.468322,0.557674,0.703709}%
\pgfsetfillcolor{currentfill}%
\pgfsetlinewidth{0.602250pt}%
\definecolor{currentstroke}{rgb}{0.296471,0.296471,0.296471}%
\pgfsetstrokecolor{currentstroke}%
\pgfsetdash{}{0pt}%
\pgfpathmoveto{\pgfqpoint{1.035614in}{0.737643in}}%
\pgfpathlineto{\pgfqpoint{1.380746in}{0.737643in}}%
\pgfpathlineto{\pgfqpoint{1.380746in}{0.663997in}}%
\pgfpathlineto{\pgfqpoint{1.035614in}{0.663997in}}%
\pgfpathlineto{\pgfqpoint{1.035614in}{0.737643in}}%
\pgfpathclose%
\pgfusepath{stroke,fill}%
\end{pgfscope}%
\begin{pgfscope}%
\pgfpathrectangle{\pgfqpoint{0.793996in}{0.443060in}}{\pgfqpoint{2.041610in}{2.577607in}}%
\pgfusepath{clip}%
\pgfsetbuttcap%
\pgfsetroundjoin%
\definecolor{currentfill}{rgb}{0.566266,0.637515,0.754216}%
\pgfsetfillcolor{currentfill}%
\pgfsetlinewidth{0.602250pt}%
\definecolor{currentstroke}{rgb}{0.296471,0.296471,0.296471}%
\pgfsetstrokecolor{currentstroke}%
\pgfsetdash{}{0pt}%
\pgfpathmoveto{\pgfqpoint{1.380746in}{0.719232in}}%
\pgfpathlineto{\pgfqpoint{1.758339in}{0.719232in}}%
\pgfpathlineto{\pgfqpoint{1.758339in}{0.682409in}}%
\pgfpathlineto{\pgfqpoint{1.380746in}{0.682409in}}%
\pgfpathlineto{\pgfqpoint{1.380746in}{0.719232in}}%
\pgfpathclose%
\pgfusepath{stroke,fill}%
\end{pgfscope}%
\begin{pgfscope}%
\pgfpathrectangle{\pgfqpoint{0.793996in}{0.443060in}}{\pgfqpoint{2.041610in}{2.577607in}}%
\pgfusepath{clip}%
\pgfsetbuttcap%
\pgfsetroundjoin%
\definecolor{currentfill}{rgb}{0.643221,0.700246,0.793900}%
\pgfsetfillcolor{currentfill}%
\pgfsetlinewidth{0.602250pt}%
\definecolor{currentstroke}{rgb}{0.296471,0.296471,0.296471}%
\pgfsetstrokecolor{currentstroke}%
\pgfsetdash{}{0pt}%
\pgfpathmoveto{\pgfqpoint{1.758339in}{0.710026in}}%
\pgfpathlineto{\pgfqpoint{1.917193in}{0.710026in}}%
\pgfpathlineto{\pgfqpoint{1.917193in}{0.691615in}}%
\pgfpathlineto{\pgfqpoint{1.758339in}{0.691615in}}%
\pgfpathlineto{\pgfqpoint{1.758339in}{0.710026in}}%
\pgfpathclose%
\pgfusepath{stroke,fill}%
\end{pgfscope}%
\begin{pgfscope}%
\pgfpathrectangle{\pgfqpoint{0.793996in}{0.443060in}}{\pgfqpoint{2.041610in}{2.577607in}}%
\pgfusepath{clip}%
\pgfsetbuttcap%
\pgfsetroundjoin%
\definecolor{currentfill}{rgb}{0.706185,0.751573,0.826368}%
\pgfsetfillcolor{currentfill}%
\pgfsetlinewidth{0.602250pt}%
\definecolor{currentstroke}{rgb}{0.296471,0.296471,0.296471}%
\pgfsetstrokecolor{currentstroke}%
\pgfsetdash{}{0pt}%
\pgfpathmoveto{\pgfqpoint{1.917193in}{0.705423in}}%
\pgfpathlineto{\pgfqpoint{2.046327in}{0.705423in}}%
\pgfpathlineto{\pgfqpoint{2.046327in}{0.696217in}}%
\pgfpathlineto{\pgfqpoint{1.917193in}{0.696217in}}%
\pgfpathlineto{\pgfqpoint{1.917193in}{0.705423in}}%
\pgfpathclose%
\pgfusepath{stroke,fill}%
\end{pgfscope}%
\begin{pgfscope}%
\pgfpathrectangle{\pgfqpoint{0.793996in}{0.443060in}}{\pgfqpoint{2.041610in}{2.577607in}}%
\pgfusepath{clip}%
\pgfsetbuttcap%
\pgfsetroundjoin%
\definecolor{currentfill}{rgb}{0.755157,0.791493,0.851622}%
\pgfsetfillcolor{currentfill}%
\pgfsetlinewidth{0.602250pt}%
\definecolor{currentstroke}{rgb}{0.296471,0.296471,0.296471}%
\pgfsetstrokecolor{currentstroke}%
\pgfsetdash{}{0pt}%
\pgfpathmoveto{\pgfqpoint{2.046327in}{0.703122in}}%
\pgfpathlineto{\pgfqpoint{2.129845in}{0.703122in}}%
\pgfpathlineto{\pgfqpoint{2.129845in}{0.698519in}}%
\pgfpathlineto{\pgfqpoint{2.046327in}{0.698519in}}%
\pgfpathlineto{\pgfqpoint{2.046327in}{0.703122in}}%
\pgfpathclose%
\pgfusepath{stroke,fill}%
\end{pgfscope}%
\begin{pgfscope}%
\pgfpathrectangle{\pgfqpoint{0.793996in}{0.443060in}}{\pgfqpoint{2.041610in}{2.577607in}}%
\pgfusepath{clip}%
\pgfsetbuttcap%
\pgfsetroundjoin%
\definecolor{currentfill}{rgb}{0.792469,0.821908,0.870863}%
\pgfsetfillcolor{currentfill}%
\pgfsetlinewidth{0.602250pt}%
\definecolor{currentstroke}{rgb}{0.296471,0.296471,0.296471}%
\pgfsetstrokecolor{currentstroke}%
\pgfsetdash{}{0pt}%
\pgfpathmoveto{\pgfqpoint{2.129845in}{0.701971in}}%
\pgfpathlineto{\pgfqpoint{2.216087in}{0.701971in}}%
\pgfpathlineto{\pgfqpoint{2.216087in}{0.699670in}}%
\pgfpathlineto{\pgfqpoint{2.129845in}{0.699670in}}%
\pgfpathlineto{\pgfqpoint{2.129845in}{0.701971in}}%
\pgfpathclose%
\pgfusepath{stroke,fill}%
\end{pgfscope}%
\begin{pgfscope}%
\pgfpathrectangle{\pgfqpoint{0.793996in}{0.443060in}}{\pgfqpoint{2.041610in}{2.577607in}}%
\pgfusepath{clip}%
\pgfsetbuttcap%
\pgfsetroundjoin%
\definecolor{currentfill}{rgb}{0.825117,0.848522,0.887698}%
\pgfsetfillcolor{currentfill}%
\pgfsetlinewidth{0.602250pt}%
\definecolor{currentstroke}{rgb}{0.296471,0.296471,0.296471}%
\pgfsetstrokecolor{currentstroke}%
\pgfsetdash{}{0pt}%
\pgfpathmoveto{\pgfqpoint{2.216087in}{0.701396in}}%
\pgfpathlineto{\pgfqpoint{2.226977in}{0.701396in}}%
\pgfpathlineto{\pgfqpoint{2.226977in}{0.700245in}}%
\pgfpathlineto{\pgfqpoint{2.216087in}{0.700245in}}%
\pgfpathlineto{\pgfqpoint{2.216087in}{0.701396in}}%
\pgfpathclose%
\pgfusepath{stroke,fill}%
\end{pgfscope}%
\begin{pgfscope}%
\pgfpathrectangle{\pgfqpoint{0.793996in}{0.443060in}}{\pgfqpoint{2.041610in}{2.577607in}}%
\pgfusepath{clip}%
\pgfsetbuttcap%
\pgfsetroundjoin%
\definecolor{currentfill}{rgb}{0.848437,0.867532,0.899724}%
\pgfsetfillcolor{currentfill}%
\pgfsetlinewidth{0.602250pt}%
\definecolor{currentstroke}{rgb}{0.296471,0.296471,0.296471}%
\pgfsetstrokecolor{currentstroke}%
\pgfsetdash{}{0pt}%
\pgfpathmoveto{\pgfqpoint{2.226977in}{0.701108in}}%
\pgfpathlineto{\pgfqpoint{2.285813in}{0.701108in}}%
\pgfpathlineto{\pgfqpoint{2.285813in}{0.700533in}}%
\pgfpathlineto{\pgfqpoint{2.226977in}{0.700533in}}%
\pgfpathlineto{\pgfqpoint{2.226977in}{0.701108in}}%
\pgfpathclose%
\pgfusepath{stroke,fill}%
\end{pgfscope}%
\begin{pgfscope}%
\pgfpathrectangle{\pgfqpoint{0.793996in}{0.443060in}}{\pgfqpoint{2.041610in}{2.577607in}}%
\pgfusepath{clip}%
\pgfsetbuttcap%
\pgfsetroundjoin%
\pgfsetlinewidth{0.803000pt}%
\definecolor{currentstroke}{rgb}{0.450000,0.450000,0.450000}%
\pgfsetstrokecolor{currentstroke}%
\pgfsetdash{}{0pt}%
\pgfpathmoveto{\pgfqpoint{0.000000in}{-0.034722in}}%
\pgfpathcurveto{\pgfqpoint{0.009208in}{-0.034722in}}{\pgfqpoint{0.018041in}{-0.031064in}}{\pgfqpoint{0.024552in}{-0.024552in}}%
\pgfpathcurveto{\pgfqpoint{0.031064in}{-0.018041in}}{\pgfqpoint{0.034722in}{-0.009208in}}{\pgfqpoint{0.034722in}{0.000000in}}%
\pgfpathcurveto{\pgfqpoint{0.034722in}{0.009208in}}{\pgfqpoint{0.031064in}{0.018041in}}{\pgfqpoint{0.024552in}{0.024552in}}%
\pgfpathcurveto{\pgfqpoint{0.018041in}{0.031064in}}{\pgfqpoint{0.009208in}{0.034722in}}{\pgfqpoint{0.000000in}{0.034722in}}%
\pgfpathcurveto{\pgfqpoint{-0.009208in}{0.034722in}}{\pgfqpoint{-0.018041in}{0.031064in}}{\pgfqpoint{-0.024552in}{0.024552in}}%
\pgfpathcurveto{\pgfqpoint{-0.031064in}{0.018041in}}{\pgfqpoint{-0.034722in}{0.009208in}}{\pgfqpoint{-0.034722in}{0.000000in}}%
\pgfpathcurveto{\pgfqpoint{-0.034722in}{-0.009208in}}{\pgfqpoint{-0.031064in}{-0.018041in}}{\pgfqpoint{-0.024552in}{-0.024552in}}%
\pgfpathcurveto{\pgfqpoint{-0.018041in}{-0.031064in}}{\pgfqpoint{-0.009208in}{-0.034722in}}{\pgfqpoint{0.000000in}{-0.034722in}}%
\pgfusepath{stroke}%
\end{pgfscope}%
\begin{pgfscope}%
\pgfpathrectangle{\pgfqpoint{0.793996in}{0.443060in}}{\pgfqpoint{2.041610in}{2.577607in}}%
\pgfusepath{clip}%
\pgfsetbuttcap%
\pgfsetroundjoin%
\definecolor{currentfill}{rgb}{0.919097,0.862812,0.832112}%
\pgfsetfillcolor{currentfill}%
\pgfsetlinewidth{0.602250pt}%
\definecolor{currentstroke}{rgb}{0.296471,0.296471,0.296471}%
\pgfsetstrokecolor{currentstroke}%
\pgfsetdash{}{0pt}%
\pgfpathmoveto{\pgfqpoint{0.793996in}{0.554104in}}%
\pgfpathlineto{\pgfqpoint{0.793996in}{0.554104in}}%
\pgfpathlineto{\pgfqpoint{0.793996in}{0.552953in}}%
\pgfpathlineto{\pgfqpoint{0.793996in}{0.552953in}}%
\pgfpathlineto{\pgfqpoint{0.793996in}{0.554104in}}%
\pgfpathclose%
\pgfusepath{stroke,fill}%
\end{pgfscope}%
\begin{pgfscope}%
\pgfpathrectangle{\pgfqpoint{0.793996in}{0.443060in}}{\pgfqpoint{2.041610in}{2.577607in}}%
\pgfusepath{clip}%
\pgfsetbuttcap%
\pgfsetroundjoin%
\definecolor{currentfill}{rgb}{0.910863,0.840546,0.801899}%
\pgfsetfillcolor{currentfill}%
\pgfsetlinewidth{0.602250pt}%
\definecolor{currentstroke}{rgb}{0.296471,0.296471,0.296471}%
\pgfsetstrokecolor{currentstroke}%
\pgfsetdash{}{0pt}%
\pgfpathmoveto{\pgfqpoint{0.793996in}{0.554679in}}%
\pgfpathlineto{\pgfqpoint{0.793996in}{0.554679in}}%
\pgfpathlineto{\pgfqpoint{0.793996in}{0.552378in}}%
\pgfpathlineto{\pgfqpoint{0.793996in}{0.552378in}}%
\pgfpathlineto{\pgfqpoint{0.793996in}{0.554679in}}%
\pgfpathclose%
\pgfusepath{stroke,fill}%
\end{pgfscope}%
\begin{pgfscope}%
\pgfpathrectangle{\pgfqpoint{0.793996in}{0.443060in}}{\pgfqpoint{2.041610in}{2.577607in}}%
\pgfusepath{clip}%
\pgfsetbuttcap%
\pgfsetroundjoin%
\definecolor{currentfill}{rgb}{0.901453,0.815098,0.767370}%
\pgfsetfillcolor{currentfill}%
\pgfsetlinewidth{0.602250pt}%
\definecolor{currentstroke}{rgb}{0.296471,0.296471,0.296471}%
\pgfsetstrokecolor{currentstroke}%
\pgfsetdash{}{0pt}%
\pgfpathmoveto{\pgfqpoint{0.793996in}{0.555830in}}%
\pgfpathlineto{\pgfqpoint{0.793996in}{0.555830in}}%
\pgfpathlineto{\pgfqpoint{0.793996in}{0.551227in}}%
\pgfpathlineto{\pgfqpoint{0.793996in}{0.551227in}}%
\pgfpathlineto{\pgfqpoint{0.793996in}{0.555830in}}%
\pgfpathclose%
\pgfusepath{stroke,fill}%
\end{pgfscope}%
\begin{pgfscope}%
\pgfpathrectangle{\pgfqpoint{0.793996in}{0.443060in}}{\pgfqpoint{2.041610in}{2.577607in}}%
\pgfusepath{clip}%
\pgfsetbuttcap%
\pgfsetroundjoin%
\definecolor{currentfill}{rgb}{0.889102,0.781698,0.722050}%
\pgfsetfillcolor{currentfill}%
\pgfsetlinewidth{0.602250pt}%
\definecolor{currentstroke}{rgb}{0.296471,0.296471,0.296471}%
\pgfsetstrokecolor{currentstroke}%
\pgfsetdash{}{0pt}%
\pgfpathmoveto{\pgfqpoint{0.793996in}{0.558131in}}%
\pgfpathlineto{\pgfqpoint{0.793996in}{0.558131in}}%
\pgfpathlineto{\pgfqpoint{0.793996in}{0.548926in}}%
\pgfpathlineto{\pgfqpoint{0.793996in}{0.548926in}}%
\pgfpathlineto{\pgfqpoint{0.793996in}{0.558131in}}%
\pgfpathclose%
\pgfusepath{stroke,fill}%
\end{pgfscope}%
\begin{pgfscope}%
\pgfpathrectangle{\pgfqpoint{0.793996in}{0.443060in}}{\pgfqpoint{2.041610in}{2.577607in}}%
\pgfusepath{clip}%
\pgfsetbuttcap%
\pgfsetroundjoin%
\definecolor{currentfill}{rgb}{0.873223,0.738755,0.663782}%
\pgfsetfillcolor{currentfill}%
\pgfsetlinewidth{0.602250pt}%
\definecolor{currentstroke}{rgb}{0.296471,0.296471,0.296471}%
\pgfsetstrokecolor{currentstroke}%
\pgfsetdash{}{0pt}%
\pgfpathmoveto{\pgfqpoint{0.793996in}{0.562734in}}%
\pgfpathlineto{\pgfqpoint{0.793996in}{0.562734in}}%
\pgfpathlineto{\pgfqpoint{0.793996in}{0.544323in}}%
\pgfpathlineto{\pgfqpoint{0.793996in}{0.544323in}}%
\pgfpathlineto{\pgfqpoint{0.793996in}{0.562734in}}%
\pgfpathclose%
\pgfusepath{stroke,fill}%
\end{pgfscope}%
\begin{pgfscope}%
\pgfpathrectangle{\pgfqpoint{0.793996in}{0.443060in}}{\pgfqpoint{2.041610in}{2.577607in}}%
\pgfusepath{clip}%
\pgfsetbuttcap%
\pgfsetroundjoin%
\definecolor{currentfill}{rgb}{0.853814,0.686269,0.592565}%
\pgfsetfillcolor{currentfill}%
\pgfsetlinewidth{0.602250pt}%
\definecolor{currentstroke}{rgb}{0.296471,0.296471,0.296471}%
\pgfsetstrokecolor{currentstroke}%
\pgfsetdash{}{0pt}%
\pgfpathmoveto{\pgfqpoint{0.793996in}{0.571940in}}%
\pgfpathlineto{\pgfqpoint{0.793996in}{0.571940in}}%
\pgfpathlineto{\pgfqpoint{0.793996in}{0.535117in}}%
\pgfpathlineto{\pgfqpoint{0.793996in}{0.535117in}}%
\pgfpathlineto{\pgfqpoint{0.793996in}{0.571940in}}%
\pgfpathclose%
\pgfusepath{stroke,fill}%
\end{pgfscope}%
\begin{pgfscope}%
\pgfpathrectangle{\pgfqpoint{0.793996in}{0.443060in}}{\pgfqpoint{2.041610in}{2.577607in}}%
\pgfusepath{clip}%
\pgfsetbuttcap%
\pgfsetroundjoin%
\definecolor{currentfill}{rgb}{0.829112,0.619469,0.501926}%
\pgfsetfillcolor{currentfill}%
\pgfsetlinewidth{0.602250pt}%
\definecolor{currentstroke}{rgb}{0.296471,0.296471,0.296471}%
\pgfsetstrokecolor{currentstroke}%
\pgfsetdash{}{0pt}%
\pgfpathmoveto{\pgfqpoint{0.793996in}{0.590351in}}%
\pgfpathlineto{\pgfqpoint{0.793996in}{0.590351in}}%
\pgfpathlineto{\pgfqpoint{0.793996in}{0.516706in}}%
\pgfpathlineto{\pgfqpoint{0.793996in}{0.516706in}}%
\pgfpathlineto{\pgfqpoint{0.793996in}{0.590351in}}%
\pgfpathclose%
\pgfusepath{stroke,fill}%
\end{pgfscope}%
\begin{pgfscope}%
\pgfpathrectangle{\pgfqpoint{0.793996in}{0.443060in}}{\pgfqpoint{2.041610in}{2.577607in}}%
\pgfusepath{clip}%
\pgfsetbuttcap%
\pgfsetroundjoin%
\definecolor{currentfill}{rgb}{0.798529,0.536765,0.389706}%
\pgfsetfillcolor{currentfill}%
\pgfsetlinewidth{0.602250pt}%
\definecolor{currentstroke}{rgb}{0.296471,0.296471,0.296471}%
\pgfsetstrokecolor{currentstroke}%
\pgfsetdash{}{0pt}%
\pgfpathmoveto{\pgfqpoint{0.793996in}{0.627174in}}%
\pgfpathlineto{\pgfqpoint{1.210535in}{0.627174in}}%
\pgfpathlineto{\pgfqpoint{1.210535in}{0.479883in}}%
\pgfpathlineto{\pgfqpoint{0.793996in}{0.479883in}}%
\pgfpathlineto{\pgfqpoint{0.793996in}{0.627174in}}%
\pgfpathclose%
\pgfusepath{stroke,fill}%
\end{pgfscope}%
\begin{pgfscope}%
\pgfpathrectangle{\pgfqpoint{0.793996in}{0.443060in}}{\pgfqpoint{2.041610in}{2.577607in}}%
\pgfusepath{clip}%
\pgfsetbuttcap%
\pgfsetroundjoin%
\definecolor{currentfill}{rgb}{0.829112,0.619469,0.501926}%
\pgfsetfillcolor{currentfill}%
\pgfsetlinewidth{0.602250pt}%
\definecolor{currentstroke}{rgb}{0.296471,0.296471,0.296471}%
\pgfsetstrokecolor{currentstroke}%
\pgfsetdash{}{0pt}%
\pgfpathmoveto{\pgfqpoint{1.210535in}{0.590351in}}%
\pgfpathlineto{\pgfqpoint{1.724513in}{0.590351in}}%
\pgfpathlineto{\pgfqpoint{1.724513in}{0.516706in}}%
\pgfpathlineto{\pgfqpoint{1.210535in}{0.516706in}}%
\pgfpathlineto{\pgfqpoint{1.210535in}{0.590351in}}%
\pgfpathclose%
\pgfusepath{stroke,fill}%
\end{pgfscope}%
\begin{pgfscope}%
\pgfpathrectangle{\pgfqpoint{0.793996in}{0.443060in}}{\pgfqpoint{2.041610in}{2.577607in}}%
\pgfusepath{clip}%
\pgfsetbuttcap%
\pgfsetroundjoin%
\definecolor{currentfill}{rgb}{0.853814,0.686269,0.592565}%
\pgfsetfillcolor{currentfill}%
\pgfsetlinewidth{0.602250pt}%
\definecolor{currentstroke}{rgb}{0.296471,0.296471,0.296471}%
\pgfsetstrokecolor{currentstroke}%
\pgfsetdash{}{0pt}%
\pgfpathmoveto{\pgfqpoint{1.724513in}{0.571940in}}%
\pgfpathlineto{\pgfqpoint{1.909842in}{0.571940in}}%
\pgfpathlineto{\pgfqpoint{1.909842in}{0.535117in}}%
\pgfpathlineto{\pgfqpoint{1.724513in}{0.535117in}}%
\pgfpathlineto{\pgfqpoint{1.724513in}{0.571940in}}%
\pgfpathclose%
\pgfusepath{stroke,fill}%
\end{pgfscope}%
\begin{pgfscope}%
\pgfpathrectangle{\pgfqpoint{0.793996in}{0.443060in}}{\pgfqpoint{2.041610in}{2.577607in}}%
\pgfusepath{clip}%
\pgfsetbuttcap%
\pgfsetroundjoin%
\definecolor{currentfill}{rgb}{0.873223,0.738755,0.663782}%
\pgfsetfillcolor{currentfill}%
\pgfsetlinewidth{0.602250pt}%
\definecolor{currentstroke}{rgb}{0.296471,0.296471,0.296471}%
\pgfsetstrokecolor{currentstroke}%
\pgfsetdash{}{0pt}%
\pgfpathmoveto{\pgfqpoint{1.909842in}{0.562734in}}%
\pgfpathlineto{\pgfqpoint{2.024756in}{0.562734in}}%
\pgfpathlineto{\pgfqpoint{2.024756in}{0.544323in}}%
\pgfpathlineto{\pgfqpoint{1.909842in}{0.544323in}}%
\pgfpathlineto{\pgfqpoint{1.909842in}{0.562734in}}%
\pgfpathclose%
\pgfusepath{stroke,fill}%
\end{pgfscope}%
\begin{pgfscope}%
\pgfpathrectangle{\pgfqpoint{0.793996in}{0.443060in}}{\pgfqpoint{2.041610in}{2.577607in}}%
\pgfusepath{clip}%
\pgfsetbuttcap%
\pgfsetroundjoin%
\definecolor{currentfill}{rgb}{0.889102,0.781698,0.722050}%
\pgfsetfillcolor{currentfill}%
\pgfsetlinewidth{0.602250pt}%
\definecolor{currentstroke}{rgb}{0.296471,0.296471,0.296471}%
\pgfsetstrokecolor{currentstroke}%
\pgfsetdash{}{0pt}%
\pgfpathmoveto{\pgfqpoint{2.024756in}{0.558131in}}%
\pgfpathlineto{\pgfqpoint{2.108214in}{0.558131in}}%
\pgfpathlineto{\pgfqpoint{2.108214in}{0.548926in}}%
\pgfpathlineto{\pgfqpoint{2.024756in}{0.548926in}}%
\pgfpathlineto{\pgfqpoint{2.024756in}{0.558131in}}%
\pgfpathclose%
\pgfusepath{stroke,fill}%
\end{pgfscope}%
\begin{pgfscope}%
\pgfpathrectangle{\pgfqpoint{0.793996in}{0.443060in}}{\pgfqpoint{2.041610in}{2.577607in}}%
\pgfusepath{clip}%
\pgfsetbuttcap%
\pgfsetroundjoin%
\definecolor{currentfill}{rgb}{0.901453,0.815098,0.767370}%
\pgfsetfillcolor{currentfill}%
\pgfsetlinewidth{0.602250pt}%
\definecolor{currentstroke}{rgb}{0.296471,0.296471,0.296471}%
\pgfsetstrokecolor{currentstroke}%
\pgfsetdash{}{0pt}%
\pgfpathmoveto{\pgfqpoint{2.108214in}{0.555830in}}%
\pgfpathlineto{\pgfqpoint{2.216118in}{0.555830in}}%
\pgfpathlineto{\pgfqpoint{2.216118in}{0.551227in}}%
\pgfpathlineto{\pgfqpoint{2.108214in}{0.551227in}}%
\pgfpathlineto{\pgfqpoint{2.108214in}{0.555830in}}%
\pgfpathclose%
\pgfusepath{stroke,fill}%
\end{pgfscope}%
\begin{pgfscope}%
\pgfpathrectangle{\pgfqpoint{0.793996in}{0.443060in}}{\pgfqpoint{2.041610in}{2.577607in}}%
\pgfusepath{clip}%
\pgfsetbuttcap%
\pgfsetroundjoin%
\definecolor{currentfill}{rgb}{0.910863,0.840546,0.801899}%
\pgfsetfillcolor{currentfill}%
\pgfsetlinewidth{0.602250pt}%
\definecolor{currentstroke}{rgb}{0.296471,0.296471,0.296471}%
\pgfsetstrokecolor{currentstroke}%
\pgfsetdash{}{0pt}%
\pgfpathmoveto{\pgfqpoint{2.216118in}{0.554679in}}%
\pgfpathlineto{\pgfqpoint{2.227100in}{0.554679in}}%
\pgfpathlineto{\pgfqpoint{2.227100in}{0.552378in}}%
\pgfpathlineto{\pgfqpoint{2.216118in}{0.552378in}}%
\pgfpathlineto{\pgfqpoint{2.216118in}{0.554679in}}%
\pgfpathclose%
\pgfusepath{stroke,fill}%
\end{pgfscope}%
\begin{pgfscope}%
\pgfpathrectangle{\pgfqpoint{0.793996in}{0.443060in}}{\pgfqpoint{2.041610in}{2.577607in}}%
\pgfusepath{clip}%
\pgfsetbuttcap%
\pgfsetroundjoin%
\definecolor{currentfill}{rgb}{0.919097,0.862812,0.832112}%
\pgfsetfillcolor{currentfill}%
\pgfsetlinewidth{0.602250pt}%
\definecolor{currentstroke}{rgb}{0.296471,0.296471,0.296471}%
\pgfsetstrokecolor{currentstroke}%
\pgfsetdash{}{0pt}%
\pgfpathmoveto{\pgfqpoint{2.227100in}{0.554104in}}%
\pgfpathlineto{\pgfqpoint{2.285813in}{0.554104in}}%
\pgfpathlineto{\pgfqpoint{2.285813in}{0.552953in}}%
\pgfpathlineto{\pgfqpoint{2.227100in}{0.552953in}}%
\pgfpathlineto{\pgfqpoint{2.227100in}{0.554104in}}%
\pgfpathclose%
\pgfusepath{stroke,fill}%
\end{pgfscope}%
\begin{pgfscope}%
\pgfpathrectangle{\pgfqpoint{0.793996in}{0.443060in}}{\pgfqpoint{2.041610in}{2.577607in}}%
\pgfusepath{clip}%
\pgfsetbuttcap%
\pgfsetroundjoin%
\pgfsetlinewidth{0.803000pt}%
\definecolor{currentstroke}{rgb}{0.450000,0.450000,0.450000}%
\pgfsetstrokecolor{currentstroke}%
\pgfsetdash{}{0pt}%
\pgfpathmoveto{\pgfqpoint{0.000000in}{-0.034722in}}%
\pgfpathcurveto{\pgfqpoint{0.009208in}{-0.034722in}}{\pgfqpoint{0.018041in}{-0.031064in}}{\pgfqpoint{0.024552in}{-0.024552in}}%
\pgfpathcurveto{\pgfqpoint{0.031064in}{-0.018041in}}{\pgfqpoint{0.034722in}{-0.009208in}}{\pgfqpoint{0.034722in}{0.000000in}}%
\pgfpathcurveto{\pgfqpoint{0.034722in}{0.009208in}}{\pgfqpoint{0.031064in}{0.018041in}}{\pgfqpoint{0.024552in}{0.024552in}}%
\pgfpathcurveto{\pgfqpoint{0.018041in}{0.031064in}}{\pgfqpoint{0.009208in}{0.034722in}}{\pgfqpoint{0.000000in}{0.034722in}}%
\pgfpathcurveto{\pgfqpoint{-0.009208in}{0.034722in}}{\pgfqpoint{-0.018041in}{0.031064in}}{\pgfqpoint{-0.024552in}{0.024552in}}%
\pgfpathcurveto{\pgfqpoint{-0.031064in}{0.018041in}}{\pgfqpoint{-0.034722in}{0.009208in}}{\pgfqpoint{-0.034722in}{0.000000in}}%
\pgfpathcurveto{\pgfqpoint{-0.034722in}{-0.009208in}}{\pgfqpoint{-0.031064in}{-0.018041in}}{\pgfqpoint{-0.024552in}{-0.024552in}}%
\pgfpathcurveto{\pgfqpoint{-0.018041in}{-0.031064in}}{\pgfqpoint{-0.009208in}{-0.034722in}}{\pgfqpoint{0.000000in}{-0.034722in}}%
\pgfusepath{stroke}%
\end{pgfscope}%
\begin{pgfscope}%
\pgfpathrectangle{\pgfqpoint{0.793996in}{0.443060in}}{\pgfqpoint{2.041610in}{2.577607in}}%
\pgfusepath{clip}%
\pgfsetbuttcap%
\pgfsetmiterjoin%
\definecolor{currentfill}{rgb}{0.347059,0.458824,0.641176}%
\pgfsetfillcolor{currentfill}%
\pgfsetlinewidth{0.602250pt}%
\definecolor{currentstroke}{rgb}{0.296471,0.296471,0.296471}%
\pgfsetstrokecolor{currentstroke}%
\pgfsetdash{}{0pt}%
\pgfpathmoveto{\pgfqpoint{0.793996in}{2.836552in}}%
\pgfpathlineto{\pgfqpoint{0.793996in}{2.836552in}}%
\pgfpathlineto{\pgfqpoint{0.793996in}{2.836552in}}%
\pgfpathlineto{\pgfqpoint{0.793996in}{2.836552in}}%
\pgfpathlineto{\pgfqpoint{0.793996in}{2.836552in}}%
\pgfpathclose%
\pgfusepath{stroke,fill}%
\end{pgfscope}%
\begin{pgfscope}%
\pgfpathrectangle{\pgfqpoint{0.793996in}{0.443060in}}{\pgfqpoint{2.041610in}{2.577607in}}%
\pgfusepath{clip}%
\pgfsetbuttcap%
\pgfsetmiterjoin%
\definecolor{currentfill}{rgb}{0.798529,0.536765,0.389706}%
\pgfsetfillcolor{currentfill}%
\pgfsetlinewidth{0.602250pt}%
\definecolor{currentstroke}{rgb}{0.296471,0.296471,0.296471}%
\pgfsetstrokecolor{currentstroke}%
\pgfsetdash{}{0pt}%
\pgfpathmoveto{\pgfqpoint{0.793996in}{2.836552in}}%
\pgfpathlineto{\pgfqpoint{0.793996in}{2.836552in}}%
\pgfpathlineto{\pgfqpoint{0.793996in}{2.836552in}}%
\pgfpathlineto{\pgfqpoint{0.793996in}{2.836552in}}%
\pgfpathlineto{\pgfqpoint{0.793996in}{2.836552in}}%
\pgfpathclose%
\pgfusepath{stroke,fill}%
\end{pgfscope}%
\begin{pgfscope}%
\pgfpathrectangle{\pgfqpoint{0.793996in}{0.443060in}}{\pgfqpoint{2.041610in}{2.577607in}}%
\pgfusepath{clip}%
\pgfsetbuttcap%
\pgfsetroundjoin%
\pgfsetlinewidth{0.752812pt}%
\definecolor{currentstroke}{rgb}{0.296471,0.296471,0.296471}%
\pgfsetstrokecolor{currentstroke}%
\pgfsetdash{}{0pt}%
\pgfpathmoveto{\pgfqpoint{2.537243in}{2.983844in}}%
\pgfpathlineto{\pgfqpoint{2.537243in}{2.836552in}}%
\pgfusepath{stroke}%
\end{pgfscope}%
\begin{pgfscope}%
\pgfpathrectangle{\pgfqpoint{0.793996in}{0.443060in}}{\pgfqpoint{2.041610in}{2.577607in}}%
\pgfusepath{clip}%
\pgfsetbuttcap%
\pgfsetroundjoin%
\pgfsetlinewidth{0.752812pt}%
\definecolor{currentstroke}{rgb}{0.296471,0.296471,0.296471}%
\pgfsetstrokecolor{currentstroke}%
\pgfsetdash{}{0pt}%
\pgfpathmoveto{\pgfqpoint{2.683462in}{2.836552in}}%
\pgfpathlineto{\pgfqpoint{2.683462in}{2.689260in}}%
\pgfusepath{stroke}%
\end{pgfscope}%
\begin{pgfscope}%
\pgfpathrectangle{\pgfqpoint{0.793996in}{0.443060in}}{\pgfqpoint{2.041610in}{2.577607in}}%
\pgfusepath{clip}%
\pgfsetbuttcap%
\pgfsetroundjoin%
\pgfsetlinewidth{0.752812pt}%
\definecolor{currentstroke}{rgb}{0.296471,0.296471,0.296471}%
\pgfsetstrokecolor{currentstroke}%
\pgfsetdash{}{0pt}%
\pgfpathmoveto{\pgfqpoint{1.372466in}{2.615614in}}%
\pgfpathlineto{\pgfqpoint{1.372466in}{2.468322in}}%
\pgfusepath{stroke}%
\end{pgfscope}%
\begin{pgfscope}%
\pgfpathrectangle{\pgfqpoint{0.793996in}{0.443060in}}{\pgfqpoint{2.041610in}{2.577607in}}%
\pgfusepath{clip}%
\pgfsetbuttcap%
\pgfsetroundjoin%
\pgfsetlinewidth{0.752812pt}%
\definecolor{currentstroke}{rgb}{0.296471,0.296471,0.296471}%
\pgfsetstrokecolor{currentstroke}%
\pgfsetdash{}{0pt}%
\pgfpathmoveto{\pgfqpoint{1.936436in}{2.468322in}}%
\pgfpathlineto{\pgfqpoint{1.936436in}{2.321030in}}%
\pgfusepath{stroke}%
\end{pgfscope}%
\begin{pgfscope}%
\pgfpathrectangle{\pgfqpoint{0.793996in}{0.443060in}}{\pgfqpoint{2.041610in}{2.577607in}}%
\pgfusepath{clip}%
\pgfsetbuttcap%
\pgfsetroundjoin%
\pgfsetlinewidth{0.752812pt}%
\definecolor{currentstroke}{rgb}{0.296471,0.296471,0.296471}%
\pgfsetstrokecolor{currentstroke}%
\pgfsetdash{}{0pt}%
\pgfpathmoveto{\pgfqpoint{1.341532in}{2.247385in}}%
\pgfpathlineto{\pgfqpoint{1.341532in}{2.100093in}}%
\pgfusepath{stroke}%
\end{pgfscope}%
\begin{pgfscope}%
\pgfpathrectangle{\pgfqpoint{0.793996in}{0.443060in}}{\pgfqpoint{2.041610in}{2.577607in}}%
\pgfusepath{clip}%
\pgfsetbuttcap%
\pgfsetroundjoin%
\pgfsetlinewidth{0.752812pt}%
\definecolor{currentstroke}{rgb}{0.296471,0.296471,0.296471}%
\pgfsetstrokecolor{currentstroke}%
\pgfsetdash{}{0pt}%
\pgfpathmoveto{\pgfqpoint{1.728546in}{2.100093in}}%
\pgfpathlineto{\pgfqpoint{1.728546in}{1.952801in}}%
\pgfusepath{stroke}%
\end{pgfscope}%
\begin{pgfscope}%
\pgfpathrectangle{\pgfqpoint{0.793996in}{0.443060in}}{\pgfqpoint{2.041610in}{2.577607in}}%
\pgfusepath{clip}%
\pgfsetbuttcap%
\pgfsetroundjoin%
\pgfsetlinewidth{0.752812pt}%
\definecolor{currentstroke}{rgb}{0.296471,0.296471,0.296471}%
\pgfsetstrokecolor{currentstroke}%
\pgfsetdash{}{0pt}%
\pgfpathmoveto{\pgfqpoint{0.793996in}{1.879155in}}%
\pgfpathlineto{\pgfqpoint{0.793996in}{1.731863in}}%
\pgfusepath{stroke}%
\end{pgfscope}%
\begin{pgfscope}%
\pgfpathrectangle{\pgfqpoint{0.793996in}{0.443060in}}{\pgfqpoint{2.041610in}{2.577607in}}%
\pgfusepath{clip}%
\pgfsetbuttcap%
\pgfsetroundjoin%
\pgfsetlinewidth{0.752812pt}%
\definecolor{currentstroke}{rgb}{0.296471,0.296471,0.296471}%
\pgfsetstrokecolor{currentstroke}%
\pgfsetdash{}{0pt}%
\pgfpathmoveto{\pgfqpoint{2.655909in}{1.731863in}}%
\pgfpathlineto{\pgfqpoint{2.655909in}{1.584571in}}%
\pgfusepath{stroke}%
\end{pgfscope}%
\begin{pgfscope}%
\pgfpathrectangle{\pgfqpoint{0.793996in}{0.443060in}}{\pgfqpoint{2.041610in}{2.577607in}}%
\pgfusepath{clip}%
\pgfsetbuttcap%
\pgfsetroundjoin%
\pgfsetlinewidth{0.752812pt}%
\definecolor{currentstroke}{rgb}{0.296471,0.296471,0.296471}%
\pgfsetstrokecolor{currentstroke}%
\pgfsetdash{}{0pt}%
\pgfpathmoveto{\pgfqpoint{0.793996in}{1.510925in}}%
\pgfpathlineto{\pgfqpoint{0.793996in}{1.363634in}}%
\pgfusepath{stroke}%
\end{pgfscope}%
\begin{pgfscope}%
\pgfpathrectangle{\pgfqpoint{0.793996in}{0.443060in}}{\pgfqpoint{2.041610in}{2.577607in}}%
\pgfusepath{clip}%
\pgfsetbuttcap%
\pgfsetroundjoin%
\pgfsetlinewidth{0.752812pt}%
\definecolor{currentstroke}{rgb}{0.296471,0.296471,0.296471}%
\pgfsetstrokecolor{currentstroke}%
\pgfsetdash{}{0pt}%
\pgfpathmoveto{\pgfqpoint{2.039398in}{1.363634in}}%
\pgfpathlineto{\pgfqpoint{2.039398in}{1.216342in}}%
\pgfusepath{stroke}%
\end{pgfscope}%
\begin{pgfscope}%
\pgfpathrectangle{\pgfqpoint{0.793996in}{0.443060in}}{\pgfqpoint{2.041610in}{2.577607in}}%
\pgfusepath{clip}%
\pgfsetbuttcap%
\pgfsetroundjoin%
\pgfsetlinewidth{0.752812pt}%
\definecolor{currentstroke}{rgb}{0.296471,0.296471,0.296471}%
\pgfsetstrokecolor{currentstroke}%
\pgfsetdash{}{0pt}%
\pgfpathmoveto{\pgfqpoint{0.793996in}{1.142696in}}%
\pgfpathlineto{\pgfqpoint{0.793996in}{0.995404in}}%
\pgfusepath{stroke}%
\end{pgfscope}%
\begin{pgfscope}%
\pgfpathrectangle{\pgfqpoint{0.793996in}{0.443060in}}{\pgfqpoint{2.041610in}{2.577607in}}%
\pgfusepath{clip}%
\pgfsetbuttcap%
\pgfsetroundjoin%
\pgfsetlinewidth{0.752812pt}%
\definecolor{currentstroke}{rgb}{0.296471,0.296471,0.296471}%
\pgfsetstrokecolor{currentstroke}%
\pgfsetdash{}{0pt}%
\pgfpathmoveto{\pgfqpoint{2.129429in}{0.995404in}}%
\pgfpathlineto{\pgfqpoint{2.129429in}{0.848112in}}%
\pgfusepath{stroke}%
\end{pgfscope}%
\begin{pgfscope}%
\pgfpathrectangle{\pgfqpoint{0.793996in}{0.443060in}}{\pgfqpoint{2.041610in}{2.577607in}}%
\pgfusepath{clip}%
\pgfsetbuttcap%
\pgfsetroundjoin%
\pgfsetlinewidth{0.752812pt}%
\definecolor{currentstroke}{rgb}{0.296471,0.296471,0.296471}%
\pgfsetstrokecolor{currentstroke}%
\pgfsetdash{}{0pt}%
\pgfpathmoveto{\pgfqpoint{0.849301in}{0.774466in}}%
\pgfpathlineto{\pgfqpoint{0.849301in}{0.627174in}}%
\pgfusepath{stroke}%
\end{pgfscope}%
\begin{pgfscope}%
\pgfpathrectangle{\pgfqpoint{0.793996in}{0.443060in}}{\pgfqpoint{2.041610in}{2.577607in}}%
\pgfusepath{clip}%
\pgfsetbuttcap%
\pgfsetroundjoin%
\pgfsetlinewidth{0.752812pt}%
\definecolor{currentstroke}{rgb}{0.296471,0.296471,0.296471}%
\pgfsetstrokecolor{currentstroke}%
\pgfsetdash{}{0pt}%
\pgfpathmoveto{\pgfqpoint{0.879577in}{0.627174in}}%
\pgfpathlineto{\pgfqpoint{0.879577in}{0.479883in}}%
\pgfusepath{stroke}%
\end{pgfscope}%
\begin{pgfscope}%
\pgfsetrectcap%
\pgfsetmiterjoin%
\pgfsetlinewidth{1.003750pt}%
\definecolor{currentstroke}{rgb}{0.800000,0.800000,0.800000}%
\pgfsetstrokecolor{currentstroke}%
\pgfsetdash{}{0pt}%
\pgfpathmoveto{\pgfqpoint{0.793996in}{0.443060in}}%
\pgfpathlineto{\pgfqpoint{0.793996in}{3.020667in}}%
\pgfusepath{stroke}%
\end{pgfscope}%
\begin{pgfscope}%
\pgfsetrectcap%
\pgfsetmiterjoin%
\pgfsetlinewidth{1.003750pt}%
\definecolor{currentstroke}{rgb}{0.800000,0.800000,0.800000}%
\pgfsetstrokecolor{currentstroke}%
\pgfsetdash{}{0pt}%
\pgfpathmoveto{\pgfqpoint{0.793996in}{0.443060in}}%
\pgfpathlineto{\pgfqpoint{2.835606in}{0.443060in}}%
\pgfusepath{stroke}%
\end{pgfscope}%
\begin{pgfscope}%
\pgfsetbuttcap%
\pgfsetmiterjoin%
\definecolor{currentfill}{rgb}{1.000000,1.000000,1.000000}%
\pgfsetfillcolor{currentfill}%
\pgfsetfillopacity{0.800000}%
\pgfsetlinewidth{0.803000pt}%
\definecolor{currentstroke}{rgb}{0.800000,0.800000,0.800000}%
\pgfsetstrokecolor{currentstroke}%
\pgfsetstrokeopacity{0.800000}%
\pgfsetdash{}{0pt}%
\pgfpathmoveto{\pgfqpoint{1.721918in}{0.495004in}}%
\pgfpathlineto{\pgfqpoint{2.762884in}{0.495004in}}%
\pgfpathquadraticcurveto{\pgfqpoint{2.783662in}{0.495004in}}{\pgfqpoint{2.783662in}{0.515782in}}%
\pgfpathlineto{\pgfqpoint{2.783662in}{0.797217in}}%
\pgfpathquadraticcurveto{\pgfqpoint{2.783662in}{0.817994in}}{\pgfqpoint{2.762884in}{0.817994in}}%
\pgfpathlineto{\pgfqpoint{1.721918in}{0.817994in}}%
\pgfpathquadraticcurveto{\pgfqpoint{1.701140in}{0.817994in}}{\pgfqpoint{1.701140in}{0.797217in}}%
\pgfpathlineto{\pgfqpoint{1.701140in}{0.515782in}}%
\pgfpathquadraticcurveto{\pgfqpoint{1.701140in}{0.495004in}}{\pgfqpoint{1.721918in}{0.495004in}}%
\pgfpathlineto{\pgfqpoint{1.721918in}{0.495004in}}%
\pgfpathclose%
\pgfusepath{stroke,fill}%
\end{pgfscope}%
\begin{pgfscope}%
\pgfsetbuttcap%
\pgfsetmiterjoin%
\definecolor{currentfill}{rgb}{0.347059,0.458824,0.641176}%
\pgfsetfillcolor{currentfill}%
\pgfsetlinewidth{0.602250pt}%
\definecolor{currentstroke}{rgb}{0.296471,0.296471,0.296471}%
\pgfsetstrokecolor{currentstroke}%
\pgfsetdash{}{0pt}%
\pgfpathmoveto{\pgfqpoint{1.742695in}{0.703717in}}%
\pgfpathlineto{\pgfqpoint{1.950473in}{0.703717in}}%
\pgfpathlineto{\pgfqpoint{1.950473in}{0.776439in}}%
\pgfpathlineto{\pgfqpoint{1.742695in}{0.776439in}}%
\pgfpathlineto{\pgfqpoint{1.742695in}{0.703717in}}%
\pgfpathclose%
\pgfusepath{stroke,fill}%
\end{pgfscope}%
\begin{pgfscope}%
\definecolor{textcolor}{rgb}{0.150000,0.150000,0.150000}%
\pgfsetstrokecolor{textcolor}%
\pgfsetfillcolor{textcolor}%
\pgftext[x=2.033584in,y=0.703717in,left,base]{\color{textcolor}{\sffamily\fontsize{7.480000}{8.976000}\selectfont\catcode`\^=\active\def^{\ifmmode\sp\else\^{}\fi}\catcode`\%=\active\def%{\%}Form+meaning}}%
\end{pgfscope}%
\begin{pgfscope}%
\pgfsetbuttcap%
\pgfsetmiterjoin%
\definecolor{currentfill}{rgb}{0.798529,0.536765,0.389706}%
\pgfsetfillcolor{currentfill}%
\pgfsetlinewidth{0.602250pt}%
\definecolor{currentstroke}{rgb}{0.296471,0.296471,0.296471}%
\pgfsetstrokecolor{currentstroke}%
\pgfsetdash{}{0pt}%
\pgfpathmoveto{\pgfqpoint{1.742695in}{0.557753in}}%
\pgfpathlineto{\pgfqpoint{1.950473in}{0.557753in}}%
\pgfpathlineto{\pgfqpoint{1.950473in}{0.630475in}}%
\pgfpathlineto{\pgfqpoint{1.742695in}{0.630475in}}%
\pgfpathlineto{\pgfqpoint{1.742695in}{0.557753in}}%
\pgfpathclose%
\pgfusepath{stroke,fill}%
\end{pgfscope}%
\begin{pgfscope}%
\definecolor{textcolor}{rgb}{0.150000,0.150000,0.150000}%
\pgfsetstrokecolor{textcolor}%
\pgfsetfillcolor{textcolor}%
\pgftext[x=2.033584in,y=0.557753in,left,base]{\color{textcolor}{\sffamily\fontsize{7.480000}{8.976000}\selectfont\catcode`\^=\active\def^{\ifmmode\sp\else\^{}\fi}\catcode`\%=\active\def%{\%}Form-only}}%
\end{pgfscope}%
\end{pgfpicture}%
\makeatother%
\endgroup%

\caption{Exact $F_1$ scores of baseline methods on the procedural datasets}%
\label{fig:baseline-exact}
\end{figure}

\section{Analysis of Emergent Languages}

\subsection{Emergent language hyperparameters}%
\label{app:el-hparams}

The following hyperparameters were used for the vector observation environment:
\smallskip
\begin{description}[nosep,itemindent=-1em]
  \item[\textit{\textbf{n}} values] $4$, $2$ (sparse)
  \item[\textit{\textbf{n}} attributes] $4$, $8$ (sparse)
  \item[\textit{\textbf{n}} distractors] $3$
  \item[vocab size] $32$
  \item[max sequence length] $10$
  \item[dataset size (CSAR input)] $10\,000$ records
\end{description}
\medskip

The ShapeWorld observation environment uses the following hyperparameters
\smallskip
\begin{description}[nosep,itemindent=-1em]
  \item[observations] $5$ shapes, $6$ colors, $3$ operators (and, or, not); \emph{and} or \emph{or} may only be used once
  \item[\textit{n} examples] $20$ total; $10$ correct targets, $10$ distractors
  \item[vocab size] $32$
  \item[max sequence length] $8$
  \item[dataset size (CSAR input)] $20\,000$ records
\end{description}
\medskip

Both environments had any beginning-of-sentence and end-of-sentence tokens removed before being fed into CSAR\@.
Running the above experiments requires about $25$ GPU-hours on NVIDIA GeForce RTX 2080Ti.


\section{Morfessor Results on Emergent Language}
\unskip\label{sec:morf-ec}
\begin{table}
  \centering
  \begin{tabular}{lrr}
  \toprule
                 & $|\text{Inv.}|$ & $|\text{Form}|$ \\
  \midrule
  Vector, AV     &            $94$ & $3.59$ \\
  Vector, sparse &           $126$ & $3.51$ \\
  SW, ref        &          $2898$ & $6.93$ \\
  SW, setref     &          $2920$ & $8.13$ \\
  SW, concept    &          $1565$ & $7.77$ \\
  \bottomrule
  \end{tabular}
  \caption{Metrics for form-only morpheme inventories generated by Morfessor across various emergent languages.}
  \unskip\label{tab:morf-ec}
\end{table}

In \cref{tab:morf-ec} we show the results of running Morfessor on various emergent language corpora.
Compared to the metrics for CSAR's output on the same corpora (\cref{tab:ec-quant}), Morfessor's results do not match or even differ consistently (although Morfessor's forms do not have prevalence weighting like CSAR's).
For the vector environments, Morfessor yields smaller inventories than CSAR yet larger inventories for ShapeWorld.
Form lengths are similar for the vector environment, but for ShapeWorld, CSAR yields shorter forms than the vector environment while Morfessor yields much longer forms.
Since we do not have ground truth morphemes for these emergent language corpora, we cannot definitively say one algorithm has performed better than the other.
Yet Morfessor here is at a disadvantage here as it is not able to use the meanings of the utterances to guide its induction.

\section{Morpheme Inventories}
Top $100$ morphemes induced by CSAR from human and emergent language datasets.

\subsection{Human languages}%
\label{app:human-language}


\paragraph{Morpho Challenge}
(``''', \{+GEN\})
(``ing\$'', \{+PCP1\})
(``ed\$'', \{+PAST\})
(``s'', \{+PL\})
(``er'', \{er\_s\})
(``ly\$'', \{ly\_s\})
(``s\$'', \{+3SG\})
(``ist'', \{ist\_s\})
(``iz'', \{ize\_s\})
(``ness'', \{ness\_s\})
(``ion'', \{ion\_s\})
(``\^{}re'', \{re\_p\})
(``\^{}de'', \{de\_p\})
(``ation'', \{ation\_s\})
(``est\$'', \{+SUP\})
(``\^{}un'', \{un\_p\})
(``less'', \{less\_s\})
(``ful'', \{ful\_s\})
(``\^{}mis'', \{mis\_p\})
(``head'', \{head\_N\})
(``way'', \{way\_N\})
(``ment'', \{ment\_s\})
(``al'', \{al\_s\})
(``it'', \{ity\_s\})
(``\^{}fire'', \{fire\_N\})
(``ency\$'', \{ency\_s\})
(``hook'', \{hook\_N\})
(``ish\$'', \{ish\_s\})
(``mind'', \{mind\_N\})
(``\^{}in'', \{in\_p\})
(``at'', \{ate\_s\})
(``if'', \{ify\_s\})
(``able\$'', \{able\_s\})
(``ically\$'', \{ally\_s\})
(``\^{}inter'', \{inter\_p\})
(``\^{}photo'', \{photo\_p\})
(``\^{}hand'', \{hand\_N\})
(``\^{}scho'', \{school\_N\})
(``house'', \{house\_N\})
(``ical\$'', \{ical\_s\})
(``hold'', \{hold\_V\})
(``long'', \{long\_A\})
(``work'', \{work\_V\})
(``up'', \{up\_B\})
(``ag'', \{age\_s\})
(``ant'', \{ant\_s\})
(``ib'', \{ible\_s\})
(``line'', \{line\_N\})
(``ed\$'', \{ed\_s\})
(``er\$'', \{+CMP\})
(``\^{}over'', \{over\_p\})
(``\^{}dis'', \{dis\_p\})
(``\^{}sea'', \{sea\_N\})
(``\^{}im'', \{im\_p\})
(``or'', \{or\_s\})
(``pos'', \{pose\_V\})
(``ence'', \{ence\_s\})
(``\^{}cardinal'', \{cardinal\_A\})
(``\^{}rational'', \{rational\_A\})
(``\^{}shoplift'', \{shop\_N\})
(``conciliat'', \{conciliate\_V\})
(``\^{}manicur'', \{manicure\_N\})
(``\^{}predict'', \{predict\_V\})
(``dressing'', \{dressing\_V\})
(``\^{}buffet'', \{buffet\_V\})
(``\^{}crimin'', \{crime\_N\})
(``\^{}entitl'', \{entitle\_V\})
(``\^{}frivol'', \{frivolous\_A\})
(``\^{}heartb'', \{heart\_N\})
(``\^{}maroon'', \{maroon\_A\})
(``\^{}ribald'', \{ribald\_A\})
(``\^{}spread'', \{spread\_V\})
(``\^{}squeak'', \{squeak\_V\})
(``\^{}squint'', \{squint\_V\})
(``\^{}statue'', \{statue\_N\})
(``\^{}summar'', \{summary\_A\})
(``whisper'', \{whisper\_V\})
(``\^{}blink'', \{blink\_V\})
(``\^{}carri'', \{carry\_V\})
(``\^{}cheer'', \{cheer\_V\})
(``\^{}four-'', \{four\_Q\})
(``\^{}hitch'', \{hitch\_V\})
(``\^{}louvr'', \{louvre\_N\})
(``\^{}muzzl'', \{muzzle\_N\})
(``\^{}nihil'', \{nihilism\_N\})
(``\^{}tooth'', \{tooth\_N\})
(``\^{}waist'', \{waist\_N\})
(``guard\$'', \{guard\_N\})
(``\^{}bull'', \{bull\_N\})
(``\^{}rail'', \{rail\_V\})
(``\^{}seri'', \{series\_N\})
(``\^{}test'', \{test\_N\})
(``\^{}two-'', \{two\_Q\})
(``ance\$'', \{ance\_s\})
(``board'', \{board\_N\})
(``chain'', \{chain\_N\})
(``eroom'', \{room\_N\})
(``grand'', \{grand\_A\})
(``order'', \{order\_V\})
(``power'', \{power\_N\})

\paragraph{Image captions}
(``tennis'', \{tennis racket\})
(``cat'', \{cat\})
(``train'', \{train\})
(``dog'', \{dog\})
(``pizza'', \{pizza\})
(``toilet'', \{toilet\})
(``man'', \{person\})
(``bus'', \{bus\})
(``clock'', \{clock\})
(``baseball'', \{baseball glove\})
(``frisbee'', \{frisbee\})
(``bed'', \{bed\})
(``horse'', \{horse\})
(``skateboard'', \{skateboard\})
(``laptop'', \{laptop\})
(``cake'', \{cake\})
(``giraffe'', \{giraffe\})
(``table'', \{dining table\})
(``bench'', \{bench\})
(``motorcycle'', \{motorcycle\})
(``bathroom'', \{sink\})
(``elephant'', \{elephant\})
(``umbrella'', \{umbrella\})
(``kitchen'', \{oven\})
(``kite'', \{kite\})
(``people'', \{person\})
(``ball'', \{sports ball\})
(``sheep'', \{sheep\})
(``zebra'', \{zebra\})
(``phone'', \{cell phone\})
(``surfboard'', \{surfboard\})
(``hydrant'', \{fire hydrant\})
(``zebras'', \{zebra\})
(``teddy'', \{teddy bear\})
(``truck'', \{truck\})
(``stop sign'', \{stop sign\})
(``sandwich'', \{sandwich\})
(``boat'', \{boat\})
(``street'', \{car\})
(``bat'', \{baseball bat\})
(``bananas'', \{banana\})
(``giraffes'', \{giraffe\})
(``living'', \{couch\})
(``snow'', \{skis\})
(``bird'', \{bird\})
(``elephants'', \{elephant\})
(``vase'', \{vase\})
(``cows'', \{cow\})
(``broccoli'', \{broccoli\})
(``computer'', \{keyboard\})
(``woman'', \{person\})
(``tie'', \{tie\})
(``horses'', \{horse\})
(``bear'', \{bear\})
(``desk'', \{mouse\})
(``plane'', \{airplane\})
(``luggage'', \{suitcase\})
(``airplane'', \{airplane\})
(``person'', \{person\})
(``hot'', \{hot dog\})
(``refrigerator'', \{refrigerator\})
(``wii'', \{remote\})
(``kites'', \{kite\})
(``boats'', \{boat\})
(``couch'', \{couch\})
(``traffic'', \{traffic light\})
(``plate'', \{fork\})
(``surf'', \{surfboard\})
(``umbrellas'', \{umbrella\})
(``wine'', \{wine glass\})
(``skate'', \{skateboard\})
(``bowl'', \{bowl\})
(``stuffed'', \{teddy bear\})
(``room'', \{tv\})
(``cow'', \{cow\})
(``scissors'', \{scissors\})
(``snowboard'', \{snowboard\})
(``chair'', \{chair\})
(``car'', \{car\})
(``banana'', \{banana\})
(``bicycle'', \{bicycle\})
(``birds'', \{bird\})
(``vegetables'', \{broccoli\})
(``microwave'', \{microwave\})
(``donuts'', \{donut\})
(``video'', \{remote\})
(``batter'', \{baseball bat, person\})
(``skateboarder'', \{person, skateboard\})
(``surfer'', \{person, surfboard\})
(``skis'', \{skis\})
(``motorcycles'', \{motorcycle\})
(``meter'', \{parking meter\})
(``suitcase'', \{suitcase\})
(``sink'', \{sink\})
(``bike'', \{bicycle\})
(``chairs'', \{chair\})
(``food'', \{bowl\})
(``dogs'', \{dog\})
(``oven'', \{oven\})
(``court'', \{sports ball\})

\paragraph{Machine translation}
(``and'', \{und\})
(``Commission'', \{Kommission\})
(``not'', \{nicht\})
(``Union'', \{Union\})
(``we'', \{wir\})
(``I'', \{ich\})
(``that'', \{daß\})
(``Mr'', \{Herr\})
(``I'', \{Ich\})
(``Parliament'', \{Parlament\})
(``President'', \{Präsident\})
(``Member States'', \{Mitgliedstaaten\})
(``report'', \{Bericht\})
(``European'', \{Europäischen\})
(``We'', \{Wir\})
(``or'', \{oder\})
(``in'', \{in\})
(``Europe'', \{Europa\})
(``the'', \{der\})
(``Council'', \{Rat\})
(``between'', \{zwischen\})
(``is'', \{ist\})
(``2000'', \{2000\})
(``Commissioner'', \{Kommissar\})
(``EU'', \{EU\})
(``for'', \{für\})
(``the'', \{die\})
(``The'', \{Die\})
(``also'', \{auch\})
(``with'', \{mit\})
(``like to'', \{möchte\})
(``you'', \{Sie\})
(``1999'', \{1999\})
(``directive'', \{Richtlinie\})
(``only'', \{nur\})
(``proposal'', \{Vorschlag\})
(``European'', \{Europäische\})
(``Madam'', \{Präsidentin\})
(``Mrs'', \{Frau\})
(``Kosovo'', \{Kosovo\})
(``but'', \{aber\})
(``new'', \{neuen\})
(``Group'', \{Fraktion\})
(``have'', \{haben\})
(``behalf'', \{Namen\})
(``Mr'', \{Herrn\})
(``women'', \{Frauen\})
(``has'', \{hat\})
(``regions'', \{Regionen\})
(``years'', \{Jahren\})
(``all'', \{alle\})
(``two'', \{zwei\})
(``cooperation'', \{Zusammenarbeit\})
(``if'', \{wenn\})
(``1'', \{1\})
(``new'', \{neue\})
(``Article'', \{Artikel\})
(``because'', \{weil\})
(``whether'', \{ob\})
(``Parliament'', \{Parlaments\})
(``a'', \{eine\})
(``measures'', \{Maßnahmen\})
(``but'', \{sondern\})
(``institutions'', \{Institutionen\})
(``social'', \{sozialen\})
(``to'', \{zu\})
(``political'', \{politischen\})
(``development'', \{Entwicklung\})
(``national'', \{nationalen\})
(``today'', \{heute\})
(``countries'', \{Länder\})
(``European'', \{europäischen\})
(``must'', \{muß\})
(``our'', \{unsere\})
(``as'', \{wie\})
(``problems'', \{Probleme\})
(``initiative'', \{Initiative\})
(``work'', \{Arbeit\})
(``be'', \{werden\})
(``very'', \{sehr\})
(``human rights'', \{Menschenrechte\})
(``of the'', \{des\})
(``us'', \{uns\})
(``three'', \{drei\})
(``debate'', \{Aussprache\})
(``other'', \{anderen\})
(``hope'', \{hoffe\})
(``already'', \{bereits\})
(``question'', \{Frage\})
(``this'', \{diesem\})
(``debate'', \{Debatte\})
(``are'', \{sind\})
(``will'', \{wird\})
(``proposals'', \{Vorschläge\})
(``If'', \{Wenn\})
(``Prodi'', \{Prodi\})
(``Council'', \{Rates\})
(``rapporteur'', \{Berichterstatter\})
(``INTERREG'', \{INTERREG\})
(``role'', \{Rolle\})


\subsection{Emergent languages}%
\label{app:ec-inv}

\paragraph{Vector, attribute--value}
Note that meanings are in the format \emph{\mbox{attribute\_value}} meaning that 1\_2 means the $1$\textsuperscript{st} attribute has value $2$.

\medskip
\noindent
{(``15'', \{3\_3\})
(``25 25'', \{3\_0\})
(``3'', \{2\_3\})
(``20 20'', \{0\_3, 1\_0\})
(``7 7'', \{0\_3, 1\_3\})
(``4'', \{2\_0\})
(``16 16 16 16 16 16 16 16 16'', \{0\_3, 3\_0\})
(``2'', \{0\_0, 2\_0\})
(``13 13 13 13 13 13'', \{2\_0\})
(``23 23 23 23 23 23 23'', \{0\_0, 2\_3\})
(``28'', \{0\_0, 1\_3\})
(``27 27'', \{1\_0\})
(``17 17 17'', \{0\_0\})
(``31'', \{2\_0, 3\_2\})
(``22 22 22 22 22'', \{1\_3\})
(``22 25 25 25 25'', \{0\_1, 1\_3\})
(``30 30'', \{2\_1\})
(``22 22 13'', \{1\_3, 3\_3\})
(``26 26 26 26 26 26 26 26'', \{1\_3, 2\_0, 3\_0\})
(``15'', \{3\_1\})
(``15 27 27 27 27 27 27 27'', \{0\_1, 3\_2\})
(``8'', \{0\_0\})
(``3 3 3 30 30'', \{0\_1, 1\_1, 2\_2\})
(``16 16'', \{3\_0\})
(``3 3 3 3 30'', \{0\_2, 1\_2, 2\_2\})
(``7 7'', \{0\_2, 3\_1\})
(``15 3 3 3 3'', \{0\_2, 3\_2\})
(``15 7 20 27 27 27 27 27 27 27'', \{0\_2, 1\_1, 2\_1, 3\_2\})
(``20 27 27 27 27 27 27'', \{0\_2, 3\_2\})
(``15 15 15 3 27 27 27 27 27 27'', \{0\_2, 1\_1, 2\_2, 3\_2\})
(``22 22 22 22 22 30 30 30 30 30'', \{0\_1, 1\_2, 2\_2, 3\_2\})
(``8 1 23'', \{1\_1, 3\_0\})
(``22 22 22 25 3 30 30 30 30 30'', \{0\_1, 1\_2, 2\_2, 3\_1\})
(``28 28 2 2 2 2 2 2'', \{1\_2, 3\_1\})
(``22 22 22 17 17 17 17'', \{1\_2, 3\_3\})
(``26 26 6 4 4 4'', \{0\_1, 3\_0\})
(``23'', \{1\_0, 2\_3\})
(``15 15 15 15 15 15 15 15 15'', \{1\_2, 2\_3\})
(``22 22 22 3'', \{0\_1, 1\_2\})
(``7 7 20 20 20 20 20 20 20 20'', \{1\_1\})
(``7 7 7 7 7 20 7'', \{1\_2, 3\_2\})
(``31 31'', \{0\_3, 3\_3\})
(``28 28 28 8 8 8 12 12 12 12'', \{1\_2, 2\_1, 3\_0\})
(``15 15 15 15 15 17 17 17 17'', \{0\_1, 1\_2, 2\_2\})
(``3 27 27 27'', \{2\_2\})
(``7 15 15 15 15 15 15 15 15'', \{0\_3, 1\_2, 2\_2\})
(``3 3 3 3 3 3 3 3 3 3'', \{0\_2, 1\_0, 3\_0\})
(``7 13'', \{0\_1, 1\_2, 3\_2\})
(``28 26'', \{3\_0\})
(``15 15 7'', \{0\_2, 1\_3, 2\_2\})
(``22 22 23 23 23 23 23 23 23 23'', \{1\_1, 2\_2, 3\_1\})
(``7'', \{1\_2\})
(``13 13 13'', \{3\_3\})
(``7 7 7 7'', \{3\_2\})
(``15 17 17 17 17 17'', \{1\_1\})
(``22 22 22 17'', \{1\_2\})
(``15 15 15 13 13 13 13 13 17'', \{0\_1, 1\_2, 2\_1\})
(``3 3 3 3 3 23 23 23'', \{0\_1, 1\_1, 3\_1\})
(``7 16 16 16 16 4'', \{0\_3, 1\_1, 3\_1\})
(``26 26 26 26 6'', \{1\_2, 3\_0\})
(``22 22 22 22'', \{2\_2, 3\_2\})
(``25 25 25 25 25'', \{2\_2\})
(``7 25 25 25 25'', \{0\_2, 1\_3, 2\_3\})
(``22 7 26 13'', \{0\_1, 1\_3, 3\_2\})
(``15 15 15 15 17'', \{0\_1, 1\_2\})
(``23 23 17'', \{2\_2, 3\_2\})
(``7 26 26 26 26 26'', \{0\_2, 1\_3, 2\_1\})
(``8 8 8 23 23 23'', \{1\_2, 2\_2, 3\_1\})
(``7 26 26 26 26 26 26 26'', \{2\_0, 3\_1\})
(``22 7 26 26 26 26 28 28 28 28'', \{0\_1, 2\_1, 3\_1\})
(``15 15 15 15 15 15 15 15'', \{0\_2, 2\_3\})
(``5 4 4 4 4'', \{0\_2, 1\_0, 3\_1\})
(``26 26 26'', \{2\_0, 3\_0\})
(``22 13 13 13 13 13 13 13 2'', \{1\_2, 3\_2\})
(``15 15 31 31 31 31 31 31 31'', \{0\_2, 1\_1, 2\_1\})
(``22 28 28 28 28 28 28 28 28'', \{2\_1, 3\_1\})
(``15'', \{1\_1\})
(``13 13 13 13 2 2 2 2'', \{1\_1, 3\_2\})
(``1 1 1 1 1 1'', \{1\_2, 2\_3\})
(``8 8 30 30'', \{1\_1, 3\_0\})
(``4 4 4 27 27'', \{0\_1, 3\_1\})
(``17 17 17 17'', \{1\_0, 3\_3\})
(``23 23 23 23 23 23 23 27'', \{2\_2\})
(``15 31 31'', \{0\_2, 2\_1\})
(``5 27 27 27 27 27'', \{0\_2, 2\_1\})
(``22'', \{0\_0, 2\_3\})
(``28 8 8 8 8'', \{2\_2, 3\_0\})
(``17 2 2 2 2 2'', \{2\_1, 3\_2\})
(``22 22 2'', \{1\_1, 3\_1\})
(``3 3 23 23 23'', \{0\_1, 3\_1\})
(``28 28 28 28'', \{2\_1\})
(``26 26 26 4 4 4 4 4'', \{0\_1, 3\_1\})
(``17 17 27 27 27'', \{2\_1, 3\_2\})
(``15 13 13 13 13'', \{1\_2, 2\_1\})
(``25 3 25 3 25'', \{0\_1, 1\_2\})
(``20 20 20 27 27 27 27 27 27 27'', \{2\_1, 3\_1\})
(``3 3 3 3'', \{0\_2\})
(``31 31 31 31 31 31 31 31 31 31'', \{1\_0\})
(``17 13'', \{0\_0, 2\_1\})
(``3 3 3 3 3'', \{1\_0, 3\_0\})
}

\paragraph{Vector, bag of meanings}
{(``22'', \{4, 7\})
(``24'', \{0, 3, 6\})
(``16'', \{1, 5\})
(``11'', \{6\})
(``26'', \{0, 4, 6\})
(``1 1'', \{3\})
(``6'', \{0, 7\})
(``17'', \{5, 7\})
(``28 28'', \{0, 2\})
(``18'', \{0, 2\})
(``21 21'', \{0, 6\})
(``14 14 14 14 14'', \{1, 3, 6, 7\})
(``16'', \{4, 7\})
(``24'', \{1, 5\})
(``1 28'', \{3, 4\})
(``31 31'', \{4, 7\})
(``12'', \{4\})
(``3'', \{4, 5\})
(``22 22 22'', \{0, 3, 6\})
(``4'', \{3\})
(``28'', \{2\})
(``12 12 12 12 12'', \{2\})
(``28 27 27 27 27 27 27'', \{0, 4\})
(``28 9'', \{0, 3, 4\})
(``11 11 11'', \{3\})
(``7'', \{3\})
(``12 12 12 12'', \{2\})
(``24 5'', \{4\})
(``30 30'', \{1, 7\})
(``14 14 14 14'', \{1, 2, 7\})
(``25 25 25 25 25 25'', \{1\})
(``4 5 5 5 5'', \{2, 4\})
(``26'', \{2\})
(``25 25 27 27 27 27 27'', \{0, 5\})
(``1 29 29 29 29 29 29 29 12 12'', \{1, 3, 6\})
(``5 5 5 5 5 5 5 5'', \{4\})
(``1 1 1 1 1 1 1'', \{1, 7\})
(``1 12 12 12 12 12 12 12 12'', \{1, 3\})
(``6 6 6 6'', \{2\})
(``1 1 1 1'', \{0, 1\})
(``1 30'', \{2\})
(``1 18'', \{3, 5\})
(``23'', \{7\})
(``16 16 16 16'', \{0\})
(``21 21 21 21'', \{7\})
(``5 5 5 27 27'', \{1\})
(``4 5'', \{4\})
(``5 12'', \{2, 3\})
(``1 12'', \{3, 5\})
(``18 27'', \{5\})
(``11 22 22'', \{2, 3\})
(``22'', \{6\})
(``22 22 11'', \{1, 2\})
(``4 4 27 27'', \{5\})
(``1 29 29'', \{3\})
(``16 16'', \{0\})
(``12'', \{1, 3\})
(``9 9 9 9 9 9 9 9'', \{0\})
(``6 1 1 1'', \{1, 2\})
(``20 20 20 20 20 20 20 20'', \{1, 7\})
(``26 26 26 26 28 28 28'', \{1, 7\})
(``29 29 4 12 12 12 12 12 12'', \{5\})
(``1 1'', \{1, 2\})
(``29 29 4 4 4 4 4 4 4 4'', \{1, 6\})
(``11 11 23 23 23'', \{3\})
(``12 12 12'', \{2\})
(``22 22'', \{3\})
(``28 12'', \{0\})
(``21 21 21 8'', \{2\})
(``1 4 4'', \{2\})
(``21 21 23'', \{2\})
(``21 21 21 21 21 21 21 21 6'', \{1, 2\})
(``12 16'', \{2\})
(``10 10 10 10 10 10 10 25 25 25'', \{2, 7\})
(``28 28 28'', \{1, 4\})
(``24 24 24 24 24 24 24 24'', \{7\})
(``21 21 6 6'', \{2\})
(``21 21'', \{2\})
(``9 5'', \{4\})
(``31 31 31 31'', \{3\})
(``28 28 28 28 28 28 28'', \{3\})
(``10 10 10 10 10'', \{7\})
(``13 13 13'', \{7\})
(``11 14 14 22 22 22 17 17 17'', \{2\})
(``7 7 7 7'', \{2\})
(``22 26 26 5 5 5 5 5 5'', \{1\})
(``22 22 22'', \{2, 5\})
(``11 11'', \{7\})
(``12 12 27 27 27 27 27'', \{2\})
(``30 30 30 30 30 30 27 27 27 27'', \{5\})
(``2 9 12 12'', \{2\})
(``11 11 23 17 17 17'', \{3\})
(``18 18 18 18 18 18 28 18'', \{1\})
(``25'', \{0\})
(``23 23 23 23 19 19 19 19'', \{1\})
(``6 6 10 10 10 10 10 10 10'', \{3\})
(``6 6 6 6 6 6 6 17 17'', \{3\})
(``24 24 24 4 4 12 12'', \{2\})
(``22 22 22 22 28 28 5'', \{1\})
(``24 4 4 4 4 4 4 4 4 4'', \{2\})
}
\paragraph{Shapeworld, reference}
{(``29'', \{ellipse\})
(``29'', \{gray\})
(``29'', \{green\})
(``29'', \{rectangle\})
(``29'', \{triangle\})
(``29'', \{white\})
(``30 2'', \{ellipse\})
(``30'', \{square\})
(``29 29'', \{blue\})
(``18 4 18'', \{white\})
(``29 29'', \{circle\})
(``5 3'', \{square\})
(``5 3'', \{rectangle\})
(``6'', \{square\})
(``24 18'', \{ellipse\})
(``11 4'', \{white\})
(``30 5'', \{triangle\})
(``2 2 2 2 2'', \{white\})
(``29'', \{yellow\})
(``11'', \{ellipse\})
(``2 2 3'', \{ellipse\})
(``4'', \{rectangle\})
(``30 30 3'', \{ellipse\})
(``2'', \{square, yellow\})
(``30'', \{circle\})
(``2 2 2 2 2'', \{ellipse\})
(``3'', \{gray\})
(``23 5'', \{rectangle\})
(``4'', \{green\})
(``13 6 13 2'', \{ellipse\})
(``23 18 23 23'', \{white\})
(``3'', \{rectangle\})
(``18 3 18'', \{white\})
(``2 2 5'', \{ellipse\})
(``24 6 24 6'', \{white\})
(``6 2'', \{ellipse\})
(``3'', \{green\})
(``13 13 23'', \{square\})
(``24 24'', \{white\})
(``18 18 18 23 18'', \{white\})
(``30 5 2'', \{ellipse\})
(``23 24 23'', \{ellipse\})
(``18 4 18 5'', \{ellipse\})
(``4 18 4 5'', \{yellow\})
(``13'', \{gray\})
(``18 5 4 18'', \{ellipse\})
(``2 18'', \{white\})
(``4 5 4'', \{ellipse\})
(``18 18'', \{yellow\})
(``23 13 23'', \{square\})
(``6 3'', \{triangle\})
(``23 3 23 3 23'', \{yellow\})
(``13 6 13 24'', \{ellipse\})
(``24 24'', \{yellow\})
(``13 13 24 13 13 13 24'', \{ellipse\})
(``6 3'', \{circle\})
(``23 18 23'', \{ellipse\})
(``2 2 23 2'', \{white\})
(``5 23'', \{green\})
(``30 30'', \{red\})
(``18 5 4'', \{yellow\})
(``3 23 3 3 3'', \{square, yellow\})
(``23 24 23'', \{white\})
(``18 18 3 18'', \{ellipse\})
(``3 3 3 3'', \{square\})
(``24 6 13 6'', \{white\})
(``30 6 30 3'', \{ellipse\})
(``18 3 18'', \{yellow\})
(``5 30'', \{blue\})
(``24 2'', \{ellipse\})
(``24 13 24 13 13'', \{square, white\})
(``23 18 23'', \{square, yellow\})
(``4 18 18 4'', \{ellipse\})
(``2 2 3'', \{white\})
(``13 6 13'', \{ellipse\})
(``13 13 24'', \{white\})
(``18 23 18'', \{white\})
(``13'', \{rectangle\})
(``13 13 24 13 13'', \{ellipse\})
(``30 24 30'', \{white\})
(``13 13 13 13'', \{white\})
(``23 3 23 3 23'', \{ellipse\})
(``5 18 18 5'', \{ellipse\})
(``24'', \{green\})
(``13 13 23 13 13 13 24'', \{circle, red\})
(``30 6 30 6'', \{blue\})
(``5 5 4'', \{ellipse\})
(``13 24'', \{blue\})
(``5'', \{circle, red\})
(``30 6 30 2'', \{white\})
(``3 3 3 3 3 3'', \{yellow\})
(``18 5 18 5 18'', \{ellipse\})
(``3 3 3'', \{white\})
(``18'', \{square\})
(``18 5 18 4 2'', \{ellipse\})
(``13 2'', \{ellipse\})
(``3 3 23 3 3 23 3 3 2'', \{circle, red\})
(``18 3 18 5 18 3'', \{ellipse\})
(``4 4'', \{blue\})
(``13 24 13'', \{yellow\})
}
\paragraph{Shapeworld, set reference}
{(``3 3'', \{circle, not\})
(``21 21'', \{circle\})
(``23 23'', \{gray, not\})
(``20 20'', \{blue, not\})
(``5 5'', \{and, green, not\})
(``28'', \{square\})
(``26'', \{or, yellow\})
(``28'', \{ellipse, not\})
(``4 4'', \{white\})
(``11'', \{not, rectangle\})
(``11 11'', \{ellipse\})
(``25 25 25'', \{blue, red\})
(``25 4'', \{blue\})
(``3 28'', \{triangle\})
(``22 26'', \{not, red\})
(``25 23'', \{green, or, red\})
(``5 4 5'', \{gray, or, white\})
(``12 23'', \{yellow\})
(``12 18'', \{or, red, white\})
(``23 25'', \{and, gray, white\})
(``5 20'', \{gray, or, red\})
(``4 23'', \{and, red\})
(``12 12'', \{yellow\})
(``3 11'', \{and, circle\})
(``21'', \{circle, or\})
(``28'', \{rectangle\})
(``23 26 23'', \{green\})
(``26 20'', \{and, white\})
(``11'', \{ellipse\})
(``21 22'', \{and, triangle\})
(``22 5'', \{blue, green\})
(``28 25'', \{triangle\})
(``5 26 5'', \{gray\})
(``3'', \{and, circle\})
(``25 20'', \{white\})
(``25 26 4'', \{blue\})
(``28 3'', \{triangle\})
(``21'', \{square\})
(``12'', \{yellow\})
(``11 22'', \{triangle\})
(``12 25'', \{or, red, white\})
(``21'', \{and\})
(``3'', \{square\})
(``20 12 20 12'', \{blue, not\})
(``20 22 22'', \{or, triangle, white\})
(``18'', \{rectangle\})
(``5 5'', \{gray\})
(``5'', \{red\})
(``23 22'', \{red\})
(``23 23'', \{green\})
(``26 22'', \{and, white\})
(``3'', \{rectangle\})
(``20 5 5'', \{white\})
(``22'', \{or\})
(``5 5'', \{triangle\})
(``12 12 23'', \{not\})
(``27'', \{triangle\})
(``12 5'', \{green, not\})
(``25 23 20'', \{and, gray, white\})
(``25 21 4'', \{blue\})
(``4 25 12'', \{blue, or\})
(``22 22'', \{triangle\})
(``20 3 20'', \{white\})
(``22 20'', \{blue, not\})
(``12 22'', \{triangle\})
(``28 26'', \{triangle\})
(``21 23'', \{green\})
(``20 20 20'', \{and, white\})
(``25'', \{blue\})
(``28 20'', \{triangle\})
(``22 22'', \{not\})
(``5'', \{gray\})
(``12'', \{or\})
(``5 11'', \{and, green\})
(``3'', \{triangle\})
(``4 22'', \{red\})
(``23'', \{not\})
(``23 4'', \{green\})
(``18'', \{triangle\})
(``27'', \{rectangle\})
(``12 20 12 20 23'', \{and\})
(``20 22'', \{blue\})
(``25 23'', \{and, gray\})
(``4 12 23'', \{and\})
(``23 3 23'', \{green\})
(``20 22 20'', \{white\})
(``12 20 23'', \{and\})
(``12 20 12 20'', \{and\})
(``8 5 12'', \{and, not\})
(``23 11 23'', \{green\})
(``23 20'', \{white\})
(``28 4 28 4 25'', \{and, triangle, white\})
(``21'', \{triangle\})
(``25 5 20'', \{white\})
(``22 26'', \{gray\})
(``28 4'', \{triangle\})
(``26 18'', \{and\})
(``5 4 18'', \{white\})
(``12 20 12'', \{and\})
(``28'', \{or\})
}
\paragraph{Shapeworld, concept}
{(``3 6'', \{gray, not\})
(``7 7'', \{blue, not\})
(``32'', \{circle\})
(``4 5'', \{not, yellow\})
(``6 12 6 12'', \{green, or, yellow\})
(``3 12 3'', \{blue, or, yellow\})
(``3 7 3 3'', \{green, white\})
(``6 6'', \{not, red\})
(``4 7 4'', \{green, or, red\})
(``6 28 6 28'', \{or, white, yellow\})
(``25 5 25'', \{blue, or, white\})
(``32 32 32'', \{ellipse\})
(``5 5 5 5 5 5 5 5'', \{green, not, yellow\})
(``25 25 25 25 25'', \{green, not, red\})
(``3 4 3'', \{and, white, yellow\})
(``28 28 28 28'', \{white\})
(``32'', \{rectangle\})
(``5 3 5'', \{blue, red\})
(``12 28'', \{yellow\})
(``22'', \{square\})
(``22'', \{triangle\})
(``7 28'', \{or, red, yellow\})
(``3 6 3'', \{white\})
(``5 32'', \{or, red, white\})
(``28 28 5'', \{gray, or, white\})
(``7 28 7 28 7 28 7'', \{blue, green\})
(``3 31 3'', \{blue\})
(``12 4 12'', \{and, green\})
(``28 3'', \{gray\})
(``5 6 5 6 5'', \{and, red, yellow\})
(``25 7 25'', \{green, not, or\})
(``7 5 7 5 7 5 7 5'', \{blue, not, yellow\})
(``4 4 4'', \{not, or, yellow\})
(``22'', \{and, ellipse, not\})
(``6 12 6 6'', \{and, gray\})
(``31 31 31 31'', \{and, blue\})
(``7 12 7'', \{and, white\})
(``5 5 3 28 7'', \{gray, or, red\})
(``28 28 31'', \{gray\})
(``25 3'', \{red, white\})
(``7 6 7 6 7'', \{blue, not, or\})
(``32'', \{triangle\})
(``22 22'', \{or, rectangle\})
(``12 12 32'', \{and, yellow\})
(``5 28'', \{red\})
(``4 6 4'', \{or\})
(``5 7 5 7'', \{and, yellow\})
(``32'', \{and, square\})
(``4 12'', \{green\})
(``3 3 3 3'', \{and, not\})
(``3 3 6'', \{and\})
(``32 32'', \{ellipse, or\})
(``6 5 5'', \{and, gray\})
(``7'', \{or, red\})
(``4 3'', \{and, gray\})
(``12 12 12'', \{yellow\})
(``28 7 28 7'', \{and, green\})
(``25 25 25'', \{and\})
(``28 3 3'', \{or\})
(``23 23'', \{blue, not\})
(``4 4 32 4 32 4'', \{not, or, yellow\})
(``4 3 4 3'', \{not\})
(``6 4'', \{or\})
(``25 7 32'', \{green, not\})
(``31'', \{blue, or\})
(``4'', \{not\})
(``5 5 5 25'', \{green, not, yellow\})
(``28 31 28 31'', \{gray\})
(``12 22 12'', \{rectangle, yellow\})
(``3 3 3'', \{and, not\})
(``5 3 5'', \{or\})
(``28'', \{or, white\})
(``5 5 3'', \{gray\})
(``31 28'', \{red\})
(``6 25'', \{not, red\})
(``32 27 27'', \{ellipse, yellow\})
(``7 6 7 32 7'', \{blue, not\})
(``32 6'', \{not\})
(``4'', \{green, or\})
(``12 6 28'', \{yellow\})
(``32'', \{ellipse\})
(``27 27'', \{yellow\})
(``25 32'', \{not\})
(``3 6 7'', \{white\})
(``28 28 12 3'', \{yellow\})
(``4'', \{circle\})
(``27 32 27'', \{yellow\})
(``3 4'', \{and, gray\})
(``6 7 6'', \{square\})
(``6 12 25'', \{red\})
(``23 22 23'', \{blue\})
(``6 3'', \{white\})
(``3 3 7'', \{green, white\})
(``3 5 5'', \{or\})
(``5 27 32'', \{or, red, white\})
(``7'', \{blue, not\})
(``5'', \{and, yellow\})
(``7 12'', \{and, white\})
(``28'', \{gray\})
(``3 25 3'', \{not\})
}
