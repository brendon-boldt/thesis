\section{Empirical Validation}%
\label{sec:val}
To validate the ability of CSAR to find well-founded morpheme inventories, we test it against procedurally generated datasets as well as human languages.
Since we do not have access to ground truth morphemes for emergent languages, we gauge the effectiveness of CSAR's morpheme induction in the next best way: by testing its performance in these adjacent domains.
Procedurally generated datasets (described in \cref{sec:data-gen}) both give us access to the ``ground truth'' morphemes and allow us to vary particular facets of the languages.
Having ground truth morphemes allows us to quantitatively evaluate CSAR against baseline methods (\cref{sec:baselines}).
Fine-grained control over the facets of the languages permits us to identify particular phenomena that are challenging for CSAR to induce correctly (\cref{sec:error-ana}).
We also test CSAR against human language data (\cref{sec:human-lang}) in order to give a qualitative sense of the effectiveness of the algorithm.


\subsection{Procedural datasets}
\unskip\label{sec:data-gen}

The dataset-generating procedure has the following basic structure:
  (1) Meanings are sampled according to some structure (viz.\@ a fixed attribute--value vector).
  (2) An utterance is produced from this meaning according to a mapping of meaning components to form components.
  (3) The form--meaning pairs that were used to generate the utterance are added to the set of ground truth morphemes.
In the basic case, for example, each particular attribute and value is associated with a unique sequence of tokens which are concatenated to form an utterance, creating a one-to-one mapping from meanings to forms.

\paragraph{Variations}
Such languages are trivial to induce morphemes from, so we introduce the following variations to produce more complex datasets:
\smallskip
\begin{description}[nosep,itemindent=-1em]
  \item[Synonymy] Multiple forms may correspond to the same meaning.
  \item[Polysemy] Multiple meanings may correspond to the same form.
  \item[Multi-token forms] A form may comprise more than one token, possibly overlapping with other forms.
  \item[Vocab size] Number of unique tokens.
  \item[Sparse meanings] Meanings occur independently of each other with no additional structure (i.e., not structured as attribute--value pairs).
  \item[Distribution imbalance] Meanings are sampled from non-uniform distributions.
  \item[Dataset size] Number of records in the dataset.
  \item[Number of meanings] Total number of meanings (e.g., varying number of attributes and values).
  \item[Noise forms] Form tokens not corresponding to any meanings are added.
  \item[Shuffle form] Inter-form order is varied randomly (while maintaining intra-form order).
  \item[Non-compositionality] A given form may correspond to multiple meanings simultaneously.
\end{description}
\medskip
For the following analyses, we report values for a collection of procedural datasets built from the Cartesian product of two values for each of the above variations (one where the variation is inactive and one where it is).
See \cref{app:synth-hparams} for details.


\paragraph{Evaluation metric}
% For many of the test cases, CSAR is not able to recover the exact morpheme inventory but is able to get close to the ground truth.
We use $F_1$ score (harmonic mean of precision and recall) to assess the quality of an induced morpheme inventory given the ground truth inventory.
We define precision as
\begin{equation}
  \frac1{|\mathcal I|} \sum_{i \in \mathcal I} \max_{g\in\mathcal G} s(i, g) ,
  \label{eq:precision}
\end{equation}
where
  $\mathcal I$ is the set of induced morphemes,
  $\mathcal G$ is the set of ground truth morphemes,
  and
  $s$ is the similarity function
  \unskip.
For exact $F_1$, the similarity function is $1$ if the morphemes are identical and $0$ otherwise.
In fuzzy $F_1$, the similarity function is the minimum of form similarity (normalized insertion--deletion ratio\footnote{$1 - (\text{insertions} + \text{deletions}) / (|s_1|+|s_2|)$}) and meaning similarity (Jaccard index).
Recall is defined similarly to precision except that the roles of $\mathcal{I}$ and $\mathcal{G}$ from \cref{eq:precision} are reversed.

% \paragraph{Results}
% Across all synthetic datasets, CSAR achieves $F_\text{exact}=0.72$ and $F_\text{fuzzy}=0.88$.
% In comparison, the naive baseline of simply using set of raw observations as the set of morphemes yields $F_\text{exact}=0.10$ and $F_\text{fuzzy}=0.55$.
% \cmt{How do we put these in perspective?}


\subsection{Comparison with baselines}
\unskip\label{sec:baselines}

\begin{figure}
  \centering
  %% Creator: Matplotlib, PGF backend
%%
%% To include the figure in your LaTeX document, write
%%   \input{<filename>.pgf}
%%
%% Make sure the required packages are loaded in your preamble
%%   \usepackage{pgf}
%%
%% Also ensure that all the required font packages are loaded; for instance,
%% the lmodern package is sometimes necessary when using math font.
%%   \usepackage{lmodern}
%%
%% Figures using additional raster images can only be included by \input if
%% they are in the same directory as the main LaTeX file. For loading figures
%% from other directories you can use the `import` package
%%   \usepackage{import}
%%
%% and then include the figures with
%%   \import{<path to file>}{<filename>.pgf}
%%
%% Matplotlib used the following preamble
%%   \def\mathdefault#1{#1}
%%   \everymath=\expandafter{\the\everymath\displaystyle}
%%   \IfFileExists{scrextend.sty}{
%%     \usepackage[fontsize=8.160000pt]{scrextend}
%%   }{
%%     \renewcommand{\normalsize}{\fontsize{8.160000}{9.792000}\selectfont}
%%     \normalsize
%%   }
%%   
%%   \makeatletter\@ifpackageloaded{underscore}{}{\usepackage[strings]{underscore}}\makeatother
%%
\begingroup%
\makeatletter%
\begin{pgfpicture}%
\pgfpathrectangle{\pgfpointorigin}{\pgfqpoint{3.190522in}{3.100000in}}%
\pgfusepath{use as bounding box, clip}%
\begin{pgfscope}%
\pgfsetbuttcap%
\pgfsetmiterjoin%
\definecolor{currentfill}{rgb}{1.000000,1.000000,1.000000}%
\pgfsetfillcolor{currentfill}%
\pgfsetlinewidth{0.000000pt}%
\definecolor{currentstroke}{rgb}{1.000000,1.000000,1.000000}%
\pgfsetstrokecolor{currentstroke}%
\pgfsetdash{}{0pt}%
\pgfpathmoveto{\pgfqpoint{0.000000in}{0.000000in}}%
\pgfpathlineto{\pgfqpoint{3.190522in}{0.000000in}}%
\pgfpathlineto{\pgfqpoint{3.190522in}{3.100000in}}%
\pgfpathlineto{\pgfqpoint{0.000000in}{3.100000in}}%
\pgfpathlineto{\pgfqpoint{0.000000in}{0.000000in}}%
\pgfpathclose%
\pgfusepath{fill}%
\end{pgfscope}%
\begin{pgfscope}%
\pgfsetbuttcap%
\pgfsetmiterjoin%
\definecolor{currentfill}{rgb}{1.000000,1.000000,1.000000}%
\pgfsetfillcolor{currentfill}%
\pgfsetlinewidth{0.000000pt}%
\definecolor{currentstroke}{rgb}{0.000000,0.000000,0.000000}%
\pgfsetstrokecolor{currentstroke}%
\pgfsetstrokeopacity{0.000000}%
\pgfsetdash{}{0pt}%
\pgfpathmoveto{\pgfqpoint{0.703330in}{0.352393in}}%
\pgfpathlineto{\pgfqpoint{2.876564in}{0.352393in}}%
\pgfpathlineto{\pgfqpoint{2.876564in}{3.111333in}}%
\pgfpathlineto{\pgfqpoint{0.703330in}{3.111333in}}%
\pgfpathlineto{\pgfqpoint{0.703330in}{0.352393in}}%
\pgfpathclose%
\pgfusepath{fill}%
\end{pgfscope}%
\begin{pgfscope}%
\pgfpathrectangle{\pgfqpoint{0.703330in}{0.352393in}}{\pgfqpoint{2.173234in}{2.758940in}}%
\pgfusepath{clip}%
\pgfsetroundcap%
\pgfsetroundjoin%
\pgfsetlinewidth{0.803000pt}%
\definecolor{currentstroke}{rgb}{0.800000,0.800000,0.800000}%
\pgfsetstrokecolor{currentstroke}%
\pgfsetdash{}{0pt}%
\pgfpathmoveto{\pgfqpoint{1.013792in}{0.352393in}}%
\pgfpathlineto{\pgfqpoint{1.013792in}{3.111333in}}%
\pgfusepath{stroke}%
\end{pgfscope}%
\begin{pgfscope}%
\definecolor{textcolor}{rgb}{0.150000,0.150000,0.150000}%
\pgfsetstrokecolor{textcolor}%
\pgfsetfillcolor{textcolor}%
\pgftext[x=1.013792in,y=0.237115in,,top]{\color{textcolor}{\sffamily\fontsize{7.480000}{8.976000}\selectfont\catcode`\^=\active\def^{\ifmmode\sp\else\^{}\fi}\catcode`\%=\active\def%{\%}0.4}}%
\end{pgfscope}%
\begin{pgfscope}%
\pgfpathrectangle{\pgfqpoint{0.703330in}{0.352393in}}{\pgfqpoint{2.173234in}{2.758940in}}%
\pgfusepath{clip}%
\pgfsetroundcap%
\pgfsetroundjoin%
\pgfsetlinewidth{0.803000pt}%
\definecolor{currentstroke}{rgb}{0.800000,0.800000,0.800000}%
\pgfsetstrokecolor{currentstroke}%
\pgfsetdash{}{0pt}%
\pgfpathmoveto{\pgfqpoint{1.634716in}{0.352393in}}%
\pgfpathlineto{\pgfqpoint{1.634716in}{3.111333in}}%
\pgfusepath{stroke}%
\end{pgfscope}%
\begin{pgfscope}%
\definecolor{textcolor}{rgb}{0.150000,0.150000,0.150000}%
\pgfsetstrokecolor{textcolor}%
\pgfsetfillcolor{textcolor}%
\pgftext[x=1.634716in,y=0.237115in,,top]{\color{textcolor}{\sffamily\fontsize{7.480000}{8.976000}\selectfont\catcode`\^=\active\def^{\ifmmode\sp\else\^{}\fi}\catcode`\%=\active\def%{\%}0.6}}%
\end{pgfscope}%
\begin{pgfscope}%
\pgfpathrectangle{\pgfqpoint{0.703330in}{0.352393in}}{\pgfqpoint{2.173234in}{2.758940in}}%
\pgfusepath{clip}%
\pgfsetroundcap%
\pgfsetroundjoin%
\pgfsetlinewidth{0.803000pt}%
\definecolor{currentstroke}{rgb}{0.800000,0.800000,0.800000}%
\pgfsetstrokecolor{currentstroke}%
\pgfsetdash{}{0pt}%
\pgfpathmoveto{\pgfqpoint{2.255640in}{0.352393in}}%
\pgfpathlineto{\pgfqpoint{2.255640in}{3.111333in}}%
\pgfusepath{stroke}%
\end{pgfscope}%
\begin{pgfscope}%
\definecolor{textcolor}{rgb}{0.150000,0.150000,0.150000}%
\pgfsetstrokecolor{textcolor}%
\pgfsetfillcolor{textcolor}%
\pgftext[x=2.255640in,y=0.237115in,,top]{\color{textcolor}{\sffamily\fontsize{7.480000}{8.976000}\selectfont\catcode`\^=\active\def^{\ifmmode\sp\else\^{}\fi}\catcode`\%=\active\def%{\%}0.8}}%
\end{pgfscope}%
\begin{pgfscope}%
\pgfpathrectangle{\pgfqpoint{0.703330in}{0.352393in}}{\pgfqpoint{2.173234in}{2.758940in}}%
\pgfusepath{clip}%
\pgfsetroundcap%
\pgfsetroundjoin%
\pgfsetlinewidth{0.803000pt}%
\definecolor{currentstroke}{rgb}{0.800000,0.800000,0.800000}%
\pgfsetstrokecolor{currentstroke}%
\pgfsetdash{}{0pt}%
\pgfpathmoveto{\pgfqpoint{2.876564in}{0.352393in}}%
\pgfpathlineto{\pgfqpoint{2.876564in}{3.111333in}}%
\pgfusepath{stroke}%
\end{pgfscope}%
\begin{pgfscope}%
\definecolor{textcolor}{rgb}{0.150000,0.150000,0.150000}%
\pgfsetstrokecolor{textcolor}%
\pgfsetfillcolor{textcolor}%
\pgftext[x=2.876564in,y=0.237115in,,top]{\color{textcolor}{\sffamily\fontsize{7.480000}{8.976000}\selectfont\catcode`\^=\active\def^{\ifmmode\sp\else\^{}\fi}\catcode`\%=\active\def%{\%}1.0}}%
\end{pgfscope}%
\begin{pgfscope}%
\definecolor{textcolor}{rgb}{0.150000,0.150000,0.150000}%
\pgfsetstrokecolor{textcolor}%
\pgfsetfillcolor{textcolor}%
\pgftext[x=1.789947in,y=0.089306in,,top]{\color{textcolor}{\sffamily\fontsize{8.160000}{9.792000}\selectfont\catcode`\^=\active\def^{\ifmmode\sp\else\^{}\fi}\catcode`\%=\active\def%{\%}F1-score}}%
\end{pgfscope}%
\begin{pgfscope}%
\definecolor{textcolor}{rgb}{0.150000,0.150000,0.150000}%
\pgfsetstrokecolor{textcolor}%
\pgfsetfillcolor{textcolor}%
\pgftext[x=0.311188in, y=2.878217in, left, base]{\color{textcolor}{\sffamily\fontsize{7.480000}{8.976000}\selectfont\catcode`\^=\active\def^{\ifmmode\sp\else\^{}\fi}\catcode`\%=\active\def%{\%}CSAR}}%
\end{pgfscope}%
\begin{pgfscope}%
\definecolor{textcolor}{rgb}{0.150000,0.150000,0.150000}%
\pgfsetstrokecolor{textcolor}%
\pgfsetfillcolor{textcolor}%
\pgftext[x=0.387546in, y=2.530209in, left, base]{\color{textcolor}{\sffamily\fontsize{7.480000}{8.976000}\selectfont\catcode`\^=\active\def^{\ifmmode\sp\else\^{}\fi}\catcode`\%=\active\def%{\%}IBM}}%
\end{pgfscope}%
\begin{pgfscope}%
\definecolor{textcolor}{rgb}{0.150000,0.150000,0.150000}%
\pgfsetstrokecolor{textcolor}%
\pgfsetfillcolor{textcolor}%
\pgftext[x=0.209169in, y=2.423536in, left, base]{\color{textcolor}{\sffamily\fontsize{7.480000}{8.976000}\selectfont\catcode`\^=\active\def^{\ifmmode\sp\else\^{}\fi}\catcode`\%=\active\def%{\%}Model 1}}%
\end{pgfscope}%
\begin{pgfscope}%
\definecolor{textcolor}{rgb}{0.150000,0.150000,0.150000}%
\pgfsetstrokecolor{textcolor}%
\pgfsetfillcolor{textcolor}%
\pgftext[x=0.387546in, y=2.136075in, left, base]{\color{textcolor}{\sffamily\fontsize{7.480000}{8.976000}\selectfont\catcode`\^=\active\def^{\ifmmode\sp\else\^{}\fi}\catcode`\%=\active\def%{\%}IBM}}%
\end{pgfscope}%
\begin{pgfscope}%
\definecolor{textcolor}{rgb}{0.150000,0.150000,0.150000}%
\pgfsetstrokecolor{textcolor}%
\pgfsetfillcolor{textcolor}%
\pgftext[x=0.209169in, y=2.029402in, left, base]{\color{textcolor}{\sffamily\fontsize{7.480000}{8.976000}\selectfont\catcode`\^=\active\def^{\ifmmode\sp\else\^{}\fi}\catcode`\%=\active\def%{\%}Model 3}}%
\end{pgfscope}%
\begin{pgfscope}%
\definecolor{textcolor}{rgb}{0.150000,0.150000,0.150000}%
\pgfsetstrokecolor{textcolor}%
\pgfsetfillcolor{textcolor}%
\pgftext[x=0.144862in, y=1.695814in, left, base]{\color{textcolor}{\sffamily\fontsize{7.480000}{8.976000}\selectfont\catcode`\^=\active\def^{\ifmmode\sp\else\^{}\fi}\catcode`\%=\active\def%{\%}Morfessor}}%
\end{pgfscope}%
\begin{pgfscope}%
\definecolor{textcolor}{rgb}{0.150000,0.150000,0.150000}%
\pgfsetstrokecolor{textcolor}%
\pgfsetfillcolor{textcolor}%
\pgftext[x=0.377781in, y=1.301679in, left, base]{\color{textcolor}{\sffamily\fontsize{7.480000}{8.976000}\selectfont\catcode`\^=\active\def^{\ifmmode\sp\else\^{}\fi}\catcode`\%=\active\def%{\%}BPE}}%
\end{pgfscope}%
\begin{pgfscope}%
\definecolor{textcolor}{rgb}{0.150000,0.150000,0.150000}%
\pgfsetstrokecolor{textcolor}%
\pgfsetfillcolor{textcolor}%
\pgftext[x=0.355964in, y=0.907545in, left, base]{\color{textcolor}{\sffamily\fontsize{7.480000}{8.976000}\selectfont\catcode`\^=\active\def^{\ifmmode\sp\else\^{}\fi}\catcode`\%=\active\def%{\%}ULM}}%
\end{pgfscope}%
\begin{pgfscope}%
\definecolor{textcolor}{rgb}{0.150000,0.150000,0.150000}%
\pgfsetstrokecolor{textcolor}%
\pgfsetfillcolor{textcolor}%
\pgftext[x=0.229428in, y=0.513411in, left, base]{\color{textcolor}{\sffamily\fontsize{7.480000}{8.976000}\selectfont\catcode`\^=\active\def^{\ifmmode\sp\else\^{}\fi}\catcode`\%=\active\def%{\%}Records}}%
\end{pgfscope}%
\begin{pgfscope}%
\definecolor{textcolor}{rgb}{0.150000,0.150000,0.150000}%
\pgfsetstrokecolor{textcolor}%
\pgfsetfillcolor{textcolor}%
\pgftext[x=0.089306in,y=1.731863in,,bottom,rotate=90.000000]{\color{textcolor}{\sffamily\fontsize{8.160000}{9.792000}\selectfont\catcode`\^=\active\def^{\ifmmode\sp\else\^{}\fi}\catcode`\%=\active\def%{\%}Model}}%
\end{pgfscope}%
\begin{pgfscope}%
\pgfpathrectangle{\pgfqpoint{0.703330in}{0.352393in}}{\pgfqpoint{2.173234in}{2.758940in}}%
\pgfusepath{clip}%
\pgfsetbuttcap%
\pgfsetroundjoin%
\definecolor{currentfill}{rgb}{0.848437,0.867532,0.899724}%
\pgfsetfillcolor{currentfill}%
\pgfsetlinewidth{0.602250pt}%
\definecolor{currentstroke}{rgb}{0.296471,0.296471,0.296471}%
\pgfsetstrokecolor{currentstroke}%
\pgfsetdash{}{0pt}%
\pgfpathmoveto{\pgfqpoint{1.710058in}{2.993401in}}%
\pgfpathlineto{\pgfqpoint{1.780056in}{2.993401in}}%
\pgfpathlineto{\pgfqpoint{1.780056in}{2.992785in}}%
\pgfpathlineto{\pgfqpoint{1.710058in}{2.992785in}}%
\pgfpathlineto{\pgfqpoint{1.710058in}{2.993401in}}%
\pgfpathclose%
\pgfusepath{stroke,fill}%
\end{pgfscope}%
\begin{pgfscope}%
\pgfpathrectangle{\pgfqpoint{0.703330in}{0.352393in}}{\pgfqpoint{2.173234in}{2.758940in}}%
\pgfusepath{clip}%
\pgfsetbuttcap%
\pgfsetroundjoin%
\definecolor{currentfill}{rgb}{0.825117,0.848522,0.887698}%
\pgfsetfillcolor{currentfill}%
\pgfsetlinewidth{0.602250pt}%
\definecolor{currentstroke}{rgb}{0.296471,0.296471,0.296471}%
\pgfsetstrokecolor{currentstroke}%
\pgfsetdash{}{0pt}%
\pgfpathmoveto{\pgfqpoint{1.780056in}{2.993709in}}%
\pgfpathlineto{\pgfqpoint{1.791506in}{2.993709in}}%
\pgfpathlineto{\pgfqpoint{1.791506in}{2.992477in}}%
\pgfpathlineto{\pgfqpoint{1.780056in}{2.992477in}}%
\pgfpathlineto{\pgfqpoint{1.780056in}{2.993709in}}%
\pgfpathclose%
\pgfusepath{stroke,fill}%
\end{pgfscope}%
\begin{pgfscope}%
\pgfpathrectangle{\pgfqpoint{0.703330in}{0.352393in}}{\pgfqpoint{2.173234in}{2.758940in}}%
\pgfusepath{clip}%
\pgfsetbuttcap%
\pgfsetroundjoin%
\definecolor{currentfill}{rgb}{0.792469,0.821908,0.870863}%
\pgfsetfillcolor{currentfill}%
\pgfsetlinewidth{0.602250pt}%
\definecolor{currentstroke}{rgb}{0.296471,0.296471,0.296471}%
\pgfsetstrokecolor{currentstroke}%
\pgfsetdash{}{0pt}%
\pgfpathmoveto{\pgfqpoint{1.791506in}{2.994325in}}%
\pgfpathlineto{\pgfqpoint{1.814608in}{2.994325in}}%
\pgfpathlineto{\pgfqpoint{1.814608in}{2.991861in}}%
\pgfpathlineto{\pgfqpoint{1.791506in}{2.991861in}}%
\pgfpathlineto{\pgfqpoint{1.791506in}{2.994325in}}%
\pgfpathclose%
\pgfusepath{stroke,fill}%
\end{pgfscope}%
\begin{pgfscope}%
\pgfpathrectangle{\pgfqpoint{0.703330in}{0.352393in}}{\pgfqpoint{2.173234in}{2.758940in}}%
\pgfusepath{clip}%
\pgfsetbuttcap%
\pgfsetroundjoin%
\definecolor{currentfill}{rgb}{0.755157,0.791493,0.851622}%
\pgfsetfillcolor{currentfill}%
\pgfsetlinewidth{0.602250pt}%
\definecolor{currentstroke}{rgb}{0.296471,0.296471,0.296471}%
\pgfsetstrokecolor{currentstroke}%
\pgfsetdash{}{0pt}%
\pgfpathmoveto{\pgfqpoint{1.814608in}{2.995556in}}%
\pgfpathlineto{\pgfqpoint{1.882572in}{2.995556in}}%
\pgfpathlineto{\pgfqpoint{1.882572in}{2.990630in}}%
\pgfpathlineto{\pgfqpoint{1.814608in}{2.990630in}}%
\pgfpathlineto{\pgfqpoint{1.814608in}{2.995556in}}%
\pgfpathclose%
\pgfusepath{stroke,fill}%
\end{pgfscope}%
\begin{pgfscope}%
\pgfpathrectangle{\pgfqpoint{0.703330in}{0.352393in}}{\pgfqpoint{2.173234in}{2.758940in}}%
\pgfusepath{clip}%
\pgfsetbuttcap%
\pgfsetroundjoin%
\definecolor{currentfill}{rgb}{0.706185,0.751573,0.826368}%
\pgfsetfillcolor{currentfill}%
\pgfsetlinewidth{0.602250pt}%
\definecolor{currentstroke}{rgb}{0.296471,0.296471,0.296471}%
\pgfsetstrokecolor{currentstroke}%
\pgfsetdash{}{0pt}%
\pgfpathmoveto{\pgfqpoint{1.882572in}{2.998020in}}%
\pgfpathlineto{\pgfqpoint{1.930675in}{2.998020in}}%
\pgfpathlineto{\pgfqpoint{1.930675in}{2.988166in}}%
\pgfpathlineto{\pgfqpoint{1.882572in}{2.988166in}}%
\pgfpathlineto{\pgfqpoint{1.882572in}{2.998020in}}%
\pgfpathclose%
\pgfusepath{stroke,fill}%
\end{pgfscope}%
\begin{pgfscope}%
\pgfpathrectangle{\pgfqpoint{0.703330in}{0.352393in}}{\pgfqpoint{2.173234in}{2.758940in}}%
\pgfusepath{clip}%
\pgfsetbuttcap%
\pgfsetroundjoin%
\definecolor{currentfill}{rgb}{0.643221,0.700246,0.793900}%
\pgfsetfillcolor{currentfill}%
\pgfsetlinewidth{0.602250pt}%
\definecolor{currentstroke}{rgb}{0.296471,0.296471,0.296471}%
\pgfsetstrokecolor{currentstroke}%
\pgfsetdash{}{0pt}%
\pgfpathmoveto{\pgfqpoint{1.930675in}{3.002946in}}%
\pgfpathlineto{\pgfqpoint{1.972042in}{3.002946in}}%
\pgfpathlineto{\pgfqpoint{1.972042in}{2.983240in}}%
\pgfpathlineto{\pgfqpoint{1.930675in}{2.983240in}}%
\pgfpathlineto{\pgfqpoint{1.930675in}{3.002946in}}%
\pgfpathclose%
\pgfusepath{stroke,fill}%
\end{pgfscope}%
\begin{pgfscope}%
\pgfpathrectangle{\pgfqpoint{0.703330in}{0.352393in}}{\pgfqpoint{2.173234in}{2.758940in}}%
\pgfusepath{clip}%
\pgfsetbuttcap%
\pgfsetroundjoin%
\definecolor{currentfill}{rgb}{0.566266,0.637515,0.754216}%
\pgfsetfillcolor{currentfill}%
\pgfsetlinewidth{0.602250pt}%
\definecolor{currentstroke}{rgb}{0.296471,0.296471,0.296471}%
\pgfsetstrokecolor{currentstroke}%
\pgfsetdash{}{0pt}%
\pgfpathmoveto{\pgfqpoint{1.972042in}{3.012800in}}%
\pgfpathlineto{\pgfqpoint{2.132743in}{3.012800in}}%
\pgfpathlineto{\pgfqpoint{2.132743in}{2.973386in}}%
\pgfpathlineto{\pgfqpoint{1.972042in}{2.973386in}}%
\pgfpathlineto{\pgfqpoint{1.972042in}{3.012800in}}%
\pgfpathclose%
\pgfusepath{stroke,fill}%
\end{pgfscope}%
\begin{pgfscope}%
\pgfpathrectangle{\pgfqpoint{0.703330in}{0.352393in}}{\pgfqpoint{2.173234in}{2.758940in}}%
\pgfusepath{clip}%
\pgfsetbuttcap%
\pgfsetroundjoin%
\definecolor{currentfill}{rgb}{0.468322,0.557674,0.703709}%
\pgfsetfillcolor{currentfill}%
\pgfsetlinewidth{0.602250pt}%
\definecolor{currentstroke}{rgb}{0.296471,0.296471,0.296471}%
\pgfsetstrokecolor{currentstroke}%
\pgfsetdash{}{0pt}%
\pgfpathmoveto{\pgfqpoint{2.132743in}{3.032506in}}%
\pgfpathlineto{\pgfqpoint{2.318851in}{3.032506in}}%
\pgfpathlineto{\pgfqpoint{2.318851in}{2.953680in}}%
\pgfpathlineto{\pgfqpoint{2.132743in}{2.953680in}}%
\pgfpathlineto{\pgfqpoint{2.132743in}{3.032506in}}%
\pgfpathclose%
\pgfusepath{stroke,fill}%
\end{pgfscope}%
\begin{pgfscope}%
\pgfpathrectangle{\pgfqpoint{0.703330in}{0.352393in}}{\pgfqpoint{2.173234in}{2.758940in}}%
\pgfusepath{clip}%
\pgfsetbuttcap%
\pgfsetroundjoin%
\definecolor{currentfill}{rgb}{0.347059,0.458824,0.641176}%
\pgfsetfillcolor{currentfill}%
\pgfsetlinewidth{0.602250pt}%
\definecolor{currentstroke}{rgb}{0.296471,0.296471,0.296471}%
\pgfsetstrokecolor{currentstroke}%
\pgfsetdash{}{0pt}%
\pgfpathmoveto{\pgfqpoint{2.318851in}{3.071920in}}%
\pgfpathlineto{\pgfqpoint{2.847153in}{3.071920in}}%
\pgfpathlineto{\pgfqpoint{2.847153in}{2.914266in}}%
\pgfpathlineto{\pgfqpoint{2.318851in}{2.914266in}}%
\pgfpathlineto{\pgfqpoint{2.318851in}{3.071920in}}%
\pgfpathclose%
\pgfusepath{stroke,fill}%
\end{pgfscope}%
\begin{pgfscope}%
\pgfpathrectangle{\pgfqpoint{0.703330in}{0.352393in}}{\pgfqpoint{2.173234in}{2.758940in}}%
\pgfusepath{clip}%
\pgfsetbuttcap%
\pgfsetroundjoin%
\definecolor{currentfill}{rgb}{0.468322,0.557674,0.703709}%
\pgfsetfillcolor{currentfill}%
\pgfsetlinewidth{0.602250pt}%
\definecolor{currentstroke}{rgb}{0.296471,0.296471,0.296471}%
\pgfsetstrokecolor{currentstroke}%
\pgfsetdash{}{0pt}%
\pgfpathmoveto{\pgfqpoint{2.847153in}{3.032506in}}%
\pgfpathlineto{\pgfqpoint{2.876564in}{3.032506in}}%
\pgfpathlineto{\pgfqpoint{2.876564in}{2.953680in}}%
\pgfpathlineto{\pgfqpoint{2.847153in}{2.953680in}}%
\pgfpathlineto{\pgfqpoint{2.847153in}{3.032506in}}%
\pgfpathclose%
\pgfusepath{stroke,fill}%
\end{pgfscope}%
\begin{pgfscope}%
\pgfpathrectangle{\pgfqpoint{0.703330in}{0.352393in}}{\pgfqpoint{2.173234in}{2.758940in}}%
\pgfusepath{clip}%
\pgfsetbuttcap%
\pgfsetroundjoin%
\definecolor{currentfill}{rgb}{0.566266,0.637515,0.754216}%
\pgfsetfillcolor{currentfill}%
\pgfsetlinewidth{0.602250pt}%
\definecolor{currentstroke}{rgb}{0.296471,0.296471,0.296471}%
\pgfsetstrokecolor{currentstroke}%
\pgfsetdash{}{0pt}%
\pgfpathmoveto{\pgfqpoint{2.876564in}{3.012800in}}%
\pgfpathlineto{\pgfqpoint{2.876564in}{3.012800in}}%
\pgfpathlineto{\pgfqpoint{2.876564in}{2.973386in}}%
\pgfpathlineto{\pgfqpoint{2.876564in}{2.973386in}}%
\pgfpathlineto{\pgfqpoint{2.876564in}{3.012800in}}%
\pgfpathclose%
\pgfusepath{stroke,fill}%
\end{pgfscope}%
\begin{pgfscope}%
\pgfpathrectangle{\pgfqpoint{0.703330in}{0.352393in}}{\pgfqpoint{2.173234in}{2.758940in}}%
\pgfusepath{clip}%
\pgfsetbuttcap%
\pgfsetroundjoin%
\definecolor{currentfill}{rgb}{0.643221,0.700246,0.793900}%
\pgfsetfillcolor{currentfill}%
\pgfsetlinewidth{0.602250pt}%
\definecolor{currentstroke}{rgb}{0.296471,0.296471,0.296471}%
\pgfsetstrokecolor{currentstroke}%
\pgfsetdash{}{0pt}%
\pgfpathmoveto{\pgfqpoint{2.876564in}{3.002946in}}%
\pgfpathlineto{\pgfqpoint{2.876564in}{3.002946in}}%
\pgfpathlineto{\pgfqpoint{2.876564in}{2.983240in}}%
\pgfpathlineto{\pgfqpoint{2.876564in}{2.983240in}}%
\pgfpathlineto{\pgfqpoint{2.876564in}{3.002946in}}%
\pgfpathclose%
\pgfusepath{stroke,fill}%
\end{pgfscope}%
\begin{pgfscope}%
\pgfpathrectangle{\pgfqpoint{0.703330in}{0.352393in}}{\pgfqpoint{2.173234in}{2.758940in}}%
\pgfusepath{clip}%
\pgfsetbuttcap%
\pgfsetroundjoin%
\definecolor{currentfill}{rgb}{0.706185,0.751573,0.826368}%
\pgfsetfillcolor{currentfill}%
\pgfsetlinewidth{0.602250pt}%
\definecolor{currentstroke}{rgb}{0.296471,0.296471,0.296471}%
\pgfsetstrokecolor{currentstroke}%
\pgfsetdash{}{0pt}%
\pgfpathmoveto{\pgfqpoint{2.876564in}{2.998020in}}%
\pgfpathlineto{\pgfqpoint{2.876564in}{2.998020in}}%
\pgfpathlineto{\pgfqpoint{2.876564in}{2.988166in}}%
\pgfpathlineto{\pgfqpoint{2.876564in}{2.988166in}}%
\pgfpathlineto{\pgfqpoint{2.876564in}{2.998020in}}%
\pgfpathclose%
\pgfusepath{stroke,fill}%
\end{pgfscope}%
\begin{pgfscope}%
\pgfpathrectangle{\pgfqpoint{0.703330in}{0.352393in}}{\pgfqpoint{2.173234in}{2.758940in}}%
\pgfusepath{clip}%
\pgfsetbuttcap%
\pgfsetroundjoin%
\definecolor{currentfill}{rgb}{0.755157,0.791493,0.851622}%
\pgfsetfillcolor{currentfill}%
\pgfsetlinewidth{0.602250pt}%
\definecolor{currentstroke}{rgb}{0.296471,0.296471,0.296471}%
\pgfsetstrokecolor{currentstroke}%
\pgfsetdash{}{0pt}%
\pgfpathmoveto{\pgfqpoint{2.876564in}{2.995556in}}%
\pgfpathlineto{\pgfqpoint{2.876564in}{2.995556in}}%
\pgfpathlineto{\pgfqpoint{2.876564in}{2.990630in}}%
\pgfpathlineto{\pgfqpoint{2.876564in}{2.990630in}}%
\pgfpathlineto{\pgfqpoint{2.876564in}{2.995556in}}%
\pgfpathclose%
\pgfusepath{stroke,fill}%
\end{pgfscope}%
\begin{pgfscope}%
\pgfpathrectangle{\pgfqpoint{0.703330in}{0.352393in}}{\pgfqpoint{2.173234in}{2.758940in}}%
\pgfusepath{clip}%
\pgfsetbuttcap%
\pgfsetroundjoin%
\definecolor{currentfill}{rgb}{0.792469,0.821908,0.870863}%
\pgfsetfillcolor{currentfill}%
\pgfsetlinewidth{0.602250pt}%
\definecolor{currentstroke}{rgb}{0.296471,0.296471,0.296471}%
\pgfsetstrokecolor{currentstroke}%
\pgfsetdash{}{0pt}%
\pgfpathmoveto{\pgfqpoint{2.876564in}{2.994325in}}%
\pgfpathlineto{\pgfqpoint{2.876564in}{2.994325in}}%
\pgfpathlineto{\pgfqpoint{2.876564in}{2.991861in}}%
\pgfpathlineto{\pgfqpoint{2.876564in}{2.991861in}}%
\pgfpathlineto{\pgfqpoint{2.876564in}{2.994325in}}%
\pgfpathclose%
\pgfusepath{stroke,fill}%
\end{pgfscope}%
\begin{pgfscope}%
\pgfpathrectangle{\pgfqpoint{0.703330in}{0.352393in}}{\pgfqpoint{2.173234in}{2.758940in}}%
\pgfusepath{clip}%
\pgfsetbuttcap%
\pgfsetroundjoin%
\definecolor{currentfill}{rgb}{0.825117,0.848522,0.887698}%
\pgfsetfillcolor{currentfill}%
\pgfsetlinewidth{0.602250pt}%
\definecolor{currentstroke}{rgb}{0.296471,0.296471,0.296471}%
\pgfsetstrokecolor{currentstroke}%
\pgfsetdash{}{0pt}%
\pgfpathmoveto{\pgfqpoint{2.876564in}{2.993709in}}%
\pgfpathlineto{\pgfqpoint{2.876564in}{2.993709in}}%
\pgfpathlineto{\pgfqpoint{2.876564in}{2.992477in}}%
\pgfpathlineto{\pgfqpoint{2.876564in}{2.992477in}}%
\pgfpathlineto{\pgfqpoint{2.876564in}{2.993709in}}%
\pgfpathclose%
\pgfusepath{stroke,fill}%
\end{pgfscope}%
\begin{pgfscope}%
\pgfpathrectangle{\pgfqpoint{0.703330in}{0.352393in}}{\pgfqpoint{2.173234in}{2.758940in}}%
\pgfusepath{clip}%
\pgfsetbuttcap%
\pgfsetroundjoin%
\definecolor{currentfill}{rgb}{0.848437,0.867532,0.899724}%
\pgfsetfillcolor{currentfill}%
\pgfsetlinewidth{0.602250pt}%
\definecolor{currentstroke}{rgb}{0.296471,0.296471,0.296471}%
\pgfsetstrokecolor{currentstroke}%
\pgfsetdash{}{0pt}%
\pgfpathmoveto{\pgfqpoint{2.876564in}{2.993401in}}%
\pgfpathlineto{\pgfqpoint{2.876564in}{2.993401in}}%
\pgfpathlineto{\pgfqpoint{2.876564in}{2.992785in}}%
\pgfpathlineto{\pgfqpoint{2.876564in}{2.992785in}}%
\pgfpathlineto{\pgfqpoint{2.876564in}{2.993401in}}%
\pgfpathclose%
\pgfusepath{stroke,fill}%
\end{pgfscope}%
\begin{pgfscope}%
\pgfpathrectangle{\pgfqpoint{0.703330in}{0.352393in}}{\pgfqpoint{2.173234in}{2.758940in}}%
\pgfusepath{clip}%
\pgfsetbuttcap%
\pgfsetroundjoin%
\pgfsetlinewidth{0.803000pt}%
\definecolor{currentstroke}{rgb}{0.450000,0.450000,0.450000}%
\pgfsetstrokecolor{currentstroke}%
\pgfsetdash{}{0pt}%
\pgfpathmoveto{\pgfqpoint{0.000000in}{-0.034722in}}%
\pgfpathcurveto{\pgfqpoint{0.009208in}{-0.034722in}}{\pgfqpoint{0.018041in}{-0.031064in}}{\pgfqpoint{0.024552in}{-0.024552in}}%
\pgfpathcurveto{\pgfqpoint{0.031064in}{-0.018041in}}{\pgfqpoint{0.034722in}{-0.009208in}}{\pgfqpoint{0.034722in}{0.000000in}}%
\pgfpathcurveto{\pgfqpoint{0.034722in}{0.009208in}}{\pgfqpoint{0.031064in}{0.018041in}}{\pgfqpoint{0.024552in}{0.024552in}}%
\pgfpathcurveto{\pgfqpoint{0.018041in}{0.031064in}}{\pgfqpoint{0.009208in}{0.034722in}}{\pgfqpoint{0.000000in}{0.034722in}}%
\pgfpathcurveto{\pgfqpoint{-0.009208in}{0.034722in}}{\pgfqpoint{-0.018041in}{0.031064in}}{\pgfqpoint{-0.024552in}{0.024552in}}%
\pgfpathcurveto{\pgfqpoint{-0.031064in}{0.018041in}}{\pgfqpoint{-0.034722in}{0.009208in}}{\pgfqpoint{-0.034722in}{0.000000in}}%
\pgfpathcurveto{\pgfqpoint{-0.034722in}{-0.009208in}}{\pgfqpoint{-0.031064in}{-0.018041in}}{\pgfqpoint{-0.024552in}{-0.024552in}}%
\pgfpathcurveto{\pgfqpoint{-0.018041in}{-0.031064in}}{\pgfqpoint{-0.009208in}{-0.034722in}}{\pgfqpoint{0.000000in}{-0.034722in}}%
\pgfusepath{stroke}%
\end{pgfscope}%
\begin{pgfscope}%
\pgfpathrectangle{\pgfqpoint{0.703330in}{0.352393in}}{\pgfqpoint{2.173234in}{2.758940in}}%
\pgfusepath{clip}%
\pgfsetbuttcap%
\pgfsetroundjoin%
\definecolor{currentfill}{rgb}{0.919097,0.862812,0.832112}%
\pgfsetfillcolor{currentfill}%
\pgfsetlinewidth{0.602250pt}%
\definecolor{currentstroke}{rgb}{0.296471,0.296471,0.296471}%
\pgfsetstrokecolor{currentstroke}%
\pgfsetdash{}{0pt}%
\pgfpathmoveto{\pgfqpoint{2.374451in}{2.836055in}}%
\pgfpathlineto{\pgfqpoint{2.414892in}{2.836055in}}%
\pgfpathlineto{\pgfqpoint{2.414892in}{2.834823in}}%
\pgfpathlineto{\pgfqpoint{2.374451in}{2.834823in}}%
\pgfpathlineto{\pgfqpoint{2.374451in}{2.836055in}}%
\pgfpathclose%
\pgfusepath{stroke,fill}%
\end{pgfscope}%
\begin{pgfscope}%
\pgfpathrectangle{\pgfqpoint{0.703330in}{0.352393in}}{\pgfqpoint{2.173234in}{2.758940in}}%
\pgfusepath{clip}%
\pgfsetbuttcap%
\pgfsetroundjoin%
\definecolor{currentfill}{rgb}{0.910863,0.840546,0.801899}%
\pgfsetfillcolor{currentfill}%
\pgfsetlinewidth{0.602250pt}%
\definecolor{currentstroke}{rgb}{0.296471,0.296471,0.296471}%
\pgfsetstrokecolor{currentstroke}%
\pgfsetdash{}{0pt}%
\pgfpathmoveto{\pgfqpoint{2.414892in}{2.836671in}}%
\pgfpathlineto{\pgfqpoint{2.443764in}{2.836671in}}%
\pgfpathlineto{\pgfqpoint{2.443764in}{2.834208in}}%
\pgfpathlineto{\pgfqpoint{2.414892in}{2.834208in}}%
\pgfpathlineto{\pgfqpoint{2.414892in}{2.836671in}}%
\pgfpathclose%
\pgfusepath{stroke,fill}%
\end{pgfscope}%
\begin{pgfscope}%
\pgfpathrectangle{\pgfqpoint{0.703330in}{0.352393in}}{\pgfqpoint{2.173234in}{2.758940in}}%
\pgfusepath{clip}%
\pgfsetbuttcap%
\pgfsetroundjoin%
\definecolor{currentfill}{rgb}{0.901453,0.815098,0.767370}%
\pgfsetfillcolor{currentfill}%
\pgfsetlinewidth{0.602250pt}%
\definecolor{currentstroke}{rgb}{0.296471,0.296471,0.296471}%
\pgfsetstrokecolor{currentstroke}%
\pgfsetdash{}{0pt}%
\pgfpathmoveto{\pgfqpoint{2.443764in}{2.837903in}}%
\pgfpathlineto{\pgfqpoint{2.467803in}{2.837903in}}%
\pgfpathlineto{\pgfqpoint{2.467803in}{2.832976in}}%
\pgfpathlineto{\pgfqpoint{2.443764in}{2.832976in}}%
\pgfpathlineto{\pgfqpoint{2.443764in}{2.837903in}}%
\pgfpathclose%
\pgfusepath{stroke,fill}%
\end{pgfscope}%
\begin{pgfscope}%
\pgfpathrectangle{\pgfqpoint{0.703330in}{0.352393in}}{\pgfqpoint{2.173234in}{2.758940in}}%
\pgfusepath{clip}%
\pgfsetbuttcap%
\pgfsetroundjoin%
\definecolor{currentfill}{rgb}{0.889102,0.781698,0.722050}%
\pgfsetfillcolor{currentfill}%
\pgfsetlinewidth{0.602250pt}%
\definecolor{currentstroke}{rgb}{0.296471,0.296471,0.296471}%
\pgfsetstrokecolor{currentstroke}%
\pgfsetdash{}{0pt}%
\pgfpathmoveto{\pgfqpoint{2.467803in}{2.840366in}}%
\pgfpathlineto{\pgfqpoint{2.502564in}{2.840366in}}%
\pgfpathlineto{\pgfqpoint{2.502564in}{2.830513in}}%
\pgfpathlineto{\pgfqpoint{2.467803in}{2.830513in}}%
\pgfpathlineto{\pgfqpoint{2.467803in}{2.840366in}}%
\pgfpathclose%
\pgfusepath{stroke,fill}%
\end{pgfscope}%
\begin{pgfscope}%
\pgfpathrectangle{\pgfqpoint{0.703330in}{0.352393in}}{\pgfqpoint{2.173234in}{2.758940in}}%
\pgfusepath{clip}%
\pgfsetbuttcap%
\pgfsetroundjoin%
\definecolor{currentfill}{rgb}{0.873223,0.738755,0.663782}%
\pgfsetfillcolor{currentfill}%
\pgfsetlinewidth{0.602250pt}%
\definecolor{currentstroke}{rgb}{0.296471,0.296471,0.296471}%
\pgfsetstrokecolor{currentstroke}%
\pgfsetdash{}{0pt}%
\pgfpathmoveto{\pgfqpoint{2.502564in}{2.845293in}}%
\pgfpathlineto{\pgfqpoint{2.526178in}{2.845293in}}%
\pgfpathlineto{\pgfqpoint{2.526178in}{2.825586in}}%
\pgfpathlineto{\pgfqpoint{2.502564in}{2.825586in}}%
\pgfpathlineto{\pgfqpoint{2.502564in}{2.845293in}}%
\pgfpathclose%
\pgfusepath{stroke,fill}%
\end{pgfscope}%
\begin{pgfscope}%
\pgfpathrectangle{\pgfqpoint{0.703330in}{0.352393in}}{\pgfqpoint{2.173234in}{2.758940in}}%
\pgfusepath{clip}%
\pgfsetbuttcap%
\pgfsetroundjoin%
\definecolor{currentfill}{rgb}{0.853814,0.686269,0.592565}%
\pgfsetfillcolor{currentfill}%
\pgfsetlinewidth{0.602250pt}%
\definecolor{currentstroke}{rgb}{0.296471,0.296471,0.296471}%
\pgfsetstrokecolor{currentstroke}%
\pgfsetdash{}{0pt}%
\pgfpathmoveto{\pgfqpoint{2.526178in}{2.855146in}}%
\pgfpathlineto{\pgfqpoint{2.573578in}{2.855146in}}%
\pgfpathlineto{\pgfqpoint{2.573578in}{2.815733in}}%
\pgfpathlineto{\pgfqpoint{2.526178in}{2.815733in}}%
\pgfpathlineto{\pgfqpoint{2.526178in}{2.855146in}}%
\pgfpathclose%
\pgfusepath{stroke,fill}%
\end{pgfscope}%
\begin{pgfscope}%
\pgfpathrectangle{\pgfqpoint{0.703330in}{0.352393in}}{\pgfqpoint{2.173234in}{2.758940in}}%
\pgfusepath{clip}%
\pgfsetbuttcap%
\pgfsetroundjoin%
\definecolor{currentfill}{rgb}{0.829112,0.619469,0.501926}%
\pgfsetfillcolor{currentfill}%
\pgfsetlinewidth{0.602250pt}%
\definecolor{currentstroke}{rgb}{0.296471,0.296471,0.296471}%
\pgfsetstrokecolor{currentstroke}%
\pgfsetdash{}{0pt}%
\pgfpathmoveto{\pgfqpoint{2.573578in}{2.874853in}}%
\pgfpathlineto{\pgfqpoint{2.646086in}{2.874853in}}%
\pgfpathlineto{\pgfqpoint{2.646086in}{2.796026in}}%
\pgfpathlineto{\pgfqpoint{2.573578in}{2.796026in}}%
\pgfpathlineto{\pgfqpoint{2.573578in}{2.874853in}}%
\pgfpathclose%
\pgfusepath{stroke,fill}%
\end{pgfscope}%
\begin{pgfscope}%
\pgfpathrectangle{\pgfqpoint{0.703330in}{0.352393in}}{\pgfqpoint{2.173234in}{2.758940in}}%
\pgfusepath{clip}%
\pgfsetbuttcap%
\pgfsetroundjoin%
\definecolor{currentfill}{rgb}{0.798529,0.536765,0.389706}%
\pgfsetfillcolor{currentfill}%
\pgfsetlinewidth{0.602250pt}%
\definecolor{currentstroke}{rgb}{0.296471,0.296471,0.296471}%
\pgfsetstrokecolor{currentstroke}%
\pgfsetdash{}{0pt}%
\pgfpathmoveto{\pgfqpoint{2.646086in}{2.914266in}}%
\pgfpathlineto{\pgfqpoint{2.876564in}{2.914266in}}%
\pgfpathlineto{\pgfqpoint{2.876564in}{2.756612in}}%
\pgfpathlineto{\pgfqpoint{2.646086in}{2.756612in}}%
\pgfpathlineto{\pgfqpoint{2.646086in}{2.914266in}}%
\pgfpathclose%
\pgfusepath{stroke,fill}%
\end{pgfscope}%
\begin{pgfscope}%
\pgfpathrectangle{\pgfqpoint{0.703330in}{0.352393in}}{\pgfqpoint{2.173234in}{2.758940in}}%
\pgfusepath{clip}%
\pgfsetbuttcap%
\pgfsetroundjoin%
\definecolor{currentfill}{rgb}{0.829112,0.619469,0.501926}%
\pgfsetfillcolor{currentfill}%
\pgfsetlinewidth{0.602250pt}%
\definecolor{currentstroke}{rgb}{0.296471,0.296471,0.296471}%
\pgfsetstrokecolor{currentstroke}%
\pgfsetdash{}{0pt}%
\pgfpathmoveto{\pgfqpoint{2.876564in}{2.874853in}}%
\pgfpathlineto{\pgfqpoint{2.876564in}{2.874853in}}%
\pgfpathlineto{\pgfqpoint{2.876564in}{2.796026in}}%
\pgfpathlineto{\pgfqpoint{2.876564in}{2.796026in}}%
\pgfpathlineto{\pgfqpoint{2.876564in}{2.874853in}}%
\pgfpathclose%
\pgfusepath{stroke,fill}%
\end{pgfscope}%
\begin{pgfscope}%
\pgfpathrectangle{\pgfqpoint{0.703330in}{0.352393in}}{\pgfqpoint{2.173234in}{2.758940in}}%
\pgfusepath{clip}%
\pgfsetbuttcap%
\pgfsetroundjoin%
\definecolor{currentfill}{rgb}{0.853814,0.686269,0.592565}%
\pgfsetfillcolor{currentfill}%
\pgfsetlinewidth{0.602250pt}%
\definecolor{currentstroke}{rgb}{0.296471,0.296471,0.296471}%
\pgfsetstrokecolor{currentstroke}%
\pgfsetdash{}{0pt}%
\pgfpathmoveto{\pgfqpoint{2.876564in}{2.855146in}}%
\pgfpathlineto{\pgfqpoint{2.876564in}{2.855146in}}%
\pgfpathlineto{\pgfqpoint{2.876564in}{2.815733in}}%
\pgfpathlineto{\pgfqpoint{2.876564in}{2.815733in}}%
\pgfpathlineto{\pgfqpoint{2.876564in}{2.855146in}}%
\pgfpathclose%
\pgfusepath{stroke,fill}%
\end{pgfscope}%
\begin{pgfscope}%
\pgfpathrectangle{\pgfqpoint{0.703330in}{0.352393in}}{\pgfqpoint{2.173234in}{2.758940in}}%
\pgfusepath{clip}%
\pgfsetbuttcap%
\pgfsetroundjoin%
\definecolor{currentfill}{rgb}{0.873223,0.738755,0.663782}%
\pgfsetfillcolor{currentfill}%
\pgfsetlinewidth{0.602250pt}%
\definecolor{currentstroke}{rgb}{0.296471,0.296471,0.296471}%
\pgfsetstrokecolor{currentstroke}%
\pgfsetdash{}{0pt}%
\pgfpathmoveto{\pgfqpoint{2.876564in}{2.845293in}}%
\pgfpathlineto{\pgfqpoint{2.876564in}{2.845293in}}%
\pgfpathlineto{\pgfqpoint{2.876564in}{2.825586in}}%
\pgfpathlineto{\pgfqpoint{2.876564in}{2.825586in}}%
\pgfpathlineto{\pgfqpoint{2.876564in}{2.845293in}}%
\pgfpathclose%
\pgfusepath{stroke,fill}%
\end{pgfscope}%
\begin{pgfscope}%
\pgfpathrectangle{\pgfqpoint{0.703330in}{0.352393in}}{\pgfqpoint{2.173234in}{2.758940in}}%
\pgfusepath{clip}%
\pgfsetbuttcap%
\pgfsetroundjoin%
\definecolor{currentfill}{rgb}{0.889102,0.781698,0.722050}%
\pgfsetfillcolor{currentfill}%
\pgfsetlinewidth{0.602250pt}%
\definecolor{currentstroke}{rgb}{0.296471,0.296471,0.296471}%
\pgfsetstrokecolor{currentstroke}%
\pgfsetdash{}{0pt}%
\pgfpathmoveto{\pgfqpoint{2.876564in}{2.840366in}}%
\pgfpathlineto{\pgfqpoint{2.876564in}{2.840366in}}%
\pgfpathlineto{\pgfqpoint{2.876564in}{2.830513in}}%
\pgfpathlineto{\pgfqpoint{2.876564in}{2.830513in}}%
\pgfpathlineto{\pgfqpoint{2.876564in}{2.840366in}}%
\pgfpathclose%
\pgfusepath{stroke,fill}%
\end{pgfscope}%
\begin{pgfscope}%
\pgfpathrectangle{\pgfqpoint{0.703330in}{0.352393in}}{\pgfqpoint{2.173234in}{2.758940in}}%
\pgfusepath{clip}%
\pgfsetbuttcap%
\pgfsetroundjoin%
\definecolor{currentfill}{rgb}{0.901453,0.815098,0.767370}%
\pgfsetfillcolor{currentfill}%
\pgfsetlinewidth{0.602250pt}%
\definecolor{currentstroke}{rgb}{0.296471,0.296471,0.296471}%
\pgfsetstrokecolor{currentstroke}%
\pgfsetdash{}{0pt}%
\pgfpathmoveto{\pgfqpoint{2.876564in}{2.837903in}}%
\pgfpathlineto{\pgfqpoint{2.876564in}{2.837903in}}%
\pgfpathlineto{\pgfqpoint{2.876564in}{2.832976in}}%
\pgfpathlineto{\pgfqpoint{2.876564in}{2.832976in}}%
\pgfpathlineto{\pgfqpoint{2.876564in}{2.837903in}}%
\pgfpathclose%
\pgfusepath{stroke,fill}%
\end{pgfscope}%
\begin{pgfscope}%
\pgfpathrectangle{\pgfqpoint{0.703330in}{0.352393in}}{\pgfqpoint{2.173234in}{2.758940in}}%
\pgfusepath{clip}%
\pgfsetbuttcap%
\pgfsetroundjoin%
\definecolor{currentfill}{rgb}{0.910863,0.840546,0.801899}%
\pgfsetfillcolor{currentfill}%
\pgfsetlinewidth{0.602250pt}%
\definecolor{currentstroke}{rgb}{0.296471,0.296471,0.296471}%
\pgfsetstrokecolor{currentstroke}%
\pgfsetdash{}{0pt}%
\pgfpathmoveto{\pgfqpoint{2.876564in}{2.836671in}}%
\pgfpathlineto{\pgfqpoint{2.876564in}{2.836671in}}%
\pgfpathlineto{\pgfqpoint{2.876564in}{2.834208in}}%
\pgfpathlineto{\pgfqpoint{2.876564in}{2.834208in}}%
\pgfpathlineto{\pgfqpoint{2.876564in}{2.836671in}}%
\pgfpathclose%
\pgfusepath{stroke,fill}%
\end{pgfscope}%
\begin{pgfscope}%
\pgfpathrectangle{\pgfqpoint{0.703330in}{0.352393in}}{\pgfqpoint{2.173234in}{2.758940in}}%
\pgfusepath{clip}%
\pgfsetbuttcap%
\pgfsetroundjoin%
\definecolor{currentfill}{rgb}{0.919097,0.862812,0.832112}%
\pgfsetfillcolor{currentfill}%
\pgfsetlinewidth{0.602250pt}%
\definecolor{currentstroke}{rgb}{0.296471,0.296471,0.296471}%
\pgfsetstrokecolor{currentstroke}%
\pgfsetdash{}{0pt}%
\pgfpathmoveto{\pgfqpoint{2.876564in}{2.836055in}}%
\pgfpathlineto{\pgfqpoint{2.876564in}{2.836055in}}%
\pgfpathlineto{\pgfqpoint{2.876564in}{2.834823in}}%
\pgfpathlineto{\pgfqpoint{2.876564in}{2.834823in}}%
\pgfpathlineto{\pgfqpoint{2.876564in}{2.836055in}}%
\pgfpathclose%
\pgfusepath{stroke,fill}%
\end{pgfscope}%
\begin{pgfscope}%
\pgfpathrectangle{\pgfqpoint{0.703330in}{0.352393in}}{\pgfqpoint{2.173234in}{2.758940in}}%
\pgfusepath{clip}%
\pgfsetbuttcap%
\pgfsetroundjoin%
\pgfsetlinewidth{0.803000pt}%
\definecolor{currentstroke}{rgb}{0.450000,0.450000,0.450000}%
\pgfsetstrokecolor{currentstroke}%
\pgfsetdash{}{0pt}%
\pgfpathmoveto{\pgfqpoint{0.000000in}{-0.034722in}}%
\pgfpathcurveto{\pgfqpoint{0.009208in}{-0.034722in}}{\pgfqpoint{0.018041in}{-0.031064in}}{\pgfqpoint{0.024552in}{-0.024552in}}%
\pgfpathcurveto{\pgfqpoint{0.031064in}{-0.018041in}}{\pgfqpoint{0.034722in}{-0.009208in}}{\pgfqpoint{0.034722in}{0.000000in}}%
\pgfpathcurveto{\pgfqpoint{0.034722in}{0.009208in}}{\pgfqpoint{0.031064in}{0.018041in}}{\pgfqpoint{0.024552in}{0.024552in}}%
\pgfpathcurveto{\pgfqpoint{0.018041in}{0.031064in}}{\pgfqpoint{0.009208in}{0.034722in}}{\pgfqpoint{0.000000in}{0.034722in}}%
\pgfpathcurveto{\pgfqpoint{-0.009208in}{0.034722in}}{\pgfqpoint{-0.018041in}{0.031064in}}{\pgfqpoint{-0.024552in}{0.024552in}}%
\pgfpathcurveto{\pgfqpoint{-0.031064in}{0.018041in}}{\pgfqpoint{-0.034722in}{0.009208in}}{\pgfqpoint{-0.034722in}{0.000000in}}%
\pgfpathcurveto{\pgfqpoint{-0.034722in}{-0.009208in}}{\pgfqpoint{-0.031064in}{-0.018041in}}{\pgfqpoint{-0.024552in}{-0.024552in}}%
\pgfpathcurveto{\pgfqpoint{-0.018041in}{-0.031064in}}{\pgfqpoint{-0.009208in}{-0.034722in}}{\pgfqpoint{0.000000in}{-0.034722in}}%
\pgfusepath{stroke}%
\end{pgfscope}%
\begin{pgfscope}%
\pgfpathrectangle{\pgfqpoint{0.703330in}{0.352393in}}{\pgfqpoint{2.173234in}{2.758940in}}%
\pgfusepath{clip}%
\pgfsetbuttcap%
\pgfsetroundjoin%
\definecolor{currentfill}{rgb}{0.848437,0.867532,0.899724}%
\pgfsetfillcolor{currentfill}%
\pgfsetlinewidth{0.602250pt}%
\definecolor{currentstroke}{rgb}{0.296471,0.296471,0.296471}%
\pgfsetstrokecolor{currentstroke}%
\pgfsetdash{}{0pt}%
\pgfpathmoveto{\pgfqpoint{0.817983in}{2.599267in}}%
\pgfpathlineto{\pgfqpoint{0.833545in}{2.599267in}}%
\pgfpathlineto{\pgfqpoint{0.833545in}{2.598651in}}%
\pgfpathlineto{\pgfqpoint{0.817983in}{2.598651in}}%
\pgfpathlineto{\pgfqpoint{0.817983in}{2.599267in}}%
\pgfpathclose%
\pgfusepath{stroke,fill}%
\end{pgfscope}%
\begin{pgfscope}%
\pgfpathrectangle{\pgfqpoint{0.703330in}{0.352393in}}{\pgfqpoint{2.173234in}{2.758940in}}%
\pgfusepath{clip}%
\pgfsetbuttcap%
\pgfsetroundjoin%
\definecolor{currentfill}{rgb}{0.825117,0.848522,0.887698}%
\pgfsetfillcolor{currentfill}%
\pgfsetlinewidth{0.602250pt}%
\definecolor{currentstroke}{rgb}{0.296471,0.296471,0.296471}%
\pgfsetstrokecolor{currentstroke}%
\pgfsetdash{}{0pt}%
\pgfpathmoveto{\pgfqpoint{0.833545in}{2.599575in}}%
\pgfpathlineto{\pgfqpoint{0.845255in}{2.599575in}}%
\pgfpathlineto{\pgfqpoint{0.845255in}{2.598343in}}%
\pgfpathlineto{\pgfqpoint{0.833545in}{2.598343in}}%
\pgfpathlineto{\pgfqpoint{0.833545in}{2.599575in}}%
\pgfpathclose%
\pgfusepath{stroke,fill}%
\end{pgfscope}%
\begin{pgfscope}%
\pgfpathrectangle{\pgfqpoint{0.703330in}{0.352393in}}{\pgfqpoint{2.173234in}{2.758940in}}%
\pgfusepath{clip}%
\pgfsetbuttcap%
\pgfsetroundjoin%
\definecolor{currentfill}{rgb}{0.792469,0.821908,0.870863}%
\pgfsetfillcolor{currentfill}%
\pgfsetlinewidth{0.602250pt}%
\definecolor{currentstroke}{rgb}{0.296471,0.296471,0.296471}%
\pgfsetstrokecolor{currentstroke}%
\pgfsetdash{}{0pt}%
\pgfpathmoveto{\pgfqpoint{0.845255in}{2.600190in}}%
\pgfpathlineto{\pgfqpoint{0.881596in}{2.600190in}}%
\pgfpathlineto{\pgfqpoint{0.881596in}{2.597727in}}%
\pgfpathlineto{\pgfqpoint{0.845255in}{2.597727in}}%
\pgfpathlineto{\pgfqpoint{0.845255in}{2.600190in}}%
\pgfpathclose%
\pgfusepath{stroke,fill}%
\end{pgfscope}%
\begin{pgfscope}%
\pgfpathrectangle{\pgfqpoint{0.703330in}{0.352393in}}{\pgfqpoint{2.173234in}{2.758940in}}%
\pgfusepath{clip}%
\pgfsetbuttcap%
\pgfsetroundjoin%
\definecolor{currentfill}{rgb}{0.755157,0.791493,0.851622}%
\pgfsetfillcolor{currentfill}%
\pgfsetlinewidth{0.602250pt}%
\definecolor{currentstroke}{rgb}{0.296471,0.296471,0.296471}%
\pgfsetstrokecolor{currentstroke}%
\pgfsetdash{}{0pt}%
\pgfpathmoveto{\pgfqpoint{0.881596in}{2.601422in}}%
\pgfpathlineto{\pgfqpoint{1.137513in}{2.601422in}}%
\pgfpathlineto{\pgfqpoint{1.137513in}{2.596495in}}%
\pgfpathlineto{\pgfqpoint{0.881596in}{2.596495in}}%
\pgfpathlineto{\pgfqpoint{0.881596in}{2.601422in}}%
\pgfpathclose%
\pgfusepath{stroke,fill}%
\end{pgfscope}%
\begin{pgfscope}%
\pgfpathrectangle{\pgfqpoint{0.703330in}{0.352393in}}{\pgfqpoint{2.173234in}{2.758940in}}%
\pgfusepath{clip}%
\pgfsetbuttcap%
\pgfsetroundjoin%
\definecolor{currentfill}{rgb}{0.706185,0.751573,0.826368}%
\pgfsetfillcolor{currentfill}%
\pgfsetlinewidth{0.602250pt}%
\definecolor{currentstroke}{rgb}{0.296471,0.296471,0.296471}%
\pgfsetstrokecolor{currentstroke}%
\pgfsetdash{}{0pt}%
\pgfpathmoveto{\pgfqpoint{1.137513in}{2.603885in}}%
\pgfpathlineto{\pgfqpoint{1.193561in}{2.603885in}}%
\pgfpathlineto{\pgfqpoint{1.193561in}{2.594032in}}%
\pgfpathlineto{\pgfqpoint{1.137513in}{2.594032in}}%
\pgfpathlineto{\pgfqpoint{1.137513in}{2.603885in}}%
\pgfpathclose%
\pgfusepath{stroke,fill}%
\end{pgfscope}%
\begin{pgfscope}%
\pgfpathrectangle{\pgfqpoint{0.703330in}{0.352393in}}{\pgfqpoint{2.173234in}{2.758940in}}%
\pgfusepath{clip}%
\pgfsetbuttcap%
\pgfsetroundjoin%
\definecolor{currentfill}{rgb}{0.643221,0.700246,0.793900}%
\pgfsetfillcolor{currentfill}%
\pgfsetlinewidth{0.602250pt}%
\definecolor{currentstroke}{rgb}{0.296471,0.296471,0.296471}%
\pgfsetstrokecolor{currentstroke}%
\pgfsetdash{}{0pt}%
\pgfpathmoveto{\pgfqpoint{1.193561in}{2.608812in}}%
\pgfpathlineto{\pgfqpoint{1.268308in}{2.608812in}}%
\pgfpathlineto{\pgfqpoint{1.268308in}{2.589105in}}%
\pgfpathlineto{\pgfqpoint{1.193561in}{2.589105in}}%
\pgfpathlineto{\pgfqpoint{1.193561in}{2.608812in}}%
\pgfpathclose%
\pgfusepath{stroke,fill}%
\end{pgfscope}%
\begin{pgfscope}%
\pgfpathrectangle{\pgfqpoint{0.703330in}{0.352393in}}{\pgfqpoint{2.173234in}{2.758940in}}%
\pgfusepath{clip}%
\pgfsetbuttcap%
\pgfsetroundjoin%
\definecolor{currentfill}{rgb}{0.566266,0.637515,0.754216}%
\pgfsetfillcolor{currentfill}%
\pgfsetlinewidth{0.602250pt}%
\definecolor{currentstroke}{rgb}{0.296471,0.296471,0.296471}%
\pgfsetstrokecolor{currentstroke}%
\pgfsetdash{}{0pt}%
\pgfpathmoveto{\pgfqpoint{1.268308in}{2.618665in}}%
\pgfpathlineto{\pgfqpoint{1.400350in}{2.618665in}}%
\pgfpathlineto{\pgfqpoint{1.400350in}{2.579252in}}%
\pgfpathlineto{\pgfqpoint{1.268308in}{2.579252in}}%
\pgfpathlineto{\pgfqpoint{1.268308in}{2.618665in}}%
\pgfpathclose%
\pgfusepath{stroke,fill}%
\end{pgfscope}%
\begin{pgfscope}%
\pgfpathrectangle{\pgfqpoint{0.703330in}{0.352393in}}{\pgfqpoint{2.173234in}{2.758940in}}%
\pgfusepath{clip}%
\pgfsetbuttcap%
\pgfsetroundjoin%
\definecolor{currentfill}{rgb}{0.468322,0.557674,0.703709}%
\pgfsetfillcolor{currentfill}%
\pgfsetlinewidth{0.602250pt}%
\definecolor{currentstroke}{rgb}{0.296471,0.296471,0.296471}%
\pgfsetstrokecolor{currentstroke}%
\pgfsetdash{}{0pt}%
\pgfpathmoveto{\pgfqpoint{1.400350in}{2.638372in}}%
\pgfpathlineto{\pgfqpoint{1.660033in}{2.638372in}}%
\pgfpathlineto{\pgfqpoint{1.660033in}{2.559545in}}%
\pgfpathlineto{\pgfqpoint{1.400350in}{2.559545in}}%
\pgfpathlineto{\pgfqpoint{1.400350in}{2.638372in}}%
\pgfpathclose%
\pgfusepath{stroke,fill}%
\end{pgfscope}%
\begin{pgfscope}%
\pgfpathrectangle{\pgfqpoint{0.703330in}{0.352393in}}{\pgfqpoint{2.173234in}{2.758940in}}%
\pgfusepath{clip}%
\pgfsetbuttcap%
\pgfsetroundjoin%
\definecolor{currentfill}{rgb}{0.347059,0.458824,0.641176}%
\pgfsetfillcolor{currentfill}%
\pgfsetlinewidth{0.602250pt}%
\definecolor{currentstroke}{rgb}{0.296471,0.296471,0.296471}%
\pgfsetstrokecolor{currentstroke}%
\pgfsetdash{}{0pt}%
\pgfpathmoveto{\pgfqpoint{1.660033in}{2.677786in}}%
\pgfpathlineto{\pgfqpoint{2.305152in}{2.677786in}}%
\pgfpathlineto{\pgfqpoint{2.305152in}{2.520132in}}%
\pgfpathlineto{\pgfqpoint{1.660033in}{2.520132in}}%
\pgfpathlineto{\pgfqpoint{1.660033in}{2.677786in}}%
\pgfpathclose%
\pgfusepath{stroke,fill}%
\end{pgfscope}%
\begin{pgfscope}%
\pgfpathrectangle{\pgfqpoint{0.703330in}{0.352393in}}{\pgfqpoint{2.173234in}{2.758940in}}%
\pgfusepath{clip}%
\pgfsetbuttcap%
\pgfsetroundjoin%
\definecolor{currentfill}{rgb}{0.468322,0.557674,0.703709}%
\pgfsetfillcolor{currentfill}%
\pgfsetlinewidth{0.602250pt}%
\definecolor{currentstroke}{rgb}{0.296471,0.296471,0.296471}%
\pgfsetstrokecolor{currentstroke}%
\pgfsetdash{}{0pt}%
\pgfpathmoveto{\pgfqpoint{2.305152in}{2.638372in}}%
\pgfpathlineto{\pgfqpoint{2.646442in}{2.638372in}}%
\pgfpathlineto{\pgfqpoint{2.646442in}{2.559545in}}%
\pgfpathlineto{\pgfqpoint{2.305152in}{2.559545in}}%
\pgfpathlineto{\pgfqpoint{2.305152in}{2.638372in}}%
\pgfpathclose%
\pgfusepath{stroke,fill}%
\end{pgfscope}%
\begin{pgfscope}%
\pgfpathrectangle{\pgfqpoint{0.703330in}{0.352393in}}{\pgfqpoint{2.173234in}{2.758940in}}%
\pgfusepath{clip}%
\pgfsetbuttcap%
\pgfsetroundjoin%
\definecolor{currentfill}{rgb}{0.566266,0.637515,0.754216}%
\pgfsetfillcolor{currentfill}%
\pgfsetlinewidth{0.602250pt}%
\definecolor{currentstroke}{rgb}{0.296471,0.296471,0.296471}%
\pgfsetstrokecolor{currentstroke}%
\pgfsetdash{}{0pt}%
\pgfpathmoveto{\pgfqpoint{2.646442in}{2.618665in}}%
\pgfpathlineto{\pgfqpoint{2.836545in}{2.618665in}}%
\pgfpathlineto{\pgfqpoint{2.836545in}{2.579252in}}%
\pgfpathlineto{\pgfqpoint{2.646442in}{2.579252in}}%
\pgfpathlineto{\pgfqpoint{2.646442in}{2.618665in}}%
\pgfpathclose%
\pgfusepath{stroke,fill}%
\end{pgfscope}%
\begin{pgfscope}%
\pgfpathrectangle{\pgfqpoint{0.703330in}{0.352393in}}{\pgfqpoint{2.173234in}{2.758940in}}%
\pgfusepath{clip}%
\pgfsetbuttcap%
\pgfsetroundjoin%
\definecolor{currentfill}{rgb}{0.643221,0.700246,0.793900}%
\pgfsetfillcolor{currentfill}%
\pgfsetlinewidth{0.602250pt}%
\definecolor{currentstroke}{rgb}{0.296471,0.296471,0.296471}%
\pgfsetstrokecolor{currentstroke}%
\pgfsetdash{}{0pt}%
\pgfpathmoveto{\pgfqpoint{2.836545in}{2.608812in}}%
\pgfpathlineto{\pgfqpoint{2.876564in}{2.608812in}}%
\pgfpathlineto{\pgfqpoint{2.876564in}{2.589105in}}%
\pgfpathlineto{\pgfqpoint{2.836545in}{2.589105in}}%
\pgfpathlineto{\pgfqpoint{2.836545in}{2.608812in}}%
\pgfpathclose%
\pgfusepath{stroke,fill}%
\end{pgfscope}%
\begin{pgfscope}%
\pgfpathrectangle{\pgfqpoint{0.703330in}{0.352393in}}{\pgfqpoint{2.173234in}{2.758940in}}%
\pgfusepath{clip}%
\pgfsetbuttcap%
\pgfsetroundjoin%
\definecolor{currentfill}{rgb}{0.706185,0.751573,0.826368}%
\pgfsetfillcolor{currentfill}%
\pgfsetlinewidth{0.602250pt}%
\definecolor{currentstroke}{rgb}{0.296471,0.296471,0.296471}%
\pgfsetstrokecolor{currentstroke}%
\pgfsetdash{}{0pt}%
\pgfpathmoveto{\pgfqpoint{2.876564in}{2.603885in}}%
\pgfpathlineto{\pgfqpoint{2.876564in}{2.603885in}}%
\pgfpathlineto{\pgfqpoint{2.876564in}{2.594032in}}%
\pgfpathlineto{\pgfqpoint{2.876564in}{2.594032in}}%
\pgfpathlineto{\pgfqpoint{2.876564in}{2.603885in}}%
\pgfpathclose%
\pgfusepath{stroke,fill}%
\end{pgfscope}%
\begin{pgfscope}%
\pgfpathrectangle{\pgfqpoint{0.703330in}{0.352393in}}{\pgfqpoint{2.173234in}{2.758940in}}%
\pgfusepath{clip}%
\pgfsetbuttcap%
\pgfsetroundjoin%
\definecolor{currentfill}{rgb}{0.755157,0.791493,0.851622}%
\pgfsetfillcolor{currentfill}%
\pgfsetlinewidth{0.602250pt}%
\definecolor{currentstroke}{rgb}{0.296471,0.296471,0.296471}%
\pgfsetstrokecolor{currentstroke}%
\pgfsetdash{}{0pt}%
\pgfpathmoveto{\pgfqpoint{2.876564in}{2.601422in}}%
\pgfpathlineto{\pgfqpoint{2.876564in}{2.601422in}}%
\pgfpathlineto{\pgfqpoint{2.876564in}{2.596495in}}%
\pgfpathlineto{\pgfqpoint{2.876564in}{2.596495in}}%
\pgfpathlineto{\pgfqpoint{2.876564in}{2.601422in}}%
\pgfpathclose%
\pgfusepath{stroke,fill}%
\end{pgfscope}%
\begin{pgfscope}%
\pgfpathrectangle{\pgfqpoint{0.703330in}{0.352393in}}{\pgfqpoint{2.173234in}{2.758940in}}%
\pgfusepath{clip}%
\pgfsetbuttcap%
\pgfsetroundjoin%
\definecolor{currentfill}{rgb}{0.792469,0.821908,0.870863}%
\pgfsetfillcolor{currentfill}%
\pgfsetlinewidth{0.602250pt}%
\definecolor{currentstroke}{rgb}{0.296471,0.296471,0.296471}%
\pgfsetstrokecolor{currentstroke}%
\pgfsetdash{}{0pt}%
\pgfpathmoveto{\pgfqpoint{2.876564in}{2.600190in}}%
\pgfpathlineto{\pgfqpoint{2.876564in}{2.600190in}}%
\pgfpathlineto{\pgfqpoint{2.876564in}{2.597727in}}%
\pgfpathlineto{\pgfqpoint{2.876564in}{2.597727in}}%
\pgfpathlineto{\pgfqpoint{2.876564in}{2.600190in}}%
\pgfpathclose%
\pgfusepath{stroke,fill}%
\end{pgfscope}%
\begin{pgfscope}%
\pgfpathrectangle{\pgfqpoint{0.703330in}{0.352393in}}{\pgfqpoint{2.173234in}{2.758940in}}%
\pgfusepath{clip}%
\pgfsetbuttcap%
\pgfsetroundjoin%
\definecolor{currentfill}{rgb}{0.825117,0.848522,0.887698}%
\pgfsetfillcolor{currentfill}%
\pgfsetlinewidth{0.602250pt}%
\definecolor{currentstroke}{rgb}{0.296471,0.296471,0.296471}%
\pgfsetstrokecolor{currentstroke}%
\pgfsetdash{}{0pt}%
\pgfpathmoveto{\pgfqpoint{2.876564in}{2.599575in}}%
\pgfpathlineto{\pgfqpoint{2.876564in}{2.599575in}}%
\pgfpathlineto{\pgfqpoint{2.876564in}{2.598343in}}%
\pgfpathlineto{\pgfqpoint{2.876564in}{2.598343in}}%
\pgfpathlineto{\pgfqpoint{2.876564in}{2.599575in}}%
\pgfpathclose%
\pgfusepath{stroke,fill}%
\end{pgfscope}%
\begin{pgfscope}%
\pgfpathrectangle{\pgfqpoint{0.703330in}{0.352393in}}{\pgfqpoint{2.173234in}{2.758940in}}%
\pgfusepath{clip}%
\pgfsetbuttcap%
\pgfsetroundjoin%
\definecolor{currentfill}{rgb}{0.848437,0.867532,0.899724}%
\pgfsetfillcolor{currentfill}%
\pgfsetlinewidth{0.602250pt}%
\definecolor{currentstroke}{rgb}{0.296471,0.296471,0.296471}%
\pgfsetstrokecolor{currentstroke}%
\pgfsetdash{}{0pt}%
\pgfpathmoveto{\pgfqpoint{2.876564in}{2.599267in}}%
\pgfpathlineto{\pgfqpoint{2.876564in}{2.599267in}}%
\pgfpathlineto{\pgfqpoint{2.876564in}{2.598651in}}%
\pgfpathlineto{\pgfqpoint{2.876564in}{2.598651in}}%
\pgfpathlineto{\pgfqpoint{2.876564in}{2.599267in}}%
\pgfpathclose%
\pgfusepath{stroke,fill}%
\end{pgfscope}%
\begin{pgfscope}%
\pgfpathrectangle{\pgfqpoint{0.703330in}{0.352393in}}{\pgfqpoint{2.173234in}{2.758940in}}%
\pgfusepath{clip}%
\pgfsetbuttcap%
\pgfsetroundjoin%
\pgfsetlinewidth{0.803000pt}%
\definecolor{currentstroke}{rgb}{0.450000,0.450000,0.450000}%
\pgfsetstrokecolor{currentstroke}%
\pgfsetdash{}{0pt}%
\pgfpathmoveto{\pgfqpoint{0.000000in}{-0.034722in}}%
\pgfpathcurveto{\pgfqpoint{0.009208in}{-0.034722in}}{\pgfqpoint{0.018041in}{-0.031064in}}{\pgfqpoint{0.024552in}{-0.024552in}}%
\pgfpathcurveto{\pgfqpoint{0.031064in}{-0.018041in}}{\pgfqpoint{0.034722in}{-0.009208in}}{\pgfqpoint{0.034722in}{0.000000in}}%
\pgfpathcurveto{\pgfqpoint{0.034722in}{0.009208in}}{\pgfqpoint{0.031064in}{0.018041in}}{\pgfqpoint{0.024552in}{0.024552in}}%
\pgfpathcurveto{\pgfqpoint{0.018041in}{0.031064in}}{\pgfqpoint{0.009208in}{0.034722in}}{\pgfqpoint{0.000000in}{0.034722in}}%
\pgfpathcurveto{\pgfqpoint{-0.009208in}{0.034722in}}{\pgfqpoint{-0.018041in}{0.031064in}}{\pgfqpoint{-0.024552in}{0.024552in}}%
\pgfpathcurveto{\pgfqpoint{-0.031064in}{0.018041in}}{\pgfqpoint{-0.034722in}{0.009208in}}{\pgfqpoint{-0.034722in}{0.000000in}}%
\pgfpathcurveto{\pgfqpoint{-0.034722in}{-0.009208in}}{\pgfqpoint{-0.031064in}{-0.018041in}}{\pgfqpoint{-0.024552in}{-0.024552in}}%
\pgfpathcurveto{\pgfqpoint{-0.018041in}{-0.031064in}}{\pgfqpoint{-0.009208in}{-0.034722in}}{\pgfqpoint{0.000000in}{-0.034722in}}%
\pgfusepath{stroke}%
\end{pgfscope}%
\begin{pgfscope}%
\pgfpathrectangle{\pgfqpoint{0.703330in}{0.352393in}}{\pgfqpoint{2.173234in}{2.758940in}}%
\pgfusepath{clip}%
\pgfsetbuttcap%
\pgfsetroundjoin%
\definecolor{currentfill}{rgb}{0.919097,0.862812,0.832112}%
\pgfsetfillcolor{currentfill}%
\pgfsetlinewidth{0.602250pt}%
\definecolor{currentstroke}{rgb}{0.296471,0.296471,0.296471}%
\pgfsetstrokecolor{currentstroke}%
\pgfsetdash{}{0pt}%
\pgfpathmoveto{\pgfqpoint{2.199317in}{2.441921in}}%
\pgfpathlineto{\pgfqpoint{2.199775in}{2.441921in}}%
\pgfpathlineto{\pgfqpoint{2.199775in}{2.440689in}}%
\pgfpathlineto{\pgfqpoint{2.199317in}{2.440689in}}%
\pgfpathlineto{\pgfqpoint{2.199317in}{2.441921in}}%
\pgfpathclose%
\pgfusepath{stroke,fill}%
\end{pgfscope}%
\begin{pgfscope}%
\pgfpathrectangle{\pgfqpoint{0.703330in}{0.352393in}}{\pgfqpoint{2.173234in}{2.758940in}}%
\pgfusepath{clip}%
\pgfsetbuttcap%
\pgfsetroundjoin%
\definecolor{currentfill}{rgb}{0.910863,0.840546,0.801899}%
\pgfsetfillcolor{currentfill}%
\pgfsetlinewidth{0.602250pt}%
\definecolor{currentstroke}{rgb}{0.296471,0.296471,0.296471}%
\pgfsetstrokecolor{currentstroke}%
\pgfsetdash{}{0pt}%
\pgfpathmoveto{\pgfqpoint{2.199775in}{2.442537in}}%
\pgfpathlineto{\pgfqpoint{2.205260in}{2.442537in}}%
\pgfpathlineto{\pgfqpoint{2.205260in}{2.440073in}}%
\pgfpathlineto{\pgfqpoint{2.199775in}{2.440073in}}%
\pgfpathlineto{\pgfqpoint{2.199775in}{2.442537in}}%
\pgfpathclose%
\pgfusepath{stroke,fill}%
\end{pgfscope}%
\begin{pgfscope}%
\pgfpathrectangle{\pgfqpoint{0.703330in}{0.352393in}}{\pgfqpoint{2.173234in}{2.758940in}}%
\pgfusepath{clip}%
\pgfsetbuttcap%
\pgfsetroundjoin%
\definecolor{currentfill}{rgb}{0.901453,0.815098,0.767370}%
\pgfsetfillcolor{currentfill}%
\pgfsetlinewidth{0.602250pt}%
\definecolor{currentstroke}{rgb}{0.296471,0.296471,0.296471}%
\pgfsetstrokecolor{currentstroke}%
\pgfsetdash{}{0pt}%
\pgfpathmoveto{\pgfqpoint{2.205260in}{2.443768in}}%
\pgfpathlineto{\pgfqpoint{2.213137in}{2.443768in}}%
\pgfpathlineto{\pgfqpoint{2.213137in}{2.438842in}}%
\pgfpathlineto{\pgfqpoint{2.205260in}{2.438842in}}%
\pgfpathlineto{\pgfqpoint{2.205260in}{2.443768in}}%
\pgfpathclose%
\pgfusepath{stroke,fill}%
\end{pgfscope}%
\begin{pgfscope}%
\pgfpathrectangle{\pgfqpoint{0.703330in}{0.352393in}}{\pgfqpoint{2.173234in}{2.758940in}}%
\pgfusepath{clip}%
\pgfsetbuttcap%
\pgfsetroundjoin%
\definecolor{currentfill}{rgb}{0.889102,0.781698,0.722050}%
\pgfsetfillcolor{currentfill}%
\pgfsetlinewidth{0.602250pt}%
\definecolor{currentstroke}{rgb}{0.296471,0.296471,0.296471}%
\pgfsetstrokecolor{currentstroke}%
\pgfsetdash{}{0pt}%
\pgfpathmoveto{\pgfqpoint{2.213137in}{2.446232in}}%
\pgfpathlineto{\pgfqpoint{2.252657in}{2.446232in}}%
\pgfpathlineto{\pgfqpoint{2.252657in}{2.436378in}}%
\pgfpathlineto{\pgfqpoint{2.213137in}{2.436378in}}%
\pgfpathlineto{\pgfqpoint{2.213137in}{2.446232in}}%
\pgfpathclose%
\pgfusepath{stroke,fill}%
\end{pgfscope}%
\begin{pgfscope}%
\pgfpathrectangle{\pgfqpoint{0.703330in}{0.352393in}}{\pgfqpoint{2.173234in}{2.758940in}}%
\pgfusepath{clip}%
\pgfsetbuttcap%
\pgfsetroundjoin%
\definecolor{currentfill}{rgb}{0.873223,0.738755,0.663782}%
\pgfsetfillcolor{currentfill}%
\pgfsetlinewidth{0.602250pt}%
\definecolor{currentstroke}{rgb}{0.296471,0.296471,0.296471}%
\pgfsetstrokecolor{currentstroke}%
\pgfsetdash{}{0pt}%
\pgfpathmoveto{\pgfqpoint{2.252657in}{2.451158in}}%
\pgfpathlineto{\pgfqpoint{2.292767in}{2.451158in}}%
\pgfpathlineto{\pgfqpoint{2.292767in}{2.431452in}}%
\pgfpathlineto{\pgfqpoint{2.252657in}{2.431452in}}%
\pgfpathlineto{\pgfqpoint{2.252657in}{2.451158in}}%
\pgfpathclose%
\pgfusepath{stroke,fill}%
\end{pgfscope}%
\begin{pgfscope}%
\pgfpathrectangle{\pgfqpoint{0.703330in}{0.352393in}}{\pgfqpoint{2.173234in}{2.758940in}}%
\pgfusepath{clip}%
\pgfsetbuttcap%
\pgfsetroundjoin%
\definecolor{currentfill}{rgb}{0.853814,0.686269,0.592565}%
\pgfsetfillcolor{currentfill}%
\pgfsetlinewidth{0.602250pt}%
\definecolor{currentstroke}{rgb}{0.296471,0.296471,0.296471}%
\pgfsetstrokecolor{currentstroke}%
\pgfsetdash{}{0pt}%
\pgfpathmoveto{\pgfqpoint{2.292767in}{2.461012in}}%
\pgfpathlineto{\pgfqpoint{2.327432in}{2.461012in}}%
\pgfpathlineto{\pgfqpoint{2.327432in}{2.421598in}}%
\pgfpathlineto{\pgfqpoint{2.292767in}{2.421598in}}%
\pgfpathlineto{\pgfqpoint{2.292767in}{2.461012in}}%
\pgfpathclose%
\pgfusepath{stroke,fill}%
\end{pgfscope}%
\begin{pgfscope}%
\pgfpathrectangle{\pgfqpoint{0.703330in}{0.352393in}}{\pgfqpoint{2.173234in}{2.758940in}}%
\pgfusepath{clip}%
\pgfsetbuttcap%
\pgfsetroundjoin%
\definecolor{currentfill}{rgb}{0.829112,0.619469,0.501926}%
\pgfsetfillcolor{currentfill}%
\pgfsetlinewidth{0.602250pt}%
\definecolor{currentstroke}{rgb}{0.296471,0.296471,0.296471}%
\pgfsetstrokecolor{currentstroke}%
\pgfsetdash{}{0pt}%
\pgfpathmoveto{\pgfqpoint{2.327432in}{2.480718in}}%
\pgfpathlineto{\pgfqpoint{2.387212in}{2.480718in}}%
\pgfpathlineto{\pgfqpoint{2.387212in}{2.401892in}}%
\pgfpathlineto{\pgfqpoint{2.327432in}{2.401892in}}%
\pgfpathlineto{\pgfqpoint{2.327432in}{2.480718in}}%
\pgfpathclose%
\pgfusepath{stroke,fill}%
\end{pgfscope}%
\begin{pgfscope}%
\pgfpathrectangle{\pgfqpoint{0.703330in}{0.352393in}}{\pgfqpoint{2.173234in}{2.758940in}}%
\pgfusepath{clip}%
\pgfsetbuttcap%
\pgfsetroundjoin%
\definecolor{currentfill}{rgb}{0.798529,0.536765,0.389706}%
\pgfsetfillcolor{currentfill}%
\pgfsetlinewidth{0.602250pt}%
\definecolor{currentstroke}{rgb}{0.296471,0.296471,0.296471}%
\pgfsetstrokecolor{currentstroke}%
\pgfsetdash{}{0pt}%
\pgfpathmoveto{\pgfqpoint{2.387212in}{2.520132in}}%
\pgfpathlineto{\pgfqpoint{2.790285in}{2.520132in}}%
\pgfpathlineto{\pgfqpoint{2.790285in}{2.362478in}}%
\pgfpathlineto{\pgfqpoint{2.387212in}{2.362478in}}%
\pgfpathlineto{\pgfqpoint{2.387212in}{2.520132in}}%
\pgfpathclose%
\pgfusepath{stroke,fill}%
\end{pgfscope}%
\begin{pgfscope}%
\pgfpathrectangle{\pgfqpoint{0.703330in}{0.352393in}}{\pgfqpoint{2.173234in}{2.758940in}}%
\pgfusepath{clip}%
\pgfsetbuttcap%
\pgfsetroundjoin%
\definecolor{currentfill}{rgb}{0.829112,0.619469,0.501926}%
\pgfsetfillcolor{currentfill}%
\pgfsetlinewidth{0.602250pt}%
\definecolor{currentstroke}{rgb}{0.296471,0.296471,0.296471}%
\pgfsetstrokecolor{currentstroke}%
\pgfsetdash{}{0pt}%
\pgfpathmoveto{\pgfqpoint{2.790285in}{2.480718in}}%
\pgfpathlineto{\pgfqpoint{2.854857in}{2.480718in}}%
\pgfpathlineto{\pgfqpoint{2.854857in}{2.401892in}}%
\pgfpathlineto{\pgfqpoint{2.790285in}{2.401892in}}%
\pgfpathlineto{\pgfqpoint{2.790285in}{2.480718in}}%
\pgfpathclose%
\pgfusepath{stroke,fill}%
\end{pgfscope}%
\begin{pgfscope}%
\pgfpathrectangle{\pgfqpoint{0.703330in}{0.352393in}}{\pgfqpoint{2.173234in}{2.758940in}}%
\pgfusepath{clip}%
\pgfsetbuttcap%
\pgfsetroundjoin%
\definecolor{currentfill}{rgb}{0.853814,0.686269,0.592565}%
\pgfsetfillcolor{currentfill}%
\pgfsetlinewidth{0.602250pt}%
\definecolor{currentstroke}{rgb}{0.296471,0.296471,0.296471}%
\pgfsetstrokecolor{currentstroke}%
\pgfsetdash{}{0pt}%
\pgfpathmoveto{\pgfqpoint{2.854857in}{2.461012in}}%
\pgfpathlineto{\pgfqpoint{2.876564in}{2.461012in}}%
\pgfpathlineto{\pgfqpoint{2.876564in}{2.421598in}}%
\pgfpathlineto{\pgfqpoint{2.854857in}{2.421598in}}%
\pgfpathlineto{\pgfqpoint{2.854857in}{2.461012in}}%
\pgfpathclose%
\pgfusepath{stroke,fill}%
\end{pgfscope}%
\begin{pgfscope}%
\pgfpathrectangle{\pgfqpoint{0.703330in}{0.352393in}}{\pgfqpoint{2.173234in}{2.758940in}}%
\pgfusepath{clip}%
\pgfsetbuttcap%
\pgfsetroundjoin%
\definecolor{currentfill}{rgb}{0.873223,0.738755,0.663782}%
\pgfsetfillcolor{currentfill}%
\pgfsetlinewidth{0.602250pt}%
\definecolor{currentstroke}{rgb}{0.296471,0.296471,0.296471}%
\pgfsetstrokecolor{currentstroke}%
\pgfsetdash{}{0pt}%
\pgfpathmoveto{\pgfqpoint{2.876564in}{2.451158in}}%
\pgfpathlineto{\pgfqpoint{2.876564in}{2.451158in}}%
\pgfpathlineto{\pgfqpoint{2.876564in}{2.431452in}}%
\pgfpathlineto{\pgfqpoint{2.876564in}{2.431452in}}%
\pgfpathlineto{\pgfqpoint{2.876564in}{2.451158in}}%
\pgfpathclose%
\pgfusepath{stroke,fill}%
\end{pgfscope}%
\begin{pgfscope}%
\pgfpathrectangle{\pgfqpoint{0.703330in}{0.352393in}}{\pgfqpoint{2.173234in}{2.758940in}}%
\pgfusepath{clip}%
\pgfsetbuttcap%
\pgfsetroundjoin%
\definecolor{currentfill}{rgb}{0.889102,0.781698,0.722050}%
\pgfsetfillcolor{currentfill}%
\pgfsetlinewidth{0.602250pt}%
\definecolor{currentstroke}{rgb}{0.296471,0.296471,0.296471}%
\pgfsetstrokecolor{currentstroke}%
\pgfsetdash{}{0pt}%
\pgfpathmoveto{\pgfqpoint{2.876564in}{2.446232in}}%
\pgfpathlineto{\pgfqpoint{2.876564in}{2.446232in}}%
\pgfpathlineto{\pgfqpoint{2.876564in}{2.436378in}}%
\pgfpathlineto{\pgfqpoint{2.876564in}{2.436378in}}%
\pgfpathlineto{\pgfqpoint{2.876564in}{2.446232in}}%
\pgfpathclose%
\pgfusepath{stroke,fill}%
\end{pgfscope}%
\begin{pgfscope}%
\pgfpathrectangle{\pgfqpoint{0.703330in}{0.352393in}}{\pgfqpoint{2.173234in}{2.758940in}}%
\pgfusepath{clip}%
\pgfsetbuttcap%
\pgfsetroundjoin%
\definecolor{currentfill}{rgb}{0.901453,0.815098,0.767370}%
\pgfsetfillcolor{currentfill}%
\pgfsetlinewidth{0.602250pt}%
\definecolor{currentstroke}{rgb}{0.296471,0.296471,0.296471}%
\pgfsetstrokecolor{currentstroke}%
\pgfsetdash{}{0pt}%
\pgfpathmoveto{\pgfqpoint{2.876564in}{2.443768in}}%
\pgfpathlineto{\pgfqpoint{2.876564in}{2.443768in}}%
\pgfpathlineto{\pgfqpoint{2.876564in}{2.438842in}}%
\pgfpathlineto{\pgfqpoint{2.876564in}{2.438842in}}%
\pgfpathlineto{\pgfqpoint{2.876564in}{2.443768in}}%
\pgfpathclose%
\pgfusepath{stroke,fill}%
\end{pgfscope}%
\begin{pgfscope}%
\pgfpathrectangle{\pgfqpoint{0.703330in}{0.352393in}}{\pgfqpoint{2.173234in}{2.758940in}}%
\pgfusepath{clip}%
\pgfsetbuttcap%
\pgfsetroundjoin%
\definecolor{currentfill}{rgb}{0.910863,0.840546,0.801899}%
\pgfsetfillcolor{currentfill}%
\pgfsetlinewidth{0.602250pt}%
\definecolor{currentstroke}{rgb}{0.296471,0.296471,0.296471}%
\pgfsetstrokecolor{currentstroke}%
\pgfsetdash{}{0pt}%
\pgfpathmoveto{\pgfqpoint{2.876564in}{2.442537in}}%
\pgfpathlineto{\pgfqpoint{2.876564in}{2.442537in}}%
\pgfpathlineto{\pgfqpoint{2.876564in}{2.440073in}}%
\pgfpathlineto{\pgfqpoint{2.876564in}{2.440073in}}%
\pgfpathlineto{\pgfqpoint{2.876564in}{2.442537in}}%
\pgfpathclose%
\pgfusepath{stroke,fill}%
\end{pgfscope}%
\begin{pgfscope}%
\pgfpathrectangle{\pgfqpoint{0.703330in}{0.352393in}}{\pgfqpoint{2.173234in}{2.758940in}}%
\pgfusepath{clip}%
\pgfsetbuttcap%
\pgfsetroundjoin%
\definecolor{currentfill}{rgb}{0.919097,0.862812,0.832112}%
\pgfsetfillcolor{currentfill}%
\pgfsetlinewidth{0.602250pt}%
\definecolor{currentstroke}{rgb}{0.296471,0.296471,0.296471}%
\pgfsetstrokecolor{currentstroke}%
\pgfsetdash{}{0pt}%
\pgfpathmoveto{\pgfqpoint{2.876564in}{2.441921in}}%
\pgfpathlineto{\pgfqpoint{2.876564in}{2.441921in}}%
\pgfpathlineto{\pgfqpoint{2.876564in}{2.440689in}}%
\pgfpathlineto{\pgfqpoint{2.876564in}{2.440689in}}%
\pgfpathlineto{\pgfqpoint{2.876564in}{2.441921in}}%
\pgfpathclose%
\pgfusepath{stroke,fill}%
\end{pgfscope}%
\begin{pgfscope}%
\pgfpathrectangle{\pgfqpoint{0.703330in}{0.352393in}}{\pgfqpoint{2.173234in}{2.758940in}}%
\pgfusepath{clip}%
\pgfsetbuttcap%
\pgfsetroundjoin%
\pgfsetlinewidth{0.803000pt}%
\definecolor{currentstroke}{rgb}{0.450000,0.450000,0.450000}%
\pgfsetstrokecolor{currentstroke}%
\pgfsetdash{}{0pt}%
\pgfpathmoveto{\pgfqpoint{0.000000in}{-0.034722in}}%
\pgfpathcurveto{\pgfqpoint{0.009208in}{-0.034722in}}{\pgfqpoint{0.018041in}{-0.031064in}}{\pgfqpoint{0.024552in}{-0.024552in}}%
\pgfpathcurveto{\pgfqpoint{0.031064in}{-0.018041in}}{\pgfqpoint{0.034722in}{-0.009208in}}{\pgfqpoint{0.034722in}{0.000000in}}%
\pgfpathcurveto{\pgfqpoint{0.034722in}{0.009208in}}{\pgfqpoint{0.031064in}{0.018041in}}{\pgfqpoint{0.024552in}{0.024552in}}%
\pgfpathcurveto{\pgfqpoint{0.018041in}{0.031064in}}{\pgfqpoint{0.009208in}{0.034722in}}{\pgfqpoint{0.000000in}{0.034722in}}%
\pgfpathcurveto{\pgfqpoint{-0.009208in}{0.034722in}}{\pgfqpoint{-0.018041in}{0.031064in}}{\pgfqpoint{-0.024552in}{0.024552in}}%
\pgfpathcurveto{\pgfqpoint{-0.031064in}{0.018041in}}{\pgfqpoint{-0.034722in}{0.009208in}}{\pgfqpoint{-0.034722in}{0.000000in}}%
\pgfpathcurveto{\pgfqpoint{-0.034722in}{-0.009208in}}{\pgfqpoint{-0.031064in}{-0.018041in}}{\pgfqpoint{-0.024552in}{-0.024552in}}%
\pgfpathcurveto{\pgfqpoint{-0.018041in}{-0.031064in}}{\pgfqpoint{-0.009208in}{-0.034722in}}{\pgfqpoint{0.000000in}{-0.034722in}}%
\pgfusepath{stroke}%
\end{pgfscope}%
\begin{pgfscope}%
\pgfpathrectangle{\pgfqpoint{0.703330in}{0.352393in}}{\pgfqpoint{2.173234in}{2.758940in}}%
\pgfusepath{clip}%
\pgfsetbuttcap%
\pgfsetroundjoin%
\definecolor{currentfill}{rgb}{0.848437,0.867532,0.899724}%
\pgfsetfillcolor{currentfill}%
\pgfsetlinewidth{0.602250pt}%
\definecolor{currentstroke}{rgb}{0.296471,0.296471,0.296471}%
\pgfsetstrokecolor{currentstroke}%
\pgfsetdash{}{0pt}%
\pgfpathmoveto{\pgfqpoint{0.590438in}{2.205132in}}%
\pgfpathlineto{\pgfqpoint{0.600624in}{2.205132in}}%
\pgfpathlineto{\pgfqpoint{0.600624in}{2.204516in}}%
\pgfpathlineto{\pgfqpoint{0.590438in}{2.204516in}}%
\pgfpathlineto{\pgfqpoint{0.590438in}{2.205132in}}%
\pgfpathclose%
\pgfusepath{stroke,fill}%
\end{pgfscope}%
\begin{pgfscope}%
\pgfpathrectangle{\pgfqpoint{0.703330in}{0.352393in}}{\pgfqpoint{2.173234in}{2.758940in}}%
\pgfusepath{clip}%
\pgfsetbuttcap%
\pgfsetroundjoin%
\definecolor{currentfill}{rgb}{0.825117,0.848522,0.887698}%
\pgfsetfillcolor{currentfill}%
\pgfsetlinewidth{0.602250pt}%
\definecolor{currentstroke}{rgb}{0.296471,0.296471,0.296471}%
\pgfsetstrokecolor{currentstroke}%
\pgfsetdash{}{0pt}%
\pgfpathmoveto{\pgfqpoint{0.600624in}{2.205440in}}%
\pgfpathlineto{\pgfqpoint{0.608704in}{2.205440in}}%
\pgfpathlineto{\pgfqpoint{0.608704in}{2.204209in}}%
\pgfpathlineto{\pgfqpoint{0.600624in}{2.204209in}}%
\pgfpathlineto{\pgfqpoint{0.600624in}{2.205440in}}%
\pgfpathclose%
\pgfusepath{stroke,fill}%
\end{pgfscope}%
\begin{pgfscope}%
\pgfpathrectangle{\pgfqpoint{0.703330in}{0.352393in}}{\pgfqpoint{2.173234in}{2.758940in}}%
\pgfusepath{clip}%
\pgfsetbuttcap%
\pgfsetroundjoin%
\definecolor{currentfill}{rgb}{0.792469,0.821908,0.870863}%
\pgfsetfillcolor{currentfill}%
\pgfsetlinewidth{0.602250pt}%
\definecolor{currentstroke}{rgb}{0.296471,0.296471,0.296471}%
\pgfsetstrokecolor{currentstroke}%
\pgfsetdash{}{0pt}%
\pgfpathmoveto{\pgfqpoint{0.608704in}{2.206056in}}%
\pgfpathlineto{\pgfqpoint{0.656887in}{2.206056in}}%
\pgfpathlineto{\pgfqpoint{0.656887in}{2.203593in}}%
\pgfpathlineto{\pgfqpoint{0.608704in}{2.203593in}}%
\pgfpathlineto{\pgfqpoint{0.608704in}{2.206056in}}%
\pgfpathclose%
\pgfusepath{stroke,fill}%
\end{pgfscope}%
\begin{pgfscope}%
\pgfpathrectangle{\pgfqpoint{0.703330in}{0.352393in}}{\pgfqpoint{2.173234in}{2.758940in}}%
\pgfusepath{clip}%
\pgfsetbuttcap%
\pgfsetroundjoin%
\definecolor{currentfill}{rgb}{0.755157,0.791493,0.851622}%
\pgfsetfillcolor{currentfill}%
\pgfsetlinewidth{0.602250pt}%
\definecolor{currentstroke}{rgb}{0.296471,0.296471,0.296471}%
\pgfsetstrokecolor{currentstroke}%
\pgfsetdash{}{0pt}%
\pgfpathmoveto{\pgfqpoint{0.656887in}{2.207288in}}%
\pgfpathlineto{\pgfqpoint{1.025063in}{2.207288in}}%
\pgfpathlineto{\pgfqpoint{1.025063in}{2.202361in}}%
\pgfpathlineto{\pgfqpoint{0.656887in}{2.202361in}}%
\pgfpathlineto{\pgfqpoint{0.656887in}{2.207288in}}%
\pgfpathclose%
\pgfusepath{stroke,fill}%
\end{pgfscope}%
\begin{pgfscope}%
\pgfpathrectangle{\pgfqpoint{0.703330in}{0.352393in}}{\pgfqpoint{2.173234in}{2.758940in}}%
\pgfusepath{clip}%
\pgfsetbuttcap%
\pgfsetroundjoin%
\definecolor{currentfill}{rgb}{0.706185,0.751573,0.826368}%
\pgfsetfillcolor{currentfill}%
\pgfsetlinewidth{0.602250pt}%
\definecolor{currentstroke}{rgb}{0.296471,0.296471,0.296471}%
\pgfsetstrokecolor{currentstroke}%
\pgfsetdash{}{0pt}%
\pgfpathmoveto{\pgfqpoint{1.025063in}{2.209751in}}%
\pgfpathlineto{\pgfqpoint{1.178851in}{2.209751in}}%
\pgfpathlineto{\pgfqpoint{1.178851in}{2.199898in}}%
\pgfpathlineto{\pgfqpoint{1.025063in}{2.199898in}}%
\pgfpathlineto{\pgfqpoint{1.025063in}{2.209751in}}%
\pgfpathclose%
\pgfusepath{stroke,fill}%
\end{pgfscope}%
\begin{pgfscope}%
\pgfpathrectangle{\pgfqpoint{0.703330in}{0.352393in}}{\pgfqpoint{2.173234in}{2.758940in}}%
\pgfusepath{clip}%
\pgfsetbuttcap%
\pgfsetroundjoin%
\definecolor{currentfill}{rgb}{0.643221,0.700246,0.793900}%
\pgfsetfillcolor{currentfill}%
\pgfsetlinewidth{0.602250pt}%
\definecolor{currentstroke}{rgb}{0.296471,0.296471,0.296471}%
\pgfsetstrokecolor{currentstroke}%
\pgfsetdash{}{0pt}%
\pgfpathmoveto{\pgfqpoint{1.178851in}{2.214678in}}%
\pgfpathlineto{\pgfqpoint{1.243678in}{2.214678in}}%
\pgfpathlineto{\pgfqpoint{1.243678in}{2.194971in}}%
\pgfpathlineto{\pgfqpoint{1.178851in}{2.194971in}}%
\pgfpathlineto{\pgfqpoint{1.178851in}{2.214678in}}%
\pgfpathclose%
\pgfusepath{stroke,fill}%
\end{pgfscope}%
\begin{pgfscope}%
\pgfpathrectangle{\pgfqpoint{0.703330in}{0.352393in}}{\pgfqpoint{2.173234in}{2.758940in}}%
\pgfusepath{clip}%
\pgfsetbuttcap%
\pgfsetroundjoin%
\definecolor{currentfill}{rgb}{0.566266,0.637515,0.754216}%
\pgfsetfillcolor{currentfill}%
\pgfsetlinewidth{0.602250pt}%
\definecolor{currentstroke}{rgb}{0.296471,0.296471,0.296471}%
\pgfsetstrokecolor{currentstroke}%
\pgfsetdash{}{0pt}%
\pgfpathmoveto{\pgfqpoint{1.243678in}{2.224531in}}%
\pgfpathlineto{\pgfqpoint{1.382824in}{2.224531in}}%
\pgfpathlineto{\pgfqpoint{1.382824in}{2.185118in}}%
\pgfpathlineto{\pgfqpoint{1.243678in}{2.185118in}}%
\pgfpathlineto{\pgfqpoint{1.243678in}{2.224531in}}%
\pgfpathclose%
\pgfusepath{stroke,fill}%
\end{pgfscope}%
\begin{pgfscope}%
\pgfpathrectangle{\pgfqpoint{0.703330in}{0.352393in}}{\pgfqpoint{2.173234in}{2.758940in}}%
\pgfusepath{clip}%
\pgfsetbuttcap%
\pgfsetroundjoin%
\definecolor{currentfill}{rgb}{0.468322,0.557674,0.703709}%
\pgfsetfillcolor{currentfill}%
\pgfsetlinewidth{0.602250pt}%
\definecolor{currentstroke}{rgb}{0.296471,0.296471,0.296471}%
\pgfsetstrokecolor{currentstroke}%
\pgfsetdash{}{0pt}%
\pgfpathmoveto{\pgfqpoint{1.382824in}{2.244238in}}%
\pgfpathlineto{\pgfqpoint{1.749464in}{2.244238in}}%
\pgfpathlineto{\pgfqpoint{1.749464in}{2.165411in}}%
\pgfpathlineto{\pgfqpoint{1.382824in}{2.165411in}}%
\pgfpathlineto{\pgfqpoint{1.382824in}{2.244238in}}%
\pgfpathclose%
\pgfusepath{stroke,fill}%
\end{pgfscope}%
\begin{pgfscope}%
\pgfpathrectangle{\pgfqpoint{0.703330in}{0.352393in}}{\pgfqpoint{2.173234in}{2.758940in}}%
\pgfusepath{clip}%
\pgfsetbuttcap%
\pgfsetroundjoin%
\definecolor{currentfill}{rgb}{0.347059,0.458824,0.641176}%
\pgfsetfillcolor{currentfill}%
\pgfsetlinewidth{0.602250pt}%
\definecolor{currentstroke}{rgb}{0.296471,0.296471,0.296471}%
\pgfsetstrokecolor{currentstroke}%
\pgfsetdash{}{0pt}%
\pgfpathmoveto{\pgfqpoint{1.749464in}{2.283651in}}%
\pgfpathlineto{\pgfqpoint{2.293061in}{2.283651in}}%
\pgfpathlineto{\pgfqpoint{2.293061in}{2.125997in}}%
\pgfpathlineto{\pgfqpoint{1.749464in}{2.125997in}}%
\pgfpathlineto{\pgfqpoint{1.749464in}{2.283651in}}%
\pgfpathclose%
\pgfusepath{stroke,fill}%
\end{pgfscope}%
\begin{pgfscope}%
\pgfpathrectangle{\pgfqpoint{0.703330in}{0.352393in}}{\pgfqpoint{2.173234in}{2.758940in}}%
\pgfusepath{clip}%
\pgfsetbuttcap%
\pgfsetroundjoin%
\definecolor{currentfill}{rgb}{0.468322,0.557674,0.703709}%
\pgfsetfillcolor{currentfill}%
\pgfsetlinewidth{0.602250pt}%
\definecolor{currentstroke}{rgb}{0.296471,0.296471,0.296471}%
\pgfsetstrokecolor{currentstroke}%
\pgfsetdash{}{0pt}%
\pgfpathmoveto{\pgfqpoint{2.293061in}{2.244238in}}%
\pgfpathlineto{\pgfqpoint{2.749235in}{2.244238in}}%
\pgfpathlineto{\pgfqpoint{2.749235in}{2.165411in}}%
\pgfpathlineto{\pgfqpoint{2.293061in}{2.165411in}}%
\pgfpathlineto{\pgfqpoint{2.293061in}{2.244238in}}%
\pgfpathclose%
\pgfusepath{stroke,fill}%
\end{pgfscope}%
\begin{pgfscope}%
\pgfpathrectangle{\pgfqpoint{0.703330in}{0.352393in}}{\pgfqpoint{2.173234in}{2.758940in}}%
\pgfusepath{clip}%
\pgfsetbuttcap%
\pgfsetroundjoin%
\definecolor{currentfill}{rgb}{0.566266,0.637515,0.754216}%
\pgfsetfillcolor{currentfill}%
\pgfsetlinewidth{0.602250pt}%
\definecolor{currentstroke}{rgb}{0.296471,0.296471,0.296471}%
\pgfsetstrokecolor{currentstroke}%
\pgfsetdash{}{0pt}%
\pgfpathmoveto{\pgfqpoint{2.749235in}{2.224531in}}%
\pgfpathlineto{\pgfqpoint{2.876564in}{2.224531in}}%
\pgfpathlineto{\pgfqpoint{2.876564in}{2.185118in}}%
\pgfpathlineto{\pgfqpoint{2.749235in}{2.185118in}}%
\pgfpathlineto{\pgfqpoint{2.749235in}{2.224531in}}%
\pgfpathclose%
\pgfusepath{stroke,fill}%
\end{pgfscope}%
\begin{pgfscope}%
\pgfpathrectangle{\pgfqpoint{0.703330in}{0.352393in}}{\pgfqpoint{2.173234in}{2.758940in}}%
\pgfusepath{clip}%
\pgfsetbuttcap%
\pgfsetroundjoin%
\definecolor{currentfill}{rgb}{0.643221,0.700246,0.793900}%
\pgfsetfillcolor{currentfill}%
\pgfsetlinewidth{0.602250pt}%
\definecolor{currentstroke}{rgb}{0.296471,0.296471,0.296471}%
\pgfsetstrokecolor{currentstroke}%
\pgfsetdash{}{0pt}%
\pgfpathmoveto{\pgfqpoint{2.876564in}{2.214678in}}%
\pgfpathlineto{\pgfqpoint{2.876564in}{2.214678in}}%
\pgfpathlineto{\pgfqpoint{2.876564in}{2.194971in}}%
\pgfpathlineto{\pgfqpoint{2.876564in}{2.194971in}}%
\pgfpathlineto{\pgfqpoint{2.876564in}{2.214678in}}%
\pgfpathclose%
\pgfusepath{stroke,fill}%
\end{pgfscope}%
\begin{pgfscope}%
\pgfpathrectangle{\pgfqpoint{0.703330in}{0.352393in}}{\pgfqpoint{2.173234in}{2.758940in}}%
\pgfusepath{clip}%
\pgfsetbuttcap%
\pgfsetroundjoin%
\definecolor{currentfill}{rgb}{0.706185,0.751573,0.826368}%
\pgfsetfillcolor{currentfill}%
\pgfsetlinewidth{0.602250pt}%
\definecolor{currentstroke}{rgb}{0.296471,0.296471,0.296471}%
\pgfsetstrokecolor{currentstroke}%
\pgfsetdash{}{0pt}%
\pgfpathmoveto{\pgfqpoint{2.876564in}{2.209751in}}%
\pgfpathlineto{\pgfqpoint{2.876564in}{2.209751in}}%
\pgfpathlineto{\pgfqpoint{2.876564in}{2.199898in}}%
\pgfpathlineto{\pgfqpoint{2.876564in}{2.199898in}}%
\pgfpathlineto{\pgfqpoint{2.876564in}{2.209751in}}%
\pgfpathclose%
\pgfusepath{stroke,fill}%
\end{pgfscope}%
\begin{pgfscope}%
\pgfpathrectangle{\pgfqpoint{0.703330in}{0.352393in}}{\pgfqpoint{2.173234in}{2.758940in}}%
\pgfusepath{clip}%
\pgfsetbuttcap%
\pgfsetroundjoin%
\definecolor{currentfill}{rgb}{0.755157,0.791493,0.851622}%
\pgfsetfillcolor{currentfill}%
\pgfsetlinewidth{0.602250pt}%
\definecolor{currentstroke}{rgb}{0.296471,0.296471,0.296471}%
\pgfsetstrokecolor{currentstroke}%
\pgfsetdash{}{0pt}%
\pgfpathmoveto{\pgfqpoint{2.876564in}{2.207288in}}%
\pgfpathlineto{\pgfqpoint{2.876564in}{2.207288in}}%
\pgfpathlineto{\pgfqpoint{2.876564in}{2.202361in}}%
\pgfpathlineto{\pgfqpoint{2.876564in}{2.202361in}}%
\pgfpathlineto{\pgfqpoint{2.876564in}{2.207288in}}%
\pgfpathclose%
\pgfusepath{stroke,fill}%
\end{pgfscope}%
\begin{pgfscope}%
\pgfpathrectangle{\pgfqpoint{0.703330in}{0.352393in}}{\pgfqpoint{2.173234in}{2.758940in}}%
\pgfusepath{clip}%
\pgfsetbuttcap%
\pgfsetroundjoin%
\definecolor{currentfill}{rgb}{0.792469,0.821908,0.870863}%
\pgfsetfillcolor{currentfill}%
\pgfsetlinewidth{0.602250pt}%
\definecolor{currentstroke}{rgb}{0.296471,0.296471,0.296471}%
\pgfsetstrokecolor{currentstroke}%
\pgfsetdash{}{0pt}%
\pgfpathmoveto{\pgfqpoint{2.876564in}{2.206056in}}%
\pgfpathlineto{\pgfqpoint{2.876564in}{2.206056in}}%
\pgfpathlineto{\pgfqpoint{2.876564in}{2.203593in}}%
\pgfpathlineto{\pgfqpoint{2.876564in}{2.203593in}}%
\pgfpathlineto{\pgfqpoint{2.876564in}{2.206056in}}%
\pgfpathclose%
\pgfusepath{stroke,fill}%
\end{pgfscope}%
\begin{pgfscope}%
\pgfpathrectangle{\pgfqpoint{0.703330in}{0.352393in}}{\pgfqpoint{2.173234in}{2.758940in}}%
\pgfusepath{clip}%
\pgfsetbuttcap%
\pgfsetroundjoin%
\definecolor{currentfill}{rgb}{0.825117,0.848522,0.887698}%
\pgfsetfillcolor{currentfill}%
\pgfsetlinewidth{0.602250pt}%
\definecolor{currentstroke}{rgb}{0.296471,0.296471,0.296471}%
\pgfsetstrokecolor{currentstroke}%
\pgfsetdash{}{0pt}%
\pgfpathmoveto{\pgfqpoint{2.876564in}{2.205440in}}%
\pgfpathlineto{\pgfqpoint{2.876564in}{2.205440in}}%
\pgfpathlineto{\pgfqpoint{2.876564in}{2.204209in}}%
\pgfpathlineto{\pgfqpoint{2.876564in}{2.204209in}}%
\pgfpathlineto{\pgfqpoint{2.876564in}{2.205440in}}%
\pgfpathclose%
\pgfusepath{stroke,fill}%
\end{pgfscope}%
\begin{pgfscope}%
\pgfpathrectangle{\pgfqpoint{0.703330in}{0.352393in}}{\pgfqpoint{2.173234in}{2.758940in}}%
\pgfusepath{clip}%
\pgfsetbuttcap%
\pgfsetroundjoin%
\definecolor{currentfill}{rgb}{0.848437,0.867532,0.899724}%
\pgfsetfillcolor{currentfill}%
\pgfsetlinewidth{0.602250pt}%
\definecolor{currentstroke}{rgb}{0.296471,0.296471,0.296471}%
\pgfsetstrokecolor{currentstroke}%
\pgfsetdash{}{0pt}%
\pgfpathmoveto{\pgfqpoint{2.876564in}{2.205132in}}%
\pgfpathlineto{\pgfqpoint{2.876564in}{2.205132in}}%
\pgfpathlineto{\pgfqpoint{2.876564in}{2.204516in}}%
\pgfpathlineto{\pgfqpoint{2.876564in}{2.204516in}}%
\pgfpathlineto{\pgfqpoint{2.876564in}{2.205132in}}%
\pgfpathclose%
\pgfusepath{stroke,fill}%
\end{pgfscope}%
\begin{pgfscope}%
\pgfpathrectangle{\pgfqpoint{0.703330in}{0.352393in}}{\pgfqpoint{2.173234in}{2.758940in}}%
\pgfusepath{clip}%
\pgfsetbuttcap%
\pgfsetroundjoin%
\pgfsetlinewidth{0.803000pt}%
\definecolor{currentstroke}{rgb}{0.450000,0.450000,0.450000}%
\pgfsetstrokecolor{currentstroke}%
\pgfsetdash{}{0pt}%
\pgfpathmoveto{\pgfqpoint{0.000000in}{-0.034722in}}%
\pgfpathcurveto{\pgfqpoint{0.009208in}{-0.034722in}}{\pgfqpoint{0.018041in}{-0.031064in}}{\pgfqpoint{0.024552in}{-0.024552in}}%
\pgfpathcurveto{\pgfqpoint{0.031064in}{-0.018041in}}{\pgfqpoint{0.034722in}{-0.009208in}}{\pgfqpoint{0.034722in}{0.000000in}}%
\pgfpathcurveto{\pgfqpoint{0.034722in}{0.009208in}}{\pgfqpoint{0.031064in}{0.018041in}}{\pgfqpoint{0.024552in}{0.024552in}}%
\pgfpathcurveto{\pgfqpoint{0.018041in}{0.031064in}}{\pgfqpoint{0.009208in}{0.034722in}}{\pgfqpoint{0.000000in}{0.034722in}}%
\pgfpathcurveto{\pgfqpoint{-0.009208in}{0.034722in}}{\pgfqpoint{-0.018041in}{0.031064in}}{\pgfqpoint{-0.024552in}{0.024552in}}%
\pgfpathcurveto{\pgfqpoint{-0.031064in}{0.018041in}}{\pgfqpoint{-0.034722in}{0.009208in}}{\pgfqpoint{-0.034722in}{0.000000in}}%
\pgfpathcurveto{\pgfqpoint{-0.034722in}{-0.009208in}}{\pgfqpoint{-0.031064in}{-0.018041in}}{\pgfqpoint{-0.024552in}{-0.024552in}}%
\pgfpathcurveto{\pgfqpoint{-0.018041in}{-0.031064in}}{\pgfqpoint{-0.009208in}{-0.034722in}}{\pgfqpoint{0.000000in}{-0.034722in}}%
\pgfusepath{stroke}%
\end{pgfscope}%
\begin{pgfscope}%
\pgfpathrectangle{\pgfqpoint{0.703330in}{0.352393in}}{\pgfqpoint{2.173234in}{2.758940in}}%
\pgfusepath{clip}%
\pgfsetbuttcap%
\pgfsetroundjoin%
\definecolor{currentfill}{rgb}{0.919097,0.862812,0.832112}%
\pgfsetfillcolor{currentfill}%
\pgfsetlinewidth{0.602250pt}%
\definecolor{currentstroke}{rgb}{0.296471,0.296471,0.296471}%
\pgfsetstrokecolor{currentstroke}%
\pgfsetdash{}{0pt}%
\pgfpathmoveto{\pgfqpoint{2.059341in}{2.047786in}}%
\pgfpathlineto{\pgfqpoint{2.070438in}{2.047786in}}%
\pgfpathlineto{\pgfqpoint{2.070438in}{2.046555in}}%
\pgfpathlineto{\pgfqpoint{2.059341in}{2.046555in}}%
\pgfpathlineto{\pgfqpoint{2.059341in}{2.047786in}}%
\pgfpathclose%
\pgfusepath{stroke,fill}%
\end{pgfscope}%
\begin{pgfscope}%
\pgfpathrectangle{\pgfqpoint{0.703330in}{0.352393in}}{\pgfqpoint{2.173234in}{2.758940in}}%
\pgfusepath{clip}%
\pgfsetbuttcap%
\pgfsetroundjoin%
\definecolor{currentfill}{rgb}{0.910863,0.840546,0.801899}%
\pgfsetfillcolor{currentfill}%
\pgfsetlinewidth{0.602250pt}%
\definecolor{currentstroke}{rgb}{0.296471,0.296471,0.296471}%
\pgfsetstrokecolor{currentstroke}%
\pgfsetdash{}{0pt}%
\pgfpathmoveto{\pgfqpoint{2.070438in}{2.048402in}}%
\pgfpathlineto{\pgfqpoint{2.114013in}{2.048402in}}%
\pgfpathlineto{\pgfqpoint{2.114013in}{2.045939in}}%
\pgfpathlineto{\pgfqpoint{2.070438in}{2.045939in}}%
\pgfpathlineto{\pgfqpoint{2.070438in}{2.048402in}}%
\pgfpathclose%
\pgfusepath{stroke,fill}%
\end{pgfscope}%
\begin{pgfscope}%
\pgfpathrectangle{\pgfqpoint{0.703330in}{0.352393in}}{\pgfqpoint{2.173234in}{2.758940in}}%
\pgfusepath{clip}%
\pgfsetbuttcap%
\pgfsetroundjoin%
\definecolor{currentfill}{rgb}{0.901453,0.815098,0.767370}%
\pgfsetfillcolor{currentfill}%
\pgfsetlinewidth{0.602250pt}%
\definecolor{currentstroke}{rgb}{0.296471,0.296471,0.296471}%
\pgfsetstrokecolor{currentstroke}%
\pgfsetdash{}{0pt}%
\pgfpathmoveto{\pgfqpoint{2.114013in}{2.049634in}}%
\pgfpathlineto{\pgfqpoint{2.134820in}{2.049634in}}%
\pgfpathlineto{\pgfqpoint{2.134820in}{2.044707in}}%
\pgfpathlineto{\pgfqpoint{2.114013in}{2.044707in}}%
\pgfpathlineto{\pgfqpoint{2.114013in}{2.049634in}}%
\pgfpathclose%
\pgfusepath{stroke,fill}%
\end{pgfscope}%
\begin{pgfscope}%
\pgfpathrectangle{\pgfqpoint{0.703330in}{0.352393in}}{\pgfqpoint{2.173234in}{2.758940in}}%
\pgfusepath{clip}%
\pgfsetbuttcap%
\pgfsetroundjoin%
\definecolor{currentfill}{rgb}{0.889102,0.781698,0.722050}%
\pgfsetfillcolor{currentfill}%
\pgfsetlinewidth{0.602250pt}%
\definecolor{currentstroke}{rgb}{0.296471,0.296471,0.296471}%
\pgfsetstrokecolor{currentstroke}%
\pgfsetdash{}{0pt}%
\pgfpathmoveto{\pgfqpoint{2.134820in}{2.052097in}}%
\pgfpathlineto{\pgfqpoint{2.150653in}{2.052097in}}%
\pgfpathlineto{\pgfqpoint{2.150653in}{2.042244in}}%
\pgfpathlineto{\pgfqpoint{2.134820in}{2.042244in}}%
\pgfpathlineto{\pgfqpoint{2.134820in}{2.052097in}}%
\pgfpathclose%
\pgfusepath{stroke,fill}%
\end{pgfscope}%
\begin{pgfscope}%
\pgfpathrectangle{\pgfqpoint{0.703330in}{0.352393in}}{\pgfqpoint{2.173234in}{2.758940in}}%
\pgfusepath{clip}%
\pgfsetbuttcap%
\pgfsetroundjoin%
\definecolor{currentfill}{rgb}{0.873223,0.738755,0.663782}%
\pgfsetfillcolor{currentfill}%
\pgfsetlinewidth{0.602250pt}%
\definecolor{currentstroke}{rgb}{0.296471,0.296471,0.296471}%
\pgfsetstrokecolor{currentstroke}%
\pgfsetdash{}{0pt}%
\pgfpathmoveto{\pgfqpoint{2.150653in}{2.057024in}}%
\pgfpathlineto{\pgfqpoint{2.221545in}{2.057024in}}%
\pgfpathlineto{\pgfqpoint{2.221545in}{2.037317in}}%
\pgfpathlineto{\pgfqpoint{2.150653in}{2.037317in}}%
\pgfpathlineto{\pgfqpoint{2.150653in}{2.057024in}}%
\pgfpathclose%
\pgfusepath{stroke,fill}%
\end{pgfscope}%
\begin{pgfscope}%
\pgfpathrectangle{\pgfqpoint{0.703330in}{0.352393in}}{\pgfqpoint{2.173234in}{2.758940in}}%
\pgfusepath{clip}%
\pgfsetbuttcap%
\pgfsetroundjoin%
\definecolor{currentfill}{rgb}{0.853814,0.686269,0.592565}%
\pgfsetfillcolor{currentfill}%
\pgfsetlinewidth{0.602250pt}%
\definecolor{currentstroke}{rgb}{0.296471,0.296471,0.296471}%
\pgfsetstrokecolor{currentstroke}%
\pgfsetdash{}{0pt}%
\pgfpathmoveto{\pgfqpoint{2.221545in}{2.066877in}}%
\pgfpathlineto{\pgfqpoint{2.272224in}{2.066877in}}%
\pgfpathlineto{\pgfqpoint{2.272224in}{2.027464in}}%
\pgfpathlineto{\pgfqpoint{2.221545in}{2.027464in}}%
\pgfpathlineto{\pgfqpoint{2.221545in}{2.066877in}}%
\pgfpathclose%
\pgfusepath{stroke,fill}%
\end{pgfscope}%
\begin{pgfscope}%
\pgfpathrectangle{\pgfqpoint{0.703330in}{0.352393in}}{\pgfqpoint{2.173234in}{2.758940in}}%
\pgfusepath{clip}%
\pgfsetbuttcap%
\pgfsetroundjoin%
\definecolor{currentfill}{rgb}{0.829112,0.619469,0.501926}%
\pgfsetfillcolor{currentfill}%
\pgfsetlinewidth{0.602250pt}%
\definecolor{currentstroke}{rgb}{0.296471,0.296471,0.296471}%
\pgfsetstrokecolor{currentstroke}%
\pgfsetdash{}{0pt}%
\pgfpathmoveto{\pgfqpoint{2.272224in}{2.086584in}}%
\pgfpathlineto{\pgfqpoint{2.347357in}{2.086584in}}%
\pgfpathlineto{\pgfqpoint{2.347357in}{2.007757in}}%
\pgfpathlineto{\pgfqpoint{2.272224in}{2.007757in}}%
\pgfpathlineto{\pgfqpoint{2.272224in}{2.086584in}}%
\pgfpathclose%
\pgfusepath{stroke,fill}%
\end{pgfscope}%
\begin{pgfscope}%
\pgfpathrectangle{\pgfqpoint{0.703330in}{0.352393in}}{\pgfqpoint{2.173234in}{2.758940in}}%
\pgfusepath{clip}%
\pgfsetbuttcap%
\pgfsetroundjoin%
\definecolor{currentfill}{rgb}{0.798529,0.536765,0.389706}%
\pgfsetfillcolor{currentfill}%
\pgfsetlinewidth{0.602250pt}%
\definecolor{currentstroke}{rgb}{0.296471,0.296471,0.296471}%
\pgfsetstrokecolor{currentstroke}%
\pgfsetdash{}{0pt}%
\pgfpathmoveto{\pgfqpoint{2.347357in}{2.125997in}}%
\pgfpathlineto{\pgfqpoint{2.876564in}{2.125997in}}%
\pgfpathlineto{\pgfqpoint{2.876564in}{1.968344in}}%
\pgfpathlineto{\pgfqpoint{2.347357in}{1.968344in}}%
\pgfpathlineto{\pgfqpoint{2.347357in}{2.125997in}}%
\pgfpathclose%
\pgfusepath{stroke,fill}%
\end{pgfscope}%
\begin{pgfscope}%
\pgfpathrectangle{\pgfqpoint{0.703330in}{0.352393in}}{\pgfqpoint{2.173234in}{2.758940in}}%
\pgfusepath{clip}%
\pgfsetbuttcap%
\pgfsetroundjoin%
\definecolor{currentfill}{rgb}{0.829112,0.619469,0.501926}%
\pgfsetfillcolor{currentfill}%
\pgfsetlinewidth{0.602250pt}%
\definecolor{currentstroke}{rgb}{0.296471,0.296471,0.296471}%
\pgfsetstrokecolor{currentstroke}%
\pgfsetdash{}{0pt}%
\pgfpathmoveto{\pgfqpoint{2.876564in}{2.086584in}}%
\pgfpathlineto{\pgfqpoint{2.876564in}{2.086584in}}%
\pgfpathlineto{\pgfqpoint{2.876564in}{2.007757in}}%
\pgfpathlineto{\pgfqpoint{2.876564in}{2.007757in}}%
\pgfpathlineto{\pgfqpoint{2.876564in}{2.086584in}}%
\pgfpathclose%
\pgfusepath{stroke,fill}%
\end{pgfscope}%
\begin{pgfscope}%
\pgfpathrectangle{\pgfqpoint{0.703330in}{0.352393in}}{\pgfqpoint{2.173234in}{2.758940in}}%
\pgfusepath{clip}%
\pgfsetbuttcap%
\pgfsetroundjoin%
\definecolor{currentfill}{rgb}{0.853814,0.686269,0.592565}%
\pgfsetfillcolor{currentfill}%
\pgfsetlinewidth{0.602250pt}%
\definecolor{currentstroke}{rgb}{0.296471,0.296471,0.296471}%
\pgfsetstrokecolor{currentstroke}%
\pgfsetdash{}{0pt}%
\pgfpathmoveto{\pgfqpoint{2.876564in}{2.066877in}}%
\pgfpathlineto{\pgfqpoint{2.876564in}{2.066877in}}%
\pgfpathlineto{\pgfqpoint{2.876564in}{2.027464in}}%
\pgfpathlineto{\pgfqpoint{2.876564in}{2.027464in}}%
\pgfpathlineto{\pgfqpoint{2.876564in}{2.066877in}}%
\pgfpathclose%
\pgfusepath{stroke,fill}%
\end{pgfscope}%
\begin{pgfscope}%
\pgfpathrectangle{\pgfqpoint{0.703330in}{0.352393in}}{\pgfqpoint{2.173234in}{2.758940in}}%
\pgfusepath{clip}%
\pgfsetbuttcap%
\pgfsetroundjoin%
\definecolor{currentfill}{rgb}{0.873223,0.738755,0.663782}%
\pgfsetfillcolor{currentfill}%
\pgfsetlinewidth{0.602250pt}%
\definecolor{currentstroke}{rgb}{0.296471,0.296471,0.296471}%
\pgfsetstrokecolor{currentstroke}%
\pgfsetdash{}{0pt}%
\pgfpathmoveto{\pgfqpoint{2.876564in}{2.057024in}}%
\pgfpathlineto{\pgfqpoint{2.876564in}{2.057024in}}%
\pgfpathlineto{\pgfqpoint{2.876564in}{2.037317in}}%
\pgfpathlineto{\pgfqpoint{2.876564in}{2.037317in}}%
\pgfpathlineto{\pgfqpoint{2.876564in}{2.057024in}}%
\pgfpathclose%
\pgfusepath{stroke,fill}%
\end{pgfscope}%
\begin{pgfscope}%
\pgfpathrectangle{\pgfqpoint{0.703330in}{0.352393in}}{\pgfqpoint{2.173234in}{2.758940in}}%
\pgfusepath{clip}%
\pgfsetbuttcap%
\pgfsetroundjoin%
\definecolor{currentfill}{rgb}{0.889102,0.781698,0.722050}%
\pgfsetfillcolor{currentfill}%
\pgfsetlinewidth{0.602250pt}%
\definecolor{currentstroke}{rgb}{0.296471,0.296471,0.296471}%
\pgfsetstrokecolor{currentstroke}%
\pgfsetdash{}{0pt}%
\pgfpathmoveto{\pgfqpoint{2.876564in}{2.052097in}}%
\pgfpathlineto{\pgfqpoint{2.876564in}{2.052097in}}%
\pgfpathlineto{\pgfqpoint{2.876564in}{2.042244in}}%
\pgfpathlineto{\pgfqpoint{2.876564in}{2.042244in}}%
\pgfpathlineto{\pgfqpoint{2.876564in}{2.052097in}}%
\pgfpathclose%
\pgfusepath{stroke,fill}%
\end{pgfscope}%
\begin{pgfscope}%
\pgfpathrectangle{\pgfqpoint{0.703330in}{0.352393in}}{\pgfqpoint{2.173234in}{2.758940in}}%
\pgfusepath{clip}%
\pgfsetbuttcap%
\pgfsetroundjoin%
\definecolor{currentfill}{rgb}{0.901453,0.815098,0.767370}%
\pgfsetfillcolor{currentfill}%
\pgfsetlinewidth{0.602250pt}%
\definecolor{currentstroke}{rgb}{0.296471,0.296471,0.296471}%
\pgfsetstrokecolor{currentstroke}%
\pgfsetdash{}{0pt}%
\pgfpathmoveto{\pgfqpoint{2.876564in}{2.049634in}}%
\pgfpathlineto{\pgfqpoint{2.876564in}{2.049634in}}%
\pgfpathlineto{\pgfqpoint{2.876564in}{2.044707in}}%
\pgfpathlineto{\pgfqpoint{2.876564in}{2.044707in}}%
\pgfpathlineto{\pgfqpoint{2.876564in}{2.049634in}}%
\pgfpathclose%
\pgfusepath{stroke,fill}%
\end{pgfscope}%
\begin{pgfscope}%
\pgfpathrectangle{\pgfqpoint{0.703330in}{0.352393in}}{\pgfqpoint{2.173234in}{2.758940in}}%
\pgfusepath{clip}%
\pgfsetbuttcap%
\pgfsetroundjoin%
\definecolor{currentfill}{rgb}{0.910863,0.840546,0.801899}%
\pgfsetfillcolor{currentfill}%
\pgfsetlinewidth{0.602250pt}%
\definecolor{currentstroke}{rgb}{0.296471,0.296471,0.296471}%
\pgfsetstrokecolor{currentstroke}%
\pgfsetdash{}{0pt}%
\pgfpathmoveto{\pgfqpoint{2.876564in}{2.048402in}}%
\pgfpathlineto{\pgfqpoint{2.876564in}{2.048402in}}%
\pgfpathlineto{\pgfqpoint{2.876564in}{2.045939in}}%
\pgfpathlineto{\pgfqpoint{2.876564in}{2.045939in}}%
\pgfpathlineto{\pgfqpoint{2.876564in}{2.048402in}}%
\pgfpathclose%
\pgfusepath{stroke,fill}%
\end{pgfscope}%
\begin{pgfscope}%
\pgfpathrectangle{\pgfqpoint{0.703330in}{0.352393in}}{\pgfqpoint{2.173234in}{2.758940in}}%
\pgfusepath{clip}%
\pgfsetbuttcap%
\pgfsetroundjoin%
\definecolor{currentfill}{rgb}{0.919097,0.862812,0.832112}%
\pgfsetfillcolor{currentfill}%
\pgfsetlinewidth{0.602250pt}%
\definecolor{currentstroke}{rgb}{0.296471,0.296471,0.296471}%
\pgfsetstrokecolor{currentstroke}%
\pgfsetdash{}{0pt}%
\pgfpathmoveto{\pgfqpoint{2.876564in}{2.047786in}}%
\pgfpathlineto{\pgfqpoint{2.876564in}{2.047786in}}%
\pgfpathlineto{\pgfqpoint{2.876564in}{2.046555in}}%
\pgfpathlineto{\pgfqpoint{2.876564in}{2.046555in}}%
\pgfpathlineto{\pgfqpoint{2.876564in}{2.047786in}}%
\pgfpathclose%
\pgfusepath{stroke,fill}%
\end{pgfscope}%
\begin{pgfscope}%
\pgfpathrectangle{\pgfqpoint{0.703330in}{0.352393in}}{\pgfqpoint{2.173234in}{2.758940in}}%
\pgfusepath{clip}%
\pgfsetbuttcap%
\pgfsetroundjoin%
\pgfsetlinewidth{0.803000pt}%
\definecolor{currentstroke}{rgb}{0.450000,0.450000,0.450000}%
\pgfsetstrokecolor{currentstroke}%
\pgfsetdash{}{0pt}%
\pgfpathmoveto{\pgfqpoint{0.000000in}{-0.034722in}}%
\pgfpathcurveto{\pgfqpoint{0.009208in}{-0.034722in}}{\pgfqpoint{0.018041in}{-0.031064in}}{\pgfqpoint{0.024552in}{-0.024552in}}%
\pgfpathcurveto{\pgfqpoint{0.031064in}{-0.018041in}}{\pgfqpoint{0.034722in}{-0.009208in}}{\pgfqpoint{0.034722in}{0.000000in}}%
\pgfpathcurveto{\pgfqpoint{0.034722in}{0.009208in}}{\pgfqpoint{0.031064in}{0.018041in}}{\pgfqpoint{0.024552in}{0.024552in}}%
\pgfpathcurveto{\pgfqpoint{0.018041in}{0.031064in}}{\pgfqpoint{0.009208in}{0.034722in}}{\pgfqpoint{0.000000in}{0.034722in}}%
\pgfpathcurveto{\pgfqpoint{-0.009208in}{0.034722in}}{\pgfqpoint{-0.018041in}{0.031064in}}{\pgfqpoint{-0.024552in}{0.024552in}}%
\pgfpathcurveto{\pgfqpoint{-0.031064in}{0.018041in}}{\pgfqpoint{-0.034722in}{0.009208in}}{\pgfqpoint{-0.034722in}{0.000000in}}%
\pgfpathcurveto{\pgfqpoint{-0.034722in}{-0.009208in}}{\pgfqpoint{-0.031064in}{-0.018041in}}{\pgfqpoint{-0.024552in}{-0.024552in}}%
\pgfpathcurveto{\pgfqpoint{-0.018041in}{-0.031064in}}{\pgfqpoint{-0.009208in}{-0.034722in}}{\pgfqpoint{0.000000in}{-0.034722in}}%
\pgfusepath{stroke}%
\end{pgfscope}%
\begin{pgfscope}%
\pgfpathrectangle{\pgfqpoint{0.703330in}{0.352393in}}{\pgfqpoint{2.173234in}{2.758940in}}%
\pgfusepath{clip}%
\pgfsetbuttcap%
\pgfsetroundjoin%
\definecolor{currentfill}{rgb}{0.848437,0.867532,0.899724}%
\pgfsetfillcolor{currentfill}%
\pgfsetlinewidth{0.602250pt}%
\definecolor{currentstroke}{rgb}{0.296471,0.296471,0.296471}%
\pgfsetstrokecolor{currentstroke}%
\pgfsetdash{}{0pt}%
\pgfpathmoveto{\pgfqpoint{-0.228056in}{1.810998in}}%
\pgfpathlineto{\pgfqpoint{-0.228056in}{1.810998in}}%
\pgfpathlineto{\pgfqpoint{-0.228056in}{1.810382in}}%
\pgfpathlineto{\pgfqpoint{-0.228056in}{1.810382in}}%
\pgfpathlineto{\pgfqpoint{-0.228056in}{1.810998in}}%
\pgfpathclose%
\pgfusepath{stroke,fill}%
\end{pgfscope}%
\begin{pgfscope}%
\pgfpathrectangle{\pgfqpoint{0.703330in}{0.352393in}}{\pgfqpoint{2.173234in}{2.758940in}}%
\pgfusepath{clip}%
\pgfsetbuttcap%
\pgfsetroundjoin%
\definecolor{currentfill}{rgb}{0.825117,0.848522,0.887698}%
\pgfsetfillcolor{currentfill}%
\pgfsetlinewidth{0.602250pt}%
\definecolor{currentstroke}{rgb}{0.296471,0.296471,0.296471}%
\pgfsetstrokecolor{currentstroke}%
\pgfsetdash{}{0pt}%
\pgfpathmoveto{\pgfqpoint{-0.228056in}{1.811306in}}%
\pgfpathlineto{\pgfqpoint{-0.228056in}{1.811306in}}%
\pgfpathlineto{\pgfqpoint{-0.228056in}{1.810074in}}%
\pgfpathlineto{\pgfqpoint{-0.228056in}{1.810074in}}%
\pgfpathlineto{\pgfqpoint{-0.228056in}{1.811306in}}%
\pgfpathclose%
\pgfusepath{stroke,fill}%
\end{pgfscope}%
\begin{pgfscope}%
\pgfpathrectangle{\pgfqpoint{0.703330in}{0.352393in}}{\pgfqpoint{2.173234in}{2.758940in}}%
\pgfusepath{clip}%
\pgfsetbuttcap%
\pgfsetroundjoin%
\definecolor{currentfill}{rgb}{0.792469,0.821908,0.870863}%
\pgfsetfillcolor{currentfill}%
\pgfsetlinewidth{0.602250pt}%
\definecolor{currentstroke}{rgb}{0.296471,0.296471,0.296471}%
\pgfsetstrokecolor{currentstroke}%
\pgfsetdash{}{0pt}%
\pgfpathmoveto{\pgfqpoint{-0.228056in}{1.811922in}}%
\pgfpathlineto{\pgfqpoint{-0.228056in}{1.811922in}}%
\pgfpathlineto{\pgfqpoint{-0.228056in}{1.809458in}}%
\pgfpathlineto{\pgfqpoint{-0.228056in}{1.809458in}}%
\pgfpathlineto{\pgfqpoint{-0.228056in}{1.811922in}}%
\pgfpathclose%
\pgfusepath{stroke,fill}%
\end{pgfscope}%
\begin{pgfscope}%
\pgfpathrectangle{\pgfqpoint{0.703330in}{0.352393in}}{\pgfqpoint{2.173234in}{2.758940in}}%
\pgfusepath{clip}%
\pgfsetbuttcap%
\pgfsetroundjoin%
\definecolor{currentfill}{rgb}{0.755157,0.791493,0.851622}%
\pgfsetfillcolor{currentfill}%
\pgfsetlinewidth{0.602250pt}%
\definecolor{currentstroke}{rgb}{0.296471,0.296471,0.296471}%
\pgfsetstrokecolor{currentstroke}%
\pgfsetdash{}{0pt}%
\pgfpathmoveto{\pgfqpoint{-0.228056in}{1.813153in}}%
\pgfpathlineto{\pgfqpoint{-0.228056in}{1.813153in}}%
\pgfpathlineto{\pgfqpoint{-0.228056in}{1.808227in}}%
\pgfpathlineto{\pgfqpoint{-0.228056in}{1.808227in}}%
\pgfpathlineto{\pgfqpoint{-0.228056in}{1.813153in}}%
\pgfpathclose%
\pgfusepath{stroke,fill}%
\end{pgfscope}%
\begin{pgfscope}%
\pgfpathrectangle{\pgfqpoint{0.703330in}{0.352393in}}{\pgfqpoint{2.173234in}{2.758940in}}%
\pgfusepath{clip}%
\pgfsetbuttcap%
\pgfsetroundjoin%
\definecolor{currentfill}{rgb}{0.706185,0.751573,0.826368}%
\pgfsetfillcolor{currentfill}%
\pgfsetlinewidth{0.602250pt}%
\definecolor{currentstroke}{rgb}{0.296471,0.296471,0.296471}%
\pgfsetstrokecolor{currentstroke}%
\pgfsetdash{}{0pt}%
\pgfpathmoveto{\pgfqpoint{-0.228056in}{1.815617in}}%
\pgfpathlineto{\pgfqpoint{-0.228056in}{1.815617in}}%
\pgfpathlineto{\pgfqpoint{-0.228056in}{1.805763in}}%
\pgfpathlineto{\pgfqpoint{-0.228056in}{1.805763in}}%
\pgfpathlineto{\pgfqpoint{-0.228056in}{1.815617in}}%
\pgfpathclose%
\pgfusepath{stroke,fill}%
\end{pgfscope}%
\begin{pgfscope}%
\pgfpathrectangle{\pgfqpoint{0.703330in}{0.352393in}}{\pgfqpoint{2.173234in}{2.758940in}}%
\pgfusepath{clip}%
\pgfsetbuttcap%
\pgfsetroundjoin%
\definecolor{currentfill}{rgb}{0.643221,0.700246,0.793900}%
\pgfsetfillcolor{currentfill}%
\pgfsetlinewidth{0.602250pt}%
\definecolor{currentstroke}{rgb}{0.296471,0.296471,0.296471}%
\pgfsetstrokecolor{currentstroke}%
\pgfsetdash{}{0pt}%
\pgfpathmoveto{\pgfqpoint{-0.228056in}{1.820543in}}%
\pgfpathlineto{\pgfqpoint{-0.228056in}{1.820543in}}%
\pgfpathlineto{\pgfqpoint{-0.228056in}{1.800837in}}%
\pgfpathlineto{\pgfqpoint{-0.228056in}{1.800837in}}%
\pgfpathlineto{\pgfqpoint{-0.228056in}{1.820543in}}%
\pgfpathclose%
\pgfusepath{stroke,fill}%
\end{pgfscope}%
\begin{pgfscope}%
\pgfpathrectangle{\pgfqpoint{0.703330in}{0.352393in}}{\pgfqpoint{2.173234in}{2.758940in}}%
\pgfusepath{clip}%
\pgfsetbuttcap%
\pgfsetroundjoin%
\definecolor{currentfill}{rgb}{0.566266,0.637515,0.754216}%
\pgfsetfillcolor{currentfill}%
\pgfsetlinewidth{0.602250pt}%
\definecolor{currentstroke}{rgb}{0.296471,0.296471,0.296471}%
\pgfsetstrokecolor{currentstroke}%
\pgfsetdash{}{0pt}%
\pgfpathmoveto{\pgfqpoint{-0.228056in}{1.830397in}}%
\pgfpathlineto{\pgfqpoint{-0.228056in}{1.830397in}}%
\pgfpathlineto{\pgfqpoint{-0.228056in}{1.790983in}}%
\pgfpathlineto{\pgfqpoint{-0.228056in}{1.790983in}}%
\pgfpathlineto{\pgfqpoint{-0.228056in}{1.830397in}}%
\pgfpathclose%
\pgfusepath{stroke,fill}%
\end{pgfscope}%
\begin{pgfscope}%
\pgfpathrectangle{\pgfqpoint{0.703330in}{0.352393in}}{\pgfqpoint{2.173234in}{2.758940in}}%
\pgfusepath{clip}%
\pgfsetbuttcap%
\pgfsetroundjoin%
\definecolor{currentfill}{rgb}{0.468322,0.557674,0.703709}%
\pgfsetfillcolor{currentfill}%
\pgfsetlinewidth{0.602250pt}%
\definecolor{currentstroke}{rgb}{0.296471,0.296471,0.296471}%
\pgfsetstrokecolor{currentstroke}%
\pgfsetdash{}{0pt}%
\pgfpathmoveto{\pgfqpoint{-0.228056in}{1.850103in}}%
\pgfpathlineto{\pgfqpoint{-0.228056in}{1.850103in}}%
\pgfpathlineto{\pgfqpoint{-0.228056in}{1.771277in}}%
\pgfpathlineto{\pgfqpoint{-0.228056in}{1.771277in}}%
\pgfpathlineto{\pgfqpoint{-0.228056in}{1.850103in}}%
\pgfpathclose%
\pgfusepath{stroke,fill}%
\end{pgfscope}%
\begin{pgfscope}%
\pgfpathrectangle{\pgfqpoint{0.703330in}{0.352393in}}{\pgfqpoint{2.173234in}{2.758940in}}%
\pgfusepath{clip}%
\pgfsetbuttcap%
\pgfsetroundjoin%
\definecolor{currentfill}{rgb}{0.347059,0.458824,0.641176}%
\pgfsetfillcolor{currentfill}%
\pgfsetlinewidth{0.602250pt}%
\definecolor{currentstroke}{rgb}{0.296471,0.296471,0.296471}%
\pgfsetstrokecolor{currentstroke}%
\pgfsetdash{}{0pt}%
\pgfpathmoveto{\pgfqpoint{-0.228056in}{1.889517in}}%
\pgfpathlineto{\pgfqpoint{-0.228056in}{1.889517in}}%
\pgfpathlineto{\pgfqpoint{-0.228056in}{1.731863in}}%
\pgfpathlineto{\pgfqpoint{-0.228056in}{1.731863in}}%
\pgfpathlineto{\pgfqpoint{-0.228056in}{1.889517in}}%
\pgfpathclose%
\pgfusepath{stroke,fill}%
\end{pgfscope}%
\begin{pgfscope}%
\pgfpathrectangle{\pgfqpoint{0.703330in}{0.352393in}}{\pgfqpoint{2.173234in}{2.758940in}}%
\pgfusepath{clip}%
\pgfsetbuttcap%
\pgfsetroundjoin%
\definecolor{currentfill}{rgb}{0.468322,0.557674,0.703709}%
\pgfsetfillcolor{currentfill}%
\pgfsetlinewidth{0.602250pt}%
\definecolor{currentstroke}{rgb}{0.296471,0.296471,0.296471}%
\pgfsetstrokecolor{currentstroke}%
\pgfsetdash{}{0pt}%
\pgfpathmoveto{\pgfqpoint{-0.228056in}{1.850103in}}%
\pgfpathlineto{\pgfqpoint{-0.228056in}{1.850103in}}%
\pgfpathlineto{\pgfqpoint{-0.228056in}{1.771277in}}%
\pgfpathlineto{\pgfqpoint{-0.228056in}{1.771277in}}%
\pgfpathlineto{\pgfqpoint{-0.228056in}{1.850103in}}%
\pgfpathclose%
\pgfusepath{stroke,fill}%
\end{pgfscope}%
\begin{pgfscope}%
\pgfpathrectangle{\pgfqpoint{0.703330in}{0.352393in}}{\pgfqpoint{2.173234in}{2.758940in}}%
\pgfusepath{clip}%
\pgfsetbuttcap%
\pgfsetroundjoin%
\definecolor{currentfill}{rgb}{0.566266,0.637515,0.754216}%
\pgfsetfillcolor{currentfill}%
\pgfsetlinewidth{0.602250pt}%
\definecolor{currentstroke}{rgb}{0.296471,0.296471,0.296471}%
\pgfsetstrokecolor{currentstroke}%
\pgfsetdash{}{0pt}%
\pgfpathmoveto{\pgfqpoint{-0.228056in}{1.830397in}}%
\pgfpathlineto{\pgfqpoint{-0.228056in}{1.830397in}}%
\pgfpathlineto{\pgfqpoint{-0.228056in}{1.790983in}}%
\pgfpathlineto{\pgfqpoint{-0.228056in}{1.790983in}}%
\pgfpathlineto{\pgfqpoint{-0.228056in}{1.830397in}}%
\pgfpathclose%
\pgfusepath{stroke,fill}%
\end{pgfscope}%
\begin{pgfscope}%
\pgfpathrectangle{\pgfqpoint{0.703330in}{0.352393in}}{\pgfqpoint{2.173234in}{2.758940in}}%
\pgfusepath{clip}%
\pgfsetbuttcap%
\pgfsetroundjoin%
\definecolor{currentfill}{rgb}{0.643221,0.700246,0.793900}%
\pgfsetfillcolor{currentfill}%
\pgfsetlinewidth{0.602250pt}%
\definecolor{currentstroke}{rgb}{0.296471,0.296471,0.296471}%
\pgfsetstrokecolor{currentstroke}%
\pgfsetdash{}{0pt}%
\pgfpathmoveto{\pgfqpoint{-0.228056in}{1.820543in}}%
\pgfpathlineto{\pgfqpoint{-0.228056in}{1.820543in}}%
\pgfpathlineto{\pgfqpoint{-0.228056in}{1.800837in}}%
\pgfpathlineto{\pgfqpoint{-0.228056in}{1.800837in}}%
\pgfpathlineto{\pgfqpoint{-0.228056in}{1.820543in}}%
\pgfpathclose%
\pgfusepath{stroke,fill}%
\end{pgfscope}%
\begin{pgfscope}%
\pgfpathrectangle{\pgfqpoint{0.703330in}{0.352393in}}{\pgfqpoint{2.173234in}{2.758940in}}%
\pgfusepath{clip}%
\pgfsetbuttcap%
\pgfsetroundjoin%
\definecolor{currentfill}{rgb}{0.706185,0.751573,0.826368}%
\pgfsetfillcolor{currentfill}%
\pgfsetlinewidth{0.602250pt}%
\definecolor{currentstroke}{rgb}{0.296471,0.296471,0.296471}%
\pgfsetstrokecolor{currentstroke}%
\pgfsetdash{}{0pt}%
\pgfpathmoveto{\pgfqpoint{-0.228056in}{1.815617in}}%
\pgfpathlineto{\pgfqpoint{-0.228056in}{1.815617in}}%
\pgfpathlineto{\pgfqpoint{-0.228056in}{1.805763in}}%
\pgfpathlineto{\pgfqpoint{-0.228056in}{1.805763in}}%
\pgfpathlineto{\pgfqpoint{-0.228056in}{1.815617in}}%
\pgfpathclose%
\pgfusepath{stroke,fill}%
\end{pgfscope}%
\begin{pgfscope}%
\pgfpathrectangle{\pgfqpoint{0.703330in}{0.352393in}}{\pgfqpoint{2.173234in}{2.758940in}}%
\pgfusepath{clip}%
\pgfsetbuttcap%
\pgfsetroundjoin%
\definecolor{currentfill}{rgb}{0.755157,0.791493,0.851622}%
\pgfsetfillcolor{currentfill}%
\pgfsetlinewidth{0.602250pt}%
\definecolor{currentstroke}{rgb}{0.296471,0.296471,0.296471}%
\pgfsetstrokecolor{currentstroke}%
\pgfsetdash{}{0pt}%
\pgfpathmoveto{\pgfqpoint{-0.228056in}{1.813153in}}%
\pgfpathlineto{\pgfqpoint{-0.228056in}{1.813153in}}%
\pgfpathlineto{\pgfqpoint{-0.228056in}{1.808227in}}%
\pgfpathlineto{\pgfqpoint{-0.228056in}{1.808227in}}%
\pgfpathlineto{\pgfqpoint{-0.228056in}{1.813153in}}%
\pgfpathclose%
\pgfusepath{stroke,fill}%
\end{pgfscope}%
\begin{pgfscope}%
\pgfpathrectangle{\pgfqpoint{0.703330in}{0.352393in}}{\pgfqpoint{2.173234in}{2.758940in}}%
\pgfusepath{clip}%
\pgfsetbuttcap%
\pgfsetroundjoin%
\definecolor{currentfill}{rgb}{0.792469,0.821908,0.870863}%
\pgfsetfillcolor{currentfill}%
\pgfsetlinewidth{0.602250pt}%
\definecolor{currentstroke}{rgb}{0.296471,0.296471,0.296471}%
\pgfsetstrokecolor{currentstroke}%
\pgfsetdash{}{0pt}%
\pgfpathmoveto{\pgfqpoint{-0.228056in}{1.811922in}}%
\pgfpathlineto{\pgfqpoint{-0.228056in}{1.811922in}}%
\pgfpathlineto{\pgfqpoint{-0.228056in}{1.809458in}}%
\pgfpathlineto{\pgfqpoint{-0.228056in}{1.809458in}}%
\pgfpathlineto{\pgfqpoint{-0.228056in}{1.811922in}}%
\pgfpathclose%
\pgfusepath{stroke,fill}%
\end{pgfscope}%
\begin{pgfscope}%
\pgfpathrectangle{\pgfqpoint{0.703330in}{0.352393in}}{\pgfqpoint{2.173234in}{2.758940in}}%
\pgfusepath{clip}%
\pgfsetbuttcap%
\pgfsetroundjoin%
\definecolor{currentfill}{rgb}{0.825117,0.848522,0.887698}%
\pgfsetfillcolor{currentfill}%
\pgfsetlinewidth{0.602250pt}%
\definecolor{currentstroke}{rgb}{0.296471,0.296471,0.296471}%
\pgfsetstrokecolor{currentstroke}%
\pgfsetdash{}{0pt}%
\pgfpathmoveto{\pgfqpoint{-0.228056in}{1.811306in}}%
\pgfpathlineto{\pgfqpoint{-0.228056in}{1.811306in}}%
\pgfpathlineto{\pgfqpoint{-0.228056in}{1.810074in}}%
\pgfpathlineto{\pgfqpoint{-0.228056in}{1.810074in}}%
\pgfpathlineto{\pgfqpoint{-0.228056in}{1.811306in}}%
\pgfpathclose%
\pgfusepath{stroke,fill}%
\end{pgfscope}%
\begin{pgfscope}%
\pgfpathrectangle{\pgfqpoint{0.703330in}{0.352393in}}{\pgfqpoint{2.173234in}{2.758940in}}%
\pgfusepath{clip}%
\pgfsetbuttcap%
\pgfsetroundjoin%
\definecolor{currentfill}{rgb}{0.848437,0.867532,0.899724}%
\pgfsetfillcolor{currentfill}%
\pgfsetlinewidth{0.602250pt}%
\definecolor{currentstroke}{rgb}{0.296471,0.296471,0.296471}%
\pgfsetstrokecolor{currentstroke}%
\pgfsetdash{}{0pt}%
\pgfpathmoveto{\pgfqpoint{-0.228056in}{1.810998in}}%
\pgfpathlineto{\pgfqpoint{-0.228056in}{1.810998in}}%
\pgfpathlineto{\pgfqpoint{-0.228056in}{1.810382in}}%
\pgfpathlineto{\pgfqpoint{-0.228056in}{1.810382in}}%
\pgfpathlineto{\pgfqpoint{-0.228056in}{1.810998in}}%
\pgfpathclose%
\pgfusepath{stroke,fill}%
\end{pgfscope}%
\begin{pgfscope}%
\pgfpathrectangle{\pgfqpoint{0.703330in}{0.352393in}}{\pgfqpoint{2.173234in}{2.758940in}}%
\pgfusepath{clip}%
\pgfsetbuttcap%
\pgfsetroundjoin%
\pgfsetlinewidth{0.803000pt}%
\definecolor{currentstroke}{rgb}{0.450000,0.450000,0.450000}%
\pgfsetstrokecolor{currentstroke}%
\pgfsetdash{}{0pt}%
\pgfpathmoveto{\pgfqpoint{0.000000in}{-0.034722in}}%
\pgfpathcurveto{\pgfqpoint{0.009208in}{-0.034722in}}{\pgfqpoint{0.018041in}{-0.031064in}}{\pgfqpoint{0.024552in}{-0.024552in}}%
\pgfpathcurveto{\pgfqpoint{0.031064in}{-0.018041in}}{\pgfqpoint{0.034722in}{-0.009208in}}{\pgfqpoint{0.034722in}{0.000000in}}%
\pgfpathcurveto{\pgfqpoint{0.034722in}{0.009208in}}{\pgfqpoint{0.031064in}{0.018041in}}{\pgfqpoint{0.024552in}{0.024552in}}%
\pgfpathcurveto{\pgfqpoint{0.018041in}{0.031064in}}{\pgfqpoint{0.009208in}{0.034722in}}{\pgfqpoint{0.000000in}{0.034722in}}%
\pgfpathcurveto{\pgfqpoint{-0.009208in}{0.034722in}}{\pgfqpoint{-0.018041in}{0.031064in}}{\pgfqpoint{-0.024552in}{0.024552in}}%
\pgfpathcurveto{\pgfqpoint{-0.031064in}{0.018041in}}{\pgfqpoint{-0.034722in}{0.009208in}}{\pgfqpoint{-0.034722in}{0.000000in}}%
\pgfpathcurveto{\pgfqpoint{-0.034722in}{-0.009208in}}{\pgfqpoint{-0.031064in}{-0.018041in}}{\pgfqpoint{-0.024552in}{-0.024552in}}%
\pgfpathcurveto{\pgfqpoint{-0.018041in}{-0.031064in}}{\pgfqpoint{-0.009208in}{-0.034722in}}{\pgfqpoint{0.000000in}{-0.034722in}}%
\pgfusepath{stroke}%
\end{pgfscope}%
\begin{pgfscope}%
\pgfpathrectangle{\pgfqpoint{0.703330in}{0.352393in}}{\pgfqpoint{2.173234in}{2.758940in}}%
\pgfusepath{clip}%
\pgfsetbuttcap%
\pgfsetroundjoin%
\definecolor{currentfill}{rgb}{0.919097,0.862812,0.832112}%
\pgfsetfillcolor{currentfill}%
\pgfsetlinewidth{0.602250pt}%
\definecolor{currentstroke}{rgb}{0.296471,0.296471,0.296471}%
\pgfsetstrokecolor{currentstroke}%
\pgfsetdash{}{0pt}%
\pgfpathmoveto{\pgfqpoint{1.937483in}{1.653652in}}%
\pgfpathlineto{\pgfqpoint{1.997808in}{1.653652in}}%
\pgfpathlineto{\pgfqpoint{1.997808in}{1.652420in}}%
\pgfpathlineto{\pgfqpoint{1.937483in}{1.652420in}}%
\pgfpathlineto{\pgfqpoint{1.937483in}{1.653652in}}%
\pgfpathclose%
\pgfusepath{stroke,fill}%
\end{pgfscope}%
\begin{pgfscope}%
\pgfpathrectangle{\pgfqpoint{0.703330in}{0.352393in}}{\pgfqpoint{2.173234in}{2.758940in}}%
\pgfusepath{clip}%
\pgfsetbuttcap%
\pgfsetroundjoin%
\definecolor{currentfill}{rgb}{0.910863,0.840546,0.801899}%
\pgfsetfillcolor{currentfill}%
\pgfsetlinewidth{0.602250pt}%
\definecolor{currentstroke}{rgb}{0.296471,0.296471,0.296471}%
\pgfsetstrokecolor{currentstroke}%
\pgfsetdash{}{0pt}%
\pgfpathmoveto{\pgfqpoint{1.997808in}{1.654268in}}%
\pgfpathlineto{\pgfqpoint{2.064855in}{1.654268in}}%
\pgfpathlineto{\pgfqpoint{2.064855in}{1.651805in}}%
\pgfpathlineto{\pgfqpoint{1.997808in}{1.651805in}}%
\pgfpathlineto{\pgfqpoint{1.997808in}{1.654268in}}%
\pgfpathclose%
\pgfusepath{stroke,fill}%
\end{pgfscope}%
\begin{pgfscope}%
\pgfpathrectangle{\pgfqpoint{0.703330in}{0.352393in}}{\pgfqpoint{2.173234in}{2.758940in}}%
\pgfusepath{clip}%
\pgfsetbuttcap%
\pgfsetroundjoin%
\definecolor{currentfill}{rgb}{0.901453,0.815098,0.767370}%
\pgfsetfillcolor{currentfill}%
\pgfsetlinewidth{0.602250pt}%
\definecolor{currentstroke}{rgb}{0.296471,0.296471,0.296471}%
\pgfsetstrokecolor{currentstroke}%
\pgfsetdash{}{0pt}%
\pgfpathmoveto{\pgfqpoint{2.064855in}{1.655500in}}%
\pgfpathlineto{\pgfqpoint{2.085178in}{1.655500in}}%
\pgfpathlineto{\pgfqpoint{2.085178in}{1.650573in}}%
\pgfpathlineto{\pgfqpoint{2.064855in}{1.650573in}}%
\pgfpathlineto{\pgfqpoint{2.064855in}{1.655500in}}%
\pgfpathclose%
\pgfusepath{stroke,fill}%
\end{pgfscope}%
\begin{pgfscope}%
\pgfpathrectangle{\pgfqpoint{0.703330in}{0.352393in}}{\pgfqpoint{2.173234in}{2.758940in}}%
\pgfusepath{clip}%
\pgfsetbuttcap%
\pgfsetroundjoin%
\definecolor{currentfill}{rgb}{0.889102,0.781698,0.722050}%
\pgfsetfillcolor{currentfill}%
\pgfsetlinewidth{0.602250pt}%
\definecolor{currentstroke}{rgb}{0.296471,0.296471,0.296471}%
\pgfsetstrokecolor{currentstroke}%
\pgfsetdash{}{0pt}%
\pgfpathmoveto{\pgfqpoint{2.085178in}{1.657963in}}%
\pgfpathlineto{\pgfqpoint{2.133728in}{1.657963in}}%
\pgfpathlineto{\pgfqpoint{2.133728in}{1.648110in}}%
\pgfpathlineto{\pgfqpoint{2.085178in}{1.648110in}}%
\pgfpathlineto{\pgfqpoint{2.085178in}{1.657963in}}%
\pgfpathclose%
\pgfusepath{stroke,fill}%
\end{pgfscope}%
\begin{pgfscope}%
\pgfpathrectangle{\pgfqpoint{0.703330in}{0.352393in}}{\pgfqpoint{2.173234in}{2.758940in}}%
\pgfusepath{clip}%
\pgfsetbuttcap%
\pgfsetroundjoin%
\definecolor{currentfill}{rgb}{0.873223,0.738755,0.663782}%
\pgfsetfillcolor{currentfill}%
\pgfsetlinewidth{0.602250pt}%
\definecolor{currentstroke}{rgb}{0.296471,0.296471,0.296471}%
\pgfsetstrokecolor{currentstroke}%
\pgfsetdash{}{0pt}%
\pgfpathmoveto{\pgfqpoint{2.133728in}{1.662890in}}%
\pgfpathlineto{\pgfqpoint{2.284059in}{1.662890in}}%
\pgfpathlineto{\pgfqpoint{2.284059in}{1.643183in}}%
\pgfpathlineto{\pgfqpoint{2.133728in}{1.643183in}}%
\pgfpathlineto{\pgfqpoint{2.133728in}{1.662890in}}%
\pgfpathclose%
\pgfusepath{stroke,fill}%
\end{pgfscope}%
\begin{pgfscope}%
\pgfpathrectangle{\pgfqpoint{0.703330in}{0.352393in}}{\pgfqpoint{2.173234in}{2.758940in}}%
\pgfusepath{clip}%
\pgfsetbuttcap%
\pgfsetroundjoin%
\definecolor{currentfill}{rgb}{0.853814,0.686269,0.592565}%
\pgfsetfillcolor{currentfill}%
\pgfsetlinewidth{0.602250pt}%
\definecolor{currentstroke}{rgb}{0.296471,0.296471,0.296471}%
\pgfsetstrokecolor{currentstroke}%
\pgfsetdash{}{0pt}%
\pgfpathmoveto{\pgfqpoint{2.284059in}{1.672743in}}%
\pgfpathlineto{\pgfqpoint{2.476165in}{1.672743in}}%
\pgfpathlineto{\pgfqpoint{2.476165in}{1.633330in}}%
\pgfpathlineto{\pgfqpoint{2.284059in}{1.633330in}}%
\pgfpathlineto{\pgfqpoint{2.284059in}{1.672743in}}%
\pgfpathclose%
\pgfusepath{stroke,fill}%
\end{pgfscope}%
\begin{pgfscope}%
\pgfpathrectangle{\pgfqpoint{0.703330in}{0.352393in}}{\pgfqpoint{2.173234in}{2.758940in}}%
\pgfusepath{clip}%
\pgfsetbuttcap%
\pgfsetroundjoin%
\definecolor{currentfill}{rgb}{0.829112,0.619469,0.501926}%
\pgfsetfillcolor{currentfill}%
\pgfsetlinewidth{0.602250pt}%
\definecolor{currentstroke}{rgb}{0.296471,0.296471,0.296471}%
\pgfsetstrokecolor{currentstroke}%
\pgfsetdash{}{0pt}%
\pgfpathmoveto{\pgfqpoint{2.476165in}{1.692450in}}%
\pgfpathlineto{\pgfqpoint{2.663364in}{1.692450in}}%
\pgfpathlineto{\pgfqpoint{2.663364in}{1.613623in}}%
\pgfpathlineto{\pgfqpoint{2.476165in}{1.613623in}}%
\pgfpathlineto{\pgfqpoint{2.476165in}{1.692450in}}%
\pgfpathclose%
\pgfusepath{stroke,fill}%
\end{pgfscope}%
\begin{pgfscope}%
\pgfpathrectangle{\pgfqpoint{0.703330in}{0.352393in}}{\pgfqpoint{2.173234in}{2.758940in}}%
\pgfusepath{clip}%
\pgfsetbuttcap%
\pgfsetroundjoin%
\definecolor{currentfill}{rgb}{0.798529,0.536765,0.389706}%
\pgfsetfillcolor{currentfill}%
\pgfsetlinewidth{0.602250pt}%
\definecolor{currentstroke}{rgb}{0.296471,0.296471,0.296471}%
\pgfsetstrokecolor{currentstroke}%
\pgfsetdash{}{0pt}%
\pgfpathmoveto{\pgfqpoint{2.663364in}{1.731863in}}%
\pgfpathlineto{\pgfqpoint{2.855584in}{1.731863in}}%
\pgfpathlineto{\pgfqpoint{2.855584in}{1.574209in}}%
\pgfpathlineto{\pgfqpoint{2.663364in}{1.574209in}}%
\pgfpathlineto{\pgfqpoint{2.663364in}{1.731863in}}%
\pgfpathclose%
\pgfusepath{stroke,fill}%
\end{pgfscope}%
\begin{pgfscope}%
\pgfpathrectangle{\pgfqpoint{0.703330in}{0.352393in}}{\pgfqpoint{2.173234in}{2.758940in}}%
\pgfusepath{clip}%
\pgfsetbuttcap%
\pgfsetroundjoin%
\definecolor{currentfill}{rgb}{0.829112,0.619469,0.501926}%
\pgfsetfillcolor{currentfill}%
\pgfsetlinewidth{0.602250pt}%
\definecolor{currentstroke}{rgb}{0.296471,0.296471,0.296471}%
\pgfsetstrokecolor{currentstroke}%
\pgfsetdash{}{0pt}%
\pgfpathmoveto{\pgfqpoint{2.855584in}{1.692450in}}%
\pgfpathlineto{\pgfqpoint{2.876564in}{1.692450in}}%
\pgfpathlineto{\pgfqpoint{2.876564in}{1.613623in}}%
\pgfpathlineto{\pgfqpoint{2.855584in}{1.613623in}}%
\pgfpathlineto{\pgfqpoint{2.855584in}{1.692450in}}%
\pgfpathclose%
\pgfusepath{stroke,fill}%
\end{pgfscope}%
\begin{pgfscope}%
\pgfpathrectangle{\pgfqpoint{0.703330in}{0.352393in}}{\pgfqpoint{2.173234in}{2.758940in}}%
\pgfusepath{clip}%
\pgfsetbuttcap%
\pgfsetroundjoin%
\definecolor{currentfill}{rgb}{0.853814,0.686269,0.592565}%
\pgfsetfillcolor{currentfill}%
\pgfsetlinewidth{0.602250pt}%
\definecolor{currentstroke}{rgb}{0.296471,0.296471,0.296471}%
\pgfsetstrokecolor{currentstroke}%
\pgfsetdash{}{0pt}%
\pgfpathmoveto{\pgfqpoint{2.876564in}{1.672743in}}%
\pgfpathlineto{\pgfqpoint{2.876564in}{1.672743in}}%
\pgfpathlineto{\pgfqpoint{2.876564in}{1.633330in}}%
\pgfpathlineto{\pgfqpoint{2.876564in}{1.633330in}}%
\pgfpathlineto{\pgfqpoint{2.876564in}{1.672743in}}%
\pgfpathclose%
\pgfusepath{stroke,fill}%
\end{pgfscope}%
\begin{pgfscope}%
\pgfpathrectangle{\pgfqpoint{0.703330in}{0.352393in}}{\pgfqpoint{2.173234in}{2.758940in}}%
\pgfusepath{clip}%
\pgfsetbuttcap%
\pgfsetroundjoin%
\definecolor{currentfill}{rgb}{0.873223,0.738755,0.663782}%
\pgfsetfillcolor{currentfill}%
\pgfsetlinewidth{0.602250pt}%
\definecolor{currentstroke}{rgb}{0.296471,0.296471,0.296471}%
\pgfsetstrokecolor{currentstroke}%
\pgfsetdash{}{0pt}%
\pgfpathmoveto{\pgfqpoint{2.876564in}{1.662890in}}%
\pgfpathlineto{\pgfqpoint{2.876564in}{1.662890in}}%
\pgfpathlineto{\pgfqpoint{2.876564in}{1.643183in}}%
\pgfpathlineto{\pgfqpoint{2.876564in}{1.643183in}}%
\pgfpathlineto{\pgfqpoint{2.876564in}{1.662890in}}%
\pgfpathclose%
\pgfusepath{stroke,fill}%
\end{pgfscope}%
\begin{pgfscope}%
\pgfpathrectangle{\pgfqpoint{0.703330in}{0.352393in}}{\pgfqpoint{2.173234in}{2.758940in}}%
\pgfusepath{clip}%
\pgfsetbuttcap%
\pgfsetroundjoin%
\definecolor{currentfill}{rgb}{0.889102,0.781698,0.722050}%
\pgfsetfillcolor{currentfill}%
\pgfsetlinewidth{0.602250pt}%
\definecolor{currentstroke}{rgb}{0.296471,0.296471,0.296471}%
\pgfsetstrokecolor{currentstroke}%
\pgfsetdash{}{0pt}%
\pgfpathmoveto{\pgfqpoint{2.876564in}{1.657963in}}%
\pgfpathlineto{\pgfqpoint{2.876564in}{1.657963in}}%
\pgfpathlineto{\pgfqpoint{2.876564in}{1.648110in}}%
\pgfpathlineto{\pgfqpoint{2.876564in}{1.648110in}}%
\pgfpathlineto{\pgfqpoint{2.876564in}{1.657963in}}%
\pgfpathclose%
\pgfusepath{stroke,fill}%
\end{pgfscope}%
\begin{pgfscope}%
\pgfpathrectangle{\pgfqpoint{0.703330in}{0.352393in}}{\pgfqpoint{2.173234in}{2.758940in}}%
\pgfusepath{clip}%
\pgfsetbuttcap%
\pgfsetroundjoin%
\definecolor{currentfill}{rgb}{0.901453,0.815098,0.767370}%
\pgfsetfillcolor{currentfill}%
\pgfsetlinewidth{0.602250pt}%
\definecolor{currentstroke}{rgb}{0.296471,0.296471,0.296471}%
\pgfsetstrokecolor{currentstroke}%
\pgfsetdash{}{0pt}%
\pgfpathmoveto{\pgfqpoint{2.876564in}{1.655500in}}%
\pgfpathlineto{\pgfqpoint{2.876564in}{1.655500in}}%
\pgfpathlineto{\pgfqpoint{2.876564in}{1.650573in}}%
\pgfpathlineto{\pgfqpoint{2.876564in}{1.650573in}}%
\pgfpathlineto{\pgfqpoint{2.876564in}{1.655500in}}%
\pgfpathclose%
\pgfusepath{stroke,fill}%
\end{pgfscope}%
\begin{pgfscope}%
\pgfpathrectangle{\pgfqpoint{0.703330in}{0.352393in}}{\pgfqpoint{2.173234in}{2.758940in}}%
\pgfusepath{clip}%
\pgfsetbuttcap%
\pgfsetroundjoin%
\definecolor{currentfill}{rgb}{0.910863,0.840546,0.801899}%
\pgfsetfillcolor{currentfill}%
\pgfsetlinewidth{0.602250pt}%
\definecolor{currentstroke}{rgb}{0.296471,0.296471,0.296471}%
\pgfsetstrokecolor{currentstroke}%
\pgfsetdash{}{0pt}%
\pgfpathmoveto{\pgfqpoint{2.876564in}{1.654268in}}%
\pgfpathlineto{\pgfqpoint{2.876564in}{1.654268in}}%
\pgfpathlineto{\pgfqpoint{2.876564in}{1.651805in}}%
\pgfpathlineto{\pgfqpoint{2.876564in}{1.651805in}}%
\pgfpathlineto{\pgfqpoint{2.876564in}{1.654268in}}%
\pgfpathclose%
\pgfusepath{stroke,fill}%
\end{pgfscope}%
\begin{pgfscope}%
\pgfpathrectangle{\pgfqpoint{0.703330in}{0.352393in}}{\pgfqpoint{2.173234in}{2.758940in}}%
\pgfusepath{clip}%
\pgfsetbuttcap%
\pgfsetroundjoin%
\definecolor{currentfill}{rgb}{0.919097,0.862812,0.832112}%
\pgfsetfillcolor{currentfill}%
\pgfsetlinewidth{0.602250pt}%
\definecolor{currentstroke}{rgb}{0.296471,0.296471,0.296471}%
\pgfsetstrokecolor{currentstroke}%
\pgfsetdash{}{0pt}%
\pgfpathmoveto{\pgfqpoint{2.876564in}{1.653652in}}%
\pgfpathlineto{\pgfqpoint{2.876564in}{1.653652in}}%
\pgfpathlineto{\pgfqpoint{2.876564in}{1.652420in}}%
\pgfpathlineto{\pgfqpoint{2.876564in}{1.652420in}}%
\pgfpathlineto{\pgfqpoint{2.876564in}{1.653652in}}%
\pgfpathclose%
\pgfusepath{stroke,fill}%
\end{pgfscope}%
\begin{pgfscope}%
\pgfpathrectangle{\pgfqpoint{0.703330in}{0.352393in}}{\pgfqpoint{2.173234in}{2.758940in}}%
\pgfusepath{clip}%
\pgfsetbuttcap%
\pgfsetroundjoin%
\pgfsetlinewidth{0.803000pt}%
\definecolor{currentstroke}{rgb}{0.450000,0.450000,0.450000}%
\pgfsetstrokecolor{currentstroke}%
\pgfsetdash{}{0pt}%
\pgfpathmoveto{\pgfqpoint{0.000000in}{-0.034722in}}%
\pgfpathcurveto{\pgfqpoint{0.009208in}{-0.034722in}}{\pgfqpoint{0.018041in}{-0.031064in}}{\pgfqpoint{0.024552in}{-0.024552in}}%
\pgfpathcurveto{\pgfqpoint{0.031064in}{-0.018041in}}{\pgfqpoint{0.034722in}{-0.009208in}}{\pgfqpoint{0.034722in}{0.000000in}}%
\pgfpathcurveto{\pgfqpoint{0.034722in}{0.009208in}}{\pgfqpoint{0.031064in}{0.018041in}}{\pgfqpoint{0.024552in}{0.024552in}}%
\pgfpathcurveto{\pgfqpoint{0.018041in}{0.031064in}}{\pgfqpoint{0.009208in}{0.034722in}}{\pgfqpoint{0.000000in}{0.034722in}}%
\pgfpathcurveto{\pgfqpoint{-0.009208in}{0.034722in}}{\pgfqpoint{-0.018041in}{0.031064in}}{\pgfqpoint{-0.024552in}{0.024552in}}%
\pgfpathcurveto{\pgfqpoint{-0.031064in}{0.018041in}}{\pgfqpoint{-0.034722in}{0.009208in}}{\pgfqpoint{-0.034722in}{0.000000in}}%
\pgfpathcurveto{\pgfqpoint{-0.034722in}{-0.009208in}}{\pgfqpoint{-0.031064in}{-0.018041in}}{\pgfqpoint{-0.024552in}{-0.024552in}}%
\pgfpathcurveto{\pgfqpoint{-0.018041in}{-0.031064in}}{\pgfqpoint{-0.009208in}{-0.034722in}}{\pgfqpoint{0.000000in}{-0.034722in}}%
\pgfusepath{stroke}%
\end{pgfscope}%
\begin{pgfscope}%
\pgfpathrectangle{\pgfqpoint{0.703330in}{0.352393in}}{\pgfqpoint{2.173234in}{2.758940in}}%
\pgfusepath{clip}%
\pgfsetbuttcap%
\pgfsetroundjoin%
\definecolor{currentfill}{rgb}{0.848437,0.867532,0.899724}%
\pgfsetfillcolor{currentfill}%
\pgfsetlinewidth{0.602250pt}%
\definecolor{currentstroke}{rgb}{0.296471,0.296471,0.296471}%
\pgfsetstrokecolor{currentstroke}%
\pgfsetdash{}{0pt}%
\pgfpathmoveto{\pgfqpoint{-0.228056in}{1.416864in}}%
\pgfpathlineto{\pgfqpoint{-0.228056in}{1.416864in}}%
\pgfpathlineto{\pgfqpoint{-0.228056in}{1.416248in}}%
\pgfpathlineto{\pgfqpoint{-0.228056in}{1.416248in}}%
\pgfpathlineto{\pgfqpoint{-0.228056in}{1.416864in}}%
\pgfpathclose%
\pgfusepath{stroke,fill}%
\end{pgfscope}%
\begin{pgfscope}%
\pgfpathrectangle{\pgfqpoint{0.703330in}{0.352393in}}{\pgfqpoint{2.173234in}{2.758940in}}%
\pgfusepath{clip}%
\pgfsetbuttcap%
\pgfsetroundjoin%
\definecolor{currentfill}{rgb}{0.825117,0.848522,0.887698}%
\pgfsetfillcolor{currentfill}%
\pgfsetlinewidth{0.602250pt}%
\definecolor{currentstroke}{rgb}{0.296471,0.296471,0.296471}%
\pgfsetstrokecolor{currentstroke}%
\pgfsetdash{}{0pt}%
\pgfpathmoveto{\pgfqpoint{-0.228056in}{1.417171in}}%
\pgfpathlineto{\pgfqpoint{-0.228056in}{1.417171in}}%
\pgfpathlineto{\pgfqpoint{-0.228056in}{1.415940in}}%
\pgfpathlineto{\pgfqpoint{-0.228056in}{1.415940in}}%
\pgfpathlineto{\pgfqpoint{-0.228056in}{1.417171in}}%
\pgfpathclose%
\pgfusepath{stroke,fill}%
\end{pgfscope}%
\begin{pgfscope}%
\pgfpathrectangle{\pgfqpoint{0.703330in}{0.352393in}}{\pgfqpoint{2.173234in}{2.758940in}}%
\pgfusepath{clip}%
\pgfsetbuttcap%
\pgfsetroundjoin%
\definecolor{currentfill}{rgb}{0.792469,0.821908,0.870863}%
\pgfsetfillcolor{currentfill}%
\pgfsetlinewidth{0.602250pt}%
\definecolor{currentstroke}{rgb}{0.296471,0.296471,0.296471}%
\pgfsetstrokecolor{currentstroke}%
\pgfsetdash{}{0pt}%
\pgfpathmoveto{\pgfqpoint{-0.228056in}{1.417787in}}%
\pgfpathlineto{\pgfqpoint{-0.228056in}{1.417787in}}%
\pgfpathlineto{\pgfqpoint{-0.228056in}{1.415324in}}%
\pgfpathlineto{\pgfqpoint{-0.228056in}{1.415324in}}%
\pgfpathlineto{\pgfqpoint{-0.228056in}{1.417787in}}%
\pgfpathclose%
\pgfusepath{stroke,fill}%
\end{pgfscope}%
\begin{pgfscope}%
\pgfpathrectangle{\pgfqpoint{0.703330in}{0.352393in}}{\pgfqpoint{2.173234in}{2.758940in}}%
\pgfusepath{clip}%
\pgfsetbuttcap%
\pgfsetroundjoin%
\definecolor{currentfill}{rgb}{0.755157,0.791493,0.851622}%
\pgfsetfillcolor{currentfill}%
\pgfsetlinewidth{0.602250pt}%
\definecolor{currentstroke}{rgb}{0.296471,0.296471,0.296471}%
\pgfsetstrokecolor{currentstroke}%
\pgfsetdash{}{0pt}%
\pgfpathmoveto{\pgfqpoint{-0.228056in}{1.419019in}}%
\pgfpathlineto{\pgfqpoint{-0.228056in}{1.419019in}}%
\pgfpathlineto{\pgfqpoint{-0.228056in}{1.414092in}}%
\pgfpathlineto{\pgfqpoint{-0.228056in}{1.414092in}}%
\pgfpathlineto{\pgfqpoint{-0.228056in}{1.419019in}}%
\pgfpathclose%
\pgfusepath{stroke,fill}%
\end{pgfscope}%
\begin{pgfscope}%
\pgfpathrectangle{\pgfqpoint{0.703330in}{0.352393in}}{\pgfqpoint{2.173234in}{2.758940in}}%
\pgfusepath{clip}%
\pgfsetbuttcap%
\pgfsetroundjoin%
\definecolor{currentfill}{rgb}{0.706185,0.751573,0.826368}%
\pgfsetfillcolor{currentfill}%
\pgfsetlinewidth{0.602250pt}%
\definecolor{currentstroke}{rgb}{0.296471,0.296471,0.296471}%
\pgfsetstrokecolor{currentstroke}%
\pgfsetdash{}{0pt}%
\pgfpathmoveto{\pgfqpoint{-0.228056in}{1.421482in}}%
\pgfpathlineto{\pgfqpoint{-0.228056in}{1.421482in}}%
\pgfpathlineto{\pgfqpoint{-0.228056in}{1.411629in}}%
\pgfpathlineto{\pgfqpoint{-0.228056in}{1.411629in}}%
\pgfpathlineto{\pgfqpoint{-0.228056in}{1.421482in}}%
\pgfpathclose%
\pgfusepath{stroke,fill}%
\end{pgfscope}%
\begin{pgfscope}%
\pgfpathrectangle{\pgfqpoint{0.703330in}{0.352393in}}{\pgfqpoint{2.173234in}{2.758940in}}%
\pgfusepath{clip}%
\pgfsetbuttcap%
\pgfsetroundjoin%
\definecolor{currentfill}{rgb}{0.643221,0.700246,0.793900}%
\pgfsetfillcolor{currentfill}%
\pgfsetlinewidth{0.602250pt}%
\definecolor{currentstroke}{rgb}{0.296471,0.296471,0.296471}%
\pgfsetstrokecolor{currentstroke}%
\pgfsetdash{}{0pt}%
\pgfpathmoveto{\pgfqpoint{-0.228056in}{1.426409in}}%
\pgfpathlineto{\pgfqpoint{-0.228056in}{1.426409in}}%
\pgfpathlineto{\pgfqpoint{-0.228056in}{1.406702in}}%
\pgfpathlineto{\pgfqpoint{-0.228056in}{1.406702in}}%
\pgfpathlineto{\pgfqpoint{-0.228056in}{1.426409in}}%
\pgfpathclose%
\pgfusepath{stroke,fill}%
\end{pgfscope}%
\begin{pgfscope}%
\pgfpathrectangle{\pgfqpoint{0.703330in}{0.352393in}}{\pgfqpoint{2.173234in}{2.758940in}}%
\pgfusepath{clip}%
\pgfsetbuttcap%
\pgfsetroundjoin%
\definecolor{currentfill}{rgb}{0.566266,0.637515,0.754216}%
\pgfsetfillcolor{currentfill}%
\pgfsetlinewidth{0.602250pt}%
\definecolor{currentstroke}{rgb}{0.296471,0.296471,0.296471}%
\pgfsetstrokecolor{currentstroke}%
\pgfsetdash{}{0pt}%
\pgfpathmoveto{\pgfqpoint{-0.228056in}{1.436262in}}%
\pgfpathlineto{\pgfqpoint{-0.228056in}{1.436262in}}%
\pgfpathlineto{\pgfqpoint{-0.228056in}{1.396849in}}%
\pgfpathlineto{\pgfqpoint{-0.228056in}{1.396849in}}%
\pgfpathlineto{\pgfqpoint{-0.228056in}{1.436262in}}%
\pgfpathclose%
\pgfusepath{stroke,fill}%
\end{pgfscope}%
\begin{pgfscope}%
\pgfpathrectangle{\pgfqpoint{0.703330in}{0.352393in}}{\pgfqpoint{2.173234in}{2.758940in}}%
\pgfusepath{clip}%
\pgfsetbuttcap%
\pgfsetroundjoin%
\definecolor{currentfill}{rgb}{0.468322,0.557674,0.703709}%
\pgfsetfillcolor{currentfill}%
\pgfsetlinewidth{0.602250pt}%
\definecolor{currentstroke}{rgb}{0.296471,0.296471,0.296471}%
\pgfsetstrokecolor{currentstroke}%
\pgfsetdash{}{0pt}%
\pgfpathmoveto{\pgfqpoint{-0.228056in}{1.455969in}}%
\pgfpathlineto{\pgfqpoint{-0.228056in}{1.455969in}}%
\pgfpathlineto{\pgfqpoint{-0.228056in}{1.377142in}}%
\pgfpathlineto{\pgfqpoint{-0.228056in}{1.377142in}}%
\pgfpathlineto{\pgfqpoint{-0.228056in}{1.455969in}}%
\pgfpathclose%
\pgfusepath{stroke,fill}%
\end{pgfscope}%
\begin{pgfscope}%
\pgfpathrectangle{\pgfqpoint{0.703330in}{0.352393in}}{\pgfqpoint{2.173234in}{2.758940in}}%
\pgfusepath{clip}%
\pgfsetbuttcap%
\pgfsetroundjoin%
\definecolor{currentfill}{rgb}{0.347059,0.458824,0.641176}%
\pgfsetfillcolor{currentfill}%
\pgfsetlinewidth{0.602250pt}%
\definecolor{currentstroke}{rgb}{0.296471,0.296471,0.296471}%
\pgfsetstrokecolor{currentstroke}%
\pgfsetdash{}{0pt}%
\pgfpathmoveto{\pgfqpoint{-0.228056in}{1.495383in}}%
\pgfpathlineto{\pgfqpoint{-0.228056in}{1.495383in}}%
\pgfpathlineto{\pgfqpoint{-0.228056in}{1.337729in}}%
\pgfpathlineto{\pgfqpoint{-0.228056in}{1.337729in}}%
\pgfpathlineto{\pgfqpoint{-0.228056in}{1.495383in}}%
\pgfpathclose%
\pgfusepath{stroke,fill}%
\end{pgfscope}%
\begin{pgfscope}%
\pgfpathrectangle{\pgfqpoint{0.703330in}{0.352393in}}{\pgfqpoint{2.173234in}{2.758940in}}%
\pgfusepath{clip}%
\pgfsetbuttcap%
\pgfsetroundjoin%
\definecolor{currentfill}{rgb}{0.468322,0.557674,0.703709}%
\pgfsetfillcolor{currentfill}%
\pgfsetlinewidth{0.602250pt}%
\definecolor{currentstroke}{rgb}{0.296471,0.296471,0.296471}%
\pgfsetstrokecolor{currentstroke}%
\pgfsetdash{}{0pt}%
\pgfpathmoveto{\pgfqpoint{-0.228056in}{1.455969in}}%
\pgfpathlineto{\pgfqpoint{-0.228056in}{1.455969in}}%
\pgfpathlineto{\pgfqpoint{-0.228056in}{1.377142in}}%
\pgfpathlineto{\pgfqpoint{-0.228056in}{1.377142in}}%
\pgfpathlineto{\pgfqpoint{-0.228056in}{1.455969in}}%
\pgfpathclose%
\pgfusepath{stroke,fill}%
\end{pgfscope}%
\begin{pgfscope}%
\pgfpathrectangle{\pgfqpoint{0.703330in}{0.352393in}}{\pgfqpoint{2.173234in}{2.758940in}}%
\pgfusepath{clip}%
\pgfsetbuttcap%
\pgfsetroundjoin%
\definecolor{currentfill}{rgb}{0.566266,0.637515,0.754216}%
\pgfsetfillcolor{currentfill}%
\pgfsetlinewidth{0.602250pt}%
\definecolor{currentstroke}{rgb}{0.296471,0.296471,0.296471}%
\pgfsetstrokecolor{currentstroke}%
\pgfsetdash{}{0pt}%
\pgfpathmoveto{\pgfqpoint{-0.228056in}{1.436262in}}%
\pgfpathlineto{\pgfqpoint{-0.228056in}{1.436262in}}%
\pgfpathlineto{\pgfqpoint{-0.228056in}{1.396849in}}%
\pgfpathlineto{\pgfqpoint{-0.228056in}{1.396849in}}%
\pgfpathlineto{\pgfqpoint{-0.228056in}{1.436262in}}%
\pgfpathclose%
\pgfusepath{stroke,fill}%
\end{pgfscope}%
\begin{pgfscope}%
\pgfpathrectangle{\pgfqpoint{0.703330in}{0.352393in}}{\pgfqpoint{2.173234in}{2.758940in}}%
\pgfusepath{clip}%
\pgfsetbuttcap%
\pgfsetroundjoin%
\definecolor{currentfill}{rgb}{0.643221,0.700246,0.793900}%
\pgfsetfillcolor{currentfill}%
\pgfsetlinewidth{0.602250pt}%
\definecolor{currentstroke}{rgb}{0.296471,0.296471,0.296471}%
\pgfsetstrokecolor{currentstroke}%
\pgfsetdash{}{0pt}%
\pgfpathmoveto{\pgfqpoint{-0.228056in}{1.426409in}}%
\pgfpathlineto{\pgfqpoint{-0.228056in}{1.426409in}}%
\pgfpathlineto{\pgfqpoint{-0.228056in}{1.406702in}}%
\pgfpathlineto{\pgfqpoint{-0.228056in}{1.406702in}}%
\pgfpathlineto{\pgfqpoint{-0.228056in}{1.426409in}}%
\pgfpathclose%
\pgfusepath{stroke,fill}%
\end{pgfscope}%
\begin{pgfscope}%
\pgfpathrectangle{\pgfqpoint{0.703330in}{0.352393in}}{\pgfqpoint{2.173234in}{2.758940in}}%
\pgfusepath{clip}%
\pgfsetbuttcap%
\pgfsetroundjoin%
\definecolor{currentfill}{rgb}{0.706185,0.751573,0.826368}%
\pgfsetfillcolor{currentfill}%
\pgfsetlinewidth{0.602250pt}%
\definecolor{currentstroke}{rgb}{0.296471,0.296471,0.296471}%
\pgfsetstrokecolor{currentstroke}%
\pgfsetdash{}{0pt}%
\pgfpathmoveto{\pgfqpoint{-0.228056in}{1.421482in}}%
\pgfpathlineto{\pgfqpoint{-0.228056in}{1.421482in}}%
\pgfpathlineto{\pgfqpoint{-0.228056in}{1.411629in}}%
\pgfpathlineto{\pgfqpoint{-0.228056in}{1.411629in}}%
\pgfpathlineto{\pgfqpoint{-0.228056in}{1.421482in}}%
\pgfpathclose%
\pgfusepath{stroke,fill}%
\end{pgfscope}%
\begin{pgfscope}%
\pgfpathrectangle{\pgfqpoint{0.703330in}{0.352393in}}{\pgfqpoint{2.173234in}{2.758940in}}%
\pgfusepath{clip}%
\pgfsetbuttcap%
\pgfsetroundjoin%
\definecolor{currentfill}{rgb}{0.755157,0.791493,0.851622}%
\pgfsetfillcolor{currentfill}%
\pgfsetlinewidth{0.602250pt}%
\definecolor{currentstroke}{rgb}{0.296471,0.296471,0.296471}%
\pgfsetstrokecolor{currentstroke}%
\pgfsetdash{}{0pt}%
\pgfpathmoveto{\pgfqpoint{-0.228056in}{1.419019in}}%
\pgfpathlineto{\pgfqpoint{-0.228056in}{1.419019in}}%
\pgfpathlineto{\pgfqpoint{-0.228056in}{1.414092in}}%
\pgfpathlineto{\pgfqpoint{-0.228056in}{1.414092in}}%
\pgfpathlineto{\pgfqpoint{-0.228056in}{1.419019in}}%
\pgfpathclose%
\pgfusepath{stroke,fill}%
\end{pgfscope}%
\begin{pgfscope}%
\pgfpathrectangle{\pgfqpoint{0.703330in}{0.352393in}}{\pgfqpoint{2.173234in}{2.758940in}}%
\pgfusepath{clip}%
\pgfsetbuttcap%
\pgfsetroundjoin%
\definecolor{currentfill}{rgb}{0.792469,0.821908,0.870863}%
\pgfsetfillcolor{currentfill}%
\pgfsetlinewidth{0.602250pt}%
\definecolor{currentstroke}{rgb}{0.296471,0.296471,0.296471}%
\pgfsetstrokecolor{currentstroke}%
\pgfsetdash{}{0pt}%
\pgfpathmoveto{\pgfqpoint{-0.228056in}{1.417787in}}%
\pgfpathlineto{\pgfqpoint{-0.228056in}{1.417787in}}%
\pgfpathlineto{\pgfqpoint{-0.228056in}{1.415324in}}%
\pgfpathlineto{\pgfqpoint{-0.228056in}{1.415324in}}%
\pgfpathlineto{\pgfqpoint{-0.228056in}{1.417787in}}%
\pgfpathclose%
\pgfusepath{stroke,fill}%
\end{pgfscope}%
\begin{pgfscope}%
\pgfpathrectangle{\pgfqpoint{0.703330in}{0.352393in}}{\pgfqpoint{2.173234in}{2.758940in}}%
\pgfusepath{clip}%
\pgfsetbuttcap%
\pgfsetroundjoin%
\definecolor{currentfill}{rgb}{0.825117,0.848522,0.887698}%
\pgfsetfillcolor{currentfill}%
\pgfsetlinewidth{0.602250pt}%
\definecolor{currentstroke}{rgb}{0.296471,0.296471,0.296471}%
\pgfsetstrokecolor{currentstroke}%
\pgfsetdash{}{0pt}%
\pgfpathmoveto{\pgfqpoint{-0.228056in}{1.417171in}}%
\pgfpathlineto{\pgfqpoint{-0.228056in}{1.417171in}}%
\pgfpathlineto{\pgfqpoint{-0.228056in}{1.415940in}}%
\pgfpathlineto{\pgfqpoint{-0.228056in}{1.415940in}}%
\pgfpathlineto{\pgfqpoint{-0.228056in}{1.417171in}}%
\pgfpathclose%
\pgfusepath{stroke,fill}%
\end{pgfscope}%
\begin{pgfscope}%
\pgfpathrectangle{\pgfqpoint{0.703330in}{0.352393in}}{\pgfqpoint{2.173234in}{2.758940in}}%
\pgfusepath{clip}%
\pgfsetbuttcap%
\pgfsetroundjoin%
\definecolor{currentfill}{rgb}{0.848437,0.867532,0.899724}%
\pgfsetfillcolor{currentfill}%
\pgfsetlinewidth{0.602250pt}%
\definecolor{currentstroke}{rgb}{0.296471,0.296471,0.296471}%
\pgfsetstrokecolor{currentstroke}%
\pgfsetdash{}{0pt}%
\pgfpathmoveto{\pgfqpoint{-0.228056in}{1.416864in}}%
\pgfpathlineto{\pgfqpoint{-0.228056in}{1.416864in}}%
\pgfpathlineto{\pgfqpoint{-0.228056in}{1.416248in}}%
\pgfpathlineto{\pgfqpoint{-0.228056in}{1.416248in}}%
\pgfpathlineto{\pgfqpoint{-0.228056in}{1.416864in}}%
\pgfpathclose%
\pgfusepath{stroke,fill}%
\end{pgfscope}%
\begin{pgfscope}%
\pgfpathrectangle{\pgfqpoint{0.703330in}{0.352393in}}{\pgfqpoint{2.173234in}{2.758940in}}%
\pgfusepath{clip}%
\pgfsetbuttcap%
\pgfsetroundjoin%
\pgfsetlinewidth{0.803000pt}%
\definecolor{currentstroke}{rgb}{0.450000,0.450000,0.450000}%
\pgfsetstrokecolor{currentstroke}%
\pgfsetdash{}{0pt}%
\pgfpathmoveto{\pgfqpoint{0.000000in}{-0.034722in}}%
\pgfpathcurveto{\pgfqpoint{0.009208in}{-0.034722in}}{\pgfqpoint{0.018041in}{-0.031064in}}{\pgfqpoint{0.024552in}{-0.024552in}}%
\pgfpathcurveto{\pgfqpoint{0.031064in}{-0.018041in}}{\pgfqpoint{0.034722in}{-0.009208in}}{\pgfqpoint{0.034722in}{0.000000in}}%
\pgfpathcurveto{\pgfqpoint{0.034722in}{0.009208in}}{\pgfqpoint{0.031064in}{0.018041in}}{\pgfqpoint{0.024552in}{0.024552in}}%
\pgfpathcurveto{\pgfqpoint{0.018041in}{0.031064in}}{\pgfqpoint{0.009208in}{0.034722in}}{\pgfqpoint{0.000000in}{0.034722in}}%
\pgfpathcurveto{\pgfqpoint{-0.009208in}{0.034722in}}{\pgfqpoint{-0.018041in}{0.031064in}}{\pgfqpoint{-0.024552in}{0.024552in}}%
\pgfpathcurveto{\pgfqpoint{-0.031064in}{0.018041in}}{\pgfqpoint{-0.034722in}{0.009208in}}{\pgfqpoint{-0.034722in}{0.000000in}}%
\pgfpathcurveto{\pgfqpoint{-0.034722in}{-0.009208in}}{\pgfqpoint{-0.031064in}{-0.018041in}}{\pgfqpoint{-0.024552in}{-0.024552in}}%
\pgfpathcurveto{\pgfqpoint{-0.018041in}{-0.031064in}}{\pgfqpoint{-0.009208in}{-0.034722in}}{\pgfqpoint{0.000000in}{-0.034722in}}%
\pgfusepath{stroke}%
\end{pgfscope}%
\begin{pgfscope}%
\pgfpathrectangle{\pgfqpoint{0.703330in}{0.352393in}}{\pgfqpoint{2.173234in}{2.758940in}}%
\pgfusepath{clip}%
\pgfsetbuttcap%
\pgfsetroundjoin%
\definecolor{currentfill}{rgb}{0.919097,0.862812,0.832112}%
\pgfsetfillcolor{currentfill}%
\pgfsetlinewidth{0.602250pt}%
\definecolor{currentstroke}{rgb}{0.296471,0.296471,0.296471}%
\pgfsetstrokecolor{currentstroke}%
\pgfsetdash{}{0pt}%
\pgfpathmoveto{\pgfqpoint{1.806679in}{1.259518in}}%
\pgfpathlineto{\pgfqpoint{1.988188in}{1.259518in}}%
\pgfpathlineto{\pgfqpoint{1.988188in}{1.258286in}}%
\pgfpathlineto{\pgfqpoint{1.806679in}{1.258286in}}%
\pgfpathlineto{\pgfqpoint{1.806679in}{1.259518in}}%
\pgfpathclose%
\pgfusepath{stroke,fill}%
\end{pgfscope}%
\begin{pgfscope}%
\pgfpathrectangle{\pgfqpoint{0.703330in}{0.352393in}}{\pgfqpoint{2.173234in}{2.758940in}}%
\pgfusepath{clip}%
\pgfsetbuttcap%
\pgfsetroundjoin%
\definecolor{currentfill}{rgb}{0.910863,0.840546,0.801899}%
\pgfsetfillcolor{currentfill}%
\pgfsetlinewidth{0.602250pt}%
\definecolor{currentstroke}{rgb}{0.296471,0.296471,0.296471}%
\pgfsetstrokecolor{currentstroke}%
\pgfsetdash{}{0pt}%
\pgfpathmoveto{\pgfqpoint{1.988188in}{1.260134in}}%
\pgfpathlineto{\pgfqpoint{2.035251in}{1.260134in}}%
\pgfpathlineto{\pgfqpoint{2.035251in}{1.257670in}}%
\pgfpathlineto{\pgfqpoint{1.988188in}{1.257670in}}%
\pgfpathlineto{\pgfqpoint{1.988188in}{1.260134in}}%
\pgfpathclose%
\pgfusepath{stroke,fill}%
\end{pgfscope}%
\begin{pgfscope}%
\pgfpathrectangle{\pgfqpoint{0.703330in}{0.352393in}}{\pgfqpoint{2.173234in}{2.758940in}}%
\pgfusepath{clip}%
\pgfsetbuttcap%
\pgfsetroundjoin%
\definecolor{currentfill}{rgb}{0.901453,0.815098,0.767370}%
\pgfsetfillcolor{currentfill}%
\pgfsetlinewidth{0.602250pt}%
\definecolor{currentstroke}{rgb}{0.296471,0.296471,0.296471}%
\pgfsetstrokecolor{currentstroke}%
\pgfsetdash{}{0pt}%
\pgfpathmoveto{\pgfqpoint{2.035251in}{1.261365in}}%
\pgfpathlineto{\pgfqpoint{2.056112in}{1.261365in}}%
\pgfpathlineto{\pgfqpoint{2.056112in}{1.256439in}}%
\pgfpathlineto{\pgfqpoint{2.035251in}{1.256439in}}%
\pgfpathlineto{\pgfqpoint{2.035251in}{1.261365in}}%
\pgfpathclose%
\pgfusepath{stroke,fill}%
\end{pgfscope}%
\begin{pgfscope}%
\pgfpathrectangle{\pgfqpoint{0.703330in}{0.352393in}}{\pgfqpoint{2.173234in}{2.758940in}}%
\pgfusepath{clip}%
\pgfsetbuttcap%
\pgfsetroundjoin%
\definecolor{currentfill}{rgb}{0.889102,0.781698,0.722050}%
\pgfsetfillcolor{currentfill}%
\pgfsetlinewidth{0.602250pt}%
\definecolor{currentstroke}{rgb}{0.296471,0.296471,0.296471}%
\pgfsetstrokecolor{currentstroke}%
\pgfsetdash{}{0pt}%
\pgfpathmoveto{\pgfqpoint{2.056112in}{1.263829in}}%
\pgfpathlineto{\pgfqpoint{2.077406in}{1.263829in}}%
\pgfpathlineto{\pgfqpoint{2.077406in}{1.253975in}}%
\pgfpathlineto{\pgfqpoint{2.056112in}{1.253975in}}%
\pgfpathlineto{\pgfqpoint{2.056112in}{1.263829in}}%
\pgfpathclose%
\pgfusepath{stroke,fill}%
\end{pgfscope}%
\begin{pgfscope}%
\pgfpathrectangle{\pgfqpoint{0.703330in}{0.352393in}}{\pgfqpoint{2.173234in}{2.758940in}}%
\pgfusepath{clip}%
\pgfsetbuttcap%
\pgfsetroundjoin%
\definecolor{currentfill}{rgb}{0.873223,0.738755,0.663782}%
\pgfsetfillcolor{currentfill}%
\pgfsetlinewidth{0.602250pt}%
\definecolor{currentstroke}{rgb}{0.296471,0.296471,0.296471}%
\pgfsetstrokecolor{currentstroke}%
\pgfsetdash{}{0pt}%
\pgfpathmoveto{\pgfqpoint{2.077406in}{1.268755in}}%
\pgfpathlineto{\pgfqpoint{2.223102in}{1.268755in}}%
\pgfpathlineto{\pgfqpoint{2.223102in}{1.249049in}}%
\pgfpathlineto{\pgfqpoint{2.077406in}{1.249049in}}%
\pgfpathlineto{\pgfqpoint{2.077406in}{1.268755in}}%
\pgfpathclose%
\pgfusepath{stroke,fill}%
\end{pgfscope}%
\begin{pgfscope}%
\pgfpathrectangle{\pgfqpoint{0.703330in}{0.352393in}}{\pgfqpoint{2.173234in}{2.758940in}}%
\pgfusepath{clip}%
\pgfsetbuttcap%
\pgfsetroundjoin%
\definecolor{currentfill}{rgb}{0.853814,0.686269,0.592565}%
\pgfsetfillcolor{currentfill}%
\pgfsetlinewidth{0.602250pt}%
\definecolor{currentstroke}{rgb}{0.296471,0.296471,0.296471}%
\pgfsetstrokecolor{currentstroke}%
\pgfsetdash{}{0pt}%
\pgfpathmoveto{\pgfqpoint{2.223102in}{1.278609in}}%
\pgfpathlineto{\pgfqpoint{2.317574in}{1.278609in}}%
\pgfpathlineto{\pgfqpoint{2.317574in}{1.239195in}}%
\pgfpathlineto{\pgfqpoint{2.223102in}{1.239195in}}%
\pgfpathlineto{\pgfqpoint{2.223102in}{1.278609in}}%
\pgfpathclose%
\pgfusepath{stroke,fill}%
\end{pgfscope}%
\begin{pgfscope}%
\pgfpathrectangle{\pgfqpoint{0.703330in}{0.352393in}}{\pgfqpoint{2.173234in}{2.758940in}}%
\pgfusepath{clip}%
\pgfsetbuttcap%
\pgfsetroundjoin%
\definecolor{currentfill}{rgb}{0.829112,0.619469,0.501926}%
\pgfsetfillcolor{currentfill}%
\pgfsetlinewidth{0.602250pt}%
\definecolor{currentstroke}{rgb}{0.296471,0.296471,0.296471}%
\pgfsetstrokecolor{currentstroke}%
\pgfsetdash{}{0pt}%
\pgfpathmoveto{\pgfqpoint{2.317574in}{1.298315in}}%
\pgfpathlineto{\pgfqpoint{2.443127in}{1.298315in}}%
\pgfpathlineto{\pgfqpoint{2.443127in}{1.219488in}}%
\pgfpathlineto{\pgfqpoint{2.317574in}{1.219488in}}%
\pgfpathlineto{\pgfqpoint{2.317574in}{1.298315in}}%
\pgfpathclose%
\pgfusepath{stroke,fill}%
\end{pgfscope}%
\begin{pgfscope}%
\pgfpathrectangle{\pgfqpoint{0.703330in}{0.352393in}}{\pgfqpoint{2.173234in}{2.758940in}}%
\pgfusepath{clip}%
\pgfsetbuttcap%
\pgfsetroundjoin%
\definecolor{currentfill}{rgb}{0.798529,0.536765,0.389706}%
\pgfsetfillcolor{currentfill}%
\pgfsetlinewidth{0.602250pt}%
\definecolor{currentstroke}{rgb}{0.296471,0.296471,0.296471}%
\pgfsetstrokecolor{currentstroke}%
\pgfsetdash{}{0pt}%
\pgfpathmoveto{\pgfqpoint{2.443127in}{1.337729in}}%
\pgfpathlineto{\pgfqpoint{2.666344in}{1.337729in}}%
\pgfpathlineto{\pgfqpoint{2.666344in}{1.180075in}}%
\pgfpathlineto{\pgfqpoint{2.443127in}{1.180075in}}%
\pgfpathlineto{\pgfqpoint{2.443127in}{1.337729in}}%
\pgfpathclose%
\pgfusepath{stroke,fill}%
\end{pgfscope}%
\begin{pgfscope}%
\pgfpathrectangle{\pgfqpoint{0.703330in}{0.352393in}}{\pgfqpoint{2.173234in}{2.758940in}}%
\pgfusepath{clip}%
\pgfsetbuttcap%
\pgfsetroundjoin%
\definecolor{currentfill}{rgb}{0.829112,0.619469,0.501926}%
\pgfsetfillcolor{currentfill}%
\pgfsetlinewidth{0.602250pt}%
\definecolor{currentstroke}{rgb}{0.296471,0.296471,0.296471}%
\pgfsetstrokecolor{currentstroke}%
\pgfsetdash{}{0pt}%
\pgfpathmoveto{\pgfqpoint{2.666344in}{1.298315in}}%
\pgfpathlineto{\pgfqpoint{2.724190in}{1.298315in}}%
\pgfpathlineto{\pgfqpoint{2.724190in}{1.219488in}}%
\pgfpathlineto{\pgfqpoint{2.666344in}{1.219488in}}%
\pgfpathlineto{\pgfqpoint{2.666344in}{1.298315in}}%
\pgfpathclose%
\pgfusepath{stroke,fill}%
\end{pgfscope}%
\begin{pgfscope}%
\pgfpathrectangle{\pgfqpoint{0.703330in}{0.352393in}}{\pgfqpoint{2.173234in}{2.758940in}}%
\pgfusepath{clip}%
\pgfsetbuttcap%
\pgfsetroundjoin%
\definecolor{currentfill}{rgb}{0.853814,0.686269,0.592565}%
\pgfsetfillcolor{currentfill}%
\pgfsetlinewidth{0.602250pt}%
\definecolor{currentstroke}{rgb}{0.296471,0.296471,0.296471}%
\pgfsetstrokecolor{currentstroke}%
\pgfsetdash{}{0pt}%
\pgfpathmoveto{\pgfqpoint{2.724190in}{1.278609in}}%
\pgfpathlineto{\pgfqpoint{2.741580in}{1.278609in}}%
\pgfpathlineto{\pgfqpoint{2.741580in}{1.239195in}}%
\pgfpathlineto{\pgfqpoint{2.724190in}{1.239195in}}%
\pgfpathlineto{\pgfqpoint{2.724190in}{1.278609in}}%
\pgfpathclose%
\pgfusepath{stroke,fill}%
\end{pgfscope}%
\begin{pgfscope}%
\pgfpathrectangle{\pgfqpoint{0.703330in}{0.352393in}}{\pgfqpoint{2.173234in}{2.758940in}}%
\pgfusepath{clip}%
\pgfsetbuttcap%
\pgfsetroundjoin%
\definecolor{currentfill}{rgb}{0.873223,0.738755,0.663782}%
\pgfsetfillcolor{currentfill}%
\pgfsetlinewidth{0.602250pt}%
\definecolor{currentstroke}{rgb}{0.296471,0.296471,0.296471}%
\pgfsetstrokecolor{currentstroke}%
\pgfsetdash{}{0pt}%
\pgfpathmoveto{\pgfqpoint{2.741580in}{1.268755in}}%
\pgfpathlineto{\pgfqpoint{2.758249in}{1.268755in}}%
\pgfpathlineto{\pgfqpoint{2.758249in}{1.249049in}}%
\pgfpathlineto{\pgfqpoint{2.741580in}{1.249049in}}%
\pgfpathlineto{\pgfqpoint{2.741580in}{1.268755in}}%
\pgfpathclose%
\pgfusepath{stroke,fill}%
\end{pgfscope}%
\begin{pgfscope}%
\pgfpathrectangle{\pgfqpoint{0.703330in}{0.352393in}}{\pgfqpoint{2.173234in}{2.758940in}}%
\pgfusepath{clip}%
\pgfsetbuttcap%
\pgfsetroundjoin%
\definecolor{currentfill}{rgb}{0.889102,0.781698,0.722050}%
\pgfsetfillcolor{currentfill}%
\pgfsetlinewidth{0.602250pt}%
\definecolor{currentstroke}{rgb}{0.296471,0.296471,0.296471}%
\pgfsetstrokecolor{currentstroke}%
\pgfsetdash{}{0pt}%
\pgfpathmoveto{\pgfqpoint{2.758249in}{1.263829in}}%
\pgfpathlineto{\pgfqpoint{2.815080in}{1.263829in}}%
\pgfpathlineto{\pgfqpoint{2.815080in}{1.253975in}}%
\pgfpathlineto{\pgfqpoint{2.758249in}{1.253975in}}%
\pgfpathlineto{\pgfqpoint{2.758249in}{1.263829in}}%
\pgfpathclose%
\pgfusepath{stroke,fill}%
\end{pgfscope}%
\begin{pgfscope}%
\pgfpathrectangle{\pgfqpoint{0.703330in}{0.352393in}}{\pgfqpoint{2.173234in}{2.758940in}}%
\pgfusepath{clip}%
\pgfsetbuttcap%
\pgfsetroundjoin%
\definecolor{currentfill}{rgb}{0.901453,0.815098,0.767370}%
\pgfsetfillcolor{currentfill}%
\pgfsetlinewidth{0.602250pt}%
\definecolor{currentstroke}{rgb}{0.296471,0.296471,0.296471}%
\pgfsetstrokecolor{currentstroke}%
\pgfsetdash{}{0pt}%
\pgfpathmoveto{\pgfqpoint{2.815080in}{1.261365in}}%
\pgfpathlineto{\pgfqpoint{2.820810in}{1.261365in}}%
\pgfpathlineto{\pgfqpoint{2.820810in}{1.256439in}}%
\pgfpathlineto{\pgfqpoint{2.815080in}{1.256439in}}%
\pgfpathlineto{\pgfqpoint{2.815080in}{1.261365in}}%
\pgfpathclose%
\pgfusepath{stroke,fill}%
\end{pgfscope}%
\begin{pgfscope}%
\pgfpathrectangle{\pgfqpoint{0.703330in}{0.352393in}}{\pgfqpoint{2.173234in}{2.758940in}}%
\pgfusepath{clip}%
\pgfsetbuttcap%
\pgfsetroundjoin%
\definecolor{currentfill}{rgb}{0.910863,0.840546,0.801899}%
\pgfsetfillcolor{currentfill}%
\pgfsetlinewidth{0.602250pt}%
\definecolor{currentstroke}{rgb}{0.296471,0.296471,0.296471}%
\pgfsetstrokecolor{currentstroke}%
\pgfsetdash{}{0pt}%
\pgfpathmoveto{\pgfqpoint{2.820810in}{1.260134in}}%
\pgfpathlineto{\pgfqpoint{2.825610in}{1.260134in}}%
\pgfpathlineto{\pgfqpoint{2.825610in}{1.257670in}}%
\pgfpathlineto{\pgfqpoint{2.820810in}{1.257670in}}%
\pgfpathlineto{\pgfqpoint{2.820810in}{1.260134in}}%
\pgfpathclose%
\pgfusepath{stroke,fill}%
\end{pgfscope}%
\begin{pgfscope}%
\pgfpathrectangle{\pgfqpoint{0.703330in}{0.352393in}}{\pgfqpoint{2.173234in}{2.758940in}}%
\pgfusepath{clip}%
\pgfsetbuttcap%
\pgfsetroundjoin%
\definecolor{currentfill}{rgb}{0.919097,0.862812,0.832112}%
\pgfsetfillcolor{currentfill}%
\pgfsetlinewidth{0.602250pt}%
\definecolor{currentstroke}{rgb}{0.296471,0.296471,0.296471}%
\pgfsetstrokecolor{currentstroke}%
\pgfsetdash{}{0pt}%
\pgfpathmoveto{\pgfqpoint{2.825610in}{1.259518in}}%
\pgfpathlineto{\pgfqpoint{2.825935in}{1.259518in}}%
\pgfpathlineto{\pgfqpoint{2.825935in}{1.258286in}}%
\pgfpathlineto{\pgfqpoint{2.825610in}{1.258286in}}%
\pgfpathlineto{\pgfqpoint{2.825610in}{1.259518in}}%
\pgfpathclose%
\pgfusepath{stroke,fill}%
\end{pgfscope}%
\begin{pgfscope}%
\pgfpathrectangle{\pgfqpoint{0.703330in}{0.352393in}}{\pgfqpoint{2.173234in}{2.758940in}}%
\pgfusepath{clip}%
\pgfsetbuttcap%
\pgfsetroundjoin%
\pgfsetlinewidth{0.803000pt}%
\definecolor{currentstroke}{rgb}{0.450000,0.450000,0.450000}%
\pgfsetstrokecolor{currentstroke}%
\pgfsetdash{}{0pt}%
\pgfpathmoveto{\pgfqpoint{0.000000in}{-0.034722in}}%
\pgfpathcurveto{\pgfqpoint{0.009208in}{-0.034722in}}{\pgfqpoint{0.018041in}{-0.031064in}}{\pgfqpoint{0.024552in}{-0.024552in}}%
\pgfpathcurveto{\pgfqpoint{0.031064in}{-0.018041in}}{\pgfqpoint{0.034722in}{-0.009208in}}{\pgfqpoint{0.034722in}{0.000000in}}%
\pgfpathcurveto{\pgfqpoint{0.034722in}{0.009208in}}{\pgfqpoint{0.031064in}{0.018041in}}{\pgfqpoint{0.024552in}{0.024552in}}%
\pgfpathcurveto{\pgfqpoint{0.018041in}{0.031064in}}{\pgfqpoint{0.009208in}{0.034722in}}{\pgfqpoint{0.000000in}{0.034722in}}%
\pgfpathcurveto{\pgfqpoint{-0.009208in}{0.034722in}}{\pgfqpoint{-0.018041in}{0.031064in}}{\pgfqpoint{-0.024552in}{0.024552in}}%
\pgfpathcurveto{\pgfqpoint{-0.031064in}{0.018041in}}{\pgfqpoint{-0.034722in}{0.009208in}}{\pgfqpoint{-0.034722in}{0.000000in}}%
\pgfpathcurveto{\pgfqpoint{-0.034722in}{-0.009208in}}{\pgfqpoint{-0.031064in}{-0.018041in}}{\pgfqpoint{-0.024552in}{-0.024552in}}%
\pgfpathcurveto{\pgfqpoint{-0.018041in}{-0.031064in}}{\pgfqpoint{-0.009208in}{-0.034722in}}{\pgfqpoint{0.000000in}{-0.034722in}}%
\pgfusepath{stroke}%
\end{pgfscope}%
\begin{pgfscope}%
\pgfpathrectangle{\pgfqpoint{0.703330in}{0.352393in}}{\pgfqpoint{2.173234in}{2.758940in}}%
\pgfusepath{clip}%
\pgfsetbuttcap%
\pgfsetroundjoin%
\definecolor{currentfill}{rgb}{0.848437,0.867532,0.899724}%
\pgfsetfillcolor{currentfill}%
\pgfsetlinewidth{0.602250pt}%
\definecolor{currentstroke}{rgb}{0.296471,0.296471,0.296471}%
\pgfsetstrokecolor{currentstroke}%
\pgfsetdash{}{0pt}%
\pgfpathmoveto{\pgfqpoint{-0.228056in}{1.022729in}}%
\pgfpathlineto{\pgfqpoint{-0.228056in}{1.022729in}}%
\pgfpathlineto{\pgfqpoint{-0.228056in}{1.022113in}}%
\pgfpathlineto{\pgfqpoint{-0.228056in}{1.022113in}}%
\pgfpathlineto{\pgfqpoint{-0.228056in}{1.022729in}}%
\pgfpathclose%
\pgfusepath{stroke,fill}%
\end{pgfscope}%
\begin{pgfscope}%
\pgfpathrectangle{\pgfqpoint{0.703330in}{0.352393in}}{\pgfqpoint{2.173234in}{2.758940in}}%
\pgfusepath{clip}%
\pgfsetbuttcap%
\pgfsetroundjoin%
\definecolor{currentfill}{rgb}{0.825117,0.848522,0.887698}%
\pgfsetfillcolor{currentfill}%
\pgfsetlinewidth{0.602250pt}%
\definecolor{currentstroke}{rgb}{0.296471,0.296471,0.296471}%
\pgfsetstrokecolor{currentstroke}%
\pgfsetdash{}{0pt}%
\pgfpathmoveto{\pgfqpoint{-0.228056in}{1.023037in}}%
\pgfpathlineto{\pgfqpoint{-0.228056in}{1.023037in}}%
\pgfpathlineto{\pgfqpoint{-0.228056in}{1.021805in}}%
\pgfpathlineto{\pgfqpoint{-0.228056in}{1.021805in}}%
\pgfpathlineto{\pgfqpoint{-0.228056in}{1.023037in}}%
\pgfpathclose%
\pgfusepath{stroke,fill}%
\end{pgfscope}%
\begin{pgfscope}%
\pgfpathrectangle{\pgfqpoint{0.703330in}{0.352393in}}{\pgfqpoint{2.173234in}{2.758940in}}%
\pgfusepath{clip}%
\pgfsetbuttcap%
\pgfsetroundjoin%
\definecolor{currentfill}{rgb}{0.792469,0.821908,0.870863}%
\pgfsetfillcolor{currentfill}%
\pgfsetlinewidth{0.602250pt}%
\definecolor{currentstroke}{rgb}{0.296471,0.296471,0.296471}%
\pgfsetstrokecolor{currentstroke}%
\pgfsetdash{}{0pt}%
\pgfpathmoveto{\pgfqpoint{-0.228056in}{1.023653in}}%
\pgfpathlineto{\pgfqpoint{-0.228056in}{1.023653in}}%
\pgfpathlineto{\pgfqpoint{-0.228056in}{1.021190in}}%
\pgfpathlineto{\pgfqpoint{-0.228056in}{1.021190in}}%
\pgfpathlineto{\pgfqpoint{-0.228056in}{1.023653in}}%
\pgfpathclose%
\pgfusepath{stroke,fill}%
\end{pgfscope}%
\begin{pgfscope}%
\pgfpathrectangle{\pgfqpoint{0.703330in}{0.352393in}}{\pgfqpoint{2.173234in}{2.758940in}}%
\pgfusepath{clip}%
\pgfsetbuttcap%
\pgfsetroundjoin%
\definecolor{currentfill}{rgb}{0.755157,0.791493,0.851622}%
\pgfsetfillcolor{currentfill}%
\pgfsetlinewidth{0.602250pt}%
\definecolor{currentstroke}{rgb}{0.296471,0.296471,0.296471}%
\pgfsetstrokecolor{currentstroke}%
\pgfsetdash{}{0pt}%
\pgfpathmoveto{\pgfqpoint{-0.228056in}{1.024885in}}%
\pgfpathlineto{\pgfqpoint{-0.228056in}{1.024885in}}%
\pgfpathlineto{\pgfqpoint{-0.228056in}{1.019958in}}%
\pgfpathlineto{\pgfqpoint{-0.228056in}{1.019958in}}%
\pgfpathlineto{\pgfqpoint{-0.228056in}{1.024885in}}%
\pgfpathclose%
\pgfusepath{stroke,fill}%
\end{pgfscope}%
\begin{pgfscope}%
\pgfpathrectangle{\pgfqpoint{0.703330in}{0.352393in}}{\pgfqpoint{2.173234in}{2.758940in}}%
\pgfusepath{clip}%
\pgfsetbuttcap%
\pgfsetroundjoin%
\definecolor{currentfill}{rgb}{0.706185,0.751573,0.826368}%
\pgfsetfillcolor{currentfill}%
\pgfsetlinewidth{0.602250pt}%
\definecolor{currentstroke}{rgb}{0.296471,0.296471,0.296471}%
\pgfsetstrokecolor{currentstroke}%
\pgfsetdash{}{0pt}%
\pgfpathmoveto{\pgfqpoint{-0.228056in}{1.027348in}}%
\pgfpathlineto{\pgfqpoint{-0.228056in}{1.027348in}}%
\pgfpathlineto{\pgfqpoint{-0.228056in}{1.017495in}}%
\pgfpathlineto{\pgfqpoint{-0.228056in}{1.017495in}}%
\pgfpathlineto{\pgfqpoint{-0.228056in}{1.027348in}}%
\pgfpathclose%
\pgfusepath{stroke,fill}%
\end{pgfscope}%
\begin{pgfscope}%
\pgfpathrectangle{\pgfqpoint{0.703330in}{0.352393in}}{\pgfqpoint{2.173234in}{2.758940in}}%
\pgfusepath{clip}%
\pgfsetbuttcap%
\pgfsetroundjoin%
\definecolor{currentfill}{rgb}{0.643221,0.700246,0.793900}%
\pgfsetfillcolor{currentfill}%
\pgfsetlinewidth{0.602250pt}%
\definecolor{currentstroke}{rgb}{0.296471,0.296471,0.296471}%
\pgfsetstrokecolor{currentstroke}%
\pgfsetdash{}{0pt}%
\pgfpathmoveto{\pgfqpoint{-0.228056in}{1.032275in}}%
\pgfpathlineto{\pgfqpoint{-0.228056in}{1.032275in}}%
\pgfpathlineto{\pgfqpoint{-0.228056in}{1.012568in}}%
\pgfpathlineto{\pgfqpoint{-0.228056in}{1.012568in}}%
\pgfpathlineto{\pgfqpoint{-0.228056in}{1.032275in}}%
\pgfpathclose%
\pgfusepath{stroke,fill}%
\end{pgfscope}%
\begin{pgfscope}%
\pgfpathrectangle{\pgfqpoint{0.703330in}{0.352393in}}{\pgfqpoint{2.173234in}{2.758940in}}%
\pgfusepath{clip}%
\pgfsetbuttcap%
\pgfsetroundjoin%
\definecolor{currentfill}{rgb}{0.566266,0.637515,0.754216}%
\pgfsetfillcolor{currentfill}%
\pgfsetlinewidth{0.602250pt}%
\definecolor{currentstroke}{rgb}{0.296471,0.296471,0.296471}%
\pgfsetstrokecolor{currentstroke}%
\pgfsetdash{}{0pt}%
\pgfpathmoveto{\pgfqpoint{-0.228056in}{1.042128in}}%
\pgfpathlineto{\pgfqpoint{-0.228056in}{1.042128in}}%
\pgfpathlineto{\pgfqpoint{-0.228056in}{1.002715in}}%
\pgfpathlineto{\pgfqpoint{-0.228056in}{1.002715in}}%
\pgfpathlineto{\pgfqpoint{-0.228056in}{1.042128in}}%
\pgfpathclose%
\pgfusepath{stroke,fill}%
\end{pgfscope}%
\begin{pgfscope}%
\pgfpathrectangle{\pgfqpoint{0.703330in}{0.352393in}}{\pgfqpoint{2.173234in}{2.758940in}}%
\pgfusepath{clip}%
\pgfsetbuttcap%
\pgfsetroundjoin%
\definecolor{currentfill}{rgb}{0.468322,0.557674,0.703709}%
\pgfsetfillcolor{currentfill}%
\pgfsetlinewidth{0.602250pt}%
\definecolor{currentstroke}{rgb}{0.296471,0.296471,0.296471}%
\pgfsetstrokecolor{currentstroke}%
\pgfsetdash{}{0pt}%
\pgfpathmoveto{\pgfqpoint{-0.228056in}{1.061835in}}%
\pgfpathlineto{\pgfqpoint{-0.228056in}{1.061835in}}%
\pgfpathlineto{\pgfqpoint{-0.228056in}{0.983008in}}%
\pgfpathlineto{\pgfqpoint{-0.228056in}{0.983008in}}%
\pgfpathlineto{\pgfqpoint{-0.228056in}{1.061835in}}%
\pgfpathclose%
\pgfusepath{stroke,fill}%
\end{pgfscope}%
\begin{pgfscope}%
\pgfpathrectangle{\pgfqpoint{0.703330in}{0.352393in}}{\pgfqpoint{2.173234in}{2.758940in}}%
\pgfusepath{clip}%
\pgfsetbuttcap%
\pgfsetroundjoin%
\definecolor{currentfill}{rgb}{0.347059,0.458824,0.641176}%
\pgfsetfillcolor{currentfill}%
\pgfsetlinewidth{0.602250pt}%
\definecolor{currentstroke}{rgb}{0.296471,0.296471,0.296471}%
\pgfsetstrokecolor{currentstroke}%
\pgfsetdash{}{0pt}%
\pgfpathmoveto{\pgfqpoint{-0.228056in}{1.101248in}}%
\pgfpathlineto{\pgfqpoint{-0.228056in}{1.101248in}}%
\pgfpathlineto{\pgfqpoint{-0.228056in}{0.943594in}}%
\pgfpathlineto{\pgfqpoint{-0.228056in}{0.943594in}}%
\pgfpathlineto{\pgfqpoint{-0.228056in}{1.101248in}}%
\pgfpathclose%
\pgfusepath{stroke,fill}%
\end{pgfscope}%
\begin{pgfscope}%
\pgfpathrectangle{\pgfqpoint{0.703330in}{0.352393in}}{\pgfqpoint{2.173234in}{2.758940in}}%
\pgfusepath{clip}%
\pgfsetbuttcap%
\pgfsetroundjoin%
\definecolor{currentfill}{rgb}{0.468322,0.557674,0.703709}%
\pgfsetfillcolor{currentfill}%
\pgfsetlinewidth{0.602250pt}%
\definecolor{currentstroke}{rgb}{0.296471,0.296471,0.296471}%
\pgfsetstrokecolor{currentstroke}%
\pgfsetdash{}{0pt}%
\pgfpathmoveto{\pgfqpoint{-0.228056in}{1.061835in}}%
\pgfpathlineto{\pgfqpoint{-0.228056in}{1.061835in}}%
\pgfpathlineto{\pgfqpoint{-0.228056in}{0.983008in}}%
\pgfpathlineto{\pgfqpoint{-0.228056in}{0.983008in}}%
\pgfpathlineto{\pgfqpoint{-0.228056in}{1.061835in}}%
\pgfpathclose%
\pgfusepath{stroke,fill}%
\end{pgfscope}%
\begin{pgfscope}%
\pgfpathrectangle{\pgfqpoint{0.703330in}{0.352393in}}{\pgfqpoint{2.173234in}{2.758940in}}%
\pgfusepath{clip}%
\pgfsetbuttcap%
\pgfsetroundjoin%
\definecolor{currentfill}{rgb}{0.566266,0.637515,0.754216}%
\pgfsetfillcolor{currentfill}%
\pgfsetlinewidth{0.602250pt}%
\definecolor{currentstroke}{rgb}{0.296471,0.296471,0.296471}%
\pgfsetstrokecolor{currentstroke}%
\pgfsetdash{}{0pt}%
\pgfpathmoveto{\pgfqpoint{-0.228056in}{1.042128in}}%
\pgfpathlineto{\pgfqpoint{-0.228056in}{1.042128in}}%
\pgfpathlineto{\pgfqpoint{-0.228056in}{1.002715in}}%
\pgfpathlineto{\pgfqpoint{-0.228056in}{1.002715in}}%
\pgfpathlineto{\pgfqpoint{-0.228056in}{1.042128in}}%
\pgfpathclose%
\pgfusepath{stroke,fill}%
\end{pgfscope}%
\begin{pgfscope}%
\pgfpathrectangle{\pgfqpoint{0.703330in}{0.352393in}}{\pgfqpoint{2.173234in}{2.758940in}}%
\pgfusepath{clip}%
\pgfsetbuttcap%
\pgfsetroundjoin%
\definecolor{currentfill}{rgb}{0.643221,0.700246,0.793900}%
\pgfsetfillcolor{currentfill}%
\pgfsetlinewidth{0.602250pt}%
\definecolor{currentstroke}{rgb}{0.296471,0.296471,0.296471}%
\pgfsetstrokecolor{currentstroke}%
\pgfsetdash{}{0pt}%
\pgfpathmoveto{\pgfqpoint{-0.228056in}{1.032275in}}%
\pgfpathlineto{\pgfqpoint{-0.228056in}{1.032275in}}%
\pgfpathlineto{\pgfqpoint{-0.228056in}{1.012568in}}%
\pgfpathlineto{\pgfqpoint{-0.228056in}{1.012568in}}%
\pgfpathlineto{\pgfqpoint{-0.228056in}{1.032275in}}%
\pgfpathclose%
\pgfusepath{stroke,fill}%
\end{pgfscope}%
\begin{pgfscope}%
\pgfpathrectangle{\pgfqpoint{0.703330in}{0.352393in}}{\pgfqpoint{2.173234in}{2.758940in}}%
\pgfusepath{clip}%
\pgfsetbuttcap%
\pgfsetroundjoin%
\definecolor{currentfill}{rgb}{0.706185,0.751573,0.826368}%
\pgfsetfillcolor{currentfill}%
\pgfsetlinewidth{0.602250pt}%
\definecolor{currentstroke}{rgb}{0.296471,0.296471,0.296471}%
\pgfsetstrokecolor{currentstroke}%
\pgfsetdash{}{0pt}%
\pgfpathmoveto{\pgfqpoint{-0.228056in}{1.027348in}}%
\pgfpathlineto{\pgfqpoint{-0.228056in}{1.027348in}}%
\pgfpathlineto{\pgfqpoint{-0.228056in}{1.017495in}}%
\pgfpathlineto{\pgfqpoint{-0.228056in}{1.017495in}}%
\pgfpathlineto{\pgfqpoint{-0.228056in}{1.027348in}}%
\pgfpathclose%
\pgfusepath{stroke,fill}%
\end{pgfscope}%
\begin{pgfscope}%
\pgfpathrectangle{\pgfqpoint{0.703330in}{0.352393in}}{\pgfqpoint{2.173234in}{2.758940in}}%
\pgfusepath{clip}%
\pgfsetbuttcap%
\pgfsetroundjoin%
\definecolor{currentfill}{rgb}{0.755157,0.791493,0.851622}%
\pgfsetfillcolor{currentfill}%
\pgfsetlinewidth{0.602250pt}%
\definecolor{currentstroke}{rgb}{0.296471,0.296471,0.296471}%
\pgfsetstrokecolor{currentstroke}%
\pgfsetdash{}{0pt}%
\pgfpathmoveto{\pgfqpoint{-0.228056in}{1.024885in}}%
\pgfpathlineto{\pgfqpoint{-0.228056in}{1.024885in}}%
\pgfpathlineto{\pgfqpoint{-0.228056in}{1.019958in}}%
\pgfpathlineto{\pgfqpoint{-0.228056in}{1.019958in}}%
\pgfpathlineto{\pgfqpoint{-0.228056in}{1.024885in}}%
\pgfpathclose%
\pgfusepath{stroke,fill}%
\end{pgfscope}%
\begin{pgfscope}%
\pgfpathrectangle{\pgfqpoint{0.703330in}{0.352393in}}{\pgfqpoint{2.173234in}{2.758940in}}%
\pgfusepath{clip}%
\pgfsetbuttcap%
\pgfsetroundjoin%
\definecolor{currentfill}{rgb}{0.792469,0.821908,0.870863}%
\pgfsetfillcolor{currentfill}%
\pgfsetlinewidth{0.602250pt}%
\definecolor{currentstroke}{rgb}{0.296471,0.296471,0.296471}%
\pgfsetstrokecolor{currentstroke}%
\pgfsetdash{}{0pt}%
\pgfpathmoveto{\pgfqpoint{-0.228056in}{1.023653in}}%
\pgfpathlineto{\pgfqpoint{-0.228056in}{1.023653in}}%
\pgfpathlineto{\pgfqpoint{-0.228056in}{1.021190in}}%
\pgfpathlineto{\pgfqpoint{-0.228056in}{1.021190in}}%
\pgfpathlineto{\pgfqpoint{-0.228056in}{1.023653in}}%
\pgfpathclose%
\pgfusepath{stroke,fill}%
\end{pgfscope}%
\begin{pgfscope}%
\pgfpathrectangle{\pgfqpoint{0.703330in}{0.352393in}}{\pgfqpoint{2.173234in}{2.758940in}}%
\pgfusepath{clip}%
\pgfsetbuttcap%
\pgfsetroundjoin%
\definecolor{currentfill}{rgb}{0.825117,0.848522,0.887698}%
\pgfsetfillcolor{currentfill}%
\pgfsetlinewidth{0.602250pt}%
\definecolor{currentstroke}{rgb}{0.296471,0.296471,0.296471}%
\pgfsetstrokecolor{currentstroke}%
\pgfsetdash{}{0pt}%
\pgfpathmoveto{\pgfqpoint{-0.228056in}{1.023037in}}%
\pgfpathlineto{\pgfqpoint{-0.228056in}{1.023037in}}%
\pgfpathlineto{\pgfqpoint{-0.228056in}{1.021805in}}%
\pgfpathlineto{\pgfqpoint{-0.228056in}{1.021805in}}%
\pgfpathlineto{\pgfqpoint{-0.228056in}{1.023037in}}%
\pgfpathclose%
\pgfusepath{stroke,fill}%
\end{pgfscope}%
\begin{pgfscope}%
\pgfpathrectangle{\pgfqpoint{0.703330in}{0.352393in}}{\pgfqpoint{2.173234in}{2.758940in}}%
\pgfusepath{clip}%
\pgfsetbuttcap%
\pgfsetroundjoin%
\definecolor{currentfill}{rgb}{0.848437,0.867532,0.899724}%
\pgfsetfillcolor{currentfill}%
\pgfsetlinewidth{0.602250pt}%
\definecolor{currentstroke}{rgb}{0.296471,0.296471,0.296471}%
\pgfsetstrokecolor{currentstroke}%
\pgfsetdash{}{0pt}%
\pgfpathmoveto{\pgfqpoint{-0.228056in}{1.022729in}}%
\pgfpathlineto{\pgfqpoint{-0.228056in}{1.022729in}}%
\pgfpathlineto{\pgfqpoint{-0.228056in}{1.022113in}}%
\pgfpathlineto{\pgfqpoint{-0.228056in}{1.022113in}}%
\pgfpathlineto{\pgfqpoint{-0.228056in}{1.022729in}}%
\pgfpathclose%
\pgfusepath{stroke,fill}%
\end{pgfscope}%
\begin{pgfscope}%
\pgfpathrectangle{\pgfqpoint{0.703330in}{0.352393in}}{\pgfqpoint{2.173234in}{2.758940in}}%
\pgfusepath{clip}%
\pgfsetbuttcap%
\pgfsetroundjoin%
\pgfsetlinewidth{0.803000pt}%
\definecolor{currentstroke}{rgb}{0.450000,0.450000,0.450000}%
\pgfsetstrokecolor{currentstroke}%
\pgfsetdash{}{0pt}%
\pgfpathmoveto{\pgfqpoint{0.000000in}{-0.034722in}}%
\pgfpathcurveto{\pgfqpoint{0.009208in}{-0.034722in}}{\pgfqpoint{0.018041in}{-0.031064in}}{\pgfqpoint{0.024552in}{-0.024552in}}%
\pgfpathcurveto{\pgfqpoint{0.031064in}{-0.018041in}}{\pgfqpoint{0.034722in}{-0.009208in}}{\pgfqpoint{0.034722in}{0.000000in}}%
\pgfpathcurveto{\pgfqpoint{0.034722in}{0.009208in}}{\pgfqpoint{0.031064in}{0.018041in}}{\pgfqpoint{0.024552in}{0.024552in}}%
\pgfpathcurveto{\pgfqpoint{0.018041in}{0.031064in}}{\pgfqpoint{0.009208in}{0.034722in}}{\pgfqpoint{0.000000in}{0.034722in}}%
\pgfpathcurveto{\pgfqpoint{-0.009208in}{0.034722in}}{\pgfqpoint{-0.018041in}{0.031064in}}{\pgfqpoint{-0.024552in}{0.024552in}}%
\pgfpathcurveto{\pgfqpoint{-0.031064in}{0.018041in}}{\pgfqpoint{-0.034722in}{0.009208in}}{\pgfqpoint{-0.034722in}{0.000000in}}%
\pgfpathcurveto{\pgfqpoint{-0.034722in}{-0.009208in}}{\pgfqpoint{-0.031064in}{-0.018041in}}{\pgfqpoint{-0.024552in}{-0.024552in}}%
\pgfpathcurveto{\pgfqpoint{-0.018041in}{-0.031064in}}{\pgfqpoint{-0.009208in}{-0.034722in}}{\pgfqpoint{0.000000in}{-0.034722in}}%
\pgfusepath{stroke}%
\end{pgfscope}%
\begin{pgfscope}%
\pgfpathrectangle{\pgfqpoint{0.703330in}{0.352393in}}{\pgfqpoint{2.173234in}{2.758940in}}%
\pgfusepath{clip}%
\pgfsetbuttcap%
\pgfsetroundjoin%
\definecolor{currentfill}{rgb}{0.919097,0.862812,0.832112}%
\pgfsetfillcolor{currentfill}%
\pgfsetlinewidth{0.602250pt}%
\definecolor{currentstroke}{rgb}{0.296471,0.296471,0.296471}%
\pgfsetstrokecolor{currentstroke}%
\pgfsetdash{}{0pt}%
\pgfpathmoveto{\pgfqpoint{1.789608in}{0.865383in}}%
\pgfpathlineto{\pgfqpoint{1.871353in}{0.865383in}}%
\pgfpathlineto{\pgfqpoint{1.871353in}{0.864152in}}%
\pgfpathlineto{\pgfqpoint{1.789608in}{0.864152in}}%
\pgfpathlineto{\pgfqpoint{1.789608in}{0.865383in}}%
\pgfpathclose%
\pgfusepath{stroke,fill}%
\end{pgfscope}%
\begin{pgfscope}%
\pgfpathrectangle{\pgfqpoint{0.703330in}{0.352393in}}{\pgfqpoint{2.173234in}{2.758940in}}%
\pgfusepath{clip}%
\pgfsetbuttcap%
\pgfsetroundjoin%
\definecolor{currentfill}{rgb}{0.910863,0.840546,0.801899}%
\pgfsetfillcolor{currentfill}%
\pgfsetlinewidth{0.602250pt}%
\definecolor{currentstroke}{rgb}{0.296471,0.296471,0.296471}%
\pgfsetstrokecolor{currentstroke}%
\pgfsetdash{}{0pt}%
\pgfpathmoveto{\pgfqpoint{1.871353in}{0.865999in}}%
\pgfpathlineto{\pgfqpoint{1.888187in}{0.865999in}}%
\pgfpathlineto{\pgfqpoint{1.888187in}{0.863536in}}%
\pgfpathlineto{\pgfqpoint{1.871353in}{0.863536in}}%
\pgfpathlineto{\pgfqpoint{1.871353in}{0.865999in}}%
\pgfpathclose%
\pgfusepath{stroke,fill}%
\end{pgfscope}%
\begin{pgfscope}%
\pgfpathrectangle{\pgfqpoint{0.703330in}{0.352393in}}{\pgfqpoint{2.173234in}{2.758940in}}%
\pgfusepath{clip}%
\pgfsetbuttcap%
\pgfsetroundjoin%
\definecolor{currentfill}{rgb}{0.901453,0.815098,0.767370}%
\pgfsetfillcolor{currentfill}%
\pgfsetlinewidth{0.602250pt}%
\definecolor{currentstroke}{rgb}{0.296471,0.296471,0.296471}%
\pgfsetstrokecolor{currentstroke}%
\pgfsetdash{}{0pt}%
\pgfpathmoveto{\pgfqpoint{1.888187in}{0.867231in}}%
\pgfpathlineto{\pgfqpoint{1.903558in}{0.867231in}}%
\pgfpathlineto{\pgfqpoint{1.903558in}{0.862304in}}%
\pgfpathlineto{\pgfqpoint{1.888187in}{0.862304in}}%
\pgfpathlineto{\pgfqpoint{1.888187in}{0.867231in}}%
\pgfpathclose%
\pgfusepath{stroke,fill}%
\end{pgfscope}%
\begin{pgfscope}%
\pgfpathrectangle{\pgfqpoint{0.703330in}{0.352393in}}{\pgfqpoint{2.173234in}{2.758940in}}%
\pgfusepath{clip}%
\pgfsetbuttcap%
\pgfsetroundjoin%
\definecolor{currentfill}{rgb}{0.889102,0.781698,0.722050}%
\pgfsetfillcolor{currentfill}%
\pgfsetlinewidth{0.602250pt}%
\definecolor{currentstroke}{rgb}{0.296471,0.296471,0.296471}%
\pgfsetstrokecolor{currentstroke}%
\pgfsetdash{}{0pt}%
\pgfpathmoveto{\pgfqpoint{1.903558in}{0.869694in}}%
\pgfpathlineto{\pgfqpoint{1.968647in}{0.869694in}}%
\pgfpathlineto{\pgfqpoint{1.968647in}{0.859841in}}%
\pgfpathlineto{\pgfqpoint{1.903558in}{0.859841in}}%
\pgfpathlineto{\pgfqpoint{1.903558in}{0.869694in}}%
\pgfpathclose%
\pgfusepath{stroke,fill}%
\end{pgfscope}%
\begin{pgfscope}%
\pgfpathrectangle{\pgfqpoint{0.703330in}{0.352393in}}{\pgfqpoint{2.173234in}{2.758940in}}%
\pgfusepath{clip}%
\pgfsetbuttcap%
\pgfsetroundjoin%
\definecolor{currentfill}{rgb}{0.873223,0.738755,0.663782}%
\pgfsetfillcolor{currentfill}%
\pgfsetlinewidth{0.602250pt}%
\definecolor{currentstroke}{rgb}{0.296471,0.296471,0.296471}%
\pgfsetstrokecolor{currentstroke}%
\pgfsetdash{}{0pt}%
\pgfpathmoveto{\pgfqpoint{1.968647in}{0.874621in}}%
\pgfpathlineto{\pgfqpoint{2.118660in}{0.874621in}}%
\pgfpathlineto{\pgfqpoint{2.118660in}{0.854914in}}%
\pgfpathlineto{\pgfqpoint{1.968647in}{0.854914in}}%
\pgfpathlineto{\pgfqpoint{1.968647in}{0.874621in}}%
\pgfpathclose%
\pgfusepath{stroke,fill}%
\end{pgfscope}%
\begin{pgfscope}%
\pgfpathrectangle{\pgfqpoint{0.703330in}{0.352393in}}{\pgfqpoint{2.173234in}{2.758940in}}%
\pgfusepath{clip}%
\pgfsetbuttcap%
\pgfsetroundjoin%
\definecolor{currentfill}{rgb}{0.853814,0.686269,0.592565}%
\pgfsetfillcolor{currentfill}%
\pgfsetlinewidth{0.602250pt}%
\definecolor{currentstroke}{rgb}{0.296471,0.296471,0.296471}%
\pgfsetstrokecolor{currentstroke}%
\pgfsetdash{}{0pt}%
\pgfpathmoveto{\pgfqpoint{2.118660in}{0.884474in}}%
\pgfpathlineto{\pgfqpoint{2.237774in}{0.884474in}}%
\pgfpathlineto{\pgfqpoint{2.237774in}{0.845061in}}%
\pgfpathlineto{\pgfqpoint{2.118660in}{0.845061in}}%
\pgfpathlineto{\pgfqpoint{2.118660in}{0.884474in}}%
\pgfpathclose%
\pgfusepath{stroke,fill}%
\end{pgfscope}%
\begin{pgfscope}%
\pgfpathrectangle{\pgfqpoint{0.703330in}{0.352393in}}{\pgfqpoint{2.173234in}{2.758940in}}%
\pgfusepath{clip}%
\pgfsetbuttcap%
\pgfsetroundjoin%
\definecolor{currentfill}{rgb}{0.829112,0.619469,0.501926}%
\pgfsetfillcolor{currentfill}%
\pgfsetlinewidth{0.602250pt}%
\definecolor{currentstroke}{rgb}{0.296471,0.296471,0.296471}%
\pgfsetstrokecolor{currentstroke}%
\pgfsetdash{}{0pt}%
\pgfpathmoveto{\pgfqpoint{2.237774in}{0.904181in}}%
\pgfpathlineto{\pgfqpoint{2.398930in}{0.904181in}}%
\pgfpathlineto{\pgfqpoint{2.398930in}{0.825354in}}%
\pgfpathlineto{\pgfqpoint{2.237774in}{0.825354in}}%
\pgfpathlineto{\pgfqpoint{2.237774in}{0.904181in}}%
\pgfpathclose%
\pgfusepath{stroke,fill}%
\end{pgfscope}%
\begin{pgfscope}%
\pgfpathrectangle{\pgfqpoint{0.703330in}{0.352393in}}{\pgfqpoint{2.173234in}{2.758940in}}%
\pgfusepath{clip}%
\pgfsetbuttcap%
\pgfsetroundjoin%
\definecolor{currentfill}{rgb}{0.798529,0.536765,0.389706}%
\pgfsetfillcolor{currentfill}%
\pgfsetlinewidth{0.602250pt}%
\definecolor{currentstroke}{rgb}{0.296471,0.296471,0.296471}%
\pgfsetstrokecolor{currentstroke}%
\pgfsetdash{}{0pt}%
\pgfpathmoveto{\pgfqpoint{2.398930in}{0.943594in}}%
\pgfpathlineto{\pgfqpoint{2.754817in}{0.943594in}}%
\pgfpathlineto{\pgfqpoint{2.754817in}{0.785941in}}%
\pgfpathlineto{\pgfqpoint{2.398930in}{0.785941in}}%
\pgfpathlineto{\pgfqpoint{2.398930in}{0.943594in}}%
\pgfpathclose%
\pgfusepath{stroke,fill}%
\end{pgfscope}%
\begin{pgfscope}%
\pgfpathrectangle{\pgfqpoint{0.703330in}{0.352393in}}{\pgfqpoint{2.173234in}{2.758940in}}%
\pgfusepath{clip}%
\pgfsetbuttcap%
\pgfsetroundjoin%
\definecolor{currentfill}{rgb}{0.829112,0.619469,0.501926}%
\pgfsetfillcolor{currentfill}%
\pgfsetlinewidth{0.602250pt}%
\definecolor{currentstroke}{rgb}{0.296471,0.296471,0.296471}%
\pgfsetstrokecolor{currentstroke}%
\pgfsetdash{}{0pt}%
\pgfpathmoveto{\pgfqpoint{2.754817in}{0.904181in}}%
\pgfpathlineto{\pgfqpoint{2.821194in}{0.904181in}}%
\pgfpathlineto{\pgfqpoint{2.821194in}{0.825354in}}%
\pgfpathlineto{\pgfqpoint{2.754817in}{0.825354in}}%
\pgfpathlineto{\pgfqpoint{2.754817in}{0.904181in}}%
\pgfpathclose%
\pgfusepath{stroke,fill}%
\end{pgfscope}%
\begin{pgfscope}%
\pgfpathrectangle{\pgfqpoint{0.703330in}{0.352393in}}{\pgfqpoint{2.173234in}{2.758940in}}%
\pgfusepath{clip}%
\pgfsetbuttcap%
\pgfsetroundjoin%
\definecolor{currentfill}{rgb}{0.853814,0.686269,0.592565}%
\pgfsetfillcolor{currentfill}%
\pgfsetlinewidth{0.602250pt}%
\definecolor{currentstroke}{rgb}{0.296471,0.296471,0.296471}%
\pgfsetstrokecolor{currentstroke}%
\pgfsetdash{}{0pt}%
\pgfpathmoveto{\pgfqpoint{2.821194in}{0.884474in}}%
\pgfpathlineto{\pgfqpoint{2.849980in}{0.884474in}}%
\pgfpathlineto{\pgfqpoint{2.849980in}{0.845061in}}%
\pgfpathlineto{\pgfqpoint{2.821194in}{0.845061in}}%
\pgfpathlineto{\pgfqpoint{2.821194in}{0.884474in}}%
\pgfpathclose%
\pgfusepath{stroke,fill}%
\end{pgfscope}%
\begin{pgfscope}%
\pgfpathrectangle{\pgfqpoint{0.703330in}{0.352393in}}{\pgfqpoint{2.173234in}{2.758940in}}%
\pgfusepath{clip}%
\pgfsetbuttcap%
\pgfsetroundjoin%
\definecolor{currentfill}{rgb}{0.873223,0.738755,0.663782}%
\pgfsetfillcolor{currentfill}%
\pgfsetlinewidth{0.602250pt}%
\definecolor{currentstroke}{rgb}{0.296471,0.296471,0.296471}%
\pgfsetstrokecolor{currentstroke}%
\pgfsetdash{}{0pt}%
\pgfpathmoveto{\pgfqpoint{2.849980in}{0.874621in}}%
\pgfpathlineto{\pgfqpoint{2.876564in}{0.874621in}}%
\pgfpathlineto{\pgfqpoint{2.876564in}{0.854914in}}%
\pgfpathlineto{\pgfqpoint{2.849980in}{0.854914in}}%
\pgfpathlineto{\pgfqpoint{2.849980in}{0.874621in}}%
\pgfpathclose%
\pgfusepath{stroke,fill}%
\end{pgfscope}%
\begin{pgfscope}%
\pgfpathrectangle{\pgfqpoint{0.703330in}{0.352393in}}{\pgfqpoint{2.173234in}{2.758940in}}%
\pgfusepath{clip}%
\pgfsetbuttcap%
\pgfsetroundjoin%
\definecolor{currentfill}{rgb}{0.889102,0.781698,0.722050}%
\pgfsetfillcolor{currentfill}%
\pgfsetlinewidth{0.602250pt}%
\definecolor{currentstroke}{rgb}{0.296471,0.296471,0.296471}%
\pgfsetstrokecolor{currentstroke}%
\pgfsetdash{}{0pt}%
\pgfpathmoveto{\pgfqpoint{2.876564in}{0.869694in}}%
\pgfpathlineto{\pgfqpoint{2.876564in}{0.869694in}}%
\pgfpathlineto{\pgfqpoint{2.876564in}{0.859841in}}%
\pgfpathlineto{\pgfqpoint{2.876564in}{0.859841in}}%
\pgfpathlineto{\pgfqpoint{2.876564in}{0.869694in}}%
\pgfpathclose%
\pgfusepath{stroke,fill}%
\end{pgfscope}%
\begin{pgfscope}%
\pgfpathrectangle{\pgfqpoint{0.703330in}{0.352393in}}{\pgfqpoint{2.173234in}{2.758940in}}%
\pgfusepath{clip}%
\pgfsetbuttcap%
\pgfsetroundjoin%
\definecolor{currentfill}{rgb}{0.901453,0.815098,0.767370}%
\pgfsetfillcolor{currentfill}%
\pgfsetlinewidth{0.602250pt}%
\definecolor{currentstroke}{rgb}{0.296471,0.296471,0.296471}%
\pgfsetstrokecolor{currentstroke}%
\pgfsetdash{}{0pt}%
\pgfpathmoveto{\pgfqpoint{2.876564in}{0.867231in}}%
\pgfpathlineto{\pgfqpoint{2.876564in}{0.867231in}}%
\pgfpathlineto{\pgfqpoint{2.876564in}{0.862304in}}%
\pgfpathlineto{\pgfqpoint{2.876564in}{0.862304in}}%
\pgfpathlineto{\pgfqpoint{2.876564in}{0.867231in}}%
\pgfpathclose%
\pgfusepath{stroke,fill}%
\end{pgfscope}%
\begin{pgfscope}%
\pgfpathrectangle{\pgfqpoint{0.703330in}{0.352393in}}{\pgfqpoint{2.173234in}{2.758940in}}%
\pgfusepath{clip}%
\pgfsetbuttcap%
\pgfsetroundjoin%
\definecolor{currentfill}{rgb}{0.910863,0.840546,0.801899}%
\pgfsetfillcolor{currentfill}%
\pgfsetlinewidth{0.602250pt}%
\definecolor{currentstroke}{rgb}{0.296471,0.296471,0.296471}%
\pgfsetstrokecolor{currentstroke}%
\pgfsetdash{}{0pt}%
\pgfpathmoveto{\pgfqpoint{2.876564in}{0.865999in}}%
\pgfpathlineto{\pgfqpoint{2.876564in}{0.865999in}}%
\pgfpathlineto{\pgfqpoint{2.876564in}{0.863536in}}%
\pgfpathlineto{\pgfqpoint{2.876564in}{0.863536in}}%
\pgfpathlineto{\pgfqpoint{2.876564in}{0.865999in}}%
\pgfpathclose%
\pgfusepath{stroke,fill}%
\end{pgfscope}%
\begin{pgfscope}%
\pgfpathrectangle{\pgfqpoint{0.703330in}{0.352393in}}{\pgfqpoint{2.173234in}{2.758940in}}%
\pgfusepath{clip}%
\pgfsetbuttcap%
\pgfsetroundjoin%
\definecolor{currentfill}{rgb}{0.919097,0.862812,0.832112}%
\pgfsetfillcolor{currentfill}%
\pgfsetlinewidth{0.602250pt}%
\definecolor{currentstroke}{rgb}{0.296471,0.296471,0.296471}%
\pgfsetstrokecolor{currentstroke}%
\pgfsetdash{}{0pt}%
\pgfpathmoveto{\pgfqpoint{2.876564in}{0.865383in}}%
\pgfpathlineto{\pgfqpoint{2.876564in}{0.865383in}}%
\pgfpathlineto{\pgfqpoint{2.876564in}{0.864152in}}%
\pgfpathlineto{\pgfqpoint{2.876564in}{0.864152in}}%
\pgfpathlineto{\pgfqpoint{2.876564in}{0.865383in}}%
\pgfpathclose%
\pgfusepath{stroke,fill}%
\end{pgfscope}%
\begin{pgfscope}%
\pgfpathrectangle{\pgfqpoint{0.703330in}{0.352393in}}{\pgfqpoint{2.173234in}{2.758940in}}%
\pgfusepath{clip}%
\pgfsetbuttcap%
\pgfsetroundjoin%
\pgfsetlinewidth{0.803000pt}%
\definecolor{currentstroke}{rgb}{0.450000,0.450000,0.450000}%
\pgfsetstrokecolor{currentstroke}%
\pgfsetdash{}{0pt}%
\pgfpathmoveto{\pgfqpoint{0.000000in}{-0.034722in}}%
\pgfpathcurveto{\pgfqpoint{0.009208in}{-0.034722in}}{\pgfqpoint{0.018041in}{-0.031064in}}{\pgfqpoint{0.024552in}{-0.024552in}}%
\pgfpathcurveto{\pgfqpoint{0.031064in}{-0.018041in}}{\pgfqpoint{0.034722in}{-0.009208in}}{\pgfqpoint{0.034722in}{0.000000in}}%
\pgfpathcurveto{\pgfqpoint{0.034722in}{0.009208in}}{\pgfqpoint{0.031064in}{0.018041in}}{\pgfqpoint{0.024552in}{0.024552in}}%
\pgfpathcurveto{\pgfqpoint{0.018041in}{0.031064in}}{\pgfqpoint{0.009208in}{0.034722in}}{\pgfqpoint{0.000000in}{0.034722in}}%
\pgfpathcurveto{\pgfqpoint{-0.009208in}{0.034722in}}{\pgfqpoint{-0.018041in}{0.031064in}}{\pgfqpoint{-0.024552in}{0.024552in}}%
\pgfpathcurveto{\pgfqpoint{-0.031064in}{0.018041in}}{\pgfqpoint{-0.034722in}{0.009208in}}{\pgfqpoint{-0.034722in}{0.000000in}}%
\pgfpathcurveto{\pgfqpoint{-0.034722in}{-0.009208in}}{\pgfqpoint{-0.031064in}{-0.018041in}}{\pgfqpoint{-0.024552in}{-0.024552in}}%
\pgfpathcurveto{\pgfqpoint{-0.018041in}{-0.031064in}}{\pgfqpoint{-0.009208in}{-0.034722in}}{\pgfqpoint{0.000000in}{-0.034722in}}%
\pgfusepath{stroke}%
\end{pgfscope}%
\begin{pgfscope}%
\pgfpathrectangle{\pgfqpoint{0.703330in}{0.352393in}}{\pgfqpoint{2.173234in}{2.758940in}}%
\pgfusepath{clip}%
\pgfsetbuttcap%
\pgfsetroundjoin%
\definecolor{currentfill}{rgb}{0.848437,0.867532,0.899724}%
\pgfsetfillcolor{currentfill}%
\pgfsetlinewidth{0.602250pt}%
\definecolor{currentstroke}{rgb}{0.296471,0.296471,0.296471}%
\pgfsetstrokecolor{currentstroke}%
\pgfsetdash{}{0pt}%
\pgfpathmoveto{\pgfqpoint{0.532885in}{0.628595in}}%
\pgfpathlineto{\pgfqpoint{0.535660in}{0.628595in}}%
\pgfpathlineto{\pgfqpoint{0.535660in}{0.627979in}}%
\pgfpathlineto{\pgfqpoint{0.532885in}{0.627979in}}%
\pgfpathlineto{\pgfqpoint{0.532885in}{0.628595in}}%
\pgfpathclose%
\pgfusepath{stroke,fill}%
\end{pgfscope}%
\begin{pgfscope}%
\pgfpathrectangle{\pgfqpoint{0.703330in}{0.352393in}}{\pgfqpoint{2.173234in}{2.758940in}}%
\pgfusepath{clip}%
\pgfsetbuttcap%
\pgfsetroundjoin%
\definecolor{currentfill}{rgb}{0.825117,0.848522,0.887698}%
\pgfsetfillcolor{currentfill}%
\pgfsetlinewidth{0.602250pt}%
\definecolor{currentstroke}{rgb}{0.296471,0.296471,0.296471}%
\pgfsetstrokecolor{currentstroke}%
\pgfsetdash{}{0pt}%
\pgfpathmoveto{\pgfqpoint{0.535660in}{0.628903in}}%
\pgfpathlineto{\pgfqpoint{0.537571in}{0.628903in}}%
\pgfpathlineto{\pgfqpoint{0.537571in}{0.627671in}}%
\pgfpathlineto{\pgfqpoint{0.535660in}{0.627671in}}%
\pgfpathlineto{\pgfqpoint{0.535660in}{0.628903in}}%
\pgfpathclose%
\pgfusepath{stroke,fill}%
\end{pgfscope}%
\begin{pgfscope}%
\pgfpathrectangle{\pgfqpoint{0.703330in}{0.352393in}}{\pgfqpoint{2.173234in}{2.758940in}}%
\pgfusepath{clip}%
\pgfsetbuttcap%
\pgfsetroundjoin%
\definecolor{currentfill}{rgb}{0.792469,0.821908,0.870863}%
\pgfsetfillcolor{currentfill}%
\pgfsetlinewidth{0.602250pt}%
\definecolor{currentstroke}{rgb}{0.296471,0.296471,0.296471}%
\pgfsetstrokecolor{currentstroke}%
\pgfsetdash{}{0pt}%
\pgfpathmoveto{\pgfqpoint{0.537571in}{0.629519in}}%
\pgfpathlineto{\pgfqpoint{0.538185in}{0.629519in}}%
\pgfpathlineto{\pgfqpoint{0.538185in}{0.627055in}}%
\pgfpathlineto{\pgfqpoint{0.537571in}{0.627055in}}%
\pgfpathlineto{\pgfqpoint{0.537571in}{0.629519in}}%
\pgfpathclose%
\pgfusepath{stroke,fill}%
\end{pgfscope}%
\begin{pgfscope}%
\pgfpathrectangle{\pgfqpoint{0.703330in}{0.352393in}}{\pgfqpoint{2.173234in}{2.758940in}}%
\pgfusepath{clip}%
\pgfsetbuttcap%
\pgfsetroundjoin%
\definecolor{currentfill}{rgb}{0.755157,0.791493,0.851622}%
\pgfsetfillcolor{currentfill}%
\pgfsetlinewidth{0.602250pt}%
\definecolor{currentstroke}{rgb}{0.296471,0.296471,0.296471}%
\pgfsetstrokecolor{currentstroke}%
\pgfsetdash{}{0pt}%
\pgfpathmoveto{\pgfqpoint{0.538185in}{0.630750in}}%
\pgfpathlineto{\pgfqpoint{0.539467in}{0.630750in}}%
\pgfpathlineto{\pgfqpoint{0.539467in}{0.625824in}}%
\pgfpathlineto{\pgfqpoint{0.538185in}{0.625824in}}%
\pgfpathlineto{\pgfqpoint{0.538185in}{0.630750in}}%
\pgfpathclose%
\pgfusepath{stroke,fill}%
\end{pgfscope}%
\begin{pgfscope}%
\pgfpathrectangle{\pgfqpoint{0.703330in}{0.352393in}}{\pgfqpoint{2.173234in}{2.758940in}}%
\pgfusepath{clip}%
\pgfsetbuttcap%
\pgfsetroundjoin%
\definecolor{currentfill}{rgb}{0.706185,0.751573,0.826368}%
\pgfsetfillcolor{currentfill}%
\pgfsetlinewidth{0.602250pt}%
\definecolor{currentstroke}{rgb}{0.296471,0.296471,0.296471}%
\pgfsetstrokecolor{currentstroke}%
\pgfsetdash{}{0pt}%
\pgfpathmoveto{\pgfqpoint{0.539467in}{0.633214in}}%
\pgfpathlineto{\pgfqpoint{0.541778in}{0.633214in}}%
\pgfpathlineto{\pgfqpoint{0.541778in}{0.623360in}}%
\pgfpathlineto{\pgfqpoint{0.539467in}{0.623360in}}%
\pgfpathlineto{\pgfqpoint{0.539467in}{0.633214in}}%
\pgfpathclose%
\pgfusepath{stroke,fill}%
\end{pgfscope}%
\begin{pgfscope}%
\pgfpathrectangle{\pgfqpoint{0.703330in}{0.352393in}}{\pgfqpoint{2.173234in}{2.758940in}}%
\pgfusepath{clip}%
\pgfsetbuttcap%
\pgfsetroundjoin%
\definecolor{currentfill}{rgb}{0.643221,0.700246,0.793900}%
\pgfsetfillcolor{currentfill}%
\pgfsetlinewidth{0.602250pt}%
\definecolor{currentstroke}{rgb}{0.296471,0.296471,0.296471}%
\pgfsetstrokecolor{currentstroke}%
\pgfsetdash{}{0pt}%
\pgfpathmoveto{\pgfqpoint{0.541778in}{0.638140in}}%
\pgfpathlineto{\pgfqpoint{0.542737in}{0.638140in}}%
\pgfpathlineto{\pgfqpoint{0.542737in}{0.618434in}}%
\pgfpathlineto{\pgfqpoint{0.541778in}{0.618434in}}%
\pgfpathlineto{\pgfqpoint{0.541778in}{0.638140in}}%
\pgfpathclose%
\pgfusepath{stroke,fill}%
\end{pgfscope}%
\begin{pgfscope}%
\pgfpathrectangle{\pgfqpoint{0.703330in}{0.352393in}}{\pgfqpoint{2.173234in}{2.758940in}}%
\pgfusepath{clip}%
\pgfsetbuttcap%
\pgfsetroundjoin%
\definecolor{currentfill}{rgb}{0.566266,0.637515,0.754216}%
\pgfsetfillcolor{currentfill}%
\pgfsetlinewidth{0.602250pt}%
\definecolor{currentstroke}{rgb}{0.296471,0.296471,0.296471}%
\pgfsetstrokecolor{currentstroke}%
\pgfsetdash{}{0pt}%
\pgfpathmoveto{\pgfqpoint{0.542737in}{0.647994in}}%
\pgfpathlineto{\pgfqpoint{0.544610in}{0.647994in}}%
\pgfpathlineto{\pgfqpoint{0.544610in}{0.608580in}}%
\pgfpathlineto{\pgfqpoint{0.542737in}{0.608580in}}%
\pgfpathlineto{\pgfqpoint{0.542737in}{0.647994in}}%
\pgfpathclose%
\pgfusepath{stroke,fill}%
\end{pgfscope}%
\begin{pgfscope}%
\pgfpathrectangle{\pgfqpoint{0.703330in}{0.352393in}}{\pgfqpoint{2.173234in}{2.758940in}}%
\pgfusepath{clip}%
\pgfsetbuttcap%
\pgfsetroundjoin%
\definecolor{currentfill}{rgb}{0.468322,0.557674,0.703709}%
\pgfsetfillcolor{currentfill}%
\pgfsetlinewidth{0.602250pt}%
\definecolor{currentstroke}{rgb}{0.296471,0.296471,0.296471}%
\pgfsetstrokecolor{currentstroke}%
\pgfsetdash{}{0pt}%
\pgfpathmoveto{\pgfqpoint{0.544610in}{0.667700in}}%
\pgfpathlineto{\pgfqpoint{0.548099in}{0.667700in}}%
\pgfpathlineto{\pgfqpoint{0.548099in}{0.588874in}}%
\pgfpathlineto{\pgfqpoint{0.544610in}{0.588874in}}%
\pgfpathlineto{\pgfqpoint{0.544610in}{0.667700in}}%
\pgfpathclose%
\pgfusepath{stroke,fill}%
\end{pgfscope}%
\begin{pgfscope}%
\pgfpathrectangle{\pgfqpoint{0.703330in}{0.352393in}}{\pgfqpoint{2.173234in}{2.758940in}}%
\pgfusepath{clip}%
\pgfsetbuttcap%
\pgfsetroundjoin%
\definecolor{currentfill}{rgb}{0.347059,0.458824,0.641176}%
\pgfsetfillcolor{currentfill}%
\pgfsetlinewidth{0.602250pt}%
\definecolor{currentstroke}{rgb}{0.296471,0.296471,0.296471}%
\pgfsetstrokecolor{currentstroke}%
\pgfsetdash{}{0pt}%
\pgfpathmoveto{\pgfqpoint{0.548099in}{0.707114in}}%
\pgfpathlineto{\pgfqpoint{1.806553in}{0.707114in}}%
\pgfpathlineto{\pgfqpoint{1.806553in}{0.549460in}}%
\pgfpathlineto{\pgfqpoint{0.548099in}{0.549460in}}%
\pgfpathlineto{\pgfqpoint{0.548099in}{0.707114in}}%
\pgfpathclose%
\pgfusepath{stroke,fill}%
\end{pgfscope}%
\begin{pgfscope}%
\pgfpathrectangle{\pgfqpoint{0.703330in}{0.352393in}}{\pgfqpoint{2.173234in}{2.758940in}}%
\pgfusepath{clip}%
\pgfsetbuttcap%
\pgfsetroundjoin%
\definecolor{currentfill}{rgb}{0.468322,0.557674,0.703709}%
\pgfsetfillcolor{currentfill}%
\pgfsetlinewidth{0.602250pt}%
\definecolor{currentstroke}{rgb}{0.296471,0.296471,0.296471}%
\pgfsetstrokecolor{currentstroke}%
\pgfsetdash{}{0pt}%
\pgfpathmoveto{\pgfqpoint{1.806553in}{0.667700in}}%
\pgfpathlineto{\pgfqpoint{2.033437in}{0.667700in}}%
\pgfpathlineto{\pgfqpoint{2.033437in}{0.588874in}}%
\pgfpathlineto{\pgfqpoint{1.806553in}{0.588874in}}%
\pgfpathlineto{\pgfqpoint{1.806553in}{0.667700in}}%
\pgfpathclose%
\pgfusepath{stroke,fill}%
\end{pgfscope}%
\begin{pgfscope}%
\pgfpathrectangle{\pgfqpoint{0.703330in}{0.352393in}}{\pgfqpoint{2.173234in}{2.758940in}}%
\pgfusepath{clip}%
\pgfsetbuttcap%
\pgfsetroundjoin%
\definecolor{currentfill}{rgb}{0.566266,0.637515,0.754216}%
\pgfsetfillcolor{currentfill}%
\pgfsetlinewidth{0.602250pt}%
\definecolor{currentstroke}{rgb}{0.296471,0.296471,0.296471}%
\pgfsetstrokecolor{currentstroke}%
\pgfsetdash{}{0pt}%
\pgfpathmoveto{\pgfqpoint{2.033437in}{0.647994in}}%
\pgfpathlineto{\pgfqpoint{2.148429in}{0.647994in}}%
\pgfpathlineto{\pgfqpoint{2.148429in}{0.608580in}}%
\pgfpathlineto{\pgfqpoint{2.033437in}{0.608580in}}%
\pgfpathlineto{\pgfqpoint{2.033437in}{0.647994in}}%
\pgfpathclose%
\pgfusepath{stroke,fill}%
\end{pgfscope}%
\begin{pgfscope}%
\pgfpathrectangle{\pgfqpoint{0.703330in}{0.352393in}}{\pgfqpoint{2.173234in}{2.758940in}}%
\pgfusepath{clip}%
\pgfsetbuttcap%
\pgfsetroundjoin%
\definecolor{currentfill}{rgb}{0.643221,0.700246,0.793900}%
\pgfsetfillcolor{currentfill}%
\pgfsetlinewidth{0.602250pt}%
\definecolor{currentstroke}{rgb}{0.296471,0.296471,0.296471}%
\pgfsetstrokecolor{currentstroke}%
\pgfsetdash{}{0pt}%
\pgfpathmoveto{\pgfqpoint{2.148429in}{0.638140in}}%
\pgfpathlineto{\pgfqpoint{2.229597in}{0.638140in}}%
\pgfpathlineto{\pgfqpoint{2.229597in}{0.618434in}}%
\pgfpathlineto{\pgfqpoint{2.148429in}{0.618434in}}%
\pgfpathlineto{\pgfqpoint{2.148429in}{0.638140in}}%
\pgfpathclose%
\pgfusepath{stroke,fill}%
\end{pgfscope}%
\begin{pgfscope}%
\pgfpathrectangle{\pgfqpoint{0.703330in}{0.352393in}}{\pgfqpoint{2.173234in}{2.758940in}}%
\pgfusepath{clip}%
\pgfsetbuttcap%
\pgfsetroundjoin%
\definecolor{currentfill}{rgb}{0.706185,0.751573,0.826368}%
\pgfsetfillcolor{currentfill}%
\pgfsetlinewidth{0.602250pt}%
\definecolor{currentstroke}{rgb}{0.296471,0.296471,0.296471}%
\pgfsetstrokecolor{currentstroke}%
\pgfsetdash{}{0pt}%
\pgfpathmoveto{\pgfqpoint{2.229597in}{0.633214in}}%
\pgfpathlineto{\pgfqpoint{2.278119in}{0.633214in}}%
\pgfpathlineto{\pgfqpoint{2.278119in}{0.623360in}}%
\pgfpathlineto{\pgfqpoint{2.229597in}{0.623360in}}%
\pgfpathlineto{\pgfqpoint{2.229597in}{0.633214in}}%
\pgfpathclose%
\pgfusepath{stroke,fill}%
\end{pgfscope}%
\begin{pgfscope}%
\pgfpathrectangle{\pgfqpoint{0.703330in}{0.352393in}}{\pgfqpoint{2.173234in}{2.758940in}}%
\pgfusepath{clip}%
\pgfsetbuttcap%
\pgfsetroundjoin%
\definecolor{currentfill}{rgb}{0.755157,0.791493,0.851622}%
\pgfsetfillcolor{currentfill}%
\pgfsetlinewidth{0.602250pt}%
\definecolor{currentstroke}{rgb}{0.296471,0.296471,0.296471}%
\pgfsetstrokecolor{currentstroke}%
\pgfsetdash{}{0pt}%
\pgfpathmoveto{\pgfqpoint{2.278119in}{0.630750in}}%
\pgfpathlineto{\pgfqpoint{2.374212in}{0.630750in}}%
\pgfpathlineto{\pgfqpoint{2.374212in}{0.625824in}}%
\pgfpathlineto{\pgfqpoint{2.278119in}{0.625824in}}%
\pgfpathlineto{\pgfqpoint{2.278119in}{0.630750in}}%
\pgfpathclose%
\pgfusepath{stroke,fill}%
\end{pgfscope}%
\begin{pgfscope}%
\pgfpathrectangle{\pgfqpoint{0.703330in}{0.352393in}}{\pgfqpoint{2.173234in}{2.758940in}}%
\pgfusepath{clip}%
\pgfsetbuttcap%
\pgfsetroundjoin%
\definecolor{currentfill}{rgb}{0.792469,0.821908,0.870863}%
\pgfsetfillcolor{currentfill}%
\pgfsetlinewidth{0.602250pt}%
\definecolor{currentstroke}{rgb}{0.296471,0.296471,0.296471}%
\pgfsetstrokecolor{currentstroke}%
\pgfsetdash{}{0pt}%
\pgfpathmoveto{\pgfqpoint{2.374212in}{0.629519in}}%
\pgfpathlineto{\pgfqpoint{2.476434in}{0.629519in}}%
\pgfpathlineto{\pgfqpoint{2.476434in}{0.627055in}}%
\pgfpathlineto{\pgfqpoint{2.374212in}{0.627055in}}%
\pgfpathlineto{\pgfqpoint{2.374212in}{0.629519in}}%
\pgfpathclose%
\pgfusepath{stroke,fill}%
\end{pgfscope}%
\begin{pgfscope}%
\pgfpathrectangle{\pgfqpoint{0.703330in}{0.352393in}}{\pgfqpoint{2.173234in}{2.758940in}}%
\pgfusepath{clip}%
\pgfsetbuttcap%
\pgfsetroundjoin%
\definecolor{currentfill}{rgb}{0.825117,0.848522,0.887698}%
\pgfsetfillcolor{currentfill}%
\pgfsetlinewidth{0.602250pt}%
\definecolor{currentstroke}{rgb}{0.296471,0.296471,0.296471}%
\pgfsetstrokecolor{currentstroke}%
\pgfsetdash{}{0pt}%
\pgfpathmoveto{\pgfqpoint{2.476434in}{0.628903in}}%
\pgfpathlineto{\pgfqpoint{2.489583in}{0.628903in}}%
\pgfpathlineto{\pgfqpoint{2.489583in}{0.627671in}}%
\pgfpathlineto{\pgfqpoint{2.476434in}{0.627671in}}%
\pgfpathlineto{\pgfqpoint{2.476434in}{0.628903in}}%
\pgfpathclose%
\pgfusepath{stroke,fill}%
\end{pgfscope}%
\begin{pgfscope}%
\pgfpathrectangle{\pgfqpoint{0.703330in}{0.352393in}}{\pgfqpoint{2.173234in}{2.758940in}}%
\pgfusepath{clip}%
\pgfsetbuttcap%
\pgfsetroundjoin%
\definecolor{currentfill}{rgb}{0.848437,0.867532,0.899724}%
\pgfsetfillcolor{currentfill}%
\pgfsetlinewidth{0.602250pt}%
\definecolor{currentstroke}{rgb}{0.296471,0.296471,0.296471}%
\pgfsetstrokecolor{currentstroke}%
\pgfsetdash{}{0pt}%
\pgfpathmoveto{\pgfqpoint{2.489583in}{0.628595in}}%
\pgfpathlineto{\pgfqpoint{2.502798in}{0.628595in}}%
\pgfpathlineto{\pgfqpoint{2.502798in}{0.627979in}}%
\pgfpathlineto{\pgfqpoint{2.489583in}{0.627979in}}%
\pgfpathlineto{\pgfqpoint{2.489583in}{0.628595in}}%
\pgfpathclose%
\pgfusepath{stroke,fill}%
\end{pgfscope}%
\begin{pgfscope}%
\pgfpathrectangle{\pgfqpoint{0.703330in}{0.352393in}}{\pgfqpoint{2.173234in}{2.758940in}}%
\pgfusepath{clip}%
\pgfsetbuttcap%
\pgfsetroundjoin%
\pgfsetlinewidth{0.803000pt}%
\definecolor{currentstroke}{rgb}{0.450000,0.450000,0.450000}%
\pgfsetstrokecolor{currentstroke}%
\pgfsetdash{}{0pt}%
\pgfpathmoveto{\pgfqpoint{0.000000in}{-0.034722in}}%
\pgfpathcurveto{\pgfqpoint{0.009208in}{-0.034722in}}{\pgfqpoint{0.018041in}{-0.031064in}}{\pgfqpoint{0.024552in}{-0.024552in}}%
\pgfpathcurveto{\pgfqpoint{0.031064in}{-0.018041in}}{\pgfqpoint{0.034722in}{-0.009208in}}{\pgfqpoint{0.034722in}{0.000000in}}%
\pgfpathcurveto{\pgfqpoint{0.034722in}{0.009208in}}{\pgfqpoint{0.031064in}{0.018041in}}{\pgfqpoint{0.024552in}{0.024552in}}%
\pgfpathcurveto{\pgfqpoint{0.018041in}{0.031064in}}{\pgfqpoint{0.009208in}{0.034722in}}{\pgfqpoint{0.000000in}{0.034722in}}%
\pgfpathcurveto{\pgfqpoint{-0.009208in}{0.034722in}}{\pgfqpoint{-0.018041in}{0.031064in}}{\pgfqpoint{-0.024552in}{0.024552in}}%
\pgfpathcurveto{\pgfqpoint{-0.031064in}{0.018041in}}{\pgfqpoint{-0.034722in}{0.009208in}}{\pgfqpoint{-0.034722in}{0.000000in}}%
\pgfpathcurveto{\pgfqpoint{-0.034722in}{-0.009208in}}{\pgfqpoint{-0.031064in}{-0.018041in}}{\pgfqpoint{-0.024552in}{-0.024552in}}%
\pgfpathcurveto{\pgfqpoint{-0.018041in}{-0.031064in}}{\pgfqpoint{-0.009208in}{-0.034722in}}{\pgfqpoint{0.000000in}{-0.034722in}}%
\pgfusepath{stroke}%
\end{pgfscope}%
\begin{pgfscope}%
\pgfpathrectangle{\pgfqpoint{0.703330in}{0.352393in}}{\pgfqpoint{2.173234in}{2.758940in}}%
\pgfusepath{clip}%
\pgfsetbuttcap%
\pgfsetroundjoin%
\definecolor{currentfill}{rgb}{0.919097,0.862812,0.832112}%
\pgfsetfillcolor{currentfill}%
\pgfsetlinewidth{0.602250pt}%
\definecolor{currentstroke}{rgb}{0.296471,0.296471,0.296471}%
\pgfsetstrokecolor{currentstroke}%
\pgfsetdash{}{0pt}%
\pgfpathmoveto{\pgfqpoint{1.013792in}{0.471249in}}%
\pgfpathlineto{\pgfqpoint{1.013792in}{0.471249in}}%
\pgfpathlineto{\pgfqpoint{1.013792in}{0.470017in}}%
\pgfpathlineto{\pgfqpoint{1.013792in}{0.470017in}}%
\pgfpathlineto{\pgfqpoint{1.013792in}{0.471249in}}%
\pgfpathclose%
\pgfusepath{stroke,fill}%
\end{pgfscope}%
\begin{pgfscope}%
\pgfpathrectangle{\pgfqpoint{0.703330in}{0.352393in}}{\pgfqpoint{2.173234in}{2.758940in}}%
\pgfusepath{clip}%
\pgfsetbuttcap%
\pgfsetroundjoin%
\definecolor{currentfill}{rgb}{0.910863,0.840546,0.801899}%
\pgfsetfillcolor{currentfill}%
\pgfsetlinewidth{0.602250pt}%
\definecolor{currentstroke}{rgb}{0.296471,0.296471,0.296471}%
\pgfsetstrokecolor{currentstroke}%
\pgfsetdash{}{0pt}%
\pgfpathmoveto{\pgfqpoint{1.013792in}{0.471865in}}%
\pgfpathlineto{\pgfqpoint{1.013792in}{0.471865in}}%
\pgfpathlineto{\pgfqpoint{1.013792in}{0.469402in}}%
\pgfpathlineto{\pgfqpoint{1.013792in}{0.469402in}}%
\pgfpathlineto{\pgfqpoint{1.013792in}{0.471865in}}%
\pgfpathclose%
\pgfusepath{stroke,fill}%
\end{pgfscope}%
\begin{pgfscope}%
\pgfpathrectangle{\pgfqpoint{0.703330in}{0.352393in}}{\pgfqpoint{2.173234in}{2.758940in}}%
\pgfusepath{clip}%
\pgfsetbuttcap%
\pgfsetroundjoin%
\definecolor{currentfill}{rgb}{0.901453,0.815098,0.767370}%
\pgfsetfillcolor{currentfill}%
\pgfsetlinewidth{0.602250pt}%
\definecolor{currentstroke}{rgb}{0.296471,0.296471,0.296471}%
\pgfsetstrokecolor{currentstroke}%
\pgfsetdash{}{0pt}%
\pgfpathmoveto{\pgfqpoint{1.013792in}{0.473097in}}%
\pgfpathlineto{\pgfqpoint{1.013792in}{0.473097in}}%
\pgfpathlineto{\pgfqpoint{1.013792in}{0.468170in}}%
\pgfpathlineto{\pgfqpoint{1.013792in}{0.468170in}}%
\pgfpathlineto{\pgfqpoint{1.013792in}{0.473097in}}%
\pgfpathclose%
\pgfusepath{stroke,fill}%
\end{pgfscope}%
\begin{pgfscope}%
\pgfpathrectangle{\pgfqpoint{0.703330in}{0.352393in}}{\pgfqpoint{2.173234in}{2.758940in}}%
\pgfusepath{clip}%
\pgfsetbuttcap%
\pgfsetroundjoin%
\definecolor{currentfill}{rgb}{0.889102,0.781698,0.722050}%
\pgfsetfillcolor{currentfill}%
\pgfsetlinewidth{0.602250pt}%
\definecolor{currentstroke}{rgb}{0.296471,0.296471,0.296471}%
\pgfsetstrokecolor{currentstroke}%
\pgfsetdash{}{0pt}%
\pgfpathmoveto{\pgfqpoint{1.013792in}{0.475560in}}%
\pgfpathlineto{\pgfqpoint{1.013792in}{0.475560in}}%
\pgfpathlineto{\pgfqpoint{1.013792in}{0.465707in}}%
\pgfpathlineto{\pgfqpoint{1.013792in}{0.465707in}}%
\pgfpathlineto{\pgfqpoint{1.013792in}{0.475560in}}%
\pgfpathclose%
\pgfusepath{stroke,fill}%
\end{pgfscope}%
\begin{pgfscope}%
\pgfpathrectangle{\pgfqpoint{0.703330in}{0.352393in}}{\pgfqpoint{2.173234in}{2.758940in}}%
\pgfusepath{clip}%
\pgfsetbuttcap%
\pgfsetroundjoin%
\definecolor{currentfill}{rgb}{0.873223,0.738755,0.663782}%
\pgfsetfillcolor{currentfill}%
\pgfsetlinewidth{0.602250pt}%
\definecolor{currentstroke}{rgb}{0.296471,0.296471,0.296471}%
\pgfsetstrokecolor{currentstroke}%
\pgfsetdash{}{0pt}%
\pgfpathmoveto{\pgfqpoint{1.013792in}{0.480487in}}%
\pgfpathlineto{\pgfqpoint{1.013792in}{0.480487in}}%
\pgfpathlineto{\pgfqpoint{1.013792in}{0.460780in}}%
\pgfpathlineto{\pgfqpoint{1.013792in}{0.460780in}}%
\pgfpathlineto{\pgfqpoint{1.013792in}{0.480487in}}%
\pgfpathclose%
\pgfusepath{stroke,fill}%
\end{pgfscope}%
\begin{pgfscope}%
\pgfpathrectangle{\pgfqpoint{0.703330in}{0.352393in}}{\pgfqpoint{2.173234in}{2.758940in}}%
\pgfusepath{clip}%
\pgfsetbuttcap%
\pgfsetroundjoin%
\definecolor{currentfill}{rgb}{0.853814,0.686269,0.592565}%
\pgfsetfillcolor{currentfill}%
\pgfsetlinewidth{0.602250pt}%
\definecolor{currentstroke}{rgb}{0.296471,0.296471,0.296471}%
\pgfsetstrokecolor{currentstroke}%
\pgfsetdash{}{0pt}%
\pgfpathmoveto{\pgfqpoint{1.013792in}{0.490340in}}%
\pgfpathlineto{\pgfqpoint{1.013792in}{0.490340in}}%
\pgfpathlineto{\pgfqpoint{1.013792in}{0.450927in}}%
\pgfpathlineto{\pgfqpoint{1.013792in}{0.450927in}}%
\pgfpathlineto{\pgfqpoint{1.013792in}{0.490340in}}%
\pgfpathclose%
\pgfusepath{stroke,fill}%
\end{pgfscope}%
\begin{pgfscope}%
\pgfpathrectangle{\pgfqpoint{0.703330in}{0.352393in}}{\pgfqpoint{2.173234in}{2.758940in}}%
\pgfusepath{clip}%
\pgfsetbuttcap%
\pgfsetroundjoin%
\definecolor{currentfill}{rgb}{0.829112,0.619469,0.501926}%
\pgfsetfillcolor{currentfill}%
\pgfsetlinewidth{0.602250pt}%
\definecolor{currentstroke}{rgb}{0.296471,0.296471,0.296471}%
\pgfsetstrokecolor{currentstroke}%
\pgfsetdash{}{0pt}%
\pgfpathmoveto{\pgfqpoint{1.013792in}{0.510047in}}%
\pgfpathlineto{\pgfqpoint{1.371359in}{0.510047in}}%
\pgfpathlineto{\pgfqpoint{1.371359in}{0.431220in}}%
\pgfpathlineto{\pgfqpoint{1.013792in}{0.431220in}}%
\pgfpathlineto{\pgfqpoint{1.013792in}{0.510047in}}%
\pgfpathclose%
\pgfusepath{stroke,fill}%
\end{pgfscope}%
\begin{pgfscope}%
\pgfpathrectangle{\pgfqpoint{0.703330in}{0.352393in}}{\pgfqpoint{2.173234in}{2.758940in}}%
\pgfusepath{clip}%
\pgfsetbuttcap%
\pgfsetroundjoin%
\definecolor{currentfill}{rgb}{0.798529,0.536765,0.389706}%
\pgfsetfillcolor{currentfill}%
\pgfsetlinewidth{0.602250pt}%
\definecolor{currentstroke}{rgb}{0.296471,0.296471,0.296471}%
\pgfsetstrokecolor{currentstroke}%
\pgfsetdash{}{0pt}%
\pgfpathmoveto{\pgfqpoint{1.371359in}{0.549460in}}%
\pgfpathlineto{\pgfqpoint{2.236499in}{0.549460in}}%
\pgfpathlineto{\pgfqpoint{2.236499in}{0.391806in}}%
\pgfpathlineto{\pgfqpoint{1.371359in}{0.391806in}}%
\pgfpathlineto{\pgfqpoint{1.371359in}{0.549460in}}%
\pgfpathclose%
\pgfusepath{stroke,fill}%
\end{pgfscope}%
\begin{pgfscope}%
\pgfpathrectangle{\pgfqpoint{0.703330in}{0.352393in}}{\pgfqpoint{2.173234in}{2.758940in}}%
\pgfusepath{clip}%
\pgfsetbuttcap%
\pgfsetroundjoin%
\definecolor{currentfill}{rgb}{0.829112,0.619469,0.501926}%
\pgfsetfillcolor{currentfill}%
\pgfsetlinewidth{0.602250pt}%
\definecolor{currentstroke}{rgb}{0.296471,0.296471,0.296471}%
\pgfsetstrokecolor{currentstroke}%
\pgfsetdash{}{0pt}%
\pgfpathmoveto{\pgfqpoint{2.236499in}{0.510047in}}%
\pgfpathlineto{\pgfqpoint{2.437599in}{0.510047in}}%
\pgfpathlineto{\pgfqpoint{2.437599in}{0.431220in}}%
\pgfpathlineto{\pgfqpoint{2.236499in}{0.431220in}}%
\pgfpathlineto{\pgfqpoint{2.236499in}{0.510047in}}%
\pgfpathclose%
\pgfusepath{stroke,fill}%
\end{pgfscope}%
\begin{pgfscope}%
\pgfpathrectangle{\pgfqpoint{0.703330in}{0.352393in}}{\pgfqpoint{2.173234in}{2.758940in}}%
\pgfusepath{clip}%
\pgfsetbuttcap%
\pgfsetroundjoin%
\definecolor{currentfill}{rgb}{0.853814,0.686269,0.592565}%
\pgfsetfillcolor{currentfill}%
\pgfsetlinewidth{0.602250pt}%
\definecolor{currentstroke}{rgb}{0.296471,0.296471,0.296471}%
\pgfsetstrokecolor{currentstroke}%
\pgfsetdash{}{0pt}%
\pgfpathmoveto{\pgfqpoint{2.437599in}{0.490340in}}%
\pgfpathlineto{\pgfqpoint{2.499897in}{0.490340in}}%
\pgfpathlineto{\pgfqpoint{2.499897in}{0.450927in}}%
\pgfpathlineto{\pgfqpoint{2.437599in}{0.450927in}}%
\pgfpathlineto{\pgfqpoint{2.437599in}{0.490340in}}%
\pgfpathclose%
\pgfusepath{stroke,fill}%
\end{pgfscope}%
\begin{pgfscope}%
\pgfpathrectangle{\pgfqpoint{0.703330in}{0.352393in}}{\pgfqpoint{2.173234in}{2.758940in}}%
\pgfusepath{clip}%
\pgfsetbuttcap%
\pgfsetroundjoin%
\definecolor{currentfill}{rgb}{0.873223,0.738755,0.663782}%
\pgfsetfillcolor{currentfill}%
\pgfsetlinewidth{0.602250pt}%
\definecolor{currentstroke}{rgb}{0.296471,0.296471,0.296471}%
\pgfsetstrokecolor{currentstroke}%
\pgfsetdash{}{0pt}%
\pgfpathmoveto{\pgfqpoint{2.499897in}{0.480487in}}%
\pgfpathlineto{\pgfqpoint{2.564255in}{0.480487in}}%
\pgfpathlineto{\pgfqpoint{2.564255in}{0.460780in}}%
\pgfpathlineto{\pgfqpoint{2.499897in}{0.460780in}}%
\pgfpathlineto{\pgfqpoint{2.499897in}{0.480487in}}%
\pgfpathclose%
\pgfusepath{stroke,fill}%
\end{pgfscope}%
\begin{pgfscope}%
\pgfpathrectangle{\pgfqpoint{0.703330in}{0.352393in}}{\pgfqpoint{2.173234in}{2.758940in}}%
\pgfusepath{clip}%
\pgfsetbuttcap%
\pgfsetroundjoin%
\definecolor{currentfill}{rgb}{0.889102,0.781698,0.722050}%
\pgfsetfillcolor{currentfill}%
\pgfsetlinewidth{0.602250pt}%
\definecolor{currentstroke}{rgb}{0.296471,0.296471,0.296471}%
\pgfsetstrokecolor{currentstroke}%
\pgfsetdash{}{0pt}%
\pgfpathmoveto{\pgfqpoint{2.564255in}{0.475560in}}%
\pgfpathlineto{\pgfqpoint{2.586873in}{0.475560in}}%
\pgfpathlineto{\pgfqpoint{2.586873in}{0.465707in}}%
\pgfpathlineto{\pgfqpoint{2.564255in}{0.465707in}}%
\pgfpathlineto{\pgfqpoint{2.564255in}{0.475560in}}%
\pgfpathclose%
\pgfusepath{stroke,fill}%
\end{pgfscope}%
\begin{pgfscope}%
\pgfpathrectangle{\pgfqpoint{0.703330in}{0.352393in}}{\pgfqpoint{2.173234in}{2.758940in}}%
\pgfusepath{clip}%
\pgfsetbuttcap%
\pgfsetroundjoin%
\definecolor{currentfill}{rgb}{0.901453,0.815098,0.767370}%
\pgfsetfillcolor{currentfill}%
\pgfsetlinewidth{0.602250pt}%
\definecolor{currentstroke}{rgb}{0.296471,0.296471,0.296471}%
\pgfsetstrokecolor{currentstroke}%
\pgfsetdash{}{0pt}%
\pgfpathmoveto{\pgfqpoint{2.586873in}{0.473097in}}%
\pgfpathlineto{\pgfqpoint{2.623690in}{0.473097in}}%
\pgfpathlineto{\pgfqpoint{2.623690in}{0.468170in}}%
\pgfpathlineto{\pgfqpoint{2.586873in}{0.468170in}}%
\pgfpathlineto{\pgfqpoint{2.586873in}{0.473097in}}%
\pgfpathclose%
\pgfusepath{stroke,fill}%
\end{pgfscope}%
\begin{pgfscope}%
\pgfpathrectangle{\pgfqpoint{0.703330in}{0.352393in}}{\pgfqpoint{2.173234in}{2.758940in}}%
\pgfusepath{clip}%
\pgfsetbuttcap%
\pgfsetroundjoin%
\definecolor{currentfill}{rgb}{0.910863,0.840546,0.801899}%
\pgfsetfillcolor{currentfill}%
\pgfsetlinewidth{0.602250pt}%
\definecolor{currentstroke}{rgb}{0.296471,0.296471,0.296471}%
\pgfsetstrokecolor{currentstroke}%
\pgfsetdash{}{0pt}%
\pgfpathmoveto{\pgfqpoint{2.623690in}{0.471865in}}%
\pgfpathlineto{\pgfqpoint{2.659760in}{0.471865in}}%
\pgfpathlineto{\pgfqpoint{2.659760in}{0.469402in}}%
\pgfpathlineto{\pgfqpoint{2.623690in}{0.469402in}}%
\pgfpathlineto{\pgfqpoint{2.623690in}{0.471865in}}%
\pgfpathclose%
\pgfusepath{stroke,fill}%
\end{pgfscope}%
\begin{pgfscope}%
\pgfpathrectangle{\pgfqpoint{0.703330in}{0.352393in}}{\pgfqpoint{2.173234in}{2.758940in}}%
\pgfusepath{clip}%
\pgfsetbuttcap%
\pgfsetroundjoin%
\definecolor{currentfill}{rgb}{0.919097,0.862812,0.832112}%
\pgfsetfillcolor{currentfill}%
\pgfsetlinewidth{0.602250pt}%
\definecolor{currentstroke}{rgb}{0.296471,0.296471,0.296471}%
\pgfsetstrokecolor{currentstroke}%
\pgfsetdash{}{0pt}%
\pgfpathmoveto{\pgfqpoint{2.659760in}{0.471249in}}%
\pgfpathlineto{\pgfqpoint{2.678363in}{0.471249in}}%
\pgfpathlineto{\pgfqpoint{2.678363in}{0.470017in}}%
\pgfpathlineto{\pgfqpoint{2.659760in}{0.470017in}}%
\pgfpathlineto{\pgfqpoint{2.659760in}{0.471249in}}%
\pgfpathclose%
\pgfusepath{stroke,fill}%
\end{pgfscope}%
\begin{pgfscope}%
\pgfpathrectangle{\pgfqpoint{0.703330in}{0.352393in}}{\pgfqpoint{2.173234in}{2.758940in}}%
\pgfusepath{clip}%
\pgfsetbuttcap%
\pgfsetroundjoin%
\pgfsetlinewidth{0.803000pt}%
\definecolor{currentstroke}{rgb}{0.450000,0.450000,0.450000}%
\pgfsetstrokecolor{currentstroke}%
\pgfsetdash{}{0pt}%
\pgfpathmoveto{\pgfqpoint{0.000000in}{-0.034722in}}%
\pgfpathcurveto{\pgfqpoint{0.009208in}{-0.034722in}}{\pgfqpoint{0.018041in}{-0.031064in}}{\pgfqpoint{0.024552in}{-0.024552in}}%
\pgfpathcurveto{\pgfqpoint{0.031064in}{-0.018041in}}{\pgfqpoint{0.034722in}{-0.009208in}}{\pgfqpoint{0.034722in}{0.000000in}}%
\pgfpathcurveto{\pgfqpoint{0.034722in}{0.009208in}}{\pgfqpoint{0.031064in}{0.018041in}}{\pgfqpoint{0.024552in}{0.024552in}}%
\pgfpathcurveto{\pgfqpoint{0.018041in}{0.031064in}}{\pgfqpoint{0.009208in}{0.034722in}}{\pgfqpoint{0.000000in}{0.034722in}}%
\pgfpathcurveto{\pgfqpoint{-0.009208in}{0.034722in}}{\pgfqpoint{-0.018041in}{0.031064in}}{\pgfqpoint{-0.024552in}{0.024552in}}%
\pgfpathcurveto{\pgfqpoint{-0.031064in}{0.018041in}}{\pgfqpoint{-0.034722in}{0.009208in}}{\pgfqpoint{-0.034722in}{0.000000in}}%
\pgfpathcurveto{\pgfqpoint{-0.034722in}{-0.009208in}}{\pgfqpoint{-0.031064in}{-0.018041in}}{\pgfqpoint{-0.024552in}{-0.024552in}}%
\pgfpathcurveto{\pgfqpoint{-0.018041in}{-0.031064in}}{\pgfqpoint{-0.009208in}{-0.034722in}}{\pgfqpoint{0.000000in}{-0.034722in}}%
\pgfusepath{stroke}%
\end{pgfscope}%
\begin{pgfscope}%
\pgfpathrectangle{\pgfqpoint{0.703330in}{0.352393in}}{\pgfqpoint{2.173234in}{2.758940in}}%
\pgfusepath{clip}%
\pgfsetbuttcap%
\pgfsetmiterjoin%
\definecolor{currentfill}{rgb}{0.347059,0.458824,0.641176}%
\pgfsetfillcolor{currentfill}%
\pgfsetlinewidth{0.602250pt}%
\definecolor{currentstroke}{rgb}{0.296471,0.296471,0.296471}%
\pgfsetstrokecolor{currentstroke}%
\pgfsetdash{}{0pt}%
\pgfpathmoveto{\pgfqpoint{-0.228056in}{2.914266in}}%
\pgfpathlineto{\pgfqpoint{-0.228056in}{2.914266in}}%
\pgfpathlineto{\pgfqpoint{-0.228056in}{2.914266in}}%
\pgfpathlineto{\pgfqpoint{-0.228056in}{2.914266in}}%
\pgfpathlineto{\pgfqpoint{-0.228056in}{2.914266in}}%
\pgfpathclose%
\pgfusepath{stroke,fill}%
\end{pgfscope}%
\begin{pgfscope}%
\pgfpathrectangle{\pgfqpoint{0.703330in}{0.352393in}}{\pgfqpoint{2.173234in}{2.758940in}}%
\pgfusepath{clip}%
\pgfsetbuttcap%
\pgfsetmiterjoin%
\definecolor{currentfill}{rgb}{0.798529,0.536765,0.389706}%
\pgfsetfillcolor{currentfill}%
\pgfsetlinewidth{0.602250pt}%
\definecolor{currentstroke}{rgb}{0.296471,0.296471,0.296471}%
\pgfsetstrokecolor{currentstroke}%
\pgfsetdash{}{0pt}%
\pgfpathmoveto{\pgfqpoint{-0.228056in}{2.914266in}}%
\pgfpathlineto{\pgfqpoint{-0.228056in}{2.914266in}}%
\pgfpathlineto{\pgfqpoint{-0.228056in}{2.914266in}}%
\pgfpathlineto{\pgfqpoint{-0.228056in}{2.914266in}}%
\pgfpathlineto{\pgfqpoint{-0.228056in}{2.914266in}}%
\pgfpathclose%
\pgfusepath{stroke,fill}%
\end{pgfscope}%
\begin{pgfscope}%
\pgfpathrectangle{\pgfqpoint{0.703330in}{0.352393in}}{\pgfqpoint{2.173234in}{2.758940in}}%
\pgfusepath{clip}%
\pgfsetbuttcap%
\pgfsetroundjoin%
\pgfsetlinewidth{0.752812pt}%
\definecolor{currentstroke}{rgb}{0.296471,0.296471,0.296471}%
\pgfsetstrokecolor{currentstroke}%
\pgfsetdash{}{0pt}%
\pgfpathmoveto{\pgfqpoint{2.650716in}{3.071920in}}%
\pgfpathlineto{\pgfqpoint{2.650716in}{2.914266in}}%
\pgfusepath{stroke}%
\end{pgfscope}%
\begin{pgfscope}%
\pgfpathrectangle{\pgfqpoint{0.703330in}{0.352393in}}{\pgfqpoint{2.173234in}{2.758940in}}%
\pgfusepath{clip}%
\pgfsetbuttcap%
\pgfsetroundjoin%
\pgfsetlinewidth{0.752812pt}%
\definecolor{currentstroke}{rgb}{0.296471,0.296471,0.296471}%
\pgfsetstrokecolor{currentstroke}%
\pgfsetdash{}{0pt}%
\pgfpathmoveto{\pgfqpoint{2.817685in}{2.914266in}}%
\pgfpathlineto{\pgfqpoint{2.817685in}{2.756612in}}%
\pgfusepath{stroke}%
\end{pgfscope}%
\begin{pgfscope}%
\pgfpathrectangle{\pgfqpoint{0.703330in}{0.352393in}}{\pgfqpoint{2.173234in}{2.758940in}}%
\pgfusepath{clip}%
\pgfsetbuttcap%
\pgfsetroundjoin%
\pgfsetlinewidth{0.752812pt}%
\definecolor{currentstroke}{rgb}{0.296471,0.296471,0.296471}%
\pgfsetstrokecolor{currentstroke}%
\pgfsetdash{}{0pt}%
\pgfpathmoveto{\pgfqpoint{1.998576in}{2.677786in}}%
\pgfpathlineto{\pgfqpoint{1.998576in}{2.520132in}}%
\pgfusepath{stroke}%
\end{pgfscope}%
\begin{pgfscope}%
\pgfpathrectangle{\pgfqpoint{0.703330in}{0.352393in}}{\pgfqpoint{2.173234in}{2.758940in}}%
\pgfusepath{clip}%
\pgfsetbuttcap%
\pgfsetroundjoin%
\pgfsetlinewidth{0.752812pt}%
\definecolor{currentstroke}{rgb}{0.296471,0.296471,0.296471}%
\pgfsetstrokecolor{currentstroke}%
\pgfsetdash{}{0pt}%
\pgfpathmoveto{\pgfqpoint{2.509504in}{2.520132in}}%
\pgfpathlineto{\pgfqpoint{2.509504in}{2.362478in}}%
\pgfusepath{stroke}%
\end{pgfscope}%
\begin{pgfscope}%
\pgfpathrectangle{\pgfqpoint{0.703330in}{0.352393in}}{\pgfqpoint{2.173234in}{2.758940in}}%
\pgfusepath{clip}%
\pgfsetbuttcap%
\pgfsetroundjoin%
\pgfsetlinewidth{0.752812pt}%
\definecolor{currentstroke}{rgb}{0.296471,0.296471,0.296471}%
\pgfsetstrokecolor{currentstroke}%
\pgfsetdash{}{0pt}%
\pgfpathmoveto{\pgfqpoint{2.029495in}{2.283651in}}%
\pgfpathlineto{\pgfqpoint{2.029495in}{2.125997in}}%
\pgfusepath{stroke}%
\end{pgfscope}%
\begin{pgfscope}%
\pgfpathrectangle{\pgfqpoint{0.703330in}{0.352393in}}{\pgfqpoint{2.173234in}{2.758940in}}%
\pgfusepath{clip}%
\pgfsetbuttcap%
\pgfsetroundjoin%
\pgfsetlinewidth{0.752812pt}%
\definecolor{currentstroke}{rgb}{0.296471,0.296471,0.296471}%
\pgfsetstrokecolor{currentstroke}%
\pgfsetdash{}{0pt}%
\pgfpathmoveto{\pgfqpoint{2.456756in}{2.125997in}}%
\pgfpathlineto{\pgfqpoint{2.456756in}{1.968344in}}%
\pgfusepath{stroke}%
\end{pgfscope}%
\begin{pgfscope}%
\pgfpathrectangle{\pgfqpoint{0.703330in}{0.352393in}}{\pgfqpoint{2.173234in}{2.758940in}}%
\pgfusepath{clip}%
\pgfsetbuttcap%
\pgfsetroundjoin%
\pgfsetlinewidth{0.752812pt}%
\definecolor{currentstroke}{rgb}{0.296471,0.296471,0.296471}%
\pgfsetstrokecolor{currentstroke}%
\pgfsetdash{}{0pt}%
\pgfusepath{stroke}%
\end{pgfscope}%
\begin{pgfscope}%
\pgfpathrectangle{\pgfqpoint{0.703330in}{0.352393in}}{\pgfqpoint{2.173234in}{2.758940in}}%
\pgfusepath{clip}%
\pgfsetbuttcap%
\pgfsetroundjoin%
\pgfsetlinewidth{0.752812pt}%
\definecolor{currentstroke}{rgb}{0.296471,0.296471,0.296471}%
\pgfsetstrokecolor{currentstroke}%
\pgfsetdash{}{0pt}%
\pgfpathmoveto{\pgfqpoint{2.799186in}{1.731863in}}%
\pgfpathlineto{\pgfqpoint{2.799186in}{1.574209in}}%
\pgfusepath{stroke}%
\end{pgfscope}%
\begin{pgfscope}%
\pgfpathrectangle{\pgfqpoint{0.703330in}{0.352393in}}{\pgfqpoint{2.173234in}{2.758940in}}%
\pgfusepath{clip}%
\pgfsetbuttcap%
\pgfsetroundjoin%
\pgfsetlinewidth{0.752812pt}%
\definecolor{currentstroke}{rgb}{0.296471,0.296471,0.296471}%
\pgfsetstrokecolor{currentstroke}%
\pgfsetdash{}{0pt}%
\pgfusepath{stroke}%
\end{pgfscope}%
\begin{pgfscope}%
\pgfpathrectangle{\pgfqpoint{0.703330in}{0.352393in}}{\pgfqpoint{2.173234in}{2.758940in}}%
\pgfusepath{clip}%
\pgfsetbuttcap%
\pgfsetroundjoin%
\pgfsetlinewidth{0.752812pt}%
\definecolor{currentstroke}{rgb}{0.296471,0.296471,0.296471}%
\pgfsetstrokecolor{currentstroke}%
\pgfsetdash{}{0pt}%
\pgfpathmoveto{\pgfqpoint{2.573674in}{1.337729in}}%
\pgfpathlineto{\pgfqpoint{2.573674in}{1.180075in}}%
\pgfusepath{stroke}%
\end{pgfscope}%
\begin{pgfscope}%
\pgfpathrectangle{\pgfqpoint{0.703330in}{0.352393in}}{\pgfqpoint{2.173234in}{2.758940in}}%
\pgfusepath{clip}%
\pgfsetbuttcap%
\pgfsetroundjoin%
\pgfsetlinewidth{0.752812pt}%
\definecolor{currentstroke}{rgb}{0.296471,0.296471,0.296471}%
\pgfsetstrokecolor{currentstroke}%
\pgfsetdash{}{0pt}%
\pgfusepath{stroke}%
\end{pgfscope}%
\begin{pgfscope}%
\pgfpathrectangle{\pgfqpoint{0.703330in}{0.352393in}}{\pgfqpoint{2.173234in}{2.758940in}}%
\pgfusepath{clip}%
\pgfsetbuttcap%
\pgfsetroundjoin%
\pgfsetlinewidth{0.752812pt}%
\definecolor{currentstroke}{rgb}{0.296471,0.296471,0.296471}%
\pgfsetstrokecolor{currentstroke}%
\pgfsetdash{}{0pt}%
\pgfpathmoveto{\pgfqpoint{2.555979in}{0.943594in}}%
\pgfpathlineto{\pgfqpoint{2.555979in}{0.785941in}}%
\pgfusepath{stroke}%
\end{pgfscope}%
\begin{pgfscope}%
\pgfpathrectangle{\pgfqpoint{0.703330in}{0.352393in}}{\pgfqpoint{2.173234in}{2.758940in}}%
\pgfusepath{clip}%
\pgfsetbuttcap%
\pgfsetroundjoin%
\pgfsetlinewidth{0.752812pt}%
\definecolor{currentstroke}{rgb}{0.296471,0.296471,0.296471}%
\pgfsetstrokecolor{currentstroke}%
\pgfsetdash{}{0pt}%
\pgfpathmoveto{\pgfqpoint{0.992962in}{0.707114in}}%
\pgfpathlineto{\pgfqpoint{0.992962in}{0.549460in}}%
\pgfusepath{stroke}%
\end{pgfscope}%
\begin{pgfscope}%
\pgfpathrectangle{\pgfqpoint{0.703330in}{0.352393in}}{\pgfqpoint{2.173234in}{2.758940in}}%
\pgfusepath{clip}%
\pgfsetbuttcap%
\pgfsetroundjoin%
\pgfsetlinewidth{0.752812pt}%
\definecolor{currentstroke}{rgb}{0.296471,0.296471,0.296471}%
\pgfsetstrokecolor{currentstroke}%
\pgfsetdash{}{0pt}%
\pgfpathmoveto{\pgfqpoint{1.631713in}{0.549460in}}%
\pgfpathlineto{\pgfqpoint{1.631713in}{0.391806in}}%
\pgfusepath{stroke}%
\end{pgfscope}%
\begin{pgfscope}%
\pgfsetrectcap%
\pgfsetmiterjoin%
\pgfsetlinewidth{1.003750pt}%
\definecolor{currentstroke}{rgb}{0.800000,0.800000,0.800000}%
\pgfsetstrokecolor{currentstroke}%
\pgfsetdash{}{0pt}%
\pgfpathmoveto{\pgfqpoint{0.703330in}{0.352393in}}%
\pgfpathlineto{\pgfqpoint{0.703330in}{3.111333in}}%
\pgfusepath{stroke}%
\end{pgfscope}%
\begin{pgfscope}%
\pgfsetrectcap%
\pgfsetmiterjoin%
\pgfsetlinewidth{1.003750pt}%
\definecolor{currentstroke}{rgb}{0.800000,0.800000,0.800000}%
\pgfsetstrokecolor{currentstroke}%
\pgfsetdash{}{0pt}%
\pgfpathmoveto{\pgfqpoint{0.703330in}{0.352393in}}%
\pgfpathlineto{\pgfqpoint{2.876564in}{0.352393in}}%
\pgfusepath{stroke}%
\end{pgfscope}%
\begin{pgfscope}%
\pgfsetbuttcap%
\pgfsetmiterjoin%
\definecolor{currentfill}{rgb}{1.000000,1.000000,1.000000}%
\pgfsetfillcolor{currentfill}%
\pgfsetfillopacity{0.800000}%
\pgfsetlinewidth{0.803000pt}%
\definecolor{currentstroke}{rgb}{0.800000,0.800000,0.800000}%
\pgfsetstrokecolor{currentstroke}%
\pgfsetstrokeopacity{0.800000}%
\pgfsetdash{}{0pt}%
\pgfpathmoveto{\pgfqpoint{0.776052in}{1.570368in}}%
\pgfpathlineto{\pgfqpoint{1.817018in}{1.570368in}}%
\pgfpathquadraticcurveto{\pgfqpoint{1.837796in}{1.570368in}}{\pgfqpoint{1.837796in}{1.591146in}}%
\pgfpathlineto{\pgfqpoint{1.837796in}{1.872580in}}%
\pgfpathquadraticcurveto{\pgfqpoint{1.837796in}{1.893358in}}{\pgfqpoint{1.817018in}{1.893358in}}%
\pgfpathlineto{\pgfqpoint{0.776052in}{1.893358in}}%
\pgfpathquadraticcurveto{\pgfqpoint{0.755274in}{1.893358in}}{\pgfqpoint{0.755274in}{1.872580in}}%
\pgfpathlineto{\pgfqpoint{0.755274in}{1.591146in}}%
\pgfpathquadraticcurveto{\pgfqpoint{0.755274in}{1.570368in}}{\pgfqpoint{0.776052in}{1.570368in}}%
\pgfpathlineto{\pgfqpoint{0.776052in}{1.570368in}}%
\pgfpathclose%
\pgfusepath{stroke,fill}%
\end{pgfscope}%
\begin{pgfscope}%
\pgfsetbuttcap%
\pgfsetmiterjoin%
\definecolor{currentfill}{rgb}{0.347059,0.458824,0.641176}%
\pgfsetfillcolor{currentfill}%
\pgfsetlinewidth{0.602250pt}%
\definecolor{currentstroke}{rgb}{0.296471,0.296471,0.296471}%
\pgfsetstrokecolor{currentstroke}%
\pgfsetdash{}{0pt}%
\pgfpathmoveto{\pgfqpoint{0.796830in}{1.779080in}}%
\pgfpathlineto{\pgfqpoint{1.004607in}{1.779080in}}%
\pgfpathlineto{\pgfqpoint{1.004607in}{1.851803in}}%
\pgfpathlineto{\pgfqpoint{0.796830in}{1.851803in}}%
\pgfpathlineto{\pgfqpoint{0.796830in}{1.779080in}}%
\pgfpathclose%
\pgfusepath{stroke,fill}%
\end{pgfscope}%
\begin{pgfscope}%
\definecolor{textcolor}{rgb}{0.150000,0.150000,0.150000}%
\pgfsetstrokecolor{textcolor}%
\pgfsetfillcolor{textcolor}%
\pgftext[x=1.087718in,y=1.779080in,left,base]{\color{textcolor}{\sffamily\fontsize{7.480000}{8.976000}\selectfont\catcode`\^=\active\def^{\ifmmode\sp\else\^{}\fi}\catcode`\%=\active\def%{\%}Form+meaning}}%
\end{pgfscope}%
\begin{pgfscope}%
\pgfsetbuttcap%
\pgfsetmiterjoin%
\definecolor{currentfill}{rgb}{0.798529,0.536765,0.389706}%
\pgfsetfillcolor{currentfill}%
\pgfsetlinewidth{0.602250pt}%
\definecolor{currentstroke}{rgb}{0.296471,0.296471,0.296471}%
\pgfsetstrokecolor{currentstroke}%
\pgfsetdash{}{0pt}%
\pgfpathmoveto{\pgfqpoint{0.796830in}{1.633117in}}%
\pgfpathlineto{\pgfqpoint{1.004607in}{1.633117in}}%
\pgfpathlineto{\pgfqpoint{1.004607in}{1.705839in}}%
\pgfpathlineto{\pgfqpoint{0.796830in}{1.705839in}}%
\pgfpathlineto{\pgfqpoint{0.796830in}{1.633117in}}%
\pgfpathclose%
\pgfusepath{stroke,fill}%
\end{pgfscope}%
\begin{pgfscope}%
\definecolor{textcolor}{rgb}{0.150000,0.150000,0.150000}%
\pgfsetstrokecolor{textcolor}%
\pgfsetfillcolor{textcolor}%
\pgftext[x=1.087718in,y=1.633117in,left,base]{\color{textcolor}{\sffamily\fontsize{7.480000}{8.976000}\selectfont\catcode`\^=\active\def^{\ifmmode\sp\else\^{}\fi}\catcode`\%=\active\def%{\%}Form-only}}%
\end{pgfscope}%
\end{pgfpicture}%
\makeatother%
\endgroup%

  \caption{Fuzzy $F_1$ scores for CSAR and baseline methods across procedural datasets.
    Results reported for form--meaning inventories and form-only inventories.}
  \unskip\label{fig:baseline}
\end{figure}

Below we describe the baseline methods we use for comparison.
\smallskip
\begin{description}[nosep,itemindent=-1em]
  \item[IBM Model 1]
    Simple expectation-maximization approach to machine translation primarily through aligning words in a sentence-parallel corpus. \citep{ibm-model-1}
  \item[IBM Model 3]
    Built on top of the IBM Model 1 to handle phenomena such as allowing a form to align to no meaning.
  \item[Morfessor]
    A form-only segmentation algorithm built to handle human language; also uses an EM algorithm.
  \item[Byte pair encoding]
    A greedy form-only tokenization method which recursively merges frequently occurring pairs of tokens.
    Vocabulary size is selected according to a simple heuristic (see \Cref{app:vocab-size}).
    \citep{gage1994bpe,sennrich-etal-2016-neural}
  \item[Unigram LM]
    An EM-based form-only tokenization method which starts with a large vocabulary and iteratively removes tokens contributing least to the likelihood of the data.
    Vocabulary size is selected according to a simple heuristic (see \Cref{app:vocab-size}).
    \citep{kudo-2018-subword}
  \item[Record]
    A trivial baseline where the inventory is just the set of all records.
\end{description}
\medskip
For the baseline methods which do not handle meanings and only produce forms, we report the form-only $F_1$ score (i.e., $s(i,g)$ only takes the form into account), though CSAR and IBM models still have access to meanings.
For form-only metrics, we exclude datasets which include noise forms as form-only methods cannot identify which forms are noise.

\paragraph{Results}
The results of CSAR and the baselines on the procedural datasets are presented in \cref{fig:baseline},
  which shows the distributions of mean scores for each hyperparameter setting for the procedural datasets. Each setting was repeated over $3$ random seeds. Additional results are given in \cref{app:proc-table}.
For inducing full morphemes (form and meaning), CSAR performs the best by a large margin over the baselines (and even greater margin when considering exact $F_1$).
The IBM alignment models perform better than the trivial record-based baseline but still perform noticeably worse than CSAR\@.
While CSAR yields roughly equal precision and recall, the IBM models' precision is lower than their recall suggesting that they are more prone to inducing spurious morphemes than CSAR\@.

When evaluating the forms only, we find that CSAR is the best method with Morfessor exhibiting comparable performance.
The IBM alignment models exhibit roughly the same performance as the tokenization methods (BPE and Unigram LM).
As with the full morpheme results, CSAR is the only method to achieve comparable precision and recall with all other baselines having precisions lower than their recalls.



\subsection{Error Analysis}
\unskip\label{sec:error-ana}
For the most part, the errors CSAR makes are ``edit errors'': identifying a correct morpheme but adding or removing a form or meaning token.
This is reflected in the near-parity between precision and recall.
This is in contrast to the baseline methods which are more prone to inducing too many morphemes, leading to lower precision.

Generally speaking, as more variations are added to a dataset, the performance degrades further.
In particular, CSAR's performance decreases the most with small corpus sizes, overlapping multi-token forms, and non-compositional mappings.
On the other hand, using sparse meanings, shuffling the forms, and using a non-uniform meaning distribution have relatively little effect.

\begin{figure}
\centering
\begin{tabular}{lp{10em}}
  \toprule
  Dataset & Induced Morpheme \\
  \midrule
  \multirow2*{Morphology}     & (``ed\$'', \{PAST\}) \\
                              & (``\,'\,'', \{POSSESSIVE\}) \\
  \midrule
  \multirow3*{Image captions} & (``stop sign'', \{stop sign\}) \\
                              & (``woman'', \{person\}) \\
                              & (``skier'', \{person, skis\}) \\
  \midrule
  \multirow2*{Translation}    & (``Member States'', \{Mitgliedstaaten\}) \\
  \bottomrule
\end{tabular}
\caption{Examples of morphemes induced from various human language datasets and tasks.}%
\label{fig:human}
\end{figure}


\subsection{Human language data}
\unskip\label{sec:human-lang}

In this section we discuss the results of running CSAR on three different human language datasets.
While these datasets are not the intended domain of CSAR---and CSAR is certainly not the best algorithm for the tasks---the point of these experiments is to demonstrate the general effectiveness of the algorithm qualitatively (examples shown in \cref{fig:human}).
Since these datasets are larger, we employed heuristic optimizations to CSAR to reduce their runtime (described in \cref{app:opt}).
The top $100$ induced morphemes for each human language dataset are given in \cref{app:human-language}.

\paragraph{Morpho Challenge}
The first human language dataset we use is from the Morpho Challenge \citep{kurimo-etal-2010-morpho}.
This dataset is a human language approximation of the task of morpheme induction for emergent language.
Concretely, the utterances are single English words, divided up at the character level, while the meanings are the constituent morphemes.
% Thus, most morphemes have form sizes greater than what need to be recognized in order to be induced correctly.

CSAR is able to recover a wide variety of morphemes including:
  roots like (``\^{}fire'', \{fire\}),
  prefixes like (``\^{}re'', \{re-\}),
  suffixes like (``ed\$'', \{PAST\}),
  and other affixes like (``\,'\,'', \{POSSESSIVE\}).
While the vast majority of morphemes CSAR induces are accurate, a handful of the lowest-weighted morphemes are spurious (e.g., (``s\$'', \{boy\})) likely due to inaccurate decoding earlier in the process (i.e., part of the true form for a given meaning was included in a prior meaning).


\paragraph{Image captions}
The next dataset we employ is the MS COCO dataset \citep[CC BY 4.0]{lin2015microsoftcococommonobjects}.
In particular, we take the image captions to be the utterances, treating words as atomic units, and the meaning to be the labeled objects in the image (e.g., person, cat).
% This dataset presents a wide variety of forms with a relatively small selection of meanings, making for an asymmetric mapping.

The bulk of highest weighted induced morphemes are direct equivalents of the objects they describe (e.g., (``cat'', \{cat\})).
% Further down, we begin to see closely associated words that are not direct equivalents such as (``court'', \{sports ball\}).
% and (``wii'', \{remote\})\footnotemark.  \footnotetext{As in the Nintendo Wii game console whose controller is often called a ``Wii Remote''.}
% We also find synonyms induced such as (``bicycle'', \{bicycle\}) and (``bike'', \{bicycle\}) while polysemic mappings we did not observe likely in part due to the small number of semantic categories in the dataset.
We find instances of synonymy (e.g., (``bicycle'', \{bicycle\}) and (``bike'', \{bicycle\})) as well as polysemy (e.g., (``animals'', \{cow\}) and (``animals'', \{sheep\})).
Finally, we also observe compound forms like (``stop sign'', \{stop sign\}) as well as compound meanings such as (``skier'', \{person, skis\}).
As we go beyond the top $100$ or so, the associations between forms and meanings remain reasonable but become looser such as (``bride'', \{dining table, tie\}) or (``sink'', \{toothbrush\}).

\paragraph{Machine translation}
For machine translation, we use the WMT16 dataset and the English--German split, in particular \citep{bojar-EtAl:2016:WMT1}.
In this case, the English text is considered to be the utterance and the German text to be the meaning, with words being the atomic units on both sides.
% While the image caption dataset was asymmetric with a wide variety of utterances corresponding to a handful of meanings, machine translation presents a more symmetric mapping, presenting a wide variety of utterances and meanings.

As with the image caption results, the bulk of induced morphemes are direct equivalents (e.g., (``and'', \{und\})).
Beyond these simple one-to-one mappings, CSAR induces the polysemic relationship (``the'', \{der\}) and (``the'', \{die\}).
Finally, CSAR also picks up on multi-token forms like (``Member States'', \{Mitgliedstaaten\}).
