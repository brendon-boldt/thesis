\section{Experiments}
\unskip\label{hpo:sec:experiments}
The code to run the experiments and analyses is publicly available at [supplementary material for review] under the MIT license.

\subsection{Hyperparameter searches}
In this paper, we present four main searches (Searches 1--4, parameters given in \Cref{tab:hp-search}) with two additional searches (Searches 5r and 6e) for use in later analyses (\Cref{hpo:sec:analysis}).
The following is a summary of the hyperparameter searches:
\smallskip
\begin{description}[nosep,leftmargin=0.2in]
  \item[Search 1] Large number of hyperparameters varied with a wide range; used small version of XferBench-da ($1$M train tokens for $1$ epoch, $200$k test tokens for $2$ epochs).
  \item[Search 2] Same number of hyperparameters varied with smaller or larger ranges depending on results of Search 1; used medium version of XferBench-da ($4$M train tokens for $2$ epochs, $1$M test tokens for $3$ epochs)
  \item[Search 3] Same parameters as Search 2 while allowing number of epochs to go higher and using the full version of XferBench-da ($15$M train tokens for $5$ epochs, $2$M test tokens for $10$ epochs).
  \item[Search 4] Reduces ranges or fixes parameters from Search 3 to maximize exploitation of good parameters; 4* in \Cref{tab:hp-search} is the best-performing trial from Search 4.
  \item[Search 5r] Most parameters varied with wide ranges except using \emph{random sampling} to remove sampling bias; similar to Search 1 with narrower ranges on learning rate. Discussed in \Cref{hpo:sec:ent-xb}.
  \item[Search 6e] Optimized for maximizing entropy after a number of previous searches (not discussed in the paper); similar to Search 4 in this regard. Discussed in \Cref{hpo:sec:ent-xb}.
\end{description}
The parameters of Searches 1--4 are given in \Cref{tab:hp-search} (for complete table, see \Cref{tab:hp-search-all}).
The implementation defaults for other hyperparameters were used unless otherwise specified.
Optuna's default parameters for TPE were used across all experiments.

The signaling game takes $5$ to $40$ minutes to run (depending primarily on the number of epochs, and, to a lesser extent, the message length), and the full version of XferBench-da takes approximately $40$ minutes to run.
Thus, the average trial (for the latter searches) takes approximately $[0.75,1.5]$ hours.
Parallelization was used to run multiple trials within a search at a time.
See \Cref{hpo:sec:resources} for a discussion of computing resources used.


\paragraph{Search design}
For each iteration of the primary searches (i.e., 1--4), we changed the search parameters based on their correlation with the objective function.
We observed four main univariate patterns\footnotemark{}, illustrated in \Cref{hpo:fig:hpo-slice}.
\footnotetext{While we did look for multivariate effects (i.e., hyperparameters that are \emph{not} independent), we did not observe any notable trends.}
For parameters with a clear trend toward the center (\Cref{hpo:fig:hpo-slice}a), we narrowed the range to encourage exploiting good values.
Some parameters trended to one side of the range (\Cref{hpo:fig:hpo-slice}b), which indicated needing to extend the range.
Parameters with weak to no trend (Figures~\ref{hpo:fig:hpo-slice}c and \ref{hpo:fig:hpo-slice}d) were left unchanged for the initial searches and given an arbitrary value for the final search to reduce additional noise.
Full hyperparameter plots given in \Cref{hpo:sec:hp-scatter}.

Searches 1 and 2 used a reduced version of XferBench to execute more trials quickly and prune the less promising hyperparameter ranges; nevertheless, caution was exercised in pruning since scaling up XferBench could change optimal hyperparameter values.
The irregular number of trials per search were due to executing as many trials as possible within a certain time (rather than aiming for a particular number of trials).


\subsection{Languages evaluated}

% For our primary evaluation of the success of hyperparameter search in finding the best emergent languages for transfer learning, we select from three categories of languages: human languages, those generated with the hyperparameter search discussed above, and extant emergent language corpora from ELCC \citep[\smallish\url{https://huggingface.co/datasets/bboldt/elcc}, CC BY 4.0]{elcc}.
We select three categories of languages to evaluate with XferBench:
  human languages, those generated with the hyperparameter search discussed above, and extant emergent language corpora from ELCC (\cref{ch:elcc}).
The primary goal is for the search-derived languages to outperform all existing emergent languages and get as close to human language performance as possible.
For the human languages, we use a subset of the baselines provided in \cref{ch:xferbench}.
In particular, we use Mandarin and Hindi because they were the best- and worst-performing human languages, respectively, and French and Arabic to round out the language families represented.

For the search-derived languages, we selected the three best languages from the final primary run of hyperparameter search (Search 4) and evaluate them on the full set of evaluation languages in XferBench.
We additionally include the three highest entropy languages from the entropy-maximizing search (Search 6e, discussed further in \Cref{hpo:sec:ent-xb}).

Finally, for the emergent language-based points of comparison, we select three of the best performing languages from ELCC\@.
Most notably, this includes Yao+ (\texttt{\smallish corpus-transfer-\allowbreak yao-et-al/\mbox{coco\_2014}} \citep{yao2022linking}) which performed far better than all other emergent languages on XferBench.
Mu+ (\texttt{\smallish generalizations-\allowbreak mu-goodman/\allowbreak cub-reference} \citep{mu2021generalizations}) and Chaabouni+ (\texttt{\smallish ec-at-scale/imagenet-10x10} \citep{chaabouni2022emergent}) were also included as more typical high-performing emergent languages on XferBench.


\subsection{Results}
\begin{figure}
  \centering
  % \includegraphics[width=0.9\linewidth]{example-image-duck}
  \inputHpo{assets/xb-bar.pgf}
  % Languages
  % - XB human languages
  % - ELCC top 3
  % - HPO top 3
  \caption{Bar chart of XferBench scores on emergent and human languages.  XB 1--3 are emergent language corpora derived from Search 4 and Entropy 1--3 from Search 6e.}
  \unskip\label{hpo:fig:bar}
\end{figure}

\Cref{hpo:fig:bar} shows $3$ randomly seeded runs of the full XferBench score for each corpus.
For the emergent languages from hyperparameter search, the models restored from checkpoints saved during the search, but the corpora were generated independently of the search.
First, we see that the emergent languages from the XferBench-based search (XB 1--3) outperform all other emergent languages and even the Hindi corpus.
While it is indeed significant that these emergent languages outperform a human language corpus, this corpus is also an outlier, and the emergent languages are still relatively far from matching the performance of the rest of the human language corpora.
Nevertheless, these figures show that the XB 1--3 languages achieve state-of-the-art levels of similarity to human language.
The corpora from the entropy-based search (Entropy 1--3) perform well, comparably to Yao+, but significantly worse than the XferBench-search languages.

% Mandarin      5.850
% Arabic        5.869
% French        5.893
% Hindi         5.958
% XB 1          5.940
% XB 2          5.938
% XB 3          5.945
% Entropy 1     5.974
% Entropy 2     5.980
% Entropy 3     5.975
% Yao+          5.971
% Mu+           6.021
% Chaabouni+    6.033
