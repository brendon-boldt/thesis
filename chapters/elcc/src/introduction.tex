\begin{abstract}
This chapter introduces the Emergent Language Corpus Collection (ELCC):
  a collection of corpora generated from open source implementations of emergent communication systems across the literature.
These systems include a variety of signalling game environments as well as more complex environments like a social deduction game and embodied navigation.
Each corpus is annotated with metadata describing the characteristics of the source system as well as a suite of analyses of the corpus (e.g., size, entropy, average message length, performance as transfer learning data).
Currently, research studying emergent languages requires directly running different systems which
  takes time away from actual analyses of such languages,
  makes studies which compare diverse emergent languages rare,
  and presents a barrier to entry for researchers without a background in deep learning.
The availability of a substantial collection of well-documented emergent language corpora, then, will enable research which can analyze a wider variety of emergent languages, which more effectively uncovers general principles in emergent communication rather than artifacts of particular environments.
We provide some quantitative and qualitative analyses with ELCC to demonstrate potential use cases of the resource in this vein.
\unskip\footnote{Based on ``ELCC: the Emergent Language Corpora Collection'' currently under review at ICLR 2025 \citep{boldt2024elcc}.}
\end{abstract}


\section{Introduction}
\unskip\label{elcc:sec:introduction}

When \citet{xferbench} introduced the metric called XferBench, they raised a question that they apparently could not answer:
how do emergent languages---communication systems that emerge from scratch in agent-based simulations---differ in their ``humanlikeness'' (as measured by their utility as pretraining data for NLP tasks).
It seems likely that they were unable to answer this question because no representative collection of samples from emergent languages existed.
The same problem plagues other research programs that seek to make generalizations about emergent languages, as a whole, rather than using a single type of environment as a proof of concept.
These include the degree to which emergent languages display entropic patterns similar to those that characterize words in human languages \citep{ueda2023on} and the kind of syntax that can be detected in emergent languages through grammar induction \citep{van-der-wal-etal-2020-grammar}.
We present an initial solution to this problem, namely the Emergent Language Corpus Collection (\theLib{}): a collection of 73 corpora generated from 7 representative emergent communication systems (ECSs).\footnotemark{}
\footnotetext{Emergent communications systems are more commonly referred to as simply ``environments''; we choose to use the term ``system'' in order to emphasize that what goes into producing an emergent language is more than just an environment including also the architecture of the agents, optimization procedure, datasets, and more.}
Prior to this work, comparing emergent languages entailed extensive work getting free and open source simulations to run---managing dependencies, manipulating output formats, etc.---before any data could even be generated.
The current work allows investigators, even those with very limited software engineering knowledge, to analyze a wide range of emergent languages straightforwardly, plowing over a barrier that has held back comparative emergent language research from its inception.
\theLib{} is published at \url{https://huggingface.co/datasets/bboldt/elcc} with data and code licensed under the CC BY 4.0 and  MIT licenses, respectively.

We discuss related work in \Cref{elcc:sec:related-work}.
\Cref{elcc:sec:design} lays out the design of \theLib{} while \Cref{elcc:sec:content} describes the content of the collection.
\Cref{elcc:sec:analysis} demonstrates some of the types of analyses enabled by ELCC\@.
\Cref{elcc:sec:discussion} presents some brief analyses, discussion, and future work related to \theLib{}.
Finally, we conclude in \Cref{elcc:sec:conclusion}.


\paragraph{Contributions}
The primary contribution of this paper is as a first-of-its kind data resource which will enable broader engagement and new research directions within the field of emergent communication.
Additionally, code published for reproducing the data resource also improve the reproducibility of existing ECS implementations in the literature, supporting further research beyond just the data resource itself.
Finally, the paper demonstrates some of the analyses uniquely made possible by a resource such as ELCC\@.

\section{Related Work}
\unskip\label{elcc:sec:related-work}

\paragraph{Emergent communication}
There is no direct precedent for this work in the emergent communication literature that we are aware of.
\citet{perkins2021texrel} introduces the TexRel dataset, but this is a dataset of observations for training ECSs, not data generated by them.
Some papers do provide the emergent language corpora generated from their experiments (e.g., \citet{yao2022linking}), although these papers are few in number and only include the particular ECS used in the paper.
At a high level, the EGG framework \citep{egg} strives to make emergent languages easily accessible,
  though instead of providing corpora directly, it provides a framework for implementing ECSs.
Thus, while EGG is useful for someone building new systems entirely, it is not geared towards research projects aiming directly at analyzing emergent languages themselves.

\paragraph{Data resources}
At a high level, \theLib{} is a collection of datasets, each of which represent a particular instance of a phenomenon (emergent communication, in this case).
On a structural level, \theLib{} is analogous to a collection of different human languages in a multi-lingual dataset.
\theLib{}, though, focuses more on a particular phenomenon of scientific interest, and, in this way, would be more analogous to work such as \citet{blum2023grammars}, which presents a collection of grammar snapshot pairs for $52$ different languages as instances of diachronic language change.
Similarly, \citet{zheng2024judging} present a dataset of conversations from Chatbot Arena, where ``text generated by different LLMs'' is the phenomenon of interest.
% \cmt{sort of like this \url{https://trainingdata.pro/datasets/llm-text-generation}; not sure if there is an academic version}
Furthermore, insofar as \theLib{} documents the basic typology of different ECSs, it is similar to the World Atlas of Language Structures (WALS) \citep{wals}.


%%% Local Variables:
%%% mode: latex
%%% TeX-master: t
%%% End:
