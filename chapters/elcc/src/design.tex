\section{Design}
\unskip\label{elcc:sec:design}

\subsection{Format}
\unskip\label{elcc:sec:format}

\theLib{} is a collection of ECSs, each of which has one or more associated \emph{variants} which correspond to runs of the system with different hyperparameter settings (e.g., different random seed, message length, dataset).
Each variant has metadata along with the corpus generated from its settings.
Each ECS has its own metadata as well and code to generate the corpus and metadata of each variant.
The file structure of \theLib{} is illustrated in \Cref{elcc:fig:structure}.

\begin{figure}
  \centering
  \small
  \fbox{\begin{minipage}{0.85\linewidth}
    \texttt{systems/}\dotfill top-level directory
    \\\null\quad\texttt{ecs-1/}\dotfill  directory for a particular ECS
    \\\null\quad\quad\texttt{metadta.yml}\dotfill metadata about the ECS
    \\\null\quad\quad\texttt{code/}\dotfill directory containing files to produce the data
    \\\null\quad\quad\texttt{data/}\dotfill directory containing corpus and metadata files
    \\\null\quad\quad\quad\texttt{hparams-1/}\dotfill directory for run with specific hyperparameters
    \\\null\quad\quad\quad\quad\texttt{corpus.jsonl}\dotfill corpus data
    \\\null\quad\quad\quad\quad\texttt{metadata.json}\dotfill metadata specific for corpus (e.g., metrics)
    \\\null\quad\quad\quad\texttt{hparams-2/}\dotfill \emph{as above}
    \\\null\quad\quad\quad\texttt{hparams-n/}\dotfill \emph{as above}
    \\\null\quad\texttt{ecs-2/}\dotfill \emph{as above}
    \\\null\quad\texttt{ecs-n/}\dotfill \emph{as above}
  \end{minipage}}
  \caption{The file structure of \theLib{}.}
  \unskip\label{elcc:fig:structure}
\end{figure}


\paragraph{ECS metadata}
Environment metadata provides a basic snapshot of a given system and where it falls in the taxonomy of ECSs.
As the collection grows, this structure makes it easier to ascertain the contents of the collection and easily find the most relevant corpora for a given purpose.
This metadata will also serve as the foundation for future analyses of the corpora by looking at how the characteristics of an ECS relate to the properties of its output.
These metadata include:
\begin{itemize}
  \item Source information including the original repository and paper of the ECS\@.
  \item High-level taxonomic information like game type and subtype.
  \item Characteristics of observation; including natural versus synthetic data, continuous versus discrete observations.
  \item Characteristics of the agents; including population size, presence of multiple utterances per episode, presence of agents that send \emph{and} receive messages.
  \item Free-form information specifying the particular variants of the ECS and general notes about the ELCC entry.
\end{itemize}
A complete description is given in \Cref{elcc:sec:md-spec}.
These metadata are stored as YAML files in each ECS directory.
A Python script is provided to validate these entries against a schema.
See \Cref{elcc:sec:ecs-md} for an example of such a metadata file.

\paragraph{Corpus}
Each \emph{corpus} comprises a list of \emph{lines} each of which is, itself, an array of \emph{tokens} represented as integers.
Each line corresponds to a single episode or round in the particular ECS.
In the case of multi-step or multi-agent systems, this might comprise multiple individual utterances which are then concatenated together to form the line (no separation tokens are added).
Each corpus is generated from a single run of the ECS; that is, they are never aggregated from distinct runs of the ECS\@.

Concretely, a \emph{corpus} is formatted as a JSON lines (JSONL) file where each \emph{line} is a JSON array of integer \emph{tokens} (see \Cref{elcc:fig:quale} for an example of the format).
There are a few advantages of JSONL:
  (1) it is a human-readable format,
  (2) it is JSON-based, meaning it is standardized and has wide support across programming languages,
  and (3) it is line-based, meaning it is easy to process with command line tools.\footnote{E.g., Creating a $100$-line random sample of a dataset could be done with \texttt{\small shuf dataset.jsonl | head -n 100 > sample.jsonl}}
Corpora are also available as single JSON objects (i.e., and array of arrays), accessible via the Croissant ecosystem \citep{croissant}.


\paragraph{Corpus analysis}
For each corpus in \theLib{} we run a suite of analyses to produce a quantitative snapshot.
This suite metrics is intended not only to paint a robust a picture of the corpus but also to serve as jumping-off point for future analyses on the corpora.
Specifically, we apply the following to each corpus:
  token count,
  unique tokens,
  line count,
  unique lines,
  tokens per line,
  tokens per line stand deviation,
  $1$-gram entropy,
  normalized $1$-gram entropy,
  entropy per line,
  $2$-gram entropy,
  $2$-gram conditional entropy,
  EoS token present,
  and EoS padding.
\emph{Normalized $1$-gram entropy} is computed as \emph{$1$-gram entropy} divided by the maximum entropy given the number of unique tokens in that corpus.

We consider an EoS (end-of-sentence) token to be present when:
  (1) every line ends with token consistent across the entire corpora,
  and (2) the first occurrence of this token in a line is only ever followed by more of the same token.
For example, \texttt{\small0} could be an EoS token in the corpus \texttt{\small[[1,2,0],[1,0,0]]} but not \texttt{\small[[1,2,0],[0,1,0]]}.
EoS padding is defined as a corpus having an EoS token, all lines being the same length, and the EoS token occurs more than once in a line at least once in the corpus.

Additionally, each corpus also has a small amount of metadata copied directly from the output of the ECS\@; for example, this might include the success rate in a signaling game environment.
We do not standardize this because it can vary widely from ECS to ECS, though it can still be useful for comparison to other results among variants within an ECS\@.

\paragraph{Reproducibility}
\theLib{} is designed with reproducibility in mind.
With each ECS, code is included to reproduce the corpora and analysis metadata.
Not only does this make \theLib{} reproducible, but it sometimes helps the reproducibility of the underlying implementation insofar as it fixes bugs, specifies Python environments, and provides examples of how to run an experiment with a certain set of hyperparameters.
Nevertheless, in this code, we have tried to keep as close to the original implementations as possible.
When the underlying implementation supports it, we set the random seed (or keep the default) for the sake of consistency, although many systems do not provide a way to easily set this.

% The web interface (discussed below) will additionally be made available on the World Wide Web.
% \cmt{Remove if not present.}

% \paragraph{Web interface}
% In order to make \theLib{} more user-friendly, we develop a simple web interface (i.e., website) for interacting with the data in the collection.
% This includes displaying entries in a tabular format with the accompanying metadata so that it is possible to search, sort, and filter the entries.
% Selecting the individual entries shows the user details about the entry, an in-browser sample of the emergent language, and the ability to download the particular entry.
% \cmt{Remove if not present.}
