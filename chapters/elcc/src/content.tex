\section{Content}
\unskip\label{elcc:sec:content}

\begin{table}
  \centering
  \begin{tabular}{lllllr}
    \toprule
    Source & Type & Data source & Multi-agent & Multi-step & $n$ corp. \\
    \midrule
    % \citet{egg}, discrim.         & signalling   & synthetic & $3$ \\
    \citet{egg}                       & signalling   & synthetic & No & No & $15$ \\
    \citet{yao2022linking}            & signalling   & natural   & No & No & $2$ \\
    \citet{mu2021generalizations}     & signalling   & both      & No & No & $6$ \\
    \citet{chaabouni2022emergent}     & signalling   & natural   & Yes & No & $5$ \\
    \citet{unger2020GeneralizingEC}   & navigation   & synthetic & No & Yes & $18$ \\
    \citet{boldt2022MathematicallyMT} & navigation   & synthetic & No & Yes & $20$ \\
    \citet{brandizzi2022rlupus}       & conversation & ---       & Yes & Yes & $7$ \\
    \bottomrule
  \end{tabular}
  \medskip
  \caption{Taxonomic summary the contents of \theLib{}.}
  \unskip\label{tab:tax-sum}
\end{table}


\theLib{} contains $73$ corpora across $8$ ECSs taken from the literature for which free and open source implementations were available.
With our selection we sought to capture variation across a three distinct dimensions:
\begin{enumerate}
  \item Variation across ECSs generally, including elements like game types, message structure, data sources, and implementation details.
  \item Variation among different hyperparameter settings within an ECS, including message length, vocabulary size, dataset, and game difficulty.
  \item Variation within a particular hyperparameter setting that comes from inherent stochasticity in the system; this is useful for gauging the stability or convergence of an ECS\@.
\end{enumerate}
\Cref{tab:tax-sum} shows an overview of the taxonomy of \theLib{} based on the ECS-level metadata.
In addition to this, \Cref{tab:quant-sum} provides a quantitative summary of the corpus-level metrics described in \Cref{elcc:sec:format}.
% The following sections describe the individual ECSs in more detail.
We separate the discussion of particular systems into two subsections: signalling games (\Cref{elcc:sec:signalling}) and its variations which represent a large proportion of system discussed in the literature and other games (\Cref{elcc:sec:other-games}) which go beyond the standard signalling framework.


\subsection{Scope}
The scope of the contents of \theLib{} is largely the same as discussed in reviews such as \citet{lazaridou2020review} and \citet[Section 1.2]{boldt2024review}.
This comprises agent-based models for simulating the formation of ``natural'' language from scratch using deep neural networks.
Importantly, \emph{from scratch} means that the models are not pretrained or tuned on human language.
Typically, such simulations make use of reinforcement learning to train the neural networks, though this is not a requirement in principle.

One criterion that we do use to filter ECSs for inclusion is its suitability for generating corpora as described above.
This requires that the communication channel is discrete, analogous to the distinct words/morphemes which for the units of human language.
This excludes a small number of emergent communication papers have approached emergent communication through constrained continuous channels like sketching \citep{mihai2021learning} or acoustic-like signals \citep{eloff2023learning}.
Other systems use discrete communication but have episodes with only a single, one-token message (e.g., \citet{tucker2021discrete}), which would have limited applicability to many research questions in emergent communication.


\subsection{Signalling games}
\unskip\label{elcc:sec:signalling}
The \emph{signalling game} (or \emph{reference game}) \citep{lewis1970ConventionAP} represents a plurality, if not majority, of the systems present in the literature.
A brief, non-exhaustive review of the literature yielded $43$ papers which use minor variations of the signalling game, a large number considering the modest body of emergent communication literature (see \Cref{elcc:sec:sg-list}).
% \unskip\footnote{\protect\citet{2106.02668, 2203.13176, 2203.13344, 2204.12982, 2112.14518, 2111.06464, 2109.06232, 2108.01828, 2106.04258, 2103.08067, 2103.04180, 2101.10253, 2012.10776, 2012.02875, 2011.00890, 2010.01878, 2008.09866, 2008.09860, 2005.00110, 2004.09124, 2004.03868, 2004.03420, 2002.01365, 2002.01335, 2002.01093, 2001.08618, 2001.03361, 1911.05546, 1911.01892, 1910.05291, 1909.11060, 1906.02403, 1905.13687, 1905.12561, 1812.01431, 1808.10696, 1804.03984, 1705.11192, 1612.07182, 2302.08913, 2211.02412, 2209.15342, 2207.07025}}
The basic format of the signalling game is a single round of the \emph{sender} agent making an observation, passing a message to the \emph{receiver} agent, and the receiver performing an action based on the information from the message.
The popularity of this game is, in large part, because of its simplicity in both concept and implementation.
Experimental variables can be manipulated easily while introducing minimal confounding factors.
Furthermore, the implementations can entirely avoid the difficulties of reinforcement learning by treating the sender and receiver agents as a single neural network, resulting in autoencoder with a discrete bottleneck which can be trained with backpropagation and supervised learning. 

The two major subtypes of the signalling game are the \emph{discrimination game} and the \emph{reconstruction game}.
In the discrimination game, the receiver must answer a multiple-choice question, that is, select the correct observation from among incorrect ``distractors''.
In the reconstruction game, the receiver must recreate the input directly, similar to the decoder of an autoencoder.

\paragraph{Vanilla}
For the most basic form of the signalling game, which we term ``vanilla'', we use the implementation provided in the Emergence of lanGuage in Games (EGG) framework \citep[MIT license]{egg}.
It is vanilla insofar as it comprises the signalling game with the simplest possible observations (synthetic, concatenated one-hot vectors), a standard agent architecture (i.e., RNNs), and no additional dynamics or variations on the game.
Both the discrimination game and the reconstruction game are included.
This system provides a good point of comparison for other ECSs which introduce variations on the signalling game.
The simplicity of the system additionally makes it easier to vary hyperparameters: for example, the size of the dataset can be scaled arbitrarily and there is no reliance on pretrained embedding models.
% both Gumbel-Softmax and REINFORCE \cmt{check} optimization methods are available,

\paragraph{Natural images}
``Linking emergent and natural languages via corpus transfer'' \citep[MIT license]{yao2022linking} presents a variant of the signalling game which uses embeddings of natural images as the observations.
In particular, the system uses embedded images from the MS-COCO and Conceptual Captions datasets consisting of pictures of everyday scenes.
Compared to the uniformly sampled one-hot vectors in the vanilla setting,
  natural image embeddings are real-valued with a generally smooth probability distribution rather than being binary or categorical.
Furthermore, natural data distributions are not uniform  and instead have concentrations of probability mass on particular elements; this non-uniform distribution is associated with various features of human language (e.g., human languages' bias towards describing warm colors \citep{gibson2017color,zaslavsky2018color}).


\paragraph{Concept-based observations}
``Emergent communication of generalizations'' \citep[MIT license]{mu2021generalizations} presents a variant of the discrimination signalling game which they term the \emph{concept game}.
The concept game changes the way that the sender's observation corresponds with the receiver's observations.
In the vanilla discrimination game, the observation the sender sees is exactly the same as the correct observation that the receiver sees.
In the concept game, the sender instead observes a set of inputs which share a particular concept (e.g., red triangle and red circle are both red), and the correct observation (among distractors) shown to the receiver contains the same concept (i.e., red) while not being identical to those observed by the sender.
The rationale for this system is that the differing observations will encourage the sender to communicate about abstract concepts rather than low-level details about the observation.
This ECS also presents the vanilla discrimination game as well as the \emph{set reference game}, which is similar to the reference game except that the whole object is consistent (e.g., different sizes and locations of a red triangle).

\paragraph{Multi-agent population}
``Emergent communication at scale'' \citep[Apache 2.0-license]{chaabouni2022emergent} presents a signalling game system with populations of agents instead of the standard fixed pair of sender and receiver.
For each round of the game, then, a random sender is paired with a random receiver.
This adds a degree of realism to the system, as natural human languages are developed within a population and not just between two speakers (cf.\@ idioglossia).
More specifically, language developing among a population of agents prevents some degree ``overfitting'' between sender and receiver;
  in this context, having a population of agents functions as an ensembling approach to regularization.

\subsection{Other games}
\unskip\label{elcc:sec:other-games}
Considering that the signalling game is close to the simplest possible game for an ECS, moving beyond the signalling game generally entails an increase in complexity.
There is no limit to the theoretical diversity of games, although some of the most common games that we see in the literature are
  conversation-based games (e.g., negotiation, social deduction) and navigation games.
These games often introduce new aspects to agent interactions like:
  multi-step episodes,
  multi-agent interactions,
  non-linguistic actions,
  and embodiment.

These kinds of systems, as a whole, are somewhat less popular in the literature.
On a practical level, more complex systems are more difficult to implement and even harder to get to converge reliably---many higher-level behaviors, such as planning or inferring other agent's knowledge, are difficult problems for reinforcement learning in general, let alone with discrete multi-agent emergent communication.
On a methodological level, more complexity in the ECS makes it harder to formally analyze the system as well as eliminate confounding factors in empirical investigation.
With so many moving parts, it can be difficult to prove that some observed effect is not just a result of some seemingly innocent hyperparameter choice (e.g., learning rate, samples in the rollout buffer) \citep{boldt2022MathematicallyMT}.
Nevertheless, we have reason to believe that these complexities are critical to understanding and learning human language as a whole \citep{bisk-etal-2020-experience}, meaning that the difficulties of more complex systems are worth overcoming as they are part of the process of creating more human-like emergent languages, which are more informative for learning about human language and more suitable for applications in NLP\@.


\paragraph{Grid-world navigation}
``Generalizing Emergent Communication'' \citep[BSD-3-clause license]{unger2020GeneralizingEC} introduces an ECS which takes some of the basic structure of the signalling game and applies it to a navigation-based system derived from the synthetic Minigrid/BabyAI environment \citep{chevalier2018babyai,MinigridMiniworld23}.
A sender with a bird's-eye view of the environment sends messages to a receiver with a limited view who has to navigate to a goal location.
Beyond navigation, some environments present a locked door for which the receiver must first pick up a key in order to open.
What distinguishes this system most from the signalling game is that it is multi-step and embodied such that the  utterances within an episodes are dependent on each other.
Among other things, this changes the distribution properties of the utterances.
For example, if the receiver is in Room A at timestep $T$, it is more likely to be in Room A at timestep $T+1$; thus if utterances are describing what room the receiver is in, this means that an utterance at $T+1$ has less uncertainty given the content of an utterance at $T$.
Practically speaking, the multiple utterances in a given episode are concatenated together to form a single line in the corpus in order to maintain the dependence of later utterances on previous ones.

\paragraph{Continuous navigation}
``Mathematically Modeling the Lexicon Entropy of Emergent Language'' \citep[GPL-3.0 license]{boldt2022MathematicallyMT} introduces a simple navigation-based ECS which is situated in a continuous environment.
A ``blind'' receiver is randomly initialized in an obstacle-free environment and must navigate toward a goal zone guided by messages from the sender which observes the position of the receiver relative to the goal.
The sender sends a single discrete token at each timestep, and a line in the dataset consists of the utterances from each timestep concatenated together.
This system shares the time-dependence between utterances of the grid-world navigation system although with no additional complexity of navigating around obstacle, opening doors, etc.
On the other hand, the continuous nature of this environment provides built-in stochasticity since there are (theoretically) infinitely many distinct arrangements of the environment that are possible, allowing for more natural variability in the resulting language.

\paragraph{Social deduction}
``RLupus: Cooperation through the emergent communication in The Werewolf social deduction game'' \citep[GPL-3.0 license]{brandizzi2022rlupus} introduces an ECS based on the social deduction game \emph{Werewolf} (a.k.a., \emph{Mafia}) where, through successive rounds of voting and discussion, the ``werewolves'' try to eliminate the ``villagers'' before the villagers figure out who the werewolves are.
In a given round, the discussion takes the form of all agents broadcasting a message to all other agents after which a vote is taken on whom to eliminate.
As there are multiple rounds in a given game, this system introduces multi-step as well as multi-speaker dynamics into the language.
Furthermore, the messages also influence distinct actions in the system (i.e., voting).
These additional features in the system add the potential for communication strategies that are shaped by a variety of heterogeneous factors rather than simply the distribution of observations (as in the signalling game).
