\usepackage{hyperref}
\usepackage{url}
\usepackage{xurl}
\usepackage{latexsym,amsmath}
\usepackage{booktabs}
\usepackage{graphicx}
\usepackage{subcaption}
\usepackage[nameinlink]{cleveref}
\usepackage{enumitem}
\usepackage{xcolor}
\usepackage{colortbl}
\usepackage{pgfplots}
\usepackage{pgfplotstable}
\usetikzlibrary{shapes,arrows,shadows.blur,calc,positioning,fit}

\pgfplotsset{compat=1.18}

\usepackage{enumitem}
\usepackage{multirow}

\DeclareMathOperator*\mean{\text{mean}}
\DeclareMathOperator*\stdev{\text{stdev}}

\pgfplotsset{
  discard if not/.style 2 args={
    x filter/.code={
      \edef\tempa{\thisrow{#1}}
      \edef\tempb{#2}
      \ifx\tempa\tempb
      \else
        \def\pgfmathresult{inf}
      \fi
    }
  }
}


%%% End Commands %%%

% From Siavoosh Payandeh Azad: https://www.siavoosh.com/blog/2019/01/05/latex-table-cell-coloring-based-on-values-in-the-cell/

% \colorlet{BestColor}{blue!40}
% \colorlet{WorstColor}{red!40}
\definecolor{BestColor}{rgb}{0.5, 0.7, 1.0}
\definecolor{WorstColor}{rgb}{1.0, 0.7, 0.5}

\newcommand{\gradientcell}[3]{
    % The values are calculated linearly between \minval and \maxval
    % \pgfmathparse{int(round(100*(#1/(#3-#2))-(\minval*(100/(#3-#2)))))}
    \pgfmathparse{int(round(100*(#3-#1)/(#2-#1)))))}
      \xdef\tempa{\pgfmathresult}
      % \cellcolor{blue!40!white!\tempa!red!40}
      \cellcolor{BestColor!\tempa!WorstColor}
      % \cellcolor{#5!\tempa!#4!#6} #1
 }

\tikzset{/num grad/.style n args={3}{
  column name=#1,
  column type=r, fixed, fixed zerofill, precision=2,
  postproc cell content/.append style={
    /pgfplots/table/@cell content/.add={\gradientcell{#2}{#3}{##1}}{},
  },
}}

