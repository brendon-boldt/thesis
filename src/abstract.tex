\null\vfill

\section*{Thesis Statement}
Advancing science and engineering with deep learning-based emergent language requires goal-oriented, environment-agnostic analytical algorithms.

\section*{Abstract}
Emergent communication is the field of research which studies how human language-like communication systems evolve from scratch in agent-based simulations.
The most recent incarnation of this topic, starting in 2016, has focused on leveraging recent advancements in deep neural networks, reinforcement learning, and natural language processing.
Emergent communication, as a paradigm, has significant potential applications from powering unprecedentedly detailed simulations of how humans invent, acquire, and use language
  to providing rich, embodied language data for training and evaluating models free of contamination or surveillance.
Despite this potential, the field has yet to make progress towards the more revolutionary applications because it lacks the methodological resources to enable cumulative research.
The design space of emergent language is dizzyingly large, and the outputs of said environments are similarly difficult to interpret.
As a result, analyses of emergent languages, and consequently their findings findings, are often ``one-off'', unable build on prior work or produce generalizable contributions.

This thesis, then, advances the field of emergent communication by developing the resources necessary for a cumulative research paradigm, something that is critical to the advancement of any area of science or engineering.
Specifically, it introduces analytical methods for deep learning-based emergent communication that enable measurable progress in the field with a particular mind towards solving practical applications and improving scientific understanding.
I first review the particular ways in which emergent communication can solve practical applications and improve scientific understanding.
This is followed by introducing emergent language data resources which enable empirical evaluation across a variety of emergent languages.
These resources are then used to develop
  (1) a deep transfer learning-based evaluation metric for emergent communication to measure the practical applicability of emergent language
  and (2) algorithms for discovering the morphology of emergent languages as a foundation for further linguistic analysis.

\vfill
