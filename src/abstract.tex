\null\vfill

\section*{Thesis Statement}
Advancing science and engineering with emergent language requires principled, environment-agnostic analytical algorithms.

\section*{Abstract}
Emergent communication is the field of research which studies how human language-like communication systems evolve from scratch in agent-based simulations.
The most recent incarnation of this topic, starting in 2016, has focused on leveraging recent advancements in deep neural network, reinforcement learning, and natural language processing.
Emergent communication, as a method, has significant potential applications from powering unprecedentedly detailed simulations of how humans invent, acquire, and use language
  to providing an alternative to extracting language data from humans to train large language models.
Despite this potential, the field has yet to make any significant progress towards these applications largely because it lacks any methodological resources to unify research efforts within the field;
  that is, research findings are often ``one-off'', lacking any way of making general claims or building on prior work.

This thesis, then, advances the field of emergent communication by developing the resources that are necessary for forging a unified research program, something that is critical to the advancement of any area of science or engineering.
Specifically, it establishes methods in emergent communication that enable measurable progress in emergent language research so as to move the field towards solving practical applications and improving scientific understanding.
It does this by first introducing emergent language data resources which enable empirical evaluation across a variety of emergent languages.
These resources are then used to develop
  (1) a deep transfer learning-based evaluation metric for emergent communication to measure the practical applicability of emergent language
  and (2) algorithms for discovering the morphology of emergent languages as a foundation for further linguistic analysis.

\vfill
