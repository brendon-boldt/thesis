\chapter{Introduction}

Modern-day large language model-based AI systems are good a mimicking human language.
Some might even say they are good at \emph{using} human language, but this either imprecise or inaccurate:
  \cmt{why}.

\cmt{emergent communication a solution}

\cmt{EC is not in a great place}

\section{Background}

Emergent communication (also known as ``emergent language'') is the area of research which studies systems of communication between virtual agents that has emerged \emph{de novo} from interactions and optimizations (in the machine learning sense) in simulated environments.
Interest in this field from a deep learning perspective---which this thesis focuses on, although not exclusively---appeared in 2016 and 2017 with \cmt{papers}, and has maintained a niche though consistent following since that time.

Going backwards in time, there are papers addressing emergent communication predating deep learning methods \cmt{examples}.
More importantly, this field is strongly rooted (conceptually, though often not methodologically) in research on the origin and evolution of language in humans \cmt{citations}.

This thesis is primarily concerned with the scope of deep learning-based emergent communication.
These related fields are critical resources in developing the ideas presented hear and we hope, also, that this work will be helpful to these fields.
Nevertheless, there are many particular concerns and techniques that are unique to the deep learning-based approaches, and these are the concerns and techniques that I will primarily address.



\section{Motivation}

\cmt{discuss importance of metrics}
\cmt{emphasis of evaluation}
\cmt{discuss metrics from TMLR paper}
\cmt{why principled?}


\section{Summary of Thesis}

\begin{enumerate}
  \item library and resources
  \item deep learning-based evaluation
  \item linguistics-based evaluation
\end{enumerate}

\begin{table}
  \centering
  \begin{tabular}{lll}
  \toprule
  2024 & August & Proposal \\
  & August--September & XferBench eval (\Cref{ch:xferbench-analysis}) \\
  & October--December & Rich corpora (\Cref{sec:rich-corpora}) \\
  2025 & January--March & Linguistic universals (\Cref{ch:universals}) \\
  & April--June & Meta-analysis (\Cref{ch:meta-analysis}) \\
  & July & Defense \\
  \bottomrule
  \end{tabular}

  \caption{Timeline}
\end{table}

%\lipsum[1-7]{}

